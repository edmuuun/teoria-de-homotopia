\documentclass[brazilian]{book}

\usepackage{babel}
\usepackage[utf8]{inputenc}
\usepackage{csquotes}
\usepackage[margin=1.35in]{geometry}
\usepackage[backend=biber,style=alphabetic]{biblatex}
\usepackage{mathtools,amssymb,amsthm}
\usepackage{indentfirst}
\usepackage[toc,page]{appendix}
\usepackage{hyperref}
\usepackage{tikz-cd}
\usepackage{graphicx}
\usepackage[capitalize,noabbrev]{cleveref}

\usetikzlibrary{babel}

\addbibresource{bibliografia.bib}

\swapnumbers
\newtheorem{teo}{Teorema}[section]
\newtheorem{prop}[teo]{Proposição}
\newtheorem{lema}[teo]{Lema}
\newtheorem{corol}[teo]{Corolário}

\theoremstyle{definition}
\newtheorem{defin}[teo]{Definição}
\newtheorem{obs}[teo]{Observação}
\newtheorem{exem}[teo]{Exemplo}

\DeclarePairedDelimiter{\abs}{\lvert}{\rvert}
\DeclarePairedDelimiter{\Abs}{\lVert}{\rVert}

\newcommand{\id}{\mathrm{id}}
\newcommand{\ct}{\mathrm{ct}}
\newcommand{\Mor}{\mathrm{Mor}}
\newcommand{\Hom}{\mathrm{Hom}}
\newcommand{\Arr}{\mathrm{Arr}}
\newcommand{\cofib}{\rightarrowtail}
\newcommand{\fib}{\twoheadrightarrow}
\newcommand{\dom}{\mathrm{dom}}
\newcommand{\cod}{\mathrm{cod}}
\newcommand{\comp}{\mathrm{comp}}
\newcommand{\fac}{\mathrm{fac}}
\newcommand{\init}{\mathrm{init}}
\newcommand{\Fun}{\mathrm{Fun}}
\newcommand{\Cyl}{\mathrm{Cyl}}

\renewcommand{\qedsymbol}{$\blacksquare$}

\title{Teoria de Homotopia Abstrata}
\author{Edmundo B. de C. Martins}
\date{\today}

\begin{document}

\maketitle

\section{Categorias modelo}

\begin{defin}
  Seja $\mathsf{M}$ uma categoria localmente pequena, completa e co-completa.
  Uma \textbf{estrutura modelo} em $\mathsf{M}$ consiste de três classes de morfismos $\mathcal{W},\, \mathcal{F},\, \mathcal{C} \subseteq \mathrm{Mor}(\mathsf{M})$ cujos elementos são chamados, respectivamente, \textbf{equivalências fracas}, \textbf{fibrações} e \textbf{cofibrações}, as quais devem satisfazer as seguintes condições:
  \begin{enumerate}
  \item[(M1)] A categoria $\mathsf{M}$ é bicompleta, ou seja, admite todos os limites e colimites indexados por categorias pequenas.
    
  \item[(M2)] (Propriedade 2-de-3) Dados morfismos $f: X \to Y$ e $g: Y \to Z$ em $\mathsf{M}$, se dois dos morfismos do conjunto $\{f,\,g,\, g \circ f\}$ estiverem em $\mathcal{W}$, então o terceiro também deve estar.
    
  \item[(M3)] (Propriedade de retração) Se um morfismo $f: A \to X$ é retração de um outro morfismo $g: B \to Y$, ou seja, se existe um diagrama comutativo como abaixo,
    \begin{displaymath}
      \begin{tikzcd}
        A
        \arrow[r]
        \arrow[d, "f" swap]
        \arrow[rr, bend left=25, "\id_{A}"]
        & B
        \arrow[r]
        \arrow[d, "g"]
        & A
        \arrow[d, "f"]
        \\ X
        \arrow[r]
        \arrow[rr, bend right=25, "\id_{X}" swap]
        & Y
        \arrow[r]
        & X
      \end{tikzcd}
    \end{displaymath}
    e $g$ pertence a $\mathcal{W}$ (ou a $\mathcal{F}$, ou a $\mathcal{C}$), então $f$ também pertence a $\mathcal{W}$ (ou a $\mathcal{F}$, ou a $\mathcal{C}$, respectivamente).
    Em suma, as classes $\mathcal{W}$, $\mathcal{F}$ e $\mathcal{C}$ são fechadas por retrações.
    
  \item[(M4)] (Propriedade de levantamento) Dado um diagrama comutativo como abaixo,
    \begin{displaymath}
      \begin{tikzcd}
        A
        \arrow[r]
        \arrow[d, "i" swap]
        & X
        \arrow[d, "p"]
        \\ B
        \arrow[r]
        & Y
      \end{tikzcd}
    \end{displaymath}
    onde $i$ é uma cofibração, e $p$ é uma fibração; se um dos dois morfismos $i$ ou $p$ é também uma equivalência fraca, então o diagrama admite um levantamento, ou seja, existe um morfismo $f: B \to X$ que faz comutar o diagrama abaixo.
    \begin{displaymath}
      \begin{tikzcd}
        A
        \arrow[r]
        \arrow[d, "i" swap]
        & X
        \arrow[d, "p"]
        \\ B
        \arrow[r]
        \arrow[ru, dashed, "f" description]
        & Y
      \end{tikzcd}
    \end{displaymath}
    
  \item[(M5)] (Propriedade de fatoração) Qualquer morfismo $f: X \to Y$ em $\mathsf{M}$ pode ser fatorado nas duas formas mostradas abaixo,
    \begin{displaymath}
      \begin{tikzcd}
        X
        \arrow[rr, "f"]
        \arrow[rd, "i" swap]
        & & Y
        \\ & \widehat{X}
        \arrow[ru, "p" swap]
      \end{tikzcd}
      \qquad
      \begin{tikzcd}
        X
        \arrow[rr, "f"]
        \arrow[rd, "j" swap]
        & &
        Y
        \\ & \widetilde{Y}
        \arrow[ru, "q" swap]
      \end{tikzcd}
    \end{displaymath}
    onde $p$ é simultaneamente uma fibração e uma equivalência fraca, enquanto $j$ é simultaneamente uma cofibração e uma equivalência fraca.
  \end{enumerate}
\end{defin}

Vamos introduzir um pouco de terminologia antes de fazermos alguns comentários sobre a definição acima.
Os morfismos de $\mathsf{M}$ que pertencem à classe $\mathcal{W} \cap \mathcal{F}$ são chamados de \textbf{fibrações triviais} ou \textbf{fibrações acíclicas}, enquanto os morfismos que pertencem à classe $\mathcal{W} \cap \mathcal{C}$ são chamados de \textbf{cofibrações triviais} ou \textbf{cofibrações acíclicas}.
Usando essa terminologia o axioma de fatoração (M5) pode ser enunciado da seguinte forma: todo morfismo em uma categoria modelo pode ser fatorado como uma cofibração seguido de uma fibração trivial, ou como uma cofibração trivial seguido de uma fibração.

\begin{obs}\label{obs:axioma_retracao_explicacao}
  Lembremos que, dados objetos $X$ e $Y$ de uma categoria $\mathsf{C}$ qualquer, dizemos que $X$ é um \textbf{retrato} de $Y$ se existem morfismos $s: X \to Y$ e $r: Y \to X$ tais que $r \circ s = \id_{X}$.
  Comumente nos referimos ao morfismo $s$ por \textbf{seção} e ao morfismo $r$ por \textbf{retração}.
  A condição $r \circ s = \id_{X}$ garante que $s$ seja um monomorfismo.
  De fato, se $f,\, g: W \to X$ são morfismos tais que $s \circ f = s \circ g$, então
  \begin{displaymath}
    f
    = \id_{X} \circ f
    = r \circ s \circ f
    = r \circ s \circ g
    = \id_{X} \circ g
    = g.
  \end{displaymath}
  Isso nos permite encarar $X$ como um subobjeto de $Y$, e o morfismo $r$ então intuitivamente deforma $Y$ para esse subobjeto, mas de forma a mantê-lo fixado.
  Note que a condição $r \circ s = \id_{X}$ garante também que o morfismo $r$ seja um epimorfismo.

  A noção de retração que aparece no axioma (M3) de uma estrutura modelo enunciado acima pode ser interpretada nesse sentido em uma categoria adequada.
  Lembremos que toda categoria $\mathsf{C}$ dá origem a uma categoria de setas $\Arr(\mathsf{C})$.
  Os objetos dessa categorias são precisamente morfismos $f: A \to B$ na categoria incial $\mathsf{C}$, e dados dois tais objetos $f: A \to B$ e $g: X \to Y$, um morfismo do tipo $(f: A \to B) \to (g: X \to Y)$ na categoria de setas $\Arr(\mathsf{C})$ é dado por um par de morfismos $(\alpha: A \to X,\, \beta: B \to Y)$ satisfazendo a igualdade $\beta \circ f = g \circ \alpha$.
  Podemos então visualizar esse morfismo em $\Arr(\mathsf{C})$ na forma de um quadrado comutativo como mostrado abaixo.
  \begin{displaymath}
    \begin{tikzcd}
      A
      \arrow[d, "f" swap]
      \arrow[r, "\alpha"]
      & X
      \arrow[d, "g"]
      \\ B
      \arrow[r, "\beta" swap]
      & Y
    \end{tikzcd}
  \end{displaymath}
  A composição de morfismos é definida ``colando'' quadrados comutativos adjacentes.
  Mais precisamente, dados três objetos $f: X_{1} \to Y_{1},\, g: X_{2} \to Y_{2}$ e $h: X_{3} \to Y_{3}$ na categoria $\Arr(\mathsf{C})$, e dados também dois morfismos componíveis
  \begin{displaymath}
    (\alpha_{1}: X_{1} \to X_{2},\, \beta_{1}: Y_{1} \to Y_{2}) \qquad (\alpha_{2}: X_{2} \to X_{3},\, \beta_{2}: Y_{2} \to Y_{3}),
  \end{displaymath}
  sua composição é o morfismo
  \begin{displaymath}
    (\alpha_{2}, \beta_{2}) \circ (\alpha_{1},\beta_{1}): (f: X_{1} \to Y_{1}) \to (h: X_{3} \to Y_{3})
  \end{displaymath}
  em $\Arr(\mathsf{C})$ definido pelo par
  \begin{displaymath}
    (\alpha_{2},\beta_{2}) \circ (\alpha_{1},\beta_{1}) \coloneqq (\alpha_{2} \circ \alpha_{1}: X_{1} \to X_{3},\, \beta_{2} \circ \beta_{1}: Y_{1} \to Y_{3}).
  \end{displaymath}
  Essa composição pode também ser visualizada como mostrado abaixo.
  \begin{displaymath}
    \begin{tikzcd}
      X_{1}
      \arrow[d, "f" swap]
      \arrow[r, "\alpha_{1}"]
      & X_{2}
      \arrow[d, "g"]
      \arrow[r, "\alpha_{2}"]
      & X_{3}
      \arrow[d, "h"]
      \\ Y_{1}
      \arrow[r, "\beta_{1}" swap]
      & Y_{2}
      \arrow[r, "\beta_{2}" swap]
      & Y_{3}
    \end{tikzcd}
    \quad \Longrightarrow \quad
    \begin{tikzcd}
      X_{1}
      \arrow[d, "f" swap]
      \arrow[r, "\alpha_{2} \circ \alpha_{1}"]
      & X_{3}
      \arrow[d, "h"]
      \\ Y_{1}
      \arrow[r, "\beta_{2} \circ \beta_{1}" swap]
      & Y_{3}
    \end{tikzcd}
  \end{displaymath}

  A associatividade dessa composição via colagem segue diretamente da associatividade da composição na categoria inicial $\mathsf{C}$.
  Por fim, dado um objeto $f: X \to Y$ qualquer, o morfismo idêntico associado a ele é dado pelo par $\id_{f} \coloneqq (\id_{X},\id_{Y})$, conforme mostrado no quadrado comutativo abaixo.
  \begin{displaymath}
    \begin{tikzcd}
      X
      \arrow[r, "\id_{X}"]
      \arrow[d, "f" swap]
      & X
      \arrow[d, "f"]
      \\ Y
      \arrow[r, "\id_{Y}" swap]
      & Y
    \end{tikzcd}
  \end{displaymath}

  Note agora que, se o objeto $f: A \to B$ é um retrato do objeto $g: X \to Y$ \emph{na categoria de setas $\Arr(\mathsf{M})$}, então por definição existem morfismos $s_{1}: A \to X,\, s_{2}: B \to Y,\, r_{1}: X \to A$ e $r_{2}: Y \to B$ tais que $(r_{1},r_{2}) \circ (s_{1},s_{2}) = \id_{f}$, o que também pode ser expresso pelo diagrama comutativo abaixo.
  \begin{displaymath}
    \begin{tikzcd}
      A
      \arrow[r, "s_{1}"]
      \arrow[d, "f" swap]
      \arrow[rr, bend left=45, "\id_{A}"]
      & X
      \arrow[r, "r_{1}"]
      \arrow[d, "g"]
      & A
      \arrow[d, "f"]
      \\ B
      \arrow[r, "s_{2}" swap]
      \arrow[rr, bend right=45, "\id_{B}" swap]
      & Y
      \arrow[r, "r_{2}" swap]
      & B
    \end{tikzcd}
  \end{displaymath}

   Esse é precisamente o diagrama que aparece no axioma de retração na definição de uma estrutura modelo.
  Podemos então reformular tal axioma dizendo que as classes de equivalências fracas, fibrações e cofibrações são todas fechadas por \emph{retrações na categoria de setas $\Arr(\mathsf{C})$}.
\end{obs}

\begin{obs}
  Quando trabalhamos com categorias modelo, no lugar de dizermos explicitamente que um morfismo é uma equivalência fraca, ou uma cofibração, ou uma fibração, simplesmente adornarmos de alguma forma a seta que representa o morfismo em questão.
  A convenção notacional que seguiremos nesse aspecto é a seguinte:
  \begin{itemize}
  \item uma equivalência fraca será denotada por $\overset{\sim}{\rightarrow}$;
    
  \item uma cofibração será denotada por $\cofib$;
    
  \item uma fibração será denotada por $\fib$.
  \end{itemize}
  Também denotaremos cofibrações ou fibrações trivias por uma combinação dos símbolos acima:
  \begin{itemize}
  \item uma cofibração trivial será denotada por $\overset{\sim}{\cofib}$;
    
  \item uma fibração trivial será denotada por $\overset{\sim}{\fib}$.
  \end{itemize}

  Seguindo essa convenção notacional, podemos, por exemplo, enunciar o axioma de levantamento (M4) da seguinte forma: em uma categoria modelo, todo quadrado comutativo da forma
  \begin{displaymath}
    \begin{tikzcd}
      A
      \arrow[r]
      \arrow[d, tail, "i" swap, "\sim" sloped]
      & X
      \arrow[d, two heads, "p"]
      \\ B
      \arrow[r]
      & Y
    \end{tikzcd}
  \end{displaymath}
  admite um levantamento $f: B \to X$
  \begin{displaymath}
    \begin{tikzcd}
      A
      \arrow[r]
      \arrow[d, tail, "i" swap, "\sim" sloped]
      & X
      \arrow[d, two heads, "p"]
      \\ B
      \arrow[r]
      \arrow[ru, dashed, "f" description]
      & Y,
    \end{tikzcd}
  \end{displaymath}
  e todo quadrado comutativo da forma
  \begin{displaymath}
    \begin{tikzcd}
      A
      \arrow[r]
      \arrow[d, "i" swap]
      & X
      \arrow[d, two heads, "\sim" {swap, sloped}, "p"]
      \\ B
      \arrow[r]
      & Y
    \end{tikzcd}
  \end{displaymath}
  admite um levantamento $f: B \to X$
  \begin{displaymath}
    \begin{tikzcd}
      A
      \arrow[r]
      \arrow[d, "i" swap]
      & X
      \arrow[d, two heads, "\sim" {swap, sloped}, "p"]
      \\ B
      \arrow[r]
      \arrow[ru, dashed, "f" description]
      & Y.
    \end{tikzcd}
  \end{displaymath}

  Usando a mesma convenção, o axioma de fatoração (M5) pode ser enunciado da seguinte maneira: em uma categoria modelo, todo morfismo $f: X \to Y$ possui duas fotarações como mostrado abaixo.
  \begin{displaymath}
    \begin{tikzcd}
      X
      \arrow[rr, "f"]
      \arrow[rd, tail, "i" swap]
      & & Y
      \\ & \widehat{X}
      \arrow[ru, two heads, "p" {swap}, "\sim" {sloped}]
    \end{tikzcd}
    \qquad
    \begin{tikzcd}
      X
      \arrow[rr, "f"]
      \arrow[rd, tail, "j" {swap}, "\sim" {sloped}]
      & & Y
      \\ & \widetilde{Y}
      \arrow[ru, two heads, "q" swap]
    \end{tikzcd}
  \end{displaymath}
\end{obs}

\begin{obs}[Fatorações funtoriais]
  Por vezes a forma como o axioma de fatoração (M5) foi enunciado é suficiente, mas em geral é mais conveniente assumirmos a existência de \emph{fatorações funtoriais} em uma categoria modelo, e o objetivo desse comentário é justamente explicar o significado preciso disso.

  Na \cref{obs:axioma_retracao_explicacao} discutimos como toda categoria $\mathsf{C}$ dá origem a uma categoria de setas $\Arr(\mathsf{C})$ cujos objetos são morfismos de $\mathsf{C}$, e cujos morfismos sãos quadrados comutativos ``conectando'' dois morfismos de $\mathsf{C}$.
  Essa categoria vem naturalmente equipada com um \emph{funtor domínio} $\dom: \Arr(\mathsf{C}) \to \mathsf{C}$ definido da seguinte forma: dado um objeto $X \overset{f}{\to} Y$ em $\Arr(\mathsf{C})$, definimos $\dom(f) \coloneqq X$, e dado um morfismo $(\alpha: X_{1} \to X_{2},\, \beta: Y_{1} \to Y_{2})$ em $\Arr(\mathsf{C})$ entre os objetos $X_{1} \overset{f}{\to} Y_{1}$ e $X_{2} \overset{g}{\to} Y_{2}$ como mostrado abaixo,
  \begin{displaymath}
    \begin{tikzcd}
      X_{1}
      \arrow[d, "f" swap]
      \arrow[r, "\alpha"]
      & X_{2}
      \arrow[d, "g"]
      \\ Y_{1}
      \arrow[r, "\beta" swap]
      & Y_{2}
    \end{tikzcd}
  \end{displaymath}
  definimos $\dom(\alpha,\beta) \coloneqq \alpha$.
  Veja que é uma definição razoável, pois $\dom(f) = X_{1}$, e $\dom(g) = X_{2}$, portanto $\dom(\alpha,\beta)$ define um morfismo de $\dom(f)$ para $\dom(g)$ na categoria $\mathsf{C}$.
  A verificação de que isso define de fato um funtor do tipo $\Arr(\mathsf{C}) \to \mathsf{C}$ segue de um raciocínio direto usando a definição da composição em $\Arr(\mathsf{C})$ via ``colagem de quadrados comutativos adjacentes''.
  Analogamente, temos também um \emph{funtor codomínio} $\cod: \Arr(\mathsf{C}) \to \mathsf{C}$ que manda um objeto $X \overset{f}{\to} Y$ de $\Arr(\mathsf{C})$ para o objeto $\cod(f) \coloneqq Y$ de $\mathsf{C}$ e que manda um morfismo $(\alpha: X_{1} \to X_{2},\, \beta: Y_{1} \to Y_{2})$ em $\Arr(\mathsf{C})$ como acima para o morfismo $\cod(\alpha,\beta) \coloneqq \beta$ em $\mathsf{C}$.

  Tendo em mãos os funtores domínio e codomínio, podemos definir uma \textbf{fatoração funtorial} em uma categoria $\mathsf{C}$ qualquer como um par de funtores $(\iota: \Arr(\mathsf{C}) \to \mathsf{C},\, \pi: \Arr(\mathsf{C}) \to \mathsf{C})$ satisfazendo as seguintes condições:
  \begin{enumerate}
  \item[(i)] $\dom \circ \iota = \dom$;
    
  \item[(ii)] $\cod \circ \pi = \cod$;
    
  \item[(iii)] $\cod \circ \iota = \dom \circ \pi$;
    
  \item[(iv)] $f = \pi(f) \circ \iota(f)$ para todo $f \in \Arr(\mathsf{C})$.
  \end{enumerate}

  Vamos entender o que cada uma dessas condições significa.
  Os funtores $\iota$ e $\pi$ nos permitem, dado um morfismo $f$ em $\mathsf{C}$, produzir dois outros morfismos $\iota(f)$ e $\pi(f)$.
  A condição (i) diz que o domínio de $\iota(f)$ é igual ao de $f$, a condição (ii) diz que o codomínio de $\pi(f)$ ao de $f$, e a condição (iii) diz que o codomínio de $\iota(f)$ é igual ao domínio de $\pi(f)$, de forma que a composição $\pi(f) \circ \iota(f)$ faz sentido e deve ser igual ao morfismo $f$ inicial de acordo com a condição (iv) acima.
  Assim, se o morfismo $f$ é do tipo $X \to Y$, então $\iota(f)$ e $\pi(f)$ são do tipo $X \to Z$ e $Z \to Y$, respectivamente, para algum objeto $Z \in \mathsf{C}$, e pela condição (iv) esses três morfismos juntos formam o triângulo comutativo mostrado abaixo.
  \begin{displaymath}
    \begin{tikzcd}
      X
      \arrow[rr, "f"]
      \arrow[rd, "\iota(f)" swap]
      & & Y
      \\ & Z
      \arrow[ru, "\pi(f)" swap]
    \end{tikzcd}
  \end{displaymath}

  Por vezes queremos que os morfismos $\iota(f)$ e $\pi(f)$ que aparecem na fatoração de $f$ pertençam a classes específicas de morfismos de $\mathsf{C}$.
  Nesse sentido, se existem classes $\mathcal{R},\,\mathcal{L} \subseteq \Mor(\mathsf{C})$ para os quais a fatoração funtorial $(\iota,\pi)$ satisfaz as condições adicionais $\iota(f) \in \mathcal{R}$ e $\pi(f) \in \mathcal{L}$ para todo $f \in \Mor(\mathsf{C})$, diremos que o par $(\iota,\pi)$ define uma \textbf{$(\mathcal{R},\mathcal{L})$-fatoração funtorial}.
\end{obs}

\end{document}

%%% Local Variables:
%%% mode: latex
%%% TeX-master: t
%%% End:
