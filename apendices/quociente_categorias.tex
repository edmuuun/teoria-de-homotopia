\section{Quocientes de categorias}

Nessa seção discutimos como definir um quociente de uma categoria de forma a introduzir identificações entre morfismos.
A ideia básica é que temos coleções de relações de equivalências nos conjuntos de morfismos entre dois objetos quaisquer, e queremos obter uma nova categoria os conjuntos de morfismos foram quocientados pelas relações em questão.
Como precisamos conseguir compor essas classes de equivalência de morfismos, as relações na categoria inicial devem satisfazer alguma condição de compatibilidade com a operação de composição.
A definição abaixo formaliza essa condição.

\begin{defin}\label{defin:congruencia_em_categoria}
    Uma \textbf{congruência} em uma categoria $\mathsf{C}$ consiste de uma relação de equivalência $\sim_{a,b}$ no conjunto de morfismos $\mathsf{C}(a,b)$ para qualquer par de objetos $(a,b) \in \Ob(\mathsf{C}) \times \Ob(\mathsf{C})$ satisfazendo a seguinte condição: se $g_1,\, g_2: a \to b$ são dois morfismos tais que $g_1 \sim_{a,b} g_2$, então para quaisquer morfismos $f: x \to a$ e $h: b \to y$ vale também que $g_1 \circ f \sim_{x,b} g_2 \circ f$ e $h \circ g_1 \sim_{a,y} h \circ g_2$.
\end{defin}

\begin{obs}
    Alguns autores chamam uma congruência de um \emph{ideal} em uma categoria.
\end{obs}

\begin{exem}\label{exem:congruencia_em_monoide}
    Seja $M$ um monoide, ou seja, um conjunto equipado com uma operação binária $\cdot: M \times M \to M$ que é associativa e possui um elemento neutro.
    Considere a categoria $\mathsf{BM}$ que possui um único objeto $*$, cujo conjunto de morfismos $\mathsf{BM}(*,*) \coloneqq M$ é dado pelo próprio conjunto subjacente ao monoide, e cuja operação de composição é dada pela operação binária do monoide.
    A associatividade da operação $\cdot$ em $M$ garante a associatividade da composição em $\mathsf{BM}$, e o elemento neutro $e$ de $M$ define precisamente o morfismo identidade $\id_*: * \to *$, portanto $\mathsf{BM}$ é de fato uma categoria.

    Uma congruência em $\mathsf{BM}$ consiste de uma única relação de equivalência $\sim$ no conjunto $\mathsf{BM}(*,*) = M$ satisfazendo a seguinte condição: se $a,\, b \in M$ são tais que $a \sim b$, então para quaisquer outros elementos $x,\, y \in M$ vale também que $ax \sim bx$ e $ya \sim yb$.
    Isso é precisamente a definição de uma congruência em um monoide, a qual é abarcada portanto pela noção de congruência em uma categoria.
\end{exem}