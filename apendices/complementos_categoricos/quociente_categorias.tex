\section{Quocientes de categorias}

Nessa seção discutimos como definir um quociente de uma categoria de forma a introduzir identificações entre morfismos.
A ideia básica é que temos coleções de relações de equivalências nos conjuntos de morfismos entre dois objetos quaisquer, e queremos obter uma nova categoria os conjuntos de morfismos foram quocientados pelas relações em questão.
Como precisamos conseguir compor essas classes de equivalência de morfismos, as relações na categoria inicial devem satisfazer alguma condição de compatibilidade com a operação de composição.
A definição abaixo formaliza essa condição.

\begin{defin}\label{defin:congruencia_em_categoria}
    Uma \textbf{congruência} em uma categoria $\mathsf{C}$ consiste de uma relação de equivalência $\sim_{a,b}$ no conjunto de morfismos $\mathsf{C}(a,b)$ para qualquer par de objetos $(a,b) \in \Ob(\mathsf{C}) \times \Ob(\mathsf{C})$ satisfazendo a seguinte condição: se $g_1,\, g_2: a \to b$ são dois morfismos tais que $g_1 \sim_{a,b} g_2$, então para quaisquer morfismos $f: x \to a$ e $h: b \to y$ vale também que $g_1 \circ f \sim_{x,b} g_2 \circ f$ e $h \circ g_1 \sim_{a,y} h \circ g_2$.
\end{defin}

\begin{obs}
    Alguns autores chamam uma congruência de um \emph{ideal} em uma categoria.
\end{obs}

\begin{exem}\label{exem:congruencia_em_monoide}
    Seja $M$ um monoide, ou seja, um conjunto equipado com uma operação binária $\cdot: M \times M \to M$ que é associativa e possui um elemento neutro.
    Considere a categoria $\mathsf{BM}$ que possui um único objeto $*$, cujo conjunto de morfismos $\mathsf{BM}(*,*) \coloneqq M$ é dado pelo próprio conjunto subjacente ao monoide, e cuja operação de composição é dada pela operação binária do monoide.
    A associatividade da operação $\cdot$ em $M$ garante a associatividade da composição em $\mathsf{BM}$, e o elemento neutro $e$ de $M$ define precisamente o morfismo identidade $\id_*: * \to *$, portanto $\mathsf{BM}$ é de fato uma categoria.

    Uma congruência em $\mathsf{BM}$ consiste de uma única relação de equivalência $\sim$ no conjunto $\mathsf{BM}(*,*) = M$ satisfazendo a seguinte condição: se $a,\, b \in M$ são tais que $a \sim b$, então para quaisquer outros elementos $x,\, y \in M$ vale também que $ax \sim bx$ e $ya \sim yb$.
    Isso é precisamente a definição de uma congruência em um monoide, a qual é abarcada portanto pela noção de congruência em uma categoria.
\end{exem}

\begin{exem}\label{exem:congruencias_em_grupos}
    Dado um grupo $G$, como todo grupo é também um monoide, podemos aplicar a construção do \cref{exem:congruencia_em_monoide} a $G$ para obtermos a categoria $\mathsf{BG}$ tendo um único objeto $*$ e tendo o próprio $G$ como conjunto de morfismos $* \to *$.
    Note que, como todo elemento de um grupo é invertível, todo morfismo em $\mathsf{BG}$ é um isomorfismo, ou seja, a categoria $\mathsf{BG}$ define um \emph{grupoide}.

    Dado um subgrupo normal $H \trianglelefteq G$, lembremos que $H$ induz a famigerada relação de equivalência \emph{módulo $H$} $\equiv_H$ em $G = \mathsf{BG}(*,*)$ dada por $a \equiv_H b$ se, e somente se, $ab^{-1} \in H$.
    Afirmamos que essa relação de equivalência define também uma congruência em $\mathsf{BG}$.
    De fato, se $a,\,b \in G$ são tais que $a \equiv_H b$, então para qualquer outro $x \in G$ vale também que $ax \equiv_H bx$ pois
    \begin{displaymath}
        ax(bx)^{-1} = axx^{-1}b^{-1} = aeb^{-1} = ab^{-1} \in H,
    \end{displaymath}
    e para qualquer $y \in G$ vale também que $ya \equiv_H yb$ pois
    \begin{displaymath}
        ya(yb)^{-1} = yab^{-1}y^{-1} = y(ab^{-1})y^{-1} \in H;
    \end{displaymath}
    onde no último passo foi necessário usar o fato de $H$ ser um subgrupo normal.

    É interessante observar que toda congruência em $\mathsf{BG}$ é induzida por algum subgrupo normal de $G$.
    De fato, se $\sim$ é uma congruência em $\mathsf{BG}$, portanto uma relação de equivalência em $G$ ``invariante'' pela multiplicação de elementos de $G$ em ambos os lados, seja $H \coloneqq [e]$ a classe de equivalência do elemento neutro.
    Vamos mostrar que $H$ é um subgrupo normal de $G$.
    \begin{itemize}
        \item A reflexividade de $\sim$ nos dá que $e \sim e$, portanto $e \in H$.

        \item Dados dois elementos $a,b \in H$, multiplicando a relação $a \sim e$ por $b$ à direita vemos que $ab \sim b$, mas $b \sim e$ por hipótese, e a transitividade nos dá então $ab \sim e$, ou seja, $ab \in H$.
        
        \item Se $a \in H$, multiplicando a relação $a \sim e$ por $a^{-1}$ concluímos que $e \sim a^{-1}$, logo $a^{-1} \sim e$ por simetria, de onde segue que $a^{-1} \in H$.
    
        \item Suponha que $a \in H$, e seja $g \in G$ um elemento qualquer.
        A fim de mostrarmos que $H$ é normal, queremos mostrar que a conjugação $gag^{-1}$ também pertence a $H$.
        Multiplicando a relação $a \sim e$ por $g$ à esquerda concluímos que $ga \sim g$, e multiplicando esta última por $g^{-1}$ à direita concluímos que $gag^{-1} \sim gg^{-1} = e$, portanto $gag^{-1} \in H$ como queríamos mostrar.
    \end{itemize}

    Sabendo que $H$ define de fato um subgrupo normal, afirmamos que a relação $\equiv_H$ induzida por $H$ coincide com a congruência $\sim$ inicial.
    De fato, se $a,\,b \in G$ são tais que $a \equiv_H b$, então $ab^{-1} \in H$, o que pela definição de $H$ significa que $ab^{-1} \sim e$, mas isso é equivalente a $a \sim b$ graças à compatibilidade de $\sim$ com a operação de multiplicação do grupo.
\end{exem}

\begin{exem}\label{exem:homotopia_como_congruencia}
    Dados dois espaços topológicos $X$ e $Y$, considere a relação $\sim_{X,Y}$ em $\mathsf{Top}(X,Y)$ dada por $f \sim_{X,Y} g$ se, e somente se, $f$ e $g$ são homotópicos.
    É bem sabido da Teoria de Homotopia que $\sim_{X,Y}$ define uma relação de equivalência no conjunto de mapas $\mathsf{Top}(X,Y)$ que é preservada pela pré-composição ou pós-composição com outros mapas quaisquer.
    Em outras palavras, a coleção de relações de equivalência $\{\sim_{X,Y}\}_{X,Y \in \mathsf{Top}}$ define uma congruência na categoria $\mathsf{Top}$.
\end{exem}

O resultado no qual estamos interessados diz que uma congruência nos permite definir uma nova categoria cujos morfismos são obtidos identificando os morfismos da categoria inicial.

\begin{teo}\label{teo:existencia_categoria_quociente}
    Sejam $\mathsf{C}$ uma categoria e $\sim = \{\sim_{a,b}\}_{a,b \in \mathsf{C}}$ uma congruência em $\mathsf{C}$.
    Existe então um par $(\overline{\mathsf{C}},\pi)$, onde $\overline{\mathsf{C}}$ é uma categoria, e $\pi: \mathsf{C} \to \overline{\mathsf{C}}$ é um funtor, satisfazendo as seguintes condições:
    \begin{enumerate}
        \item $\pi$ identifica morfismos equivalentes em $\mathsf{C}$, ou seja, se $f,\,g: a \to b$ são morfismos em $\mathsf{C}$ tais que $f \sim_{a,b} g$, então $\pi(f) = \pi(g)$.

        \item Se $\mathsf{D}$ é uma categoria qualquer, e $F: \mathsf{C} \to \mathsf{D}$ é um funtor que identifica morfismos em $\mathsf{C}$, então existe um único funtor $\overline{F}: \overline{\mathsf{C}} \to \mathsf{D}$ satisfazendo a igualdade $\overline{F} \circ \pi = f$.
        \begin{displaymath}
            \begin{tikzcd}
                \mathsf{C}
                \arrow[r, "F"]
                \arrow[d, "\pi" swap]
                & \mathsf{D}
                \\ \overline{\mathsf{C}}
                \arrow[ru, dashed, "\overline{F}" swap]
            \end{tikzcd}
        \end{displaymath}
    \end{enumerate}
\end{teo}

\begin{proof}
    Considere a categoria $\overline{\mathsf{C}}$ tendo os mesmos objetos de $\mathsf{C}$, e dados então dois objetos $a,\,b \in \Ob(\overline{\mathsf{C}})$, considere o conjunto de morfismos do tipo $a \to b$ em $\overline{\mathsf{C}}$ definido por
    \begin{displaymath}
        \overline{\mathsf{C}}(a,b) \coloneqq \overline{C}(a,b)/\sim_{a,b},
    \end{displaymath}
    ou seja, os conjuntos de morfismos são todos quocientes pelas relações de equivalência que definem a congruência em $\sim$ em $\mathsf{C}$.
    Dado um morfismo $f \in \mathsf{C}(a,b)$, denotaremos por $\overline{f} \in \overline{\mathsf{C}}(a,b)$ o morfismo correspondente em $\overline{\mathsf{C}}$ dado pela imagem de $f$ ao longo da projeção canônica
    \begin{displaymath}
        \pi_{a,b}: \mathsf{C}(a,b) \to \overline{\mathsf{C}}(a,b).
    \end{displaymath}

    Precisamos definir uma operação de composição de morfismos em $\overline{\mathsf{C}}$ e verificar que ela possui identidades e é associativa.
    Dado então dois morfismos $\overline{f} \in \overline{\mathsf{C}}(a,b)$ e $\overline{g} \in \overline{\mathsf{C}}(b,c)$, definimos sua composição $\overline{g} \circ \overline{f} \in \overline{\mathsf{C}}(a,c)$ por meio da fórmula
    \begin{displaymath}
        \overline{g} \circ \overline{f} \coloneqq \overline{g \circ f}.
    \end{displaymath}
    O fato de tal expressão independer dos representates das classes de equivalência $\overline{f}$ e $\overline{g}$ é uma consequência direta do fato de $\sim$ ser uma \emph{congruência} na categoria $\mathsf{C}$, e não apenas uma coleção de relações de equivalência nos conjuntos de morfismos.
    
    Dado um objeto qualquer $x \in \Ob(\overline{\mathsf{C}})$, a classe $\overline{\id_x}$ do morfismo idêntico $\id_x$ em $\mathsf{C}$ é uma unidade para a composição em $\overline{\mathsf{C}}$ definida acima.
    De fato, dado um morfismo $f: w \to x$ em $\overline{\mathsf{C}}$ temos
    \begin{displaymath}
        \overline{id_x} \circ \overline{f} = \overline{\id_x \circ f} = \overline{f},
    \end{displaymath}
    e dado um morfismo $\overline{g}: x \to y$ em $\overline{\mathsf{C}}$ temos
    \begin{displaymath}
        \overline{g} \circ \overline{\id_x} = \overline{g \circ \id_x} = \overline{g}.
    \end{displaymath}
    Além disso, dados três morfismos quaisquer $\overline{f}: w \to x$, $\overline{g}: x \to y$ e $\overline{h}: y \to z$ em $\overline{\mathsf{C}}$, utlizando a associatividade da composição em $\mathsf{C}$ vemos que
    \begin{displaymath}
        (\overline{h} \circ \overline{g}) \circ \overline{f}
        = \overline{h \circ g} \circ \overline{f}
        = \overline{(h \circ g) \circ f}
        = \overline{h \circ (g \circ f)}
        = \overline{h} \circ \overline{g \circ f}
        = \overline{h} \circ (\overline{g} \circ \overline{f}),
    \end{displaymath}
    ou seja, a operação de composição em $\overline{\mathsf{C}}$ é associativa também.

    O raciocínio acima mostra que $\overline{\mathsf{C}}$ define de fato uma categoria.
    O funtor $\pi: \mathsf{C} \to \overline{\mathsf{C}}$ do enunciado pode ser definido da seguinte maneira: nos objetos $\pi_{\Ob}: \Ob(\mathsf{C}) \to \Ob(\overline{\mathsf{C}})$ é dado simplemsente pela função identidade, e para dois objetos $a,\,b \in \Ob(\mathsf{C})$ quaisquer, a compomente do funtor nos morfismos entre esses dois objetos é dado pela projeção canônica
    \begin{displaymath}
        \pi_{a,b}: \mathsf{C}(a,b) \to \overline{\mathsf{C}}(a,b)
    \end{displaymath}
    mencionada anteriormente.
    Segue diretamente da forma como a composição e os morfismos identidade em $\overline{\mathsf{C}}$ foram definidos que $\pi$ é de fato um funtor.
    Além disso, $\pi$ identifica morfismos relacionados, pois se $f \sim_{a,b} g$, temos a igualdade de clases de equivalência $\overline{f} = \overline{g}$ e, portanto, a igualdade $\pi(f) = \pi(g)$.

    suponha agora que tenhamos um funtor $F: \mathsf{C} \to \mathsf{D}$ que identifica os morfismo de $\mathsf{C}$ relacionados pela congruência $\sim$.
    Denotando por $F_{a,b}: \mathsf{C}(a,b) \to \mathsf{D}(a,b)$ a componente de $F$ nos morfismos entre $a$ e $b$, por hipótese tal função é constante nas classes de equivalência de $\sim_{a,b}$, portanto a propriedade universal do quociente em conjuntos nos fornece uma única função $\overline{F}_{a,b}: \overline{\mathsf{C}}(a,b) \to \mathsf{D}(a,b)$ satisfazendo a igualdade $\overline{F}_{a,b} \circ \pi_{a,b} = F_{a,b}$.

    Afirmamos que a função $F_{\Ob}: \Ob(\mathsf{C}) = \Ob(\overline{\mathsf{C}}) \to \Ob(\mathsf{D})$ juntamente com a família de funções $(\overline{F}_{a,b})_{a,\,b}$ juntas definem o funtor $\overline{F}: \overline{\mathsf{C}} \to \overline{D}$ desejado.
    Note que a composição de morfismos é preservada, já que para quaisquer dois morfismos $\overline{f}: a \to b$ e $\overline{g}: b \to c$ em $\overline{\mathsf{C}}$ temos
    \begin{displaymath}
        \overline{F}_{a,b}(\overline{g} \circ \overline{f})
        = \overline{F}_{a,b}(\overline{g \circ f})
        = F_{a,b}(g \circ f)
        = F_{b,c}(g) \circ F_{a,b}(f)
        = \overline{F}_{b,c}(\overline{g}) \circ \overline{F}_{a,b}(\overline{f}).
    \end{displaymath}
    Analogamente, $\overline{F}$ também preserva morfismos identidade, já que para qualquer objeto $x \in \Ob(\overline{\mathsf{C}})$ temos
    \begin{displaymath}
        \overline{F}_{x,x}(\overline{\id_x})
        = F_{x,x}(\id_x)
        = \id_{F(x)}
        = \id_{\overline{F}(x)}.
    \end{displaymath}

    Sabendo que $\overline{F}$ define de fato um funtor, segue diretamente de sua definição que a igualdade $\overline{F} \circ \pi = F$ é satisfeita.
    Por fim, se $G: \overline{\mathsf{C}} \to \mathsf{D}$ é outro funtor tal que $G \circ \pi = F$, então por um lado a nível dos objetos temos
    \begin{displaymath}
        G_{\Ob} = G_{\Ob} \circ \id_{\Ob(\mathsf{C})}
        = G_{\Ob} \circ \pi_{\Ob}
        = F_{\Ob}
        = \overline{F}_{\Ob},
    \end{displaymath}
    enquanto para quaisquer dois objetos $a,\,b \in \Ob(\overline{\mathsf{C}})$ temos a igualdade $G_{a,b} \circ \pi_{a,b} = F_{a,b}$ e, portanto, $G_{a,b} = \overline{F}_{a,b}$ graças à unicidade que caracteriza a função $\overline{F}_{a,b}$ na propreidade universal do quociente de conjuntos.
    Segue desse raciocínio que a igualdade desejada $G = \overline{F}$ é válida.
\end{proof}