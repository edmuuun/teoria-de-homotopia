\section{Categorias livres}

%TODO: Revisar essa seção.

Comumente representamos uma categoria por meio de um grafo orientado tendo os objetos da categoria como vértices e os morfismos da categoria como arestas.
Apesar dessa representação bastante útil, uma categoria possui mais estrutura do que um mero grafo orientado, visto que podemos compor os morfismos de uma categoria para obtermos novos morfismos, mas não existe uma operação análoga de composição das arestas de um grafo.
Mais precisamente, toda categoria possui um grafo orientado \emph{subjacente} que esquece da estrutura algébrica da composição.
Os leitores com alguma experiência em Teoria de Categorias já devem esperar que exista também uma construção no sentido contrário por meio da qual um grafo orientado dá origem a uma categoria obtida pela adição da forma mais simples possível de morfismos que representem a composição das arestas do grafo.
A categoria assim obtida é a chamada \emph{categoria livre} gerada pelo grafo, e essa construção é o objetivo central desta seção.

Antes de continuarmos, cabe aqui um aviso sobre terminologia.
O termo \emph{grafo} é particularmente carregado na literatura matemática, e por esse motivo escolhemos usar daqui em diante o termo \emph{aljava} (tradução do inglês \emph{quiver}) no lugar de grafo orientado.

\begin{defin}\label{defin:aljava}
    Uma \textbf{aljava} consiste de uma tupla $(V,A,s,t)$, onde $V$ e $A$ são conjuntos, e $s,\,t: A \to V$ são funções.
\end{defin}

Dada uma aljava $(V,A,s,t)$, os elementos de $V$ são chamados de \textbf{vértices} da aljava, enquanto os elementos de $A$ são chamados de \textbf{arestas} da aljava.
As funções $s$ e $t$ são chamadas de funções \textbf{source} e \textbf{target}, respectivamente.
Visualmente, cada elemento $e \in A$ deve ser visualizado como uma aresta direcionada começando no vértice $s(e)$ e terminando no vértice $t(e)$.
Por vezes será mais conveniente denotarmos uma aljava simplesmente por uma letra como $Q$, e caso seja necessário nos referirmos às estruturas constituintes dessa aljava, usaremos então notações como $V(Q)$, $A(Q)$, $s_Q$ e $t_Q$.

\begin{exem}\label{exem:aljava_discreta}
    Um conjunto $X$ pode ser usado para definir uma aljava $QX$ dada pela tupla $QX \coloneqq (X,X,\id_X,\id_X)$.
    Visualmente, tal aljava possui um vértice para cada elemento $x \in X$ e também uma aresta direcionada desse elemento para si mesmo formando um laço.
    A aljava $QX$ assim definida é chamada de \textbf{aljava discreta} gerado por $X$.
\end{exem}

\begin{exem}\label{exem:aljava_via_relacao}
    Seja $R \subseteq X \times X$ uma relação qualquer em um conjunto $X$.
    Podemos construir a partir dessa relação uma aljava $Q \coloneqq (X,R,s,t)$, onde $s,\,t: R \to X$ são as restriçõs das projeções canônicas associadas ao produto $X \times X$, ou seja, $s(x_1,x_2) \coloneqq x_1$ e $t(x_1,x_2) \coloneqq x_2$ para qualquer par $(x_1,x_2) \in R$.
    
    É especialmente comum considerarmos essa construção no caso em que a relação $R$ é uma pré-ordem $\leq$ no conjunto $X$, ou seja, uma relação reflexiva e transitiva.
    Nesse caso, as arestas da aljava``acompanham o sentido de crescimento'' da pré-ordem $\leq$.
\end{exem}

\begin{exem}\label{exem:aljava_oposta}
    Toda aljava $Q=(V,A,s,t)$ dá origem a uma \textbf{aljava oposta} $Q^{\mathsf{op}}$ definida pela tupla $(V,A,t,s)$.
    Visualmente, $Q^{\mathsf{op}}$ possui os mesmos vértices de $Q$, mas com as arestas indo no sentido oposto.
\end{exem}

\begin{exem}\label{exem:aljava_via_espaco}
    Seja $X$ um espaço topológico qualquer, e considere o conjunto\footnote{É verdade que também podemos considerar $X^I$ como um espaço equipado com a topologia compacto-aberto, por exemplo, mas aqui não existe essa necessidade.} $X^I$ formado por todas as funções contínuas do tipo $I \to X$, onde $I$ denota o intervalo unitário $[0,1]$.
    Podemos considerar as funções de avaliação $e_0,\, e_1: X^I \to X$ dadas por $e_0(\gamma) \coloneqq \gamma(0)$ e $e_1(\gamma) \coloneqq \gamma(1)$ para todo $\gamma \in X^{I}$.
    A tupla $PX \coloneqq (X,X^I,e_0,e_1)$ define então a \textbf{aljava de caminhos} do espaço $X$.
\end{exem}

\begin{exem}\label{exem:aljava_subjacente_categoria}
    Seja $\mathsf{C}$ uma categoria pequena.
    Se $\dom,\,\cod: \Mor(\mathsf{C}) \to \mathsf{C}$ são as funções que associam a cada morfismo de $\mathsf{C}$ seu domínio e codomínio, respectivamente, então a tupla $U(\mathsf{C}) \coloneqq (\mathsf{C},\Mor(\mathsf{C}),\dom,\cod)$ define a \textbf{aljava subjacente à categoria} $\mathsf{C}$.
\end{exem}

\begin{defin}\label{defin:morfismo_aljavas}
    Dadas duas aljavas $Q_1$ e $Q_2$, um \textbf{morfismo de aljavas} do tipo $Q_1 \to Q_2$ é por definição um par de funções $(f_V: V(Q_1) \to V(Q_2),\, f_A: A(Q_1) \to A(Q_2))$ satisfazendo as igualdades $s_{Q_2} \circ f_A = f_V \circ s_{Q_1}$ e $t_{Q_2} \circ f_A = f_V \circ t_{Q_1}$, ou seja, os dois diagramas abaixo são comutativos.
    \begin{displaymath}
        \begin{tikzcd}
            A(Q_1)
            \arrow[r, "f_A"]
            \arrow[d, "s_{Q_1}" swap]
            & A(Q_2)
            \arrow[d, "s_{Q_2}"]
            \\ V(Q_1)
            \arrow[r, "f_V" swap]
            & V(Q_2)
        \end{tikzcd}
        \qquad
        \begin{tikzcd}
            A(Q_1)
            \arrow[r, "f_A"]
            \arrow[d, "t_{Q_1}" swap]
            & A(Q_2)
            \arrow[d, "t_{Q_2}"]
            \\ V(Q_1)
            \arrow[r, "f_V" swap]
            & V(Q_2)
        \end{tikzcd}
    \end{displaymath}
\end{defin}

Durante as demonstrações faremos questão por motivos de clareza de utilizar símbolos diferentes para denotar as duas funções que compõem um morfismo de aljavas.
Entretanto, na prática é muito comum denotarmos ambas as funções pelo mesmo símbolo, da mesma forma como usamos um único símbolo para denotar a ação de um funtor nos objetos e nos morfismos.

\begin{exem}
    Dados dois conjuntos $X_1$ e $X_2$ e uma função $f: X_1 \to X_2$, o par $(f,f)$ define um morfismo $QX_1 \to QX_2$ entre as aljavas discretas geradas por $X_1$ e $X_2$ de acordo com o \cref{exem:aljava_discreta}.
\end{exem}

\begin{exem}
    Sejam $(X_1,R_1)$ e $(X_2,R_2)$ dois conjuntos equipados com relações, e suponha que $f: X_1 \to X_2$ seja uma função que respeita as relações, ou seja, se $(x,x') \in R_1$, então $(f(x),f(x')) \in R_2$.
    Considere as aljavas $Q_1$ e $Q_2$ associados às relações $R_1$ e $R_2$, respectivamente, conforme discutido no \cref{exem:aljava_via_relacao}.
    Como $f$ respeita as relações $R_1$ e $R_2$, a função produto $f \times f: X_1 \times X_1 \to X_2 \times X_2$ satisfaz $(f \times f)(R_1) \subseteq R_2$, portanto podemos considerar a função restrita
    \begin{displaymath}
        f \times f \rvert_{R_1}: R_1 \to R_2.
    \end{displaymath}
    O par de funções $(f,f \times f\rvert_{R_1})$ define então um morfismo de aljavas do tipo $Q_1$ para $Q_2$.
    
    No caso em que $R_1$ e $R_2$ são relações de pré-ordem $\leq_1$ e $\leq_2$, a condição de $f$ respeitar as relações diz que, se $x \leq_1 x'$, então $f(x) \leq_2 f(x')$; ou seja, $f$ é uma função monótona.
    Segue da discussão acima que funções monótonas entre conjuntos pré-ordenados dão origem a morfismos entre as aljavas obtidas a partir deles.
\end{exem}

\begin{exem}
    Uma função contínua $f: X \to Y$ entre espaços topológicos dá origem a uma função de pushforward $X^f: X^I \to Y^{I}$ definida por $X^f(\gamma) \coloneqq f \circ \gamma$ para todo $\gamma \in X^{I}$.
    O par $(f,X^f)$ define então um morfismo aljavas
    \begin{displaymath}
        (f,X^f): (X,X^I,e^X_0,e^X_1) \to (Y,Y^I,e^Y_0,e^Y_1)
    \end{displaymath}
    entre as alajvas de caminhos definidas no \cref{exem:aljava_via_espaco}.
    De fato, para qualquer $\gamma \in X^I$ por um lado temos
    \begin{displaymath}
        e^Y_0(X^f(\gamma)) = e^Y_0(f \circ \gamma) = f(\gamma(0)) = f(e^X_0(\gamma)),
    \end{displaymath}
    enquanto por outro temos
    \begin{displaymath}
        e^Y_1(X^f(\gamma)) = e^Y_1(f \circ \gamma) = f(\gamma(1)) = f(e^X_1(\gamma)).
    \end{displaymath}
\end{exem}

\begin{exem}\label{exem:funtor_induz_morfismo_de_aljavas}
    Lembre-se que um funtor $F: \mathsf{C} \to \mathsf{D}$ é dado por uma função $F: \mathsf{C} \to \mathsf{D}$ a nível de objetos e também por uma função $\Mor(F): \Mor(\mathsf{C}) \to \Mor(\mathsf{D})$ a nível de morfismos a qual é compatível com as funções de domínio e codomínio de cada uma das categorias e também com as respectivas operações de composição.
    A compatibilidade com as funções de domínio e codomínio, em particular, faz com que o par $(F,\Mor(F))$ defina um morfismo do tipo $U(\mathsf{C}) \to U(\mathsf{D})$ entre as aljavas subjacentes às categorias discutidas no \cref{exem:aljava_subjacente_categoria}.
\end{exem}

Dois morfismos de aljavas $(f_1,F_1): Q_1 \to Q_2;$ e $(f_2,F_2): Q_2 \to Q_3$ podem ser compostos para definir um outro morfismo de aljavas $(f_2,F_2) \circ (f_1,F_1): Q_1 \to Q_3$ pela regra
\begin{displaymath}
    (f_2,F_2) \circ (f_1,F_1) \coloneqq (f_2 \circ f_1,F_2 \circ F_1).
\end{displaymath}
Além disso, toda aljava $Q$ tem associada a ela um morfismo identidade $\id_Q: Q \to Q$ dado pelo par $(\id_{V(Q)},\id_{A(Q)})$.
Uma conta simples mostra que o morfismo $\id_Q$ realmente age como uma identidade para a composição de morfismos de grafos orientados introduzida acima.
Em resumo, temos uma categoria $\mathsf{Alj}$ formada por aljavas e morfismos entre elas
Diversas das construções introduzidas nos exemplos anteriores podem ser estendidas para funtores definidos na categoria $\mathsf{Alj}$, ou então tomando valores nela.

\begin{exem}
    Considere a categoria $\mathsf{SetRel}$ cujos objetos são conjuntos equipados com alguma relação e cujos morfismos são funções preservando tais relações.\footnote{É tentador usar o nome $\mathsf{Rel}$ para tal categoria, mas é comum na literatura usar esse nome para identificar a categoria onde relações aparecem como \emph{morfismos} entre conjuntos.}
    Vimos no \cref{exem:aljava_via_relacao} que todo conjunto equipado com uma relação $(X,\leq)$ dá origem a uma aljava, e que funções preservando tais relações dão origem a morfismos entre as aljavas correspondentes.
    Uma conta simples mostra que essa construção preserva composições e morfismos identidade, definindo portanto um funtor $F: \mathsf{SetRel} \to \mathsf{Alj}$.
    
    Existe também um funtor no sentido contrário $\mathsf{Alj} \to \mathsf{SetRel}$.
    Dada uma aljava $Q$, definimos uma relação $R_Q$ no conjunto de vértices $V(Q)$ da aljava por meio da fórmula
    \begin{displaymath}
        R_Q \coloneqq \{(u,v) \in V(Q) \times V(Q) \mid s_Q(e) = u \text{ e } t_Q(e) = v \text{ para alguma aresta } e \in A(Q)\}.
    \end{displaymath}
    Note também que, se $(f,F): Q_1 \to Q_2$ é um morfismo de aljavas, então a função a nível dos vértices $f: V(Q_1) \to V(Q_2)$ preservas as relações $R_{Q_1}$ e $R_{Q_2}$.
    De fato, se $(u,v) \in R_{Q_1}$, então existe alguma aresta $e \in A(Q_1)$ tal que $s_{Q_1}(e) = u$ e $t_{Q_1}(e) = v$.
    Note então que pelas condições de compatibilidade entre $f$, $F$ e as funções source e target das aljavas temos
    \begin{displaymath}
        s_{Q_2}(F(e)) = f(s_{Q_1}(e)) = f(u)
        \quad \text{e} \quad
        t_{Q_2}(F(e)) = f(t_{Q_1}(e)) = f(v),
    \end{displaymath}
    portanto $(f(u),f(v)) \in R_{Q_2}$.
    Isso nos permite então definir um funtor $\mathsf{Alj} \to \mathsf{SetRel}$ como mencionado.
\end{exem}

\begin{exem}\label{exem:funtor_esquecimento_cat_aljava}
    Vimos no \cref{exem:aljava_subjacente_categoria} que toda categoria pequena $\mathsf{C}$ possui uma aljava subjacente $U(\mathsf{C})$ e vimos também no \cref{exem:funtor_induz_morfismo_de_aljavas} que um funtor $F: \mathsf{C} \to \mathsf{D}$ induz um morfismo $U(F) \coloneqq (F,\Mor(F)): U(\mathsf{C}) \to U(\mathsf{D})$ entre as aljavas subjacentes.
    Essas duas construções combinadas nos permitem definir um \emph{funtor de esquecimento} $U: \mathsf{Cat} \to \mathsf{Alj}$ definido na categoria de categorias pequenas e tomando valores na categoria de aljavas.
    Usamos o termo funtor de esquecimento pois ao passar da categoria $\mathsf{C}$ para sua aljava subjacente $U(\mathsf{C})$ nos esquecemos da estrutura de composição de morfismos existente na categoria original.
\end{exem}

Nosso interesse é investigar a existência de um funtor no sentido contrário $\mathsf{Alj} \to \mathsf{Cat}$ que associa a cada aljava uma categoria onde podemos compor formalmente suas arestas da forma mais simples possível.
A existência de tal funtor está ligada à existência de categorias livres gerados por aljavas que definimos agora.

\begin{defin}\label{defin:categoria_livre}
    Dada uma aljava $Q$, uma \textbf{categoria livre gerada por $Q$} é um par $(\mathsf{C},i)$, onde $\mathsf{C}$ é uma categoria pequena, e $i: Q \to U(\mathsf{C})$ é um morfismo de aljavas satisfazendo a seguinte propriedade universal: se $\mathsf{D}$ é outra categoria pequena qualquer, e $j: G \to U(\mathsf{D})$ é um morfismo de aljavas, então existe um único funtor $F: \mathsf{C} \to \mathsf{D}$ tal que $U(F) \circ i = j$.
    \begin{displaymath}
        \begin{tikzcd}
            U(\mathsf{C})
            \arrow[r, dashed, "U(F)"]
            & U(\mathsf{D})
            \\ Q
            \arrow[u, "i"]
            \arrow[ru, "j" swap]
        \end{tikzcd}
    \end{displaymath}
\end{defin}

Tendo em mãos a definição de uma categoria livre, começaremos a caminhar agora em direção à construção de um modelo explícito para uma categoria livre.
Dado um natural $n \in \mathbb{N}$, considere a aljava $[n]$ definida da seguinte maneira: o conjunto de vértices é por definição $V([n]) \coloneqq \{0,\dots,n\}$, enquanto o conjunto de arestas é por definição $A([n]) \coloneqq \{(k,k+1) \mid 0 \leq k \leq n-1\}$.
As funções source e target são então restrições das projeções canônicas associadas ao produto $\{0,\dots,n\} \times \{0,\dots,n\}$, ou seja, temos
\begin{displaymath}
    s(k,k+1) \coloneqq k
    \quad \text{e} \quad
    t(k,k+1) \coloneqq k+1
\end{displaymath}
para todo $0 \leq k \leq n-1$.
Visualmente, $[n]$ possui $n+1$ vértices rotulados pelos números o conjunto $\{0,\dots,n\}$ e $n$ vértices, sendo que cada vértice liga um número desse conjunto ao seu sucessor como mostrado na figura abaixo.
Note, em particular, que a aljava $[0]$ consiste de um único vértice $0$ sem nenhuma aresta.

\begin{figure}[h]
    \centering
    \begin{tikzpicture}
        \node (zero) [circle,draw] {0};
        \node (one) [circle,draw,right=of zero] {1};
        \node (two) [circle,draw,right=of one] {2};
        \node (dots) [circle,right=of two] {$\cdots$};
        \node (ultimo) [circle,draw,right=of dots] {$n$};
        
        \draw[-stealth] (zero) to (one);
        \draw[-stealth] (one) to (two);
        \draw[-stealth] (two) to (dots);
        \draw[-stealth] (dots) to (ultimo);
    \end{tikzpicture}
    \caption{Representação visual da aljava $[n]$.}
    \label{fig:aljava_ordinal}
\end{figure}

\begin{obs}
    A aljava $[n]$ introduzida acima é \emph{diferente} da aljava associada ao conjunto parcialmente ordenado $(\{0,\dots,n\},\leq)$ como no \cref{exem:aljava_via_relacao}!
    Enquanto a primeira possui arestas apenas entre vértices sucessivos, a segunda possui arestas ligando dois vértices $m$ e $n$ sempre que $m \leq n$, ou seja, a segunda possui uma qunatidade bem maior de vértices.
    Em particular, a segunda aljava possui arestas formando laços, já que a relação $\leq$ é reflexiva, enquanto a primeira aljava não possui nenhum laço como é evidente de sua definição.
    A figura \cref{fig:aljava_ordinal_versus_aljava_poset} ilustra explicitamente a diferença entre a aljava $[2]$ e a aljava obtida a partir do conjunto $\{0,1,2\}$ munido de sua ordem parcial usual.
    
    \begin{figure}
        \centering
        \begin{tikzpicture}[->,>=stealth]
            \node (zero) [circle,draw] {0};
            \node (one) [circle,draw,right=of zero] {1};
            \node (two) [circle,draw,right=of one] {2};
            
            \path
            (zero) edge (one)
            (one) edge (two);
            
            \begin{scope}[xshift=6cm]
                \node (zero) [circle,draw] {0};
                \node (one) [circle,draw,right=of zero] {1};
                \node (two) [circle,draw,right=of one] {2};
                
                \path
                (zero) edge (one)
                (zero) edge [out=120,in=60,loop]
                (zero) edge [bend right] (two)
                (one) edge (two)
                (one) edge [out=120,in=60,loop] (one)
                (two) edge [out=120,in=60,loop] (two);
            \end{scope}
        \end{tikzpicture}
        \caption{Comparação entre a aljava $[2]$, representada à esquerda, e a aljava associada ao conjunto parcialmente ordenado $\{0,1,2\}$, representada à direita.}
        \label{fig:aljava_ordinal_versus_aljava_poset}
    \end{figure}
\end{obs}

Definimos um \textbf{caminho de comprimento $n+1$} em uma aljava $Q$ como sendo um morfismo de aljavas $\gamma: [n] \to Q$.
A justificativa para a terminologia é que a aljava $[n]$ possui $n+1$ vértices, portanto o morfismo $\gamma$ seleciona $n+1$ vértices de $Q$ conectados por uma sequência de arestas.

A ideia básica para construirmos uma categoria a partir de uma aljava é que, embora não possamos compor duas arestas consecutivas de uma aljava de forma a obtermos uma \emph{outra aresta}, podemos compor caminhos de comprimentos quaisquer, portanto podemos reinterpretar arestas como caminhos de comprimento $2$ na aljava e encarar sua composição como sendo um caminho de comprimento $3$.
O fato que nos permite compor caminhos é que duas aljavas $[m]$ e $[n]$ podem ser coladas ao longo de seus pontos extremos para formarem uma aljava do tipo $[m+n]$.
A melhor maneira de formalizarmos essa ideia de colagem é por meio de um pushout na categoria de aljavas, mas para isso precisamos antes introduzir alguns morfismos específicos.

Dados números naturais $m$ e $n$ tais que $m \leq n$, considere o morfismo de aljavas
\begin{displaymath}
    \iota_{m,n}: [m] \to [n]
\end{displaymath}
definido da seguinte maneira: nos vértices temos $\iota_{m,n}(k) \coloneqq k$ para todo $k \in \{0,\dots,m\}$, enquanto nas arestas de $[m]$ temos $\iota_{m,n}(k,k+1) \coloneqq (k,k+1)$ para qualquer $k \in \{0,\dots,m-1\}$.
Visualmente, $\iota_{m,n}$ inclui a aljava $[m]$ no \emph{segmento inicial} da aljava $[n]$ formado pelos $m+1$ primeiros vértices e todas as arestas entre eles, conforme mostrado na \cref{fig:morfismo_aljavas_segmento_inicial}.

\begin{figure}[h]
    \centering
    \begin{tikzpicture}[->,>=stealth]
        \node[draw,circle] (zero) at (0,0) {$0$};
        \node[draw,circle,right=of zero] (one) {$1$};
        \node[draw,circle,right=of one] (two) {$2$};
        
        \path
        (zero) edge (one)
        (one) edge (two);

        \begin{scope}[xshift=4cm]
            \node (seta-inicio) {};
            \node[right=of seta-inicio] (seta-fim) {};
            \draw (seta-inicio) to node[auto] {$\iota_{2,3}$} (seta-fim);
        \end{scope}
        
        \begin{scope}[xshift=6cm]
            \node[draw,red,circle] (zero) at (0,0) {$0$};
            \node[draw,red,circle,right=of zero] (one) {$1$};
            \node[draw,red,circle,right=of one] (two) {$2$};
            \node[draw,circle,right=of two] (three) {$3$};
            
            \path
            (zero) edge[red] (one)
            (one) edge[red] (two)
            (two) edge (three);
        \end{scope}
    \end{tikzpicture}
    \caption{Visualização do morfismo de aljavas $\iota_{2,3}: [2] \to [3]$. Os vértices e arestam em vermelho indicam a ``imagem'' do morfismo.}
    \label{fig:morfismo_aljavas_segmento_inicial}
\end{figure}

Considere também o morfismo $\tau_{m,n}: [m] \to [n]$ definido da seguinte forma: nos vértices temos $\tau_{m,n}(k) \coloneqq k+n-m$ para todo $k \in \{0,\dots,m\}$, enquanto nas arestas temos $\tau_{m,n}(k,k+1) \coloneqq (k+n-m,k+n-m+1)$ para todo $k \in \{0,\dots,m-1\}$.
Visualmente, $\tau_{m,n}$ inclui a aljava $[m]$ no \emph{segmento terminal} da aljava $[n]$ formado pelos seus últimos $m+1$ vértices e todas as arestas entre eles, conforme mostrado na \cref{fig:morfismo_aljavas_segmento_terminal}.

\begin{figure}[h]
    \centering
    \begin{tikzpicture}[->,>=stealth]
        \node[draw,circle] (zero) at (0,0) {$0$};
        \node[draw,circle,right=of zero] (one) {$1$};
        \node[draw,circle,right=of one] (two) {$2$};
        
        \path
        (zero) edge (one)
        (one) edge (two);

        \begin{scope}[xshift=4cm]
            \node (seta-inicio) {};
            \node[right=of seta-inicio] (seta-fim) {};
            \draw (seta-inicio) to node[auto] {$\tau_{2,3}$} (seta-fim);
        \end{scope}
        
        \begin{scope}[xshift=6cm]
            \node[draw,circle] (zero) at (0,0) {$0$};
            \node[draw,red,circle,right=of zero] (one) {$1$};
            \node[draw,red,circle,right=of one] (two) {$2$};
            \node[draw,red,circle,right=of two] (three) {$3$};
            
            \path
            (zero) edge (one)
            (one) edge[red] (two)
            (two) edge[red] (three);
        \end{scope}
    \end{tikzpicture}
    \caption{Visualização do morfismo de aljavas $\tau_{2,3}: [2] \to [3]$. Os vértices e arestas em vermelho indicam a ``imagem'' do morfismo.}
    \label{fig:morfismo_aljavas_segmento_terminal}
\end{figure}

Note que, quando $m=0$, o morfismo $\iota_{0,n}$ destaca o vértice inicial de $[n]$, enquanto o morfismo $\tau_{0,n}$ destaca o último terminal de $[n]$.

O lema abaixo descreve o processo de colagem de aljavas mencionado anteiormente em termos de um pushout.

\begin{lema}\label{lema:pushout_aljavas_ordinais}
    O diagrama comutativo abaixo define um pushout na categoria $\mathsf{Alj}$.
    \begin{equation}\label{eq:pushout_aljavas_ordinais_diag}
        \begin{tikzcd}[column sep=1.35cm]
            {[0]}
            \arrow[r,"{\tau_{0,m}}"]
            \arrow[d, "{\iota_{0,n}}" swap]
            & {[m]}
            \arrow[d, "{\iota_{m,m+n}}"]
            \\ {[n]}
            \arrow[r, "{\tau_{n,m+n}}" swap]
            & {[m+n]}
        \end{tikzcd}
    \end{equation}
\end{lema}

\begin{proof}
    Note primeiro que o diagrama acima é realmente comutativo.
    No único vértice $0$ da aljava $[0]$ temos por um lado
    \begin{displaymath}
        \iota_{m,m+n}(\tau_{0,m}(0))
        = \iota_{m,m+n}(0+m-0)
        = \iota_{m,m+n}(m)
        = m
    \end{displaymath}
    e por outro
    \begin{displaymath}
        \tau_{n,m+n}(\iota_{0,n}(0))
        = \tau_{n,m+n}(0)
        = 0 + (m+n) - n
        = m.
    \end{displaymath}
    
    Suponha agora que $Q$ seja uma aljava qualquer e que tenhamos morfismos $f: [m] \to Q$ e $g: [n] \to Q$ tais que $f \circ \tau_{0,m} = g \circ \iota_{0,n}$.
    Nosso objetivo é mostrar a existência de um único morfismo $\phi: [m+n] \to Q$ fazendo comutar o diagrama abaixo.
    \begin{displaymath}
        \begin{tikzcd}[column sep=1.35cm]
            {[0]}
            \arrow[r,"{\tau_{0,m}}"]
            \arrow[d, "{\iota_{0,n}}" swap]
            & {[m]}
            \arrow[d, "{\iota_{m,m+n}}"]
            \arrow[rdd, bend left=45, "f"]
            \\ {[n]}
            \arrow[r, "{\tau_{n,m+n}}" swap]
            \arrow[rrd, bend right=45, "g" swap]
            & {[m+n]}
            \arrow[rd, dashed, "\phi" description]
            \\ & & Q
        \end{tikzcd}
    \end{displaymath}
    Nos vértices de $[m+n]$ definimos
    \begin{displaymath}
        \phi(k) \coloneqq
        \begin{cases}
            f(k), & \text{se } 0 \leq k \leq m; \\
            g(k-m), & \text{se } m \leq k \leq m+n.
        \end{cases}
    \end{displaymath}
    Veja que $\phi$ está bem-definida nos vértices pois a condição de comutatividade $f \circ \tau_{0,m} = g \circ \iota_{0,n}$ diz precisamente que vale a igualdade $f(m) = g(0)$.
    A definição de $\phi$ nas arestas é a esperada: utilizamos o morfismo $f$ nas $m$ primeiras arestas e o morfismo $g$ nas $n$ últimas.
    Explicitamente, definimos
    \begin{displaymath}
        \phi(k,k+1) \coloneqq
        \begin{cases}
            f(k,k+1), & \text{se } 0 \leq k \leq m-1; \\
            g(k-m,k-m+1), & \text{se } m \leq k \leq m+n-1.
        \end{cases}
    \end{displaymath}
    
    Uma conta direta usando o fato de $f$ e $g$ morfismos de aljavas mostra que $\phi$ é também um morfismo de aljavas.
    Resta agora mostrarmos que $\phi$ é na verdade o único morfismo possível satisfazendo as condições de comutatividade em questão.
    Suponha então que $\psi: [m+n] \to Q$ seja um outro morfismo de aljavas tal que $\psi \circ \iota_{m,m+n} = f$ e $\psi \circ \tau_{n,m+n} = g$.
    Dado um vértice $k \in \{0,\dots,m+n\}$ qualquer, se $0 \leq k \leq m$, então
    \begin{displaymath}
        \psi(k) = \psi(\iota_{m,m+n}(k)) = f(k) = \phi(k),
    \end{displaymath}
    e se $m \leq k \leq m+n$, então
    \begin{displaymath}
        \psi(k) = \psi((k-m)+m) = \psi(\tau_{n,m+n}(k-m)) = g(k-m) = \phi(k);
    \end{displaymath}
    portanto $\psi$ coincide com $\phi$ nos vértices.
    Dada agora uma aresta $(k,k+1)$ de $[m+n]$, se $0 \leq k \leq m-1$ temos
    \begin{displaymath}
        \psi(k,k+1) = \psi(\iota_{m,m+n}(k,k+1)) = f(k,k+1) = \phi(k,k+1),
    \end{displaymath}
    e se $m \leq k \leq m+n-1$ temos
    \begin{align*}
        \psi(k,k+1)
        & = \psi((k-m)+m,(k-m+1)+m) \\
        & = \psi(\tau_{n,m+n}(k-m,k-m+1)) \\
        & = g(k-m,k-m+1) \\
        & = \phi(k,k+1);
    \end{align*}
    portanto $\psi$ coincide com $\phi$ nos vértices também como queríamos mostrar.
\end{proof}

Temos agora em mãos todos os ingredientes necessários para construirmos uma categoria que depois mostraremos que satisfaz a propriedade universal que caracteriza a categoria livremente gerada por uma aljava.

Dada uma aljava qualquer $Q$, considere a categoria $\mathsf{FQ}$ tendo como objetos os vértices $V(Q)$ da aljava inicial $Q$ e tendo como morfismos entre dois objetos $a,\, b \in V(Q)$ o conjunto definido por
\begin{equation}\label{eq:morfismo_categoria_livre_defin}
    \Hom_{\mathsf{FQ}}(a,b) \coloneqq \bigcup_{n \in \mathbb{N}} \{F: [n] \to Q \mid F(0) = a \text{ e } F(n) = b\};
\end{equation}
ou seja, um morfismo de $a$ para $b$ em $\mathsf{FQ}$ é um caminho de tamanho qualquer na aljava $Q$ começando em $a$ e terminando em $b$.

Vamos definir uma operação de composição nesses conjuntos de morfismos.
Dados três objetos $a$, $b$ e $c$ em $\mathsf{FQ}$, se $F:[m] \to Q$ é um morfismo de $a$ para $b$, e $G:[n] \to Q$ é um morfismo de $b$ para $c$, como $F(m) = b = G(0)$, os morfismos compostos $F \circ \tau_{0,m},\, G \circ \iota_{0,n}: [0] \to Q$ são iguais, portanto existe um único morfismo de aljavas $G \star F: [m+n] \to Q$ fazendo comutar o diagrama mostrado abaixo graças ao \cref{lema:pushout_aljavas_ordinais}.
\begin{displaymath}
    \begin{tikzcd}
        {[0]}
        \arrow[r, "{\tau_{0,m}}"]
        \arrow[d, "{\iota_{0,n}}" swap]
        & {[m]}
        \arrow[d, "\iota_{m,m+n}"]
        \arrow[rdd, bend left=45, "F"]
        \\ {[n]}
        \arrow[r, "{\tau_{n,m+n}}" swap]
        \arrow[rrd, bend right=45, "G" swap]
        & {[m+n]}
        \arrow[rd, dashed, "G \star F" description]
        \\ & & Q
    \end{tikzcd}
\end{displaymath}
Veja que pelas condições de comutatividade temos
\begin{displaymath}
    (G \star F)(0) = [(G \star F) \circ \iota_{m,m+n}](0) = F(0) = a
\end{displaymath}
e também
\begin{displaymath}
    (G \star F)(m+n) = [(G \star F) \circ \tau_{n,m+n}](n) = G(n) = c;
\end{displaymath}
ou seja, $G \star F$ define um caminho em $Q$ começando em $a$ e terminando em $c$, portanto um morfismo de $a$ para $c$ em $\mathsf{FQ}$ conforme definido em \eqref{eq:morfismo_categoria_livre_defin}.
É este morfismo $G \star F$ que consideraremos como sendo a composição dos morfismos $F$ e $G$.
Note que pela construção dada no \cref{lema:pushout_aljavas_ordinais} podemos descrever $G \star F$ explictamente: em um vértice $k \in \{0,\dots,m+n\}$ qualquer temos
\begin{equation}\label{eq:composicao_caminhos_aljava_1}
    (G \star F)(k) =
    \begin{cases}
        F(k), & \text{se } 0 \leq k \leq m, \\
        G(k-m), & \text{se } m \leq k \leq m+n;
    \end{cases}
\end{equation}
e em uma aresta $(k,k+1)$ qualquer temos
\begin{equation}\label{eq:composicao_caminhos_aljava_2}
    (G \star F)(k,k+1) =
    \begin{cases}
        F(k,k+1), & \text{se } 0 \leq k \leq m-1, \\
        G(k-m,k-m+1), & \text{se } m \leq k \leq m+n-1.
    \end{cases}
\end{equation}

Dado um objeto $a \in \mathsf{FQ}$ qualquer, considere o morfismo de aljavas $p_a: [0] \to Q$ que mapeia o único vértice $0 \in [0]$ para $a$.
Esse morfismo define um caminho na aljava $Q$ começando e terminando no vértice $a$, ou seja, um morfismo do tipo $a \to a$ na categoria $FQ$.
Vamos mostrar que esse morfismo é uma identidade para a operação de composição definida acima.
Considere um morfismo do tipo $a \to b$ dado por um caminho $G:[n] \to Q$ começando em $a$ e terminando em $b$ e vamos mostrar que vale a igualdade $G \star p_a = G$.
Uma maneira de fazermos isso é simplesmente compararmos diretamente ambos os lados usando a descrição explícita para a composição $G \star p_a$ dada acima, mas também podemos mostrar essa igualdade usando a unicidade que caracteriza o morfismo $G \star p_a$.
Mais precisamente, $G \star p_a$ é o \emph{único} morfismo de alajvas do tipo $[0+n] = [n] \to Q$ que faz comutar o diagrama mostrado abaio.
\begin{displaymath}
    \begin{tikzcd}
        {[0]}
        \arrow[r, "{\tau_{0,0}}"]
        \arrow[d, "{\iota_{0,n}}" swap]
        & {[0]}
        \arrow[d, "\iota_{0,n}"]
        \arrow[rdd, bend left=45, "p_a"]
        \\ {[n]}
        \arrow[r, "{\tau_{n,n}}" swap]
        \arrow[rrd, bend right=45, "G" swap]
        & {[n]}
        \arrow[rd, dashed, "G \star p_a" description]
        \\ & & Q
    \end{tikzcd}
\end{displaymath}
Segue dessa unicidade que caracteriza $G \star p_a$ que a validade da igualdade $G \star p_a = G$ é equivalente à validade das igualdades
\begin{displaymath}
    \begin{cases}
        G \circ \iota_{0,n} = p_a, \\
        G \circ \tau_{n,n} = G.
    \end{cases}
\end{displaymath}
A primeira dessas igualdade é verdadeira pois $G$ é por hipótese um caminho começando no vértice $a$, enquanto a segunda é verdadeira pois o morfismo $\tau_{n,n}$ é igual ao morfismo identidade $\id_{[n]}$.
Considere agora um morfismo do tipo $z \to a$ na categoria $FQ$ dado por um caminho $F:[m] \to Q$ começando em $z$ e terminando em $a$.
A composição $p_a \star F: [m] \to Q$ é por definição o único morfismo de aljavas do tipo $[m] \to Q$ fazendo comutar o diagrama abaixo.
\begin{displaymath}
    \begin{tikzcd}
        {[0]}
        \arrow[r, "{\tau_{0,m}}"]
        \arrow[d, "{\iota_{0,0}}" swap]
        & {[m]}
        \arrow[d, "\iota_{m,m}"]
        \arrow[rdd, bend left=45, "F"]
        \\ {[0]}
        \arrow[r, "{\tau_{0,m}}" swap]
        \arrow[rrd, bend right=45, "p_a" swap]
        & {[m]}
        \arrow[rd, dashed, "p_a \star F" description]
        \\ & & Q
    \end{tikzcd}
\end{displaymath}
A igualdade $p_a \star F = F$ seguirá então se mostrarmos que $F$ satisfaz as igualdades
\begin{displaymath}
    \begin{cases}
        F \circ \iota_{m,m} = F, \\
        F \circ \tau_{0,m} = p_a.
    \end{cases}
\end{displaymath}
A primeira das igualdades acima é válida pois o morfismo $\iota_{m,m}$ é igual ao morfismo identidade $\id_{[m]}$, enquanto a segunda é válida pois o caminho $F$ termina no vértice $a$ por hipótese.
Concluímos enfim que o morfismo do tipo $a \to a$ em $\mathsf{FQ}$ dado pelo caminho $p_a: [0] \to Q$ é uma identidade para a operação de composição $\star$ em questão.

Vejamos agora a questão da associatividade da operação de composição $\star$.
Essa propriedade pode ser verificada de uma forma categórica similar ao que fizemos acima, mas por comodidade vamos mostrar isso usando as fórmulas explícitas para a composição.
Considere então objetos $a$, $b$, $c$ e $d$ em $\mathsf{FQ}$ e dados morfismos do tipo $a \to b$, $b \to c$ e $c \to d$ representados por caminhos $F: [\ell] \to Q$, $G: [m] \to Q$ e $H: [n] \to Q$, respectivamente.
Usando \eqref{eq:composicao_caminhos_aljava_1} vemos que a ação do morfismo $(H \star G) \star F: [\ell+m+n] \to Q$ nos vértices é dada pela fórmula
\begin{align*}
    ((H \star G) \star F)(k) & =
    \begin{cases}
        F(k), & \text{se } 0 \leq k \leq \ell, \\
        (H \star G)(k-\ell), & \text{se } \ell \leq k \leq \ell+m+n
    \end{cases} \\
    & =
    \begin{cases}
        F(k), & \text{se } 0 \leq k \leq \ell, \\
        G(k-\ell), & \text{se } \ell \leq k \leq \ell+m, \\
        H(k-\ell-m), & \text{se } \ell+m \leq k \leq \ell+m+n;
    \end{cases}
\end{align*}
enquanto sua ação nas arestas é dada pelas fórmulas
\begin{align*}
    ((H \star G) \star F)(k,k+1) & =
    \begin{cases}
        F(k,k+1), & \text{se } 0 \leq k \leq \ell-1, \\
        (H \star G)(k-\ell,k-\ell+1), & \text{se } \ell \leq k \leq \ell+m+n-1,
    \end{cases} \\
    & =
    \begin{cases}
        F(k,k+1), & \text{se } 0 \leq k \leq \ell-1, \\
        G(k-\ell,k-\ell+1), & \text{se } \ell \leq k \leq \ell+m-1, \\
        H(k-\ell-m,k-\ell-m+1), & \text{se } \ell+m \leq k \leq \ell+m+n-1.
    \end{cases}
\end{align*}
Analogamente, a ação de $H \star (G \star F): [\ell+m+n] \to Q$ nos vértices é dada pelas fórmulas
\begin{align*}
    (H \star (G \star F))(k)
    & =
    \begin{cases}
        (F \star G)(k) & \text{se } 0 \leq k \leq \ell+m, \\
        H(k-\ell-m), & \text{se } \ell+m \leq k \leq \ell+m+n,
    \end{cases} \\
    & =
    \begin{cases}
        F(k), & \text{se } 0 \leq k \leq \ell, \\
        G(k-\ell), & \text{se } \ell \leq k \leq \ell+m, \\
        H(k-\ell-m), & \text{se } \ell+m \leq k \leq \ell+m+n;
    \end{cases}
\end{align*}
enquanto sua ação nas arestas é dada pelas fórmulas
\begin{align*}
    (H \star (G \star F))(k,k+1)
    & =
    \begin{cases}
        (G \star F)(k,k+1), & \text{se } 0 \leq k \leq \ell+m-1, \\
        H(k-\ell-m,k-\ell-m+1), & \text{se } \ell+m \leq k \leq \ell+m+n-1,
    \end{cases} \\
    & =
    \begin{cases}
        F(k,k+1), & \text{se } 0 \leq k \leq \ell-1, \\
        G(k-\ell,k-\ell+1), & \text{se } \ell \leq k \leq \ell+m-1, \\
        H(k-\ell-m,k-\ell-m+1), & \text{se } \ell+m \leq k \leq \ell+m+n.
    \end{cases}
\end{align*}
Comparando as fórmulas obtidas vemos que os morfismos $(H \star G) \star F$ e $H \star (G \star F)$ agem exatamente da mesma forma tanto nos vértices quanto nas arestas de $[\ell+m+n]$, portanto vale a igualdade $(H \star G) \star F = H \star (G \star F)$ desejada.

O raciocínio acima mostra a associatividade da operação de composição $\star$ definida e nos permite concluir enfim que $\mathsf{FQ}$ define de fato uma categoria.
Veja que $\mathsf{FQ}$ é uma categoria pequena, visto que seus objetos formam um conjunto - o conjunto de vértices de $Q$ por definição -  e para quaisquer dois objetos $a$ e $b$ de $\mathsf{FQ}$, os morfismos do tipo $a \to b$, ou seja, os caminhos na aljava $Q$ começando em $a$ e terminando em $b$ também constituem um conjunto.

Uma propriedade bastante útil da categoria $\mathsf{FQ}$ é que todo morfismo nela pode ser decomposto como uma composição de morfismos que são em certo sentidos mais simples.
O significado preciso disso é o conteúdo do lema abaixo.

\begin{lema}\label{lema:decompondo_caminhos_aljava}
    Seja $Q$ um aljava qualquer.
    Dado um natural $n \geq 1$ e um caminho $\gamma: [n] \to Q$, existem caminhos $\gamma_1,\dots,\gamma_n: [1] \to Q$ tais que
    \begin{displaymath}
        \gamma = \gamma_n \star \dotsm \star \gamma_1.
    \end{displaymath}
\end{lema}

\begin{proof}
    A prova é por indução sobre o comprimento do caminhos.
    Se $n = 1$ o resultado é imediato pois o caminho já está decomposto na forma necessária, ou seja, basta tomarmos $\gamma_1 \coloneqq \gamma$.

    Suponha agora que o resultado seja verdadeiro para caminhos do tipo $[n] \to Q$ e considere então um caminho $\gamma: [n+1] \to Q$.
    Lembrando dos morfismos de aljavas $\tau_{1,n+1}: [1] \to [n+1]$ e $\iota_{n,n+1}: [n] \to [n+1]$ introduzidos logo antes do \cref{lema:pushout_aljavas_ordinais}, considere os caminhos $\gamma_{n+1}: [1] \to Q$ e $\delta: [n] \to Q$ definidos pelas composições
    \begin{displaymath}
        \gamma_{n+1} \coloneqq \gamma \circ \tau_{1,n+1}
        \quad \text{e} \quad
        \delta \coloneqq \gamma \circ \iota_{n,n+1}.
    \end{displaymath}
    
    Podemos então formar o caminho composto $\gamma_{n+1} \star \delta: [n+1] \to Q$ que por construção é o único de seu tipo que faz comutar o diagrama abaixo.
    \begin{displaymath}
        \begin{tikzcd}[column sep=1.25cm]
            {[0]}
            \arrow[r, "{\tau_{0,n}}"]
            \arrow[d, "{\iota_{0,1}}" swap]
            & {[n]}
            \arrow[d, "{\iota_{n,n+1}}"]
            \arrow[rdd, bend left=45, "\delta"]
            \\ {[1]}
            \arrow[r, "{\tau_{1,n+1}}" swap]
            \arrow[rrd, bend right=45, "\gamma_{n+1}" swap]
            & {[n+1]}
            \arrow[rd, dashed, "\gamma_{n+1} \star \delta" description]
            \\ & & Q
        \end{tikzcd}
    \end{displaymath}
    Ora, fica claro da definição dos morfismos $\delta$ e $\gamma_{n+1}$ que o caminho inicial $\gamma: [n+1] \to Q$ também faz comutar o diagrama acima, portanto segue da unicidade que vale a igualdade
    \begin{displaymath}
        \gamma = \gamma_{n+1} \star \delta.
    \end{displaymath}

    Aplicando a hipótese de indução ao caminho $\delta: [n] \to Q$ obtemos caminhos $\gamma_1, \dots, \gamma_n: [1] \to Q$ tais que $\delta = \gamma_n \star \dotsm \star \gamma_1$.
    Substituindo isso na igualdade anterior obtemos a decomposição desejada
    \begin{displaymath}
        \gamma = \gamma_{n+1} \star \delta
        = \gamma_{n+1} \star \gamma_n \star \dotsm \star \gamma_1. \qedhere
    \end{displaymath}
\end{proof}

Nosso objetivo é mostrar que a categoria $\mathsf{FQ}$ que acabamos de construir fornece um modelo concreto para a categoria livremente gerada por $Q$.
Tendo em vista a \cref{defin:categoria_livre}, precisamos antes de algum morfismo de aljavas $i: Q \to U(\mathsf{FQ})$, onde $U: \mathsf{Cat} \to \mathsf{Alj}$ denota o funtor de esquecimento.
Pensando inicialmente na ação de $i$ nos vértices, lembre que pela definição do funtor $U$, a aljava $U(\mathsf{FQ})$ tem como vértices os objetos de $\mathsf{FQ}$, mas o conjunto de tais objetos é por definição o conjunto $V(Q)$ de vértices da aljava $Q$ inicial, portanto faz sentido definirmos $i: V(Q) \to V(U(\mathsf{FQ}))$ pondo $i(a) \coloneqq a$ para todo vértice $a \in V(Q)$.
Pensando agora na ação nas arestas, lembre-se que por definição o conjunto de arestas $A(U(\mathsf{FQ}))$ é dado pelo conjunto de todos os morfismos em $\mathsf{FQ}$, ou seja, todos os caminhos na aljava $Q$.
Assim, precisamos associar a cada aresta de $Q$ um caminho na própria aljava $Q$, e a ideia por trás disso é interpretar uma aresta como um caminho entre seus vértices inicial e final.
Mais precisamente, se $e \in A(Q)$ é uma aresta qualquer, considere o caminho $i_e: [1] \to Q$ definido da seguinte forma: nos vértices de $[1]$ temos $i_e(0) \coloneqq s_Q(e)$ e $i_e(1) \coloneqq t_Q(e)$, onde $s_Q,\,t_Q: A(Q) \to V(Q)$ são as funções \emph{source} e \emph{target} de $Q$; enquanto na única aresta $(0,1) \in A([1])$ temos $i_e(0,1) \coloneqq e$.
Uma conta direta mostra que $i_e$ é compatível com as funções source e target de $[1]$ e $Q$, definindo de fato um morfismo de aljavas e, portanto, um morfismo na categoria $\mathsf{FQ}$.
Tendo essa construção em mãos, definimos uma função $i: A(Q) \to A(U(\mathsf{FQ}))$ entre as arestas simplesmente por $i(e) \coloneqq i_e$ para toda aresta $e \in A(Q)$.

As funções $i$ a nível de vértices e arestas construídas acima juntas definem um morfismo de aljavas $i: Q \to U(\mathsf{FQ})$.
De fato, como as funções source e target da aljava $U(\mathsf{FQ})$ são as funções domínio e codomínio da categoria $\mathsf{FQ}$, e para toda aresta $e \in A(Q)$, $i_e$ define explicitamente um morfismo do tipo $s_Q(e) \to t_Q(e)$, fica claro que as funções $i$ são compatíveis com as funções source e target das aljavas $Q$ e $\mathsf{FQ}$.

Tendo em mãos um ``morfismo de comparação'' entre $Q$ e $U(\mathsf{FQ})$, podemos enfim demonstrar o resultado principal da seção.

\begin{teo}\label{teo:categoria_livremente_gerada_por_aljava}
    Dada uma aljava qualquer $Q$, o par $(\mathsf{FQ},i)$ satisfaz a propriedade universal de uma categoria livremente gerada por $Q$.
\end{teo}

\begin{proof}
    Seja $\mathsf{C}$ uma categoria pequena e suponha que tenhamos um morfismo de aljavas $f: Q \to U(\mathsf{C})$.
    Nosso objetivo é mostrar que esse morfismo de aljavas pode ser unicamente ``estendido'' a um funtor $F: \mathsf{FQ} \to \mathsf{C}$.
    Vamos denotar por $f_V: V(Q) \to V(U(\mathsf{C})) = \Ob(\mathsf{C})$ a ação do morfismo $f$ a nível dos vértices e por $f_A: A(Q) \to A(U(\mathsf{C})) = \Mor(\mathsf{C})$ a ação do morfismo $f$ a nível das arestas.

    Um funtor $F: \mathsf{FQ} \to \mathsf{C}$ é dado por funções $F_{\Ob}: \Ob(\mathsf{FQ}) \to \Ob(\mathsf{C})$ e $F_{\Mor}: \Mor(\mathsf{FQ}) \to \Mor(\mathsf{C})$ compatíveis com toda a estrutura presente.
    Por construção temos $\Ob(\mathsf{FQ}) = V(Q)$, logo a nível de objetos podemos definir uma função $F_{\Ob}: V(Q) \to \Ob(\mathsf{C})$ por $F_{\Ob}(a) \coloneqq f_V(a)$ para todo vértice $a \in V(Q)$.

    Resta definirmos uma função $F_{\Mor}: \Mor(\mathsf{FQ}) \to \Mor(\mathsf{C})$.
    Lembrando que por construção os morfismos em $\mathsf{FQ}$ são dados por caminhos de comprimento qualquer na aljava $Q$, a ideia é pensar em um caminho como uma ``composição de arestas'' vistas como caminhos de comprimento $2$, e transformar essa composição de arestas em uma composição de morfismos em $\mathsf{C}$ por meio da função $f_A: A(Q) \to \Mor(\mathsf{C})$.
    Mais precisamente, dado um caminho $\gamma: [n] \to Q$ com $n \geq 1$, ou seja, $\gamma$ tem comprimento pelo menos $2$, definimos um morfismo $F_{\Mor}(\gamma) \in \Mor(\mathsf{C})$ pela fórmula:
    \begin{equation}\label{eq:cat_livre_funtor_induzido}
        F(\gamma) \coloneqq f_A(\gamma_A(n-1,n)) \circ \dotsm \circ f_A(\gamma_A(1,2)) \circ f_A(\gamma_A(0,1)).
    \end{equation}
    Vejamos que a expressão acima faz sentido.
    Primeiramente, para qualquer $k \in \{0,\dots,n-1\}$, $(k,k+1)$ é uma aresta de $[n]$, logo $\gamma_A(k,k+1)$ é uma aresta de $Q$, e $f_A(\gamma_A(k,k+1))$ é uma aresta da aljava subjacente $U(\mathsf{C})$, ou seja, um morfismo da categoria $\mathsf{C}$.
    Além disso, como $f$ e $\gamma$ são morfismos de aljavas por hipótese, por um lado temos
    \begin{align*}
        \dom(f_A(\gamma_A(k+1,k+2)))
        & = f_V(s_Q(\gamma_A(k+1,k+2))) \\
        & = f_V(\gamma_V(s(k+1,k+2))) \\
        & = f_V(\gamma_V(k+1)),
    \end{align*}
    e por outro
    \begin{align*}
        \cod(f_A(\gamma_A(k,k+1)))
        & = f_V(t_Q(\gamma_A(k,k+1))) \\
        & = f_V(\gamma_V(t(k,k+1))) \\
        & = f_V(\gamma(_Vk+1));
    \end{align*}
    ou seja, as composições da forma $f_A(\gamma_A(k+1,k+2)) \circ f_A(\gamma_A(k,k+1))$ que aparecem na equação \eqref{eq:cat_livre_funtor_induzido} fazem sentido.

    É claro que a definição acima só funciona se $n \geq 1$.
    No caso de um caminho $\gamma: [0] \to Q$ de tamanho $1$, definimos à força $F_{\Mor}(\gamma) \coloneqq \id_{f_V(\gamma_V(0))}$.
    Veja que isso faz bastante sentido pois os caminhos de comprimento $1$ são os morfismos idênticos da categoria $\mathsf{FQ}$, e é bom que a função $F_{\Mor}$ preserve esses morfismos.

    Vejamos agora que as funções $F_{\Ob}$ e $F_{\Mor}$ juntas definem de fato um funtor $\mathsf{FQ} \to \mathsf{C}$.
    O primeiro passo é verificarmos a compatibilidade com as funções de domínio e codomínio das respectivas categorias.
    Veja inicialmente que pela definição de um morfismo de aljavas temos as condições de compatibilidade
    \begin{displaymath}
        f_V \circ s_Q = \dom_{\mathsf{C}} \circ f_A
        \quad \text{e} \quad
        f_V \circ t_Q = \cod_{\mathsf{C}} \circ f_A.
    \end{displaymath}
    Inicialmente, se $\gamma: [0] \to Q$ é um caminho de comprimento de $1$, temos
    \begin{align*}
        \dom_{\mathsf{C}}(F_{\Mor}(\gamma))
        & = \dom_{\mathsf{C}}(\id_{f_V(\gamma_V(0))}) \\
        & = f_V(\gamma_V(0)) \\
        & = F_{\Ob}(\gamma_V(0)) \\
        & = F_{\Ob}(\dom_{\mathsf{FQ}}(\gamma)).
    \end{align*}
    Um raciocínio completamente análogo mostra que nesse caso também temos a compatibilidade com os codomínios, ou seja, que vale a igualdade $\cod_{\mathsf{C}}(F_{\Mor}(\gamma)) = F_{\Ob}(\cod_{\mathsf{FQ}}(\gamma))$.

    Suponha agora que $\gamma: [n] \to Q$ seja um caminho tal que $n \geq 1$.
    Usando então a definição de $F$ temos
    \begin{align*}
        \dom_{\mathsf{C}}(F_{\Mor}(\gamma))
        & = \dom_{\mathsf{C}}(f_A(\gamma_A(n-1,n)) \circ \dotsm \circ f_A(\gamma_A(1,2)) \circ f_A(\gamma_A(0,1))) \\
        & = \dom_{\mathsf{C}}(f_A(\gamma_A(0,1))) \\
        & = f_V(s_Q(\gamma_A(0,1))) \\
        & = f_V(\gamma_V(s(0,1))) \\
        & = f_V(\gamma_V(0)) \\
        & = F_{\Ob}(\dom_{\mathsf{FQ}}(\gamma)).
    \end{align*}
    Analagomente, no caso das funções codomínio temos
    \begin{align*}
        \cod_{\mathsf{C}}(F_{\Mor}(\gamma))
        & = \cod_{C}(f_A(\gamma_A(n-1,n)) \circ \dotsm \circ f_A(\gamma_A(1,2)) \circ f_A(\gamma_A(0,1))) \\
        & = \cod_{\mathsf{C}}(f_A(\gamma_A(n-1,n))) \\
        & = f_V(t_Q(\gamma_A(n-1,n))) \\
        & = f_V(\gamma_V(t(0,1))) \\
        & = f_V(\gamma_v(n)) \\
        & = F_{\Ob}(\cod_{\mathsf{FQ}}(\gamma)).
    \end{align*}

    Já discutimos anteriormente que $F_{\Mor}$ por construção preserva os morfismos identidade de $\mathsf{FQ}$ dados pelos caminhos de comprimento $1$, logo resta apenas verificarmos a compatibilidade de $F_{\Mor}$ com a operação de composição.
    Considere então dois morfismos $a \to b$ e $b \to c$ dados, respectivamente, por caminhos $\gamma: [m] \to Q$ e $\delta: [n] \to Q$.
    Vamos assumir por simplicidade inicialmente que $n=1$.
    Como já sabemos que $F_{\Mor}$ preserva morfismos identidade, podemos assumir também que $m \geq 1$.
    Temos então
    \begin{align*}
        F_{Mor}(\delta) \circ F_{\Mor}(\gamma)
        & = f_A(\delta_A(0,1)) \circ f_A(\gamma_A(m-1,m)) \circ \dotsm \circ f_A(\gamma_A(0,1)) \\
        & = f_A((\delta \star \gamma)_A(m,m+1)) \circ f_A((\delta \star \gamma)_A(m-1,,m)) \circ \dotsm \circ f_A((\delta \star \gamma)_A(0,1)) \\
        & = F_{\Mor}(\delta \star \gamma).
    \end{align*}
    Isso mostra que $F_{\Mor}$ preserva todas as composições nas quais o segundo caminho é do tipo $[1] \to Q$.
    
    O caso acima nos permite demostrar o caso geral fazendo uma indução no comprimento do segundo caminho que aparece na composição.
    O caso em que $n=1$ é o que demonstramos acima.
    Supondo que o resultado seja verdadeiro para caminhos do tipo $[n] \to Q$, considere agora que o segundo caminho é do tipo $\delta: [n+1] \to Q$.
    Uma consequência do \cref{lema:decompondo_caminhos_aljava} é que podemos encontrar caminhos $\delta_{n+1}: [1] \to Q$ e $\delta': [n] \to Q$ tais que $\delta = \delta_{n+1} \star \delta'$.
    Note então que
    \begin{align*}
        F_{\Mor}(\delta) \circ F_{\Mor}(\gamma)
        & = F_{\Mor}(\delta_{n+1} \star \delta') \circ F_{\Mor}(\gamma) \\
        & = F_{\Mor}(\delta_{n+1}) \circ F_{\Mor}(\delta') \circ F_{\Mor}(\gamma)
        \tag{caso base da indução} \\
        & = F_{\Mor}(\delta_{n+1}) \circ F_{\Mor}(\delta' \star \gamma)
        \tag{hipótese de indução} \\
        & = F_{\Mor}(\delta_{n+1} \star \delta' \star \gamma)
        \tag{caso base da indução} \\
        & = F_{\Mor}(\delta \star \gamma).
    \end{align*}

    Sabendo que as funções $F_{\Ob}$ e $F_{\Mor}$ juntas definem realmente um funtor $F: \mathsf{FQ} \to \mathsf{C}$, vejamos que ele satisfaz a condição, $U(F) \circ i = f$.
    Lembremos que o morfismo de aljavas $i: Q \to U(\mathsf{FQ})$ a nível de objetos é dado pela função identidade $\id_{V(Q)}$, enquanto o morfismo $U(F): U(\mathsf{FQ}) \to U(\mathsf{C})$ é dado pela função $F_{\Ob}: \Ob(\mathsf{FQ}) \to \Ob(\mathsf{C})$, a qual foi definida como sendo igual à função $f_V: V(Q) \to \Ob(\mathsf{C})$.
    É imediato então que os morfismos de aljavas $U(F) \circ i$ e $f$ coincidem nos vértices de $Q$.
    Lembre agora que $U(F)$ é dado nas arestas pela função $F_{\Mor}: \Mor(\mathsf{FQ}) \to \Mor(\mathsf{C})$ que construímos acima, e que $i_A: A(Q) \to A(U(\mathsf{FQ}))$ mapeia uma aresta $e \in A(Q)$ para o caminho $i_e: [1] \to Q$ conectando $s_Q(e)$ a $t_Q(e)$ por meio da própria aresta $e$.
    Usando então a definição de $F_{\Mor}$ vemos que
    \begin{displaymath}
        F_{\Mor}(i_A(e))
        = F_{\Mor}(i_e)
        = f_A(i_e(0,1))
        = f_A(e);
    \end{displaymath}
    mostrando assim que $F \circ i$ e $f$ também coincidem nas arestas de $Q$, portanto vale a igualdade $F \circ i = f$ desejada.

    Enfim, resta apenas verificarmos a unicidade do funtor $F$ satisfazendo tal condição de comutatividade.
    Suponha que $G = (G_{\Ob},G_{\Mor}): \mathsf{FQ} \to \mathsf{C}$ seja um outro funtor satisfazendo $U(G) \circ i = f$.
    No nível dos vértices, isso significa que vale a igualdade $G_{\Ob} \circ i_V = f_V$, mas como $i_V$ é dado pela função idêntica $\id_{V(Q)}$, essa igualdade diz apenas que $G_{\Ob} = f_v = F_{\Ob}$, ou seja, já concluímos assim que $F$ e $G$ coincidem nos objetos da categoria $\mathsf{FQ}$.
    Pensando agora nas arestas, temos a igualdade $G_{\Mor} \circ i_A = f_A$.
    Afirmamos então que $G_{\Mor}$ e $F_{\Mor}$ coincidem nos morfismos de $\mathsf{FQ}$ dados por caminhos do tipo $[1] \to Q$.
    Dado um tal caminho $\gamma: [1] \to Q$, se definirmos a aresta $e \coloneqq \gamma_A(0,1) \in A(Q)$, então por hipótese temos $G_{\Mor}(i_A(e)) = f_A(e)$.
    Ora, o caminho $i_A(e) = i_e: [1] \to Q$ é nada mais que o próprio caminho $\gamma$ considerado inicialmente, já que no vértice $0 \in V([1])$ temos
    \begin{displaymath}
        i_e(0)
        = s_Q(e)
        = s_Q(\gamma_A(0,1))
        = \gamma_V(s(0,1))
        = \gamma_V(0),
    \end{displaymath}
    no vértice $1 \in V([1])$ temos uma igualdade análoga, e na única areta $(0,1) \in A([1])$ temos
    \begin{displaymath}
        i_e(0,1)
        = e
        = \gamma_A(0,1).
    \end{displaymath}
    Sabendo disso, a hipótese sobre $G$ nos permite concluir que $G_{\Mor}(\gamma)$ é dado por $f_A(\gamma_A(0,1))$, mas esse é justamente o valor de $F_{\Mor}(\gamma)$ também.
    Concluímos assim que $G_{\Mor}$ e $F_{\Mor}$ coincidem nos caminhos de comprimento $2$ em $Q$, e combinando isso com uma aplicação direta do \cref{lema:decompondo_caminhos_aljava} concluímos que as funções $G_{\Mor}$ e $F_{\Mor}$ coincidem na verdade em todos os caminhos.
\end{proof}

\begin{obs}[Questões de tamanho]
    \label{obs:questoes_de_tamanho}
    Frequentemente na Teoria de Categorias somos naturalmente levados a manipular coleções grandes demais para serem consideradas conjuntos legítimos segundo os axiomas de ZFC usuais da Teoria de Conjuntos, e a manipulação descuidada de tais coleções pode causar algumas dores de cabeça.
    
    Uma maneira de contornarmos esse problema é trabalharmos com um sistema de axiomas que nos permita manipular formalmente tais coleções muito grandes.
    Um exemplo possível de tal sistema são os axiomas de \emph{Neumann-Bernays-Gödel}, no qual as coleções muito grandes mencionadas anteriormente são chamadas de \emph{classes}.

    Existe uma outra estratégia - a qual vem se tornando mais e mais popular -  para contornarmos o problema em questão sem nos distanciarmos tanto da axiomática usual da Teoria de Conjuntos: suplementamos os axiomas de ZFC com novos axiomas a respeito da existência dos chamados \emph{universos de Gröthendieck}.
    Sem entrar em muitos detalhes, um universo de Gröthendieck é um conjunto $U$ grande o suficiente para que possamos ``fazer Teoria dos Conjuntos internamente a $U$'', ou seja, podemos utilizar todas as construções usuais da Teoria de Conjuntos com os elementos de $U$ e ter a certeza de que ao final obteremos um outro elemento de $U$.
    Quando fixamos um tal universo, dizemos que os conjuntos que pertencem a $U$ são \emph{pequenos}.
   
    Fixada então essa noção \emph{relativa} de tamannho, redefinimos nossos objetos de estudo como sendo os conjuntos pequenos, os grupos pequenos, os espaços topológicos pequenos, e por aí vai...
    Redefinimos então nos objetos de estudo de forma a levarmos em conta essa restrição relativa sobre o tamanho dos conjuntos subjacentes: passamoss a estudar grupos pequenos, espaços vetoriais pequenos e espaços topológicos pequenos, por exemplo.

    Poderíamos então considerar apenas categorias cujos conjuntos de objetos e morfismos são pequenos em relação ao universo fixado inicialmente.
    Infelizmente, isso não é suficiente para resolver todos os nossos problemas.
    Veja que o universo $U$ em si não é pequeno, pois isso implicaria $U \in U$, relação esta que nunca é válida em ZFC graças ao Axioma de Regularidade.
    Consequentemente, diversas coleções de objetos pequenas não são pequenas.
    Assim, embora posssamos falar de conjuntos pequenos em relação ao $U$, não podemos formar a categoria $U\mathsf{Set}$ cujos objetos são os conjuntos pequenos, já que $U$ em si não é um conjunto pequeno.
    Essa noção de categoria com uma condição de tamanho relativas nos objetos e morfismos é então restrita demais na prática.

    A solução frequentemente adotada é então trabalharmos com \emph{dois} universos de Gröthendieck $U_1$ e $U_2$ tais que $U_1 \in U_2$.
    Os conjuntos pertencentes a $U_1$ seriam então os conjuntos pequenos da teoria, os conjuntos pertencentes a $U_2$ seria os conjuntos grandes, e os conjuntos que não pertencem nem mesmo a $U_2$ seriam considerado demasiadamente grandes e estariam fora do escopo da teoria.

    Adotando então essa estratégia de dois universos de Gröthendieck, definimos uma categoria $\mathsf{C}$ como tendo conjuntos possivelmente grandes de objetos e morfismos, ou seja, consideramos que $\Ob(\mathsf{C})$ e $\Mor(\mathsf{C})$ pertencem a $U_2$.
    No caso especial em que $\Ob(\mathsf{C})$ e $\Mor(\mathsf{C})$ pertencem não só a $U_2$, mas também a $U_1$, dizemos que $\mathsf{C}$ é uma categoria pequena.
    Note que agora podemos construir uma categoria $U_1-\mathsf{Set}$ tendo como objetos os conjuntos pequenos da categoria.
    Embora tal categoria não seja pequena, a existência do segundo universo $U_2$ nos permite construí-la formalmente dentro da teoria.
    Analogamente, podemos também considerar a categorias (grandes!) $U_1-\mathsf{Grp}$ cujos objetos são os grupos pequenos da teoria, ou a categoria $U_1-\mathsf{Top}$ cujos objetos são os espaços topológicos pequenos da teoria.

    As duas noções relativas de tamanho nos permitem formalizar tambéma a noção de uma categoria localmente pequena.
    Precisamente, uma categoria $\mathsf{C}$ (possivelmente grande) é dita \emph{localmente pequena} se, dados quaisquer dois objetos $X,\, Y \in \Ob(\mathsf{C})$, o conjunto de morfismos $\Mor_{\mathsf{C}}(X,Y)$ é pequeno na teoria, ou seja, $\Mor_{\mathsf{C}}(X,Y) \in U_1$.
    Todas as categorias mencionadas no parágrafo anterior são localmente pequenas nesse sentido preciso.

    Veja que é possível então impormos condições de tamanho sobre outros objetos que tenham relação com categorias.
    Por exemplo, podemos agora considerar aljavas $Q$ pequenas ou grandes, a depender do tamanhos dos conjuntos $V(Q)$ e $A(Q)$ de vértices e arestas.
    Naturalmente, podemos também definir a noção de uma aljava localmente pequena.
    A construção apresentada na demonstração do \cref{teo:categoria_livremente_gerada_por_aljava} mostra então que, se uma aljava $Q$ é pequena em relação a algum universo fixado, então a categoria $\mathsf{FQ}$ livremente gerada por ela é também pequena em relação a esse universo.
    Note, entretanto, que a propriedade universal de uma categoria livre só faz sentido para o primeiro universo $U_1$, já que a categoria $U_1-\mathsf{Cat}$ formada pelas categorias pequenas forma uma categoria de fato, a qual é grande, mas a coleção de todas as categorias grandes é grande demais para pertencer a $U_2$, portanto não somos capazes de formar a categoria $U_2-\mathsf{Cat}$ utilizando apenas dois universos.

    Um cuidado particular deve ser tomado com as aljavas localmente pequenas.
    Parece razoável esperar que a categoria $\mathsf{FQ}$ livremente gerada por uma aljava localmente pequena $Q$ seja localmente pequena também, mas o exemplo abaixo mostra que isso não é verdade em geral!
\end{obs}

%TODO: Pensar em um exemplo disso...