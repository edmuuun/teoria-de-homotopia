\section{Categorias livres}

Comumente representamos uma categoria por meio de um grafo orientado tendo os objetos da categoria como vértices e os morfismos da categoria como arestas.
Apesar dessa representação bastante útil, uma categoria possui mais estrutura do que um mero grafo orientado, visto que podemos compor os morfismos de uma categoria para obtermos novos morfismos, mas não existe uma operação análoga de composição das arestas de um grafo.
Mais precisamente, toda categoria possui um grafo orientado \emph{subjacente} que esquece da estrutura algébrica da composição.
Os leitores com alguma experiência em Teoria de Categorias já devem esperar que exista também uma construção no sentido contrário por meio da qual um grafo orientado dá origem a uma categoria obtida pela adição da forma mais simples possível de morfismos que representem a composição das arestas do grafo.
A categoria assim obtida é a chamada \emph{categoria livre} gerada pelo grafo, e essa construção é o objetivo central desta seção.

\begin{defin}\label{defin:grafo_orientado}
    Um \textbf{grafo orientado} consiste de uma tupla $(V,A,s,t)$, onde $V$ e $A$ são conjuntos, e $s,\,t: A \to V$ são funções.
\end{defin}

Dado um grafo orientado $(V,A,s,t)$, os elementos de $V$ são chamados de \textbf{vértices} do grafo, enquanto os elementos de $A$ são chamados de \textbf{arestas} do grafo.
As funções $s$ e $t$ são chamadas de funções \textbf{source} e \textbf{target}, respectivamente.
Visualmente, cada elemento $e \in A$ deve ser visualizado como uma aresta direcionada começando no vértice $s(e)$ e terminando no vértice $t(e)$.

\begin{exem}\label{exem:grafo_orientado_discreto}
    Um conjunto $X$ pode ser usado para definir um grafo orientado $GX$ dado pela tupla $GX \coloneqq (X,X,\id_X,\id_X)$.
    Visualmente, tal grafo possui um vértice para cada elemento $x \in X$ e também uma aresta direcionada desse elemento para si mesmo formando um laço.
    O grafo orientado $GX$ assim deifnido é chamado de \textbf{grafo orientado discreto} gerado por $X$.
\end{exem}

\begin{exem}\label{exem:grafo_orientado_via_relacao}
    Seja $R \subseteq X \times X$ uma relação qualquer em um conjunto $X$.
    Podemos construir a partir dessa relação um grafo orientado $G \coloneqq (X,R,s,t)$, onde $s,\,t: R \to X$ são as restriçõs das projeções canônicas associadas ao produto $X \times X$, ou seja, $s(x_1,x_2) \coloneqq x_1$ e $t(x_1,x_2) \coloneqq x_2$ para qualquer par $(x_1,x_2) \in R$.

    É especialmente comum considerarmos essa construção no caso em que a relação $R$ é uma pré-ordem $\leq$ no conjunto $X$, ou seja, uma relação reflexiva e transitiva.
    Nesse caso, o sentido das arestas da grafo orientado ``acompanham o sentido de crescimento'' da pré-ordem $\leq$.
\end{exem}

\begin{exem}\label{exem:grafo_orientado_oposto}
    Todo grafo orientado $G=(V,A,s,t)$ dá origem a um \textbf{grafo orientado oposto} $G^{\mathsf{op}}$ definido pela tupla $(V,A,t,s)$.
    Visualmente, $G^{\mathsf{op}}$ possui os mesmos vértices de $G$, mas com as arestas indo no sentido oposto.
\end{exem}

\begin{exem}\label{exem:grafo_orientado_via_espaco_top}
    Seja $X$ um espaço topológico qualquer, e considere o conjunto\footnote{É verdade que também podemos considerar $X^I$ como um espaço equipado com a topologia compacto-aberto, por exemplo, mas aqui não existe essa necessidade.} $X^I$ formado por todas as funções contínuas do tipo $I \to X$, onde $I$ denota o intervalo unitário $[0,1]$.
    Podemos considerar as funções de avaliação $e_0,\, e_1: X^I \to X$ dadas por $e_0(\gamma) \coloneqq \gamma(0)$ e $e_1(\gamma) \coloneqq \gamma(1)$ para todo $\gamma \in X^{I}$.
    A tupla $(X,X^I,e_0,e_1)$ define então um grafo orientado.
\end{exem}

\begin{exem}\label{exem:grafo_orientado_subjacente_categoria}
    Seja $\mathsf{C}$ uma categoria pequena.
    Se $\dom,\,\cod: \Mor(\mathsf{C}) \to \mathsf{C}$ são as funções que associam a cada morfismo de $\mathsf{C}$ seu domínio e codomínio, respectivamente, então a tupla $U(\mathsf{C}) \coloneqq (\mathsf{C},\Mor(\mathsf{C}),\dom,\cod)$ define o \textbf{grafo orientado subjacente à categoria} $\mathsf{C}$.
\end{exem}

\begin{defin}\label{defin:morfismo_grafos_orientados}
    Dados dois grafos orientados $G_1 \coloneqq (V_1,A_1,s_1,t_1)$ e $G_2 \coloneqq (V_2,A_2,s_2,t_2)$, um \textbf{morfismo de grafos orientados} do tipo $G_1 \to G_2$ é por definição um par de funções $(f: V_1 \to V_2,\, F: A_1 \to A_2)$ satisfazendo as igualdades $s_2 \circ F = f \circ s_1$ e $t_2 \circ F = f \circ t_1$.
\end{defin}

\begin{exem}
    Dados dois conjuntos $X_1$ e $X_2$ e uma função $f: X_1 \to X_2$, o par $(f,f)$ define um morfismo de grafos orientados $(X_1,X_1,\id_{X_1},\id_{X_1}) \to (X_2,X_2,\id_{X_2},\id_{X_2})$ entre os grafos discretos gerados por tais conjuntos de acordo com o \cref{exem:grafo_orientado_discreto}.
\end{exem}

\begin{exem}
    Sejam $(X_1,R_1)$ e $(X_2,R_2)$ dois conjuntos equipados com relações, e suponha que $f: X_1 \to X_2$ seja uma função que respeita as relações, ou seja, se $(x,x') \in R_1$, então $(f(x),f(x')) \in R_2$.
    Considere os grafos $G_1$ e $G_2$ associados às relações $R_1$ e $R_2$, respectivamente, conforme discutido no \cref{exem:grafo_orientado_via_relacao}.
    Como $f$ respeita as relações $R_1$ e $R_2$, a função produto $f \times f: X_1 \times X_1 \to X_2 \times X_2$ satisfaz $(f \times f)(R_1) \subseteq R_2$, portanto podemos considerar a função restrita
    \begin{displaymath}
        f \times f \rvert_{R_1}: R_1 \to R_2.
    \end{displaymath}
    O par de funções $(f,f \times f\rvert_{R_1})$ define então um morfismo de grafos orientados de $G_1$ para $G_2$.

    No caso em que $R_1$ e $R_2$ são relações de pré-ordem $\leq_1$ e $\leq_2$, a condição de $f$ respeitar as relações diz que, se $x \leq_1 x'$, então $f(x) \leq_2 f(x')$; ou seja, $f$ é uma função monótona.
    Segue da discussão acima que funções monótonas entre conjuntos pré-ordenados dão origem a morfismos entre os grafos orientados obtidos a partir deles.
\end{exem}

\begin{exem}
    Uma função contínua $f: X \to Y$ entre espaços topológicos dá origem a uma função de pushforward $X^f: X^I \to Y^{I}$ definida por $X^f(\gamma) \coloneqq f \circ \gamma$ para todo $\gamma \in X^{I}$.
    O par $(f,X^f)$ define então um morfismo de grafos orientados
    \begin{displaymath}
        (f,X^f): (X,X^I,e^X_0,e^X_1) \to (Y,Y^I,e^Y_0,e^Y_1)
    \end{displaymath}
    entre os grafos orientados definidos no \cref{exem:grafo_orientado_via_espaco_top}.
    De fato, para qualquer $\gamma \in X^I$ por um lado temos
    \begin{displaymath}
        e^Y_0(X^f(\gamma)) = e^Y_0(f \circ \gamma) = f(\gamma(0)) = f(e^X_0(\gamma)),
    \end{displaymath}
    enquanto por outro temos
    \begin{displaymath}
        e^Y_1(X^f(\gamma)) = e^Y_1(f \circ \gamma) = f(\gamma(1)) = f(e^X_1(\gamma)).
    \end{displaymath}
\end{exem}

\begin{exem}\label{exem:funtor_induz_morfismo_de_grafos}
    Lembre-se que um funtor $F: \mathsf{C} \to \mathsf{D}$ é dado por uma função $F: \mathsf{C} \to \mathsf{D}$ a nível de objetos e também por uma função $\Mor(F): \Mor(\mathsf{C}) \to \Mor(\mathsf{D})$ a nível de morfismos a qual é compatível com as funções de domínio e codomínio de cada uma das categorias e também com as respectivas operações de composição.
    A compatibilidade com as funções de domínio e codomínio, em particular, faz com que o par $(F,\Mor(F))$ defina um morfismo do tipo $U(\mathsf{C}) \to U(\mathsf{D})$ entre os grafos orientados subjacentes às categorias como discutido no \cref{exem:grafo_orientado_subjacente_categoria}.
\end{exem}

Dois morfismos de grafos orientados $(f_1,F_1): G_1 \to G_2$ e $(f_2,F_2): G_2 \to G_3$ podem ser compostos para definir um outro morfismo de grafos orientados $(f_2,F_2) \circ (f_1,F_1): G_1 \to G_3$ pela regra
\begin{displaymath}
    (f_2,F_2) \circ (f_1,F_1) \coloneqq (f_2 \circ f_1,F_2 \circ F_1).
\end{displaymath}
Além disso, todo grafo orientado $G$ tem associado a ele um morfismo identidade $\id_G: G \to G$ dado pelo par $(\id_{V(G)},\id_{A(G)})$, onde $V(G)$ e $A(G)$ denotam os conjuntos de vértices e arestas de $G$, respectivamente.
Uma conta simples mostra que o morfismo $\id_G$ realmente age como uma identidade para a composição de morfismos de grafos orientados introduzida acima.
Em resumo, temos uma categoria $\mathsf{GrafOr}$ formadas por grafos orientados e morfismos entre eles.
Diversas das construções introduzidas nos exemplos anteriores podem ser estendidas para funtores que estão definidos na categoria $\mathsf{GrafOr}$, ou então que tomam valores nela.

\begin{exem}\label{exem:funtor_esquecimento_cat_grafo_orientado}
    Vimos no \cref{exem:grafo_orientado_subjacente_categoria} que toda categoria pequena $\mathsf{C}$ possui um grafo orientado subjacente $U(\mathsf{C})$ e vimos também no \cref{exem:funtor_induz_morfismo_de_grafos} que um funtor $F: \mathsf{C} \to \mathsf{D}$ induz um morfismo $U(F) \coloneqq (F,\Mor(F)): U(\mathsf{C}) \to U(\mathsf{D})$ entre os grafos orientados subjacentes.
    Essas duas construções combinadas nos permitem definir um \emph{funtor de esquecimento} $U: \mathsf{Cat} \to \mathsf{GrafOr}$ definido na categoria de categorias pequenas e tomando valores na categoria de grafos orientados definida anteriormente.
    Usamos o termo funtor de esquecimento pois ao passar da categoria $\mathsf{C}$ para seu grafo orientado subjacente $U(\mathsf{C})$ nos esquecemos da estrutura de composição de morfismos existente na categoria original.
\end{exem}

Nosso interesse é investigar a existência de um funtor no sentido contrário $\mathsf{GrafOr} \to \mathsf{Cat}$ que associa a cada grafo orientado uma categoria onde podemos compor formalmente suas arestas da forma mais simples possível.
A existência de tal funtor está ligada à existência de categorias livres gerados por grafos orientados que definimos agora.

\begin{defin}\label{defin:categoria_livre}
    Dado um grafo orientado $G$, uma \textbf{categoria livre gerada por $G$} é um par $(\mathsf{C},i)$, onde $\mathsf{C}$ é uma categoria pequena, e $i: G \to U(\mathsf{C})$ é um morfismo de grafos orientados satisfazendo a seguinte propriedade universal: se $\mathsf{D}$ é outra categoria pequena qualquer, e $j: G \to U(\mathsf{D})$ é um morfismo de grafos orientados, então existe um único funtor $F: \mathsf{C} \to \mathsf{D}$ tal que $U(F) \circ i = j$.
    \begin{displaymath}
        \begin{tikzcd}
            U(\mathsf{C})
            \arrow[r, dashed, "U(F)"]
            & U(\mathsf{D})
            \\ G
            \arrow[u, "i"]
            \arrow[ru, "j" swap]
        \end{tikzcd}
    \end{displaymath}
\end{defin}