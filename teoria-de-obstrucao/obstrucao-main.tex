\documentclass[brazilian]{article}

\usepackage{babel}
\usepackage[utf8]{inputenc}
\usepackage{csquotes}
\usepackage[margin=1.35in]{geometry}
\usepackage[backend=biber,style=alphabetic]{biblatex}
\usepackage{mathtools,amssymb,amsthm}
\usepackage{indentfirst}
\usepackage{hyperref}
\usepackage{tikz-cd}
\usepackage{graphicx}
\usepackage[capitalize,noabbrev]{cleveref}

\usetikzlibrary{babel}

\addbibresource{bibliografia.bib}

\swapnumbers
\newtheorem{teo}{Teorema}[section]
\newtheorem{prop}[teo]{Proposição}
\newtheorem{lema}[teo]{Lema}
\newtheorem{corol}[teo]{Corolário}

\theoremstyle{definition}
\newtheorem{defin}[teo]{Definição}
\newtheorem{obs}[teo]{Observação}
\newtheorem{exem}[teo]{Exemplo}

\DeclarePairedDelimiter{\abs}{\lvert}{\rvert}
\DeclarePairedDelimiter{\Abs}{\lVert}{\rVert}

\newcommand{\id}{\mathrm{id}}
\newcommand{\ct}{\mathrm{ct}}
\newcommand{\Mor}{\mathrm{Mor}}
\newcommand{\Hom}{\mathrm{Hom}}
\newcommand{\Arr}{\mathrm{Arr}}
\newcommand{\cofib}{\rightarrowtail}
\newcommand{\Cofib}{\mathrm{Cofib}}
\newcommand{\fib}{\twoheadrightarrow}
\newcommand{\Fib}{\mathrm{Fib}}
\newcommand{\dom}{\mathrm{dom}}
\newcommand{\cod}{\mathrm{cod}}
\newcommand{\comp}{\mathrm{comp}}
\newcommand{\fac}{\mathrm{fac}}
\newcommand{\init}{\mathrm{init}}
\newcommand{\Fun}{\mathrm{Fun}}
\newcommand{\Cyl}{\mathrm{Cyl}}
\newcommand{\Cone}{\mathrm{Cone}}
\newcommand{\Ho}{\mathrm{Ho}}

\renewcommand{\qedsymbol}{$\blacksquare$}

\title{Teoria de Obstrução Categórica\\Notas para o seminário}
\author{Edmundo Martins}
\date{\today}

%%% Local Variables:
%%% mode: latex
%%% TeX-master: "main"
%%% End:


\begin{document}

\maketitle

% TODO:
% - Entender o objeto final da categoria de conjuntos simpliciais.
% - Descrever conjuntos simpliciais pontuados em termos de pares formados por um conjunto simplicial e uma escolha de vértice.

\section{Introdução e motivação}

Considere o quadrado comutativo abaixo em uma categoria modelo $\mathsf{M}$ qualquer, onde $i: A \cofib B$ é uma cofibração, e $p: X \fib Y$ é uma fibração.

\begin{displaymath}
  \begin{tikzcd}
    A
    \arrow[r, "\alpha"]
    \arrow[d, tail, "i" swap]
    & X
    \arrow[d, two heads, "p"]
    \\ B
    \arrow[r, "\beta" swap]
    & Y
  \end{tikzcd}
\end{displaymath}
O \emph{axioma de levantamento} na definição de uma categoria modelo $\mathsf{M}$ garante que, quando $i$ ou $p$ são equivalências fracas, então podemos encontrar um morfismo diagonal $h: B \to X$  (um levantamento) que faz os diagramas resultantes comutarem.
\begin{displaymath}
  \begin{tikzcd}
     A
    \arrow[r, "\alpha"]
    \arrow[d, tail, "i" {swap}, "\sim" {sloped}]
    & X
    \arrow[d, two heads, "p"]
    \\ B
    \arrow[r, "\beta" swap]
    \arrow[ru, dashed, "h" description]
    & Y
  \end{tikzcd}
  \hspace{2cm}
  \begin{tikzcd}
     A
    \arrow[r, "\alpha"]
    \arrow[d, tail, "i" swap]
    & X
    \arrow[d, two heads, "p", "\sim" {swap,sloped}]
    \\ B
    \arrow[r, "\beta" swap]
    \arrow[ru, dashed, "h" description]
    & Y
  \end{tikzcd}
\end{displaymath}

Mas o que ocorre quando removemos a condição de trivialidade sobre o morfismo $i$ ou sobre o morfismo $p$?
Nesse caso, não há por que esperar que exista necesariamente um morfismo diagonal $h$ que complete o diagrama de forma comutativa.
Vejamos um exemplo mais concreto para nos convencermos disso.

\begin{exem}[Homotopias à esquerda e à direita]
  \label{exem:homotopia_como_levantamento}
  Em uma categoria modelo $\mathsf{M}$, sejam $B$ um objeto qualquer, $(\Cyl(B),i,\varepsilon)$ um objeto cilindro para $B$, e $X$ um objeto fibrante.
  Dado um par de morfismos $f,\,g: B \to X$, podemos considerar o quadrado comutativo abaixo.
  \begin{displaymath}
    \begin{tikzcd}
      B \sqcup B
      \arrow[r, "{\langle f,g \rangle}"]
      \arrow[d, tail, "i" swap]
      & X
      \arrow[d, two heads]
      \\ \Cyl(B)
      \arrow[r]
      & *
    \end{tikzcd}
  \end{displaymath}
  Veja que esse diagrama é do tipo considerado acima, já que $i: B \sqcup B \cofib \Cyl(B)$ é uma cofibração pela definição de objeto cilindro, e o morfismo único $X \fib *$ é uma fibração por conta da hipótese que fizemos sobre $X$.
  Note que um morfismo diagonal $h: \Cyl(B) \to X$ faz comtuar o diagrama acima se, e somente se, a igualdade $h \circ i = \langle f,g \rangle$ é satisfeita; ou seja, um levantamento para o diagrama acima é o mesmo que uma homotopia à esquerda entre os morfismos $f$ e $g$, o que não necessariamente vai existir sempre.

  Vale notar que a exigência de que $X$ seja fibrante não significa nada na estrutura modelo de Serre na categoria de espaços topológicos $\mathsf{Top}$, já que nessa estrutura modelo todo objeto é fibrante.
  Em outras palavras, quando lidamos com espaços topológicos, todo problema de construção de uma homotopia entre dois mapas pode ser formulado como um problema de levantamento do tipo que estamos considerando.

  É claro que também temos uma formulação análogo para o problema de construção de uma homotopia à direita entre os morfismos $f$ e $g$.
  Nesse caso, supomos que $B$ seja um objeto cofibrante, tomamos $(P(X),c,p)$ um objeto de caminhos para $X$, e consideramos o quadrado comutativo abaixo o qual é do tipo que estamos considerando.
  \begin{displaymath}
    \begin{tikzcd}
      \varnothing
      \arrow[d, tail]
      \arrow[r]
      & P(X)
      \arrow[d, two heads, "p"]
      \\ B
      \arrow[r, "{(f,g)}" swap]
      & X \times X
    \end{tikzcd}
  \end{displaymath}
  Um morfismo diagonal $h: B \to P(X)$ faz comutar o diagrama acima se, e somente se, satisfaz a igualdade $p \circ h = (f,g)$; ou seja, um levantamento para o diagrama acima é precisamente uma homotopia à direita entre $f$ e $g$, a qual não necessariamente precisa existir.
\end{exem}

O próximo exemplo exige um lema simples a respeito da relação entre produtos e fibrações.

\begin{lema}\label{lema:produto_com_fibrante_induz_fibracao}
  Em uma categoria modelo $\mathsf{M}$, suponha que $X$ seja um objeto fibrante.
  Então, dado qualquer outro objeto $B$, a projeção no primeiro fator $\pi_{1}: B \times X \to B$ é uma fibração.
\end{lema}

\begin{proof}
  Bastar notar que o diagrama abaixo é um pullback e usar o fato que fibrações são preservadas por pullbacks.
  \begin{displaymath}
    \begin{tikzcd}
      B \times X
      \arrow[d, "\pi_{1}" swap]
      \arrow[r, "\pi_{2}"]
      & X
      \arrow[d, two heads]
      \\ B
      \arrow[r]
      & *
    \end{tikzcd} \qedhere
  \end{displaymath}
\end{proof}

\begin{exem}[Extensões ao longo de cofibrações]
  \label{exem:extensao_ao_longo_de_cofibracao_como_levantamento}
  Suponha que $i: A \cofib B$ seja uma cofibração em uma categoria modelo $\mathsf{M}$, o que geralmente interpretamos como $A$ sendo um \emph{bom} subobjeto de $B$.
  Considere um morfismo $f: A \to X$, onde supomos que $X$ seja um objeto fibrante.
  Segue do \cref{lema:produto_com_fibrante_induz_fibracao} acima que a projeção $\pi_{1}: B \times X \to B$ é uma fibração, e podemos então considerar o problema de levantamento dado pelo quadrado comutativo abaixo.
  \begin{displaymath}
    \begin{tikzcd}
      A
      \arrow[d, tail, "i" swap]
      \arrow[r, "{(i,f)}"]
      & B \times X
      \arrow[d, two heads, "\pi_{1}"]
      \\ B
      \arrow[r, "\id_{B}" swap]
      & B
    \end{tikzcd}
  \end{displaymath}
  Suponha então que $h: B \to B \times X$ seja um levantamento para o diagrama acima.
  Note então que o morfismo $F \coloneqq \pi_{2} \circ h: B \to X$ satisfaz a igualdade
  \begin{displaymath}
    F \circ i = \pi_{2} \circ h \circ i = \pi_{2} \circ (i,f) = f;
  \end{displaymath}
  ou seja, $F$ é \emph{extensão do morfismo $f$ ao longo da cofibração $i$}.
  Reciprocamente, se $F: B \to X$ satisfaz $F \circ i = f$, então o morfismo $h \coloneqq (\id_{B},F): B \to B \times X$ define um levantamento para o diagrama acima.
  De fato, a igualdade $\pi_{1} \circ h = \id_{B}$ é imediata da definição, e a igualdade $h \circ i = (i,f)$ segue da sequência de igualdade
  \begin{displaymath}
    \pi_{1} \circ h \circ i = \id_{B} \circ i = i
  \end{displaymath}
  e também da sequência de igualdades
  \begin{displaymath}
    \pi_{2} \circ h \circ i = \pi_{2} \circ (\id_{B},F) \circ i = F \circ i = f.
  \end{displaymath}

  Em suma, o problema de levantamento acima admite uma solução se, e somente se, o morfismo $f: A \to X$ pode ser estendido a um morfismo $F: B \to X$ ao longo da cofibração $i: A \cofib B$.
  Pensando novamente em cofibrações como boas inclusões, temos um morfismo definido \emph{parcialmente} no subobjeto $A$ e queremos estendê-lo a um morfismo definido \emph{globalmente} no objeto $B$.

  Novamente, no caso clássico da categoria $\mathsf{Top}$, a exigência de que $X$ seja fibrante não impõe na verdade nenhuma restrição adicional sobre $X$, as projeções nos fatores de um produto são sempre fibrações de Serre, portanto qualquer problema de extensão de um mapa contínuo ao longo de uma cofibração na estrutura modelo de Serre pode ser formulado como um problema de levantamento no sentido em que estamos tratando aqui.
\end{exem}

\begin{exem}[Seções de uma fibração]
  Considere uma cofibração $i: A \cofib B$, e uma fibração $p: E \fib B$.
  Vamos supor que essa fibração admita uma \emph{seção parcial sobre $A$}, ou seja, que exista um morfismo $s: A \to E$ satisfazendo a igualdade $p \circ s = i$.
  Podemos então considerar o problema de levantamento dado pelo quadrado comutativo abaixo.
  \begin{displaymath}
    \begin{tikzcd}
      A
      \arrow[d, tail, "i" swap]
      \arrow[r, "s"]
      & E
      \arrow[d, two heads, "p"]
      \\ B
      \arrow[r, "\id_{B}" swap]
      & B
    \end{tikzcd}
  \end{displaymath}

  Suponha que $S: B \to E$ seja um levantamento para o diagrama acima,o que significa que $S$ deve satisfazer as seguintes igualdades:
  \begin{enumerate}
  \item[(i)] $p \circ S = \id_{B}$;
    
  \item[(ii)] $S \circ i = s$.
  \end{enumerate}
  A primeira igualdade diz que $S$ é uma \emph{seção} da fibração $p$, enquanto a segunda igualdade diz que $S$ \emph{estende} a seção parcial $s$ considerada inicialmente ao longo da cofibração $i$.

  No caso topológico, se considerarmos $A$ como um \emph{bom} subespaço de $B$, então no problema acima começamos com uma seção parcial da fibração $p$ definida apenas sobre o subespaço $A$, e queremos então estendê-la a uma outra seção que esteja definida globalmente no espaço $B$.
\end{exem}

Os exemplos acima ilustram algumas situações nas quais nos deparamos com problemas de levantamento não triviais.
O objetivo da Teoria de Obstrução no contexto de categorias modelo é exatamente estudar esses problemas de levantamento não-triviais e obter critérios que nos permitam inferir quando eles admitem ou não alguma solução.

\section{Categorias pontuadas}

Antes de definirmos propriamente a noção de uma teoria de obstrução para morfismos em uma categoria modelo, precisamos antes discutir a noção de uma categoria modelo pontuada, já que é nesse contexto que desenvolveremos nosso estudo.

\begin{defin}\label{defin:categoria_pontuada}
  Uma categoria $\mathsf{C}$ é dita \textbf{pontuada} se ela admite um objeto que seja tanto inicial quanto final.
\end{defin}

\begin{obs}
  Por vezes, especialmente em contextos algébricos, um objeto que seja simultanemente inicial e final é chamado de \emph{objeto zero}.
  Seguindo essa terminologia, uma categoria pontuada é então uma categoria que possui um objeto zero.
\end{obs}

\begin{exem}\label{exem:categorias_pontuadas_algebricas}
  A categoria de grupos $\mathsf{Grp}$ é pontuada, sendo seu objeto zero dado pelo grupo trivial $\{e\}$.

  Dado um anel $R$ qualquer, a categoria $\mathsf{Ch}(R)$ de complexos de cadeias de $R$-módulos é uma categoria pontuada, sendo seu objeto zero dado pelo complexo trivial $0_{\bullet}$ cujo módulo de $n$-cadeias $0_{n}$ é por definição o $R$-módulo trivial $0$ e cujo morfismo de bordo $\partial_{n}: 0_{n} \to 0_{n-1}$ é por definição o morfismo trivial.
\end{exem}

Podemos obter uma categoria pontuada a partir de uma categoria arbitrária utilizando a noção de \emph{objetos pontuados}.

\begin{exem}[Categorias de objetos pontuados]
  \label{exem:categoria_objetos_pontuados}
  Dada uma categoria $\mathsf{C}$ contendo um objeto final $*$, considere a categoria co-slice $\mathsf{C}_{*} \coloneqq * \backslash \mathsf{C}$, ou seja, os objetos de $\mathsf{C}_{*}$ são pares $(A,a)$, onde $A$ é um objeto de $\mathsf{C}$, e $a: * \to A$ é um morfismo; e um morfismo do tipo $(A,a) \to (B,b)$ é um morfismo $f: A \to B$ na categoria original $\mathsf{C}$ satisfazendo a igualdade $f \circ a = b$.
  \begin{displaymath}
    \begin{tikzcd}
      A
      \arrow[rr, dashed, "f"]
      & & B
      \\ & *
      \arrow[lu, "a"]
      \arrow[ru, "b" swap]
    \end{tikzcd}
  \end{displaymath}
  É comum interpretarmos o morfismo $a: * \to A$ como uma escolha de ponto no objeto $A$, e nos referirmos então ao morfismo $a$ como um \emph{ponto base} para $A$ e ao par $(A,a)$ como um \emph{objeto pontuado} em $\mathsf{C}$.
  Seguindo essa interpretação, a condição $f \circ a = b$ imposta sobre um morfismo $f: (A,a) \to (B,b)$ pode ser entendida como uma condição de \emph{preservação de pontos base}, razão pela qual dizemos que $f$ nesse caso é um \emph{morfismo pontuado}

  Veja que o objeto terminal vem sempre equipado com o ponto base dado pelo morfismo idêntico $\id_{*}: * \to *$.
  Afirmamos que o objeto pontuado $(*,\id_{*})$ é tanto inicial quanto final na categoria $\mathsf{C}_{*}$.
  Dado um objeto pontuado $(A,a) \in \mathsf{C}_{*}$, o fato de $*$ ser final garante a existência de um único morfismo $!_{A}: A \to *$ na categoria original $\mathsf{C}$.
  Veja que esse morfismo é automaticamente pontuado, já que a igualdade $!_{A} \circ a = \id_{*}$ segue diretamente do fato de ambos os morfismos $!_{A} \circ a$ e $\id_{*}$ terem o objeto final $*$ como codomínio.
  Em outras palavras, $!_{A}$ é o único morfismo do tipo $(A,a) \to (*,\id_{*})$ em $\mathsf{C}_{*}$, o que mostra que $(*,\id_{*})$ é um objeto final nessa categoria.
  Note agora que o morfismo $a: * \to A$ que determina o ponto base pode ser visto como um morfismo pontuado $a: (*,\id_{*}) \to (A,a)$.
  Na verdade, pela definição de morfismo pontuado, esse é na verdade o único morfismo deste tipo, já que se $a': (*,\id_{*}) \to (A,a)$ é pontuado, então pode definição $a' \circ \id_{*} = a$, ou seja, $a' = a$.
  Concluímos assim que existe um único morfismo do tipo $(*,\id_{*}) \to (A,a)$, o que mostra que $(*,\id_{*})$ é também um objeto inicial em $\mathsf{C}_{*}$.
  Em suma, a categoria $\mathsf{C}_{*}$ é sempre pontuada.
\end{exem}

\begin{exem}[Conjuntos e espaços pontuados]
  \label{exem:conjuntos_e_espacos_pontuados}
  Aplicando a construção descrita no \cref{exem:categoria_objetos_pontuados} a algumas categorias bem conhecidas recuperamos exemplos familiares de objetos pontuados.
   Essa construção geral pode ser especializada para obtermos várias categorias de objetos pontuadas com as quais estamos habituados.
  Tomando $\mathsf{C} \coloneqq \mathsf{Set}$ obtemos a categoria $\mathsf{Set_{*}}$ de \emph{conjuntos pontuados}.
  Veja que uma função $a: * \to A$ determina um elemento único $a(*) \in A$, e reciprocamente, todo elemento de $A$ determina uma função do tipo $* \to A$, o que nos permite recuperar a noção mais usual de conjunto pontuado como sendo um par $(A,a)$, onde $A$ é um conjunto e $a \in A$ é um elemento deste conjunto.
  Exatamente o mesmo raciocínio mostra que tomando $\mathsf{C} \coloneqq \mathsf{Top}$ recuperamos a categoria $\mathsf{Top_{*}}$ usual de espaços pontuados.
\end{exem}

Veremos agora como a existência de um objeto zero em uma categoria pontuada nos permite fazer uma série de construções especiais.

Dados dois objetos quaisquer $X$ e $Y$ de uma categoria pontuada $\mathsf{C}$, a existência de um objeto simultaneamente inicial e final nos permite definir um morfismo especial $\ct_{X,Y}: X \to Y$ chamado \textbf{morfismo constante de $X$ para $Y$} por meio da composição mostrada abaixo.
\begin{displaymath}
  \begin{tikzcd}
    X
    \arrow[rd, "!_{X}" swap]
    \arrow[rr, dashed]
    & & Y
    \\ & *
    \arrow[ru, "!_{Y}" swap]
  \end{tikzcd}
\end{displaymath}

\begin{exem}
  Vejamos a interpretação dessa noção de morfismo constante em alguns exemplos concretos de categorias pontuadas.

  \begin{enumerate}
  \item[(i)] No caso da categoria de grupos $\mathsf{Grp}$, dados grupos $G$ e $H$, como o morfismo $!_{G}: G \to \{e\}$ manda todos os elementos de $G$ para $e$, enquanto o morfismo $!_{H}: \{e\} \to H$ manda $e$ para a identidade $e_{H}$ do grupo $H$, o morfismo constante $\ct_{G,H}: G \to H$ é constante e igual a $e_{H}$.

  \item[(ii)] No caso da categoria de conjuntos pontuados $\mathsf{Set_{*}}$, dados $(A,a)$ e $(B,b)$, o morfismo constante $\ct_{(A,a),(B,b)}: (A,a) \to (B,b)$ manda todos os elementos de $A$ para o ponto base $b \in B$.
    A interpretação é exatamente a mesma no caso da categoria $\mathsf{Top_{*}}$ de espaços pontuados.
  \end{enumerate}
\end{exem}

Introduduziremos agora algumas construções envolvendo escolhas de pontos base que serão especialmente relevantes no estudo de categorias pontuadas.

\begin{defin}\label{defin:fibra_sobre_ponto_base}
  Sejam $f: X \to Y$ um morfismo em uma categoria $\mathsf{C}$ qualquer e $y: * \to Y$ uma escolha de ponto base no codomínio do mesmo.
  Caso o diagrama
  \begin{displaymath}
    \begin{tikzcd}
      & X
      \arrow[d, "f"]
      \\ *
      \arrow[r, "y" swap]
      & Y
    \end{tikzcd}
  \end{displaymath}
  admita um pullback $(F,i,!_{F})$, nos referiremos a este pullback por \textbf{fibra de $f$ sobre $y$}.
  \begin{displaymath}
    \begin{tikzcd}
      F
      \arrow[r, "i"]
      \arrow[d, "!_{F}" swap]
      & X
      \arrow[d, "f"]
      \\ *
      \arrow[r, "y" swap]
      & Y
    \end{tikzcd}
  \end{displaymath}
\end{defin}

Um fato que conforta o coração é que, conforme esperado, mesmo em uma categoria qualquer, a fibra de um morfismo ``vive dentro'' do domínio deste morfismo em um sentido categórico adequado como mostra o Lema abaixo.

\begin{lema}\label{lema:fibra_define_subobjeto}
  Sejam $f: X \to Y$ um morfismo em uma categoria $\mathsf{C}$ qualquer e $y: * \to Y$ um ponto base.
  Se $(F,i,!_{F})$ é uma fibra de $f$ sobre o ponto $y$, então $i: F \to X$ é um monomorfismo, ou seja, a fibra $F$ define um subobjeto do domínio $X$.
\end{lema}

\begin{proof}
  Afirmamos que o morfismo $y: * \to Y$ que define o ponto base é sempre um monomorfismo.
  Isso pode parecer surpreendente, mas não passa de uma trivialidade absoluta: se $\alpha,\, \beta: Z \to *$ são dois morfismos tais que $y \circ \alpha = y \circ \beta$, necessariamente devemos ter $\alpha = \beta$ simplesmente pelo fato de $*$ ser um objeto final da categoria $\mathsf{C}$.
  Sabendo disso, o resultado em questão segue diretamente do fato de monomorfismos serem preservados por pullbacks.
\end{proof}

\begin{exem}
  Vejamos que essa noção categórica de fibra faz sentido nas categoriais usuais com as quais estamos acostumados.

  \begin{enumerate}
  \item[(i)] Suponha que $f: X \to Y$ seja um morfismo na categoria de conjuntos $\mathsf{Set}$.
    Dado um ponto $y \in Y$, seja $p_{y}: \{*\} \to Y$ a função associada que escolhe esse elemento, ou seja, $p_{y}(*) \coloneqq y$.
    Vamos mostrar que a fibra usual $f^{-1}(y)$ juntamente com o mapa de inclusão $i: f^{-1}(y) \hookrightarrow X$ e a função terminal $f^{-1}(y) \to \{*\}$ definem uma fibra também no sentido categórico, ou seja, vamos mostrar que o diagrama abaixo é um pullback em $\mathsf{Set}$.
    \begin{displaymath}
      \begin{tikzcd}
        f^{-1}(y)
        \arrow[r, "i"]
        \arrow[d]
        & X
        \arrow[d, "f"]
        \\ \{*\}
        \arrow[r, "p_{y}" swap]
        & Y
      \end{tikzcd}
    \end{displaymath}

    Suponha então que $W$ seja outro conjunto e que $\alpha: W \to X$ seja uma função que juntamente com a função terminal $!_{W}: W \to \{*\}$ faz comutar a camada externa do diagrama abaixo.
    \begin{displaymath}
      \begin{tikzcd}
        W
        \arrow[rrd, bend left=20, "\alpha"]
        \arrow[rdd, bend right=20, "!_{W}" swap]
        \\ & f^{-1}(y)
        \arrow[r, "i"]
        \arrow[d]
        & X
        \arrow[d, "f"]
        \\ & \{*\}
        \arrow[r, "p_{y}" swap]
        & Y
      \end{tikzcd}
    \end{displaymath}
    Veja então que $\alpha$ necessariamente toma valores na fibra $f^{-1}(y)$, já que pela condição de comutatividade acima temos as igualdades
    \begin{displaymath}
      f(\alpha(w)) = p_{y}(!_{W}(w)) = p_{y}(*) = y
    \end{displaymath}
    para qualquer elemento $w \in W$.
    Podemos então fatorar unicamente $\alpha$ pela fibra e obtermos uma função $\overline{\alpha}: W \to f^{-1}(y)$ que faz comutar todo o diagrama abaixo.
    \begin{displaymath}
      \begin{tikzcd}
        W
        \arrow[rrd, bend left=20, "\alpha"]
        \arrow[rdd, bend right=20, "!_{W}" swap]
        \arrow[rd, dashed, "\overline{\alpha}" description]
        \\ & f^{-1}(y)
        \arrow[r, "i"]
        \arrow[d]
        & X
        \arrow[d, "f"]
        \\ & \{*\}
        \arrow[r, "p_{y}" swap]
        & Y
      \end{tikzcd}
    \end{displaymath}
    Exatamente o mesmo raciocínio mostra que a noção categórica de fibra coincide com a noção usual na categoria $\mathsf{Top}$.
    
  \item[(ii)] Seja $R$ um anel qualquer e consider a categoria $R-\mathsf{Mod}$ de $R$-módulos.
    Como o objeto terminal dessa categoria é dado pelo $R$-módulo trivial $\mathbf{0}$, e morfismos de $R$-módulos sempre levam zero em zero, o único ponto base que um $R$-módulo $M$ qualquer possui é seu elemento zero $0_{M}$.
    Assim, só podemos falar categoricamente de fibras sobre o zero, e como era de se esperar, dado um morfismo de $R$-módulos $f: M \to N$, um modelo concreto para tal fibra é dada pelo núcleo $\ker f$, juntamente com o morfismo de inclusão $i: \ker f \hookrightarrow M$ e o morfismo terminal $!: \ker f \to \mathbf{0}$; ou em outras palavras, o quadrado comutativo abaixo é um pullback na categoria $R-\mathsf{Mod}$.
    \begin{displaymath}
      \begin{tikzcd}
        \ker f
        \arrow[r, "i"]
        \arrow[d, "!" swap]
        & M
        \arrow[d, "f"]
        \\ \mathbf{0}
        \arrow[r, "0" swap]
        & N
      \end{tikzcd}
    \end{displaymath}
  \end{enumerate}
\end{exem}

No caso de uma categoria pontuada, como cada objeto admite um único ponto base, já que o objeto terminal é também inicial, só faz sentido falarmos da fibra de um morfismo $f: X \to Y$ sobre o ponto base único $!_{Y}: * \to Y$.
Nos referiremos a essa fibra sobre o único ponto base simplesmente como \textbf{a fibra} do morfismo $f$ e a denotaremos por $\Fib(f)$.
\begin{displaymath}
  \begin{tikzcd}
    \Fib(f)
    \arrow[r, "i_{f}"]
    \arrow[d, "!_{\Fib(f)}" swap]
    & X
    \arrow[d, "f"]
    \\ *
    \arrow[r, "!_{Y}" swap]
    & Y
  \end{tikzcd}
\end{displaymath}

\begin{exem}[Fibra do morfismo constante]
  \label{exem:fibra_morfismo_constate}
  Sejam $X$ e $Y$ objetos quaisquer de uma categoria pontuada $\mathsf{C}$, e considere o morfismo constante $\ct_{X,Y}: X \to Y$.
  Seguindo a experiência que temos com morfismos constantes em categorias concretas, parece razoável esperarmos que a fibra deste morfismo seja o próprio objeto $X$, e vamos mostrar que isso é de fato verdade, ou seja, que o diagram abaixo define um pullback em $\mathsf{C}$.
  \begin{displaymath}
    \begin{tikzcd}
      X
      \arrow[r, "\id_{X}"]
      \arrow[d, "!_{X}" swap]
      & X
      \arrow[d, "{\ct_{X,Y}}"]
      \\ *
      \arrow[r, "!_{Y}" swap]
      & Y
    \end{tikzcd}
  \end{displaymath}

  Note primeiro que o quadrado acima é comutativo, pois pela definição do morfismo constante temos
  \begin{displaymath}
    \ct_{X,Y} \circ \id_{X} = \ct_{X,Y} = !_{Y} \circ !_{X}.
  \end{displaymath}
  Se $W$ é outro objeto e $\alpha: W \to X$ é um morfismo tal que $\alpha \circ \ct_{X,Y} = !_{Y} \circ \alpha$, afirmamos que o próprio morfismo $\alpha: X \to Y$ pode ser usado para fazer o diagrama abaixo comutar.
  \begin{displaymath}
    \begin{tikzcd}
      W
      \arrow[rrd, bend left=20, "\alpha"]
      \arrow[rdd, bend right=20, "!_{W}" swap]
      \arrow[rd, dashed, "\alpha" description]
      \\ & X
      \arrow[d, "!_{X}" swap]
      \arrow[r, "\id_{X}"]
      & X
      \arrow[d, "{\ct_{X,Y}}"]
      \\ & *
      \arrow[r, "!_{Y}" swap]
      & Y
    \end{tikzcd}
  \end{displaymath}
  É claro que triângulo superior é comutativo, e a comutatividade do triângulo inferior segue simplesmente do fato de $!_{X} \circ \alpha$ e $!_{W}$ serem ambos morfismos para o objeto terminal $*$.
  É claro também que $\alpha$ é o único morfismo satisfazendo tais condições de comutatividade, já que se $\alpha': W \to X$ é outro morfismo satisfazendo as mesmas condições, então em particular $\id_{X} \circ \alpha' = \alpha$, portanto $\alpha' = \alpha$.
\end{exem}

\section{Categorias modelo pontuadas}

Após introduzirmos algumas das noções básicas associadas a categorias pontuadas, vejamos como essa estrutura se combina com a estrutura de uma categoria modelo.
Uma \textbf{categoria modelo pontuada} é nada mais que uma categoria modelo $(\mathsf{M},\mathcal{W},\mathcal{C},\mathcal{F})$ cuja categoria subjacente $\mathsf{M}$ é pontuada no sentido da seção anterior.

\end{document}

%%% Local Variables:
%%% mode: latex
%%% TeX-master: t
%%% End:
