\documentclass[brazilian]{article}

\usepackage{babel}
\usepackage[utf8]{inputenc}
\usepackage{csquotes}
\usepackage[margin=1.35in]{geometry}
\usepackage[backend=biber,style=alphabetic]{biblatex}
\usepackage{mathtools,amssymb,amsthm}
\usepackage{indentfirst}
\usepackage{hyperref}
\usepackage{tikz-cd}
\usepackage{graphicx}
\usepackage[capitalize,noabbrev]{cleveref}

\usetikzlibrary{babel}

\addbibresource{bibliografia.bib}

\swapnumbers
\newtheorem{teo}{Teorema}[section]
\newtheorem{prop}[teo]{Proposição}
\newtheorem{lema}[teo]{Lema}
\newtheorem{corol}[teo]{Corolário}

\theoremstyle{definition}
\newtheorem{defin}[teo]{Definição}
\newtheorem{obs}[teo]{Observação}
\newtheorem{exem}[teo]{Exemplo}

\DeclarePairedDelimiter{\abs}{\lvert}{\rvert}
\DeclarePairedDelimiter{\Abs}{\lVert}{\rVert}

\newcommand{\id}{\mathrm{id}}
\newcommand{\ct}{\mathrm{ct}}
\newcommand{\Mor}{\mathrm{Mor}}
\newcommand{\Hom}{\mathrm{Hom}}
\newcommand{\Arr}{\mathrm{Arr}}
\newcommand{\cofib}{\rightarrowtail}
\newcommand{\Cofib}{\mathrm{Cofib}}
\newcommand{\fib}{\twoheadrightarrow}
\newcommand{\Fib}{\mathrm{Fib}}
\newcommand{\dom}{\mathrm{dom}}
\newcommand{\cod}{\mathrm{cod}}
\newcommand{\comp}{\mathrm{comp}}
\newcommand{\fac}{\mathrm{fac}}
\newcommand{\init}{\mathrm{init}}
\newcommand{\Fun}{\mathrm{Fun}}
\newcommand{\Cyl}{\mathrm{Cyl}}
\newcommand{\Cone}{\mathrm{Cone}}
\newcommand{\Ho}{\mathrm{Ho}}

\renewcommand{\qedsymbol}{$\blacksquare$}

\title{Teoria de Obstrução Categórica\\Notas para o seminário}
\author{Edmundo Martins}
\date{\today}

%%% Local Variables:
%%% mode: latex
%%% TeX-master: "main"
%%% End:


\begin{document}

\maketitle

% TODO:
% - Entender o objeto final da categoria de conjuntos simpliciais.
% - Descrever conjuntos simpliciais pontuados em termos de pares formados por um conjunto simplicial e uma escolha de vértice.
% - Terminar a discussão sobre limites e colimites em categorias slice e co-slice.

\section{Introdução e motivação}

Considere o quadrado comutativo abaixo em uma categoria modelo $\mathsf{M}$ qualquer, onde $i: A \cofib B$ é uma cofibração, e $p: X \fib Y$ é uma fibração.

\begin{displaymath}
  \begin{tikzcd}
    A
    \arrow[r, "\alpha"]
    \arrow[d, tail, "i" swap]
    & X
    \arrow[d, two heads, "p"]
    \\ B
    \arrow[r, "\beta" swap]
    & Y
  \end{tikzcd}
\end{displaymath}
O \emph{axioma de levantamento} na definição de uma categoria modelo $\mathsf{M}$ garante que, quando $i$ ou $p$ são equivalências fracas, então podemos encontrar um morfismo diagonal $h: B \to X$  (um levantamento) que faz os diagramas resultantes comutarem.
\begin{displaymath}
  \begin{tikzcd}
     A
    \arrow[r, "\alpha"]
    \arrow[d, tail, "i" {swap}, "\sim" {sloped}]
    & X
    \arrow[d, two heads, "p"]
    \\ B
    \arrow[r, "\beta" swap]
    \arrow[ru, dashed, "h" description]
    & Y
  \end{tikzcd}
  \hspace{2cm}
  \begin{tikzcd}
     A
    \arrow[r, "\alpha"]
    \arrow[d, tail, "i" swap]
    & X
    \arrow[d, two heads, "p", "\sim" {swap,sloped}]
    \\ B
    \arrow[r, "\beta" swap]
    \arrow[ru, dashed, "h" description]
    & Y
  \end{tikzcd}
\end{displaymath}

Mas o que ocorre quando removemos a condição de trivialidade sobre o morfismo $i$ ou sobre o morfismo $p$?
Nesse caso, não há por que esperar que exista necesariamente um morfismo diagonal $h$ que complete o diagrama de forma comutativa.
Vejamos um exemplo mais concreto para nos convencermos disso.

\begin{exem}[Homotopias à esquerda e à direita]
  \label{exem:homotopia_como_levantamento}
  Em uma categoria modelo $\mathsf{M}$, sejam $B$ um objeto qualquer, $(\Cyl(B),i,\varepsilon)$ um objeto cilindro para $B$, e $X$ um objeto fibrante.
  Dado um par de morfismos $f,\,g: B \to X$, podemos considerar o quadrado comutativo abaixo.
  \begin{displaymath}
    \begin{tikzcd}
      B \sqcup B
      \arrow[r, "{\langle f,g \rangle}"]
      \arrow[d, tail, "i" swap]
      & X
      \arrow[d, two heads]
      \\ \Cyl(B)
      \arrow[r]
      & *
    \end{tikzcd}
  \end{displaymath}
  Veja que esse diagrama é do tipo considerado acima, já que $i: B \sqcup B \cofib \Cyl(B)$ é uma cofibração pela definição de objeto cilindro, e o morfismo único $X \fib *$ é uma fibração por conta da hipótese que fizemos sobre $X$.
  Note que um morfismo diagonal $h: \Cyl(B) \to X$ faz comtuar o diagrama acima se, e somente se, a igualdade $h \circ i = \langle f,g \rangle$ é satisfeita; ou seja, um levantamento para o diagrama acima é o mesmo que uma homotopia à esquerda entre os morfismos $f$ e $g$, o que não necessariamente vai existir sempre.

  Vale notar que a exigência de que $X$ seja fibrante não significa nada na estrutura modelo de Serre na categoria de espaços topológicos $\mathsf{Top}$, já que nessa estrutura modelo todo objeto é fibrante.
  Em outras palavras, quando lidamos com espaços topológicos, todo problema de construção de uma homotopia entre dois mapas pode ser formulado como um problema de levantamento do tipo que estamos considerando.

  É claro que também temos uma formulação análogo para o problema de construção de uma homotopia à direita entre os morfismos $f$ e $g$.
  Nesse caso, supomos que $B$ seja um objeto cofibrante, tomamos $(P(X),c,p)$ um objeto de caminhos para $X$, e consideramos o quadrado comutativo abaixo o qual é do tipo que estamos considerando.
  \begin{displaymath}
    \begin{tikzcd}
      \varnothing
      \arrow[d, tail]
      \arrow[r]
      & P(X)
      \arrow[d, two heads, "p"]
      \\ B
      \arrow[r, "{(f,g)}" swap]
      & X \times X
    \end{tikzcd}
  \end{displaymath}
  Um morfismo diagonal $h: B \to P(X)$ faz comutar o diagrama acima se, e somente se, satisfaz a igualdade $p \circ h = (f,g)$; ou seja, um levantamento para o diagrama acima é precisamente uma homotopia à direita entre $f$ e $g$, a qual não necessariamente precisa existir.
\end{exem}

\begin{exem}[Extensões ao longo de cofibrações]
  \label{exem:extensao_ao_longo_de_cofibracao_como_levantamento}
  Suponha que $i: A \cofib B$ seja uma cofibração em uma categoria modelo $\mathsf{M}$, o que geralmente interpretamos como $A$ sendo um \emph{bom} subobjeto de $B$.
  Considere um morfismo $f: A \to X$, onde supomos que $X$ seja um objeto fibrante.
  Podemos então considerar o problema de levantamento dado pelo quadrado comutativo abaixo.
  \begin{displaymath}
    \begin{tikzcd}
      A
      \arrow[d, tail, "i" swap]
      \arrow[r, "f"]
      & X
      \arrow[d, two heads, "!_{X}"]
      \\ B
      \arrow[r, "\id_{B}" swap]
      & *
    \end{tikzcd}
  \end{displaymath}
  Veja que uma solução para o problema de levantamento acima é precisamente um morfismo $F: B \to X$ satisfazendo a igualdade $F \circ i = f$, ou seja, $F$ é \emph{extensão do morfismo $f$ ao longo da cofibração $i$}.
  Pensando novamente em cofibrações como boas inclusões, temos um morfismo definido \emph{parcialmente} no ``subobjeto'' $A$ e queremos estendê-lo a um morfismo definido \emph{globalmente} no objeto $B$.

  Novamente, no caso clássico da categoria $\mathsf{Top}$, a exigência de que $X$ seja fibrante não impõe na verdade nenhuma restrição adicional sobre $X$, portanto qualquer problema de extensão de um mapa contínuo ao longo de uma cofibração na estrutura modelo de Serre pode ser formulado como um problema de levantamento no sentido em que estamos tratando aqui.
\end{exem}

\begin{exem}[Seções de uma fibração]
  Considere uma cofibração $i: A \cofib B$, e uma fibração $p: E \fib B$.
  Vamos supor que essa fibração admita uma \emph{seção parcial sobre $A$}, ou seja, que exista um morfismo $s: A \to E$ satisfazendo a igualdade $p \circ s = i$.
  Podemos então considerar o problema de levantamento dado pelo quadrado comutativo abaixo.
  \begin{displaymath}
    \begin{tikzcd}
      A
      \arrow[d, tail, "i" swap]
      \arrow[r, "s"]
      & E
      \arrow[d, two heads, "p"]
      \\ B
      \arrow[r, "\id_{B}" swap]
      & B
    \end{tikzcd}
  \end{displaymath}

  Suponha que $S: B \to E$ seja um levantamento para o diagrama acima,o que significa que $S$ deve satisfazer as seguintes igualdades:
  \begin{enumerate}
  \item[(i)] $p \circ S = \id_{B}$;
    
  \item[(ii)] $S \circ i = s$.
  \end{enumerate}
  A primeira igualdade diz que $S$ é uma \emph{seção} da fibração $p$, enquanto a segunda igualdade diz que $S$ \emph{estende} a seção parcial $s$ considerada inicialmente ao longo da cofibração $i$.

  No caso topológico, se considerarmos $A$ como um \emph{bom} subespaço de $B$, então no problema acima começamos com uma seção parcial da fibração $p$ definida apenas sobre o subespaço $A$, e queremos então estendê-la a uma outra seção que esteja definida globalmente no espaço $B$.
\end{exem}

Os exemplos acima ilustram algumas situações nas quais nos deparamos com problemas de levantamento não triviais.
O objetivo da Teoria de Obstrução no contexto de categorias modelo é exatamente estudar esses problemas de levantamento não-triviais e obter critérios que nos permitam inferir quando eles admitem ou não alguma solução.

\section{Categorias pontuadas}

Antes de definirmos propriamente a noção de uma teoria de obstrução para morfismos em uma categoria modelo, precisamos antes discutir a noção de uma categoria modelo pontuada, já que é nesse contexto que desenvolveremos nosso estudo.

\begin{defin}\label{defin:categoria_pontuada}
  Uma categoria $\mathsf{C}$ é dita \textbf{pontuada} se ela admite um objeto que seja tanto inicial quanto final.
\end{defin}

\begin{obs}
  Por vezes, especialmente em contextos algébricos, um objeto que seja simultanemente inicial e final é chamado de \emph{objeto zero}.
  Seguindo essa terminologia, uma categoria pontuada é então uma categoria que possui um objeto zero.
\end{obs}

\begin{exem}\label{exem:categorias_pontuadas_algebricas}
  A categoria de grupos $\mathsf{Grp}$ é pontuada, sendo seu objeto zero dado pelo grupo trivial $\{e\}$.

  Dado um anel $R$ qualquer, a categoria $\mathsf{Ch}(R)$ de complexos de cadeias de $R$-módulos é uma categoria pontuada, sendo seu objeto zero dado pelo complexo trivial $0_{\bullet}$ cujo módulo de $n$-cadeias $0_{n}$ é por definição o $R$-módulo trivial $0$ e cujo morfismo de bordo $\partial_{n}: 0_{n} \to 0_{n-1}$ é por definição o morfismo trivial.
\end{exem}

Podemos obter uma categoria pontuada a partir de uma categoria arbitrária utilizando a noção de \emph{objetos pontuados}.

\begin{exem}[Categorias de objetos pontuados]
  \label{exem:categoria_objetos_pontuados}
  Dada uma categoria $\mathsf{C}$ contendo um objeto final $*$, considere a categoria co-slice $\mathsf{C}_{*} \coloneqq * / \mathsf{C}$, ou seja, os objetos de $\mathsf{C}_{*}$ são pares $(A,a)$, onde $A$ é um objeto de $\mathsf{C}$, e $a: * \to A$ é um morfismo; e um morfismo do tipo $(A,a) \to (B,b)$ é um morfismo $f: A \to B$ na categoria original $\mathsf{C}$ satisfazendo a igualdade $f \circ a = b$.
  \begin{displaymath}
    \begin{tikzcd}
      & *
      \arrow[ld, "a" swap]
      \arrow[rd, "b"]
      \\ A
      \arrow[rr, dashed, "f" swap]
      & & B
    \end{tikzcd}
  \end{displaymath}
  É comum interpretarmos o morfismo $a: * \to A$ como uma escolha de ponto no objeto $A$, e nos referirmos então ao morfismo $a$ como um \emph{ponto base} para $A$ e ao par $(A,a)$ como um \emph{objeto pontuado} em $\mathsf{C}$.
  Seguindo essa interpretação, a condição $f \circ a = b$ imposta sobre um morfismo $f: (A,a) \to (B,b)$ pode ser entendida como uma condição de \emph{preservação de pontos base}, razão pela qual dizemos que $f$ nesse caso é um \emph{morfismo pontuado}

  Veja que o objeto terminal vem sempre equipado com o ponto base dado pelo morfismo idêntico $\id_{*}: * \to *$.
  Afirmamos que o objeto pontuado $(*,\id_{*})$ é tanto inicial quanto final na categoria $\mathsf{C}_{*}$.
  Dado um objeto pontuado $(A,a) \in \mathsf{C}_{*}$, o fato de $*$ ser final garante a existência de um único morfismo $!_{A}: A \to *$ na categoria original $\mathsf{C}$.
  Veja que esse morfismo é automaticamente pontuado, já que a igualdade $!_{A} \circ a = \id_{*}$ segue diretamente do fato de ambos os morfismos $!_{A} \circ a$ e $\id_{*}$ terem o objeto final $*$ como codomínio.
  Em outras palavras, $!_{A}$ é o único morfismo do tipo $(A,a) \to (*,\id_{*})$ em $\mathsf{C}_{*}$, o que mostra que $(*,\id_{*})$ é um objeto final nessa categoria.
  Note agora que o morfismo $a: * \to A$ que determina o ponto base pode ser visto como um morfismo pontuado $a: (*,\id_{*}) \to (A,a)$.
  Na verdade, pela definição de morfismo pontuado, esse é na verdade o único morfismo deste tipo, já que se $a': (*,\id_{*}) \to (A,a)$ é pontuado, então pode definição $a' \circ \id_{*} = a$, ou seja, $a' = a$.
  Concluímos assim que existe um único morfismo do tipo $(*,\id_{*}) \to (A,a)$, o que mostra que $(*,\id_{*})$ é também um objeto inicial em $\mathsf{C}_{*}$.
  Em suma, a categoria $\mathsf{C}_{*}$ é sempre pontuada.
\end{exem}

\begin{exem}[Conjuntos e espaços pontuados]
  \label{exem:conjuntos_e_espacos_pontuados}
  Aplicando a construção descrita no \cref{exem:categoria_objetos_pontuados} a algumas categorias bem conhecidas recuperamos exemplos familiares de objetos pontuados.
   Essa construção geral pode ser especializada para obtermos várias categorias de objetos pontuadas com as quais estamos habituados.
  Tomando $\mathsf{C} \coloneqq \mathsf{Set}$ obtemos a categoria $\mathsf{Set_{*}}$ de \emph{conjuntos pontuados}.
  Veja que uma função $a: * \to A$ determina um elemento único $a(*) \in A$, e reciprocamente, todo elemento de $A$ determina uma função do tipo $* \to A$, o que nos permite recuperar a noção mais usual de conjunto pontuado como sendo um par $(A,a)$, onde $A$ é um conjunto e $a \in A$ é um elemento deste conjunto.
  Exatamente o mesmo raciocínio mostra que tomando $\mathsf{C} \coloneqq \mathsf{Top}$ recuperamos a categoria $\mathsf{Top_{*}}$ usual de espaços pontuados.
\end{exem}

Assumindo que $\mathsf{C}$ possua um pouco mais de estrutura, podemos relacioná-la com a categoria de objetos pontuadas $\mathsf{C}_{*}$.
Mais assumindo que $\mathsf{C}$ admita coprodutos binários\footnote{A princíprio só precisamos da existência de coprodutos com o objeto terminal, mas enfim...}, podemos \emph{adjuntar um ponto base} a qualquer objeto $X \in \mathsf{C}$ formando o coproduto $X \sqcup *$, o qual e comumente denotado por $X^{+}$.
Denotando os morfismos canônicos para esse coproduto por $j^{X}_{1}: X \to X^{+}$ e $j^{X}_{2}: * \to X^{+}$, o par $(X^{+}, j^{X}_{2})$ define define um objeto na categoria $\mathsf{C}_{*}$.
Essa construção pode ser estendida aos morfismos de $\mathsf{C}$.
De fato, dado $f: X \to Y$, podemos formar o coproduto $f^{+} \coloneqq f \sqcup \id_{*}$ o qual define um morfismo do tipo $X^{+} \to Y^{+}$.
Lembremos que pela definição do coproduto de dois morfismos $f^{+}$ é o único morfismo de seu tipo que faz comutar o diagrama mostrado abaixo.
\begin{equation}\label{eq:diagrama_adjuncao_ponto_base_em_morfismo}
  \begin{tikzcd}[column sep=1.3cm]
    X
    \arrow[r, "f"]
    \arrow[d, "j^{X}_{1}" swap]
    & Y
    \arrow[d, "j^{Y}_{1}"]
    \\ X^{+}
    \arrow[r, dashed, "f^{+}" description]
    & Y^{+}
    \\ *
    \arrow[u, "j^{X}_{2}"]
    \arrow[r, "\id_{*}" swap]
    & *
    \arrow[u, "j^{Y}_{2}" swap]
  \end{tikzcd}
\end{equation}
Note que a comutatividade da parte inferior do diagrama mostra que $f^{+}$ define um morfismo pontuado do tipo $(X^{+},j^{X}_{2}) \to (Y^{+},j^{Y}_{2})$.

Lembrando que o coproduto de objetos e morfismos depende funtorialmente de ambas as variáveis, temos as igualdades
\begin{displaymath}
  \id_{X}^{+} = \id_{X} \sqcup \id_{*} = \id_{X \sqcup *} = \id_{X^{+}}
\end{displaymath}
e também as igualdades
\begin{displaymath}
  (g \circ f)^{+} = (g \circ f) \sqcup \id_{*} = (g \sqcup \id_{*}) \circ (f \sqcup \id_{*}) = g^{+} \circ f^{+}.
\end{displaymath}
Em suma, associando a cada objeto $X \in \mathsf{C}$ o objeto pontuado $(X^{+},j^{X}_{2})$, e associando a cada morfismo $f: X \to Y$ o morfismo pontuado correspondente $f^{+}: (X^{+},j^{X}_{2}) \to (Y^{+},j^{Y}_{2})$ obtemos o \emph{funtor de adjunção de ponto base} $(-)^{+}: \mathsf{C} \to \mathsf{C}_{*}$.
No caso de categorias pontuadas, o resultado abaixo mostra que essa construção identifica $\mathsf{C}$ com $\mathsf{C}_{*}$.

\begin{prop}\label{prop:cat_pontuada_implica_adjuntar_ponto_base_eh_equiv}
  Se $\mathsf{C}$ é uma categoria pontuada, então o funtor de adjunção de ponto base $(-)^{+}: \mathsf{C} \to \mathsf{C}_{*}$ é uma equivalência de categorias, sendo seu quasi-inverso dado pelo funtor de esquecimento $\mathcal{E}: \mathsf{C}_{*} \to \mathsf{C}$.
\end{prop}

\begin{proof}
  Vamos mostrar primeiro que a composição $\mathcal{E} \circ (-)^{+}: \mathsf{C} \to \mathsf{C}$ é naturalmente isomorfa ao funtor idêntico.
  Considere a transformação natural $\theta: \id_{\mathsf{C}} \Rightarrow \mathcal{E} \circ (-)^{+}$ cuja componente no objeto $X \in \mathsf{C}$ é dada pela injeção canônica $j^{X}_{1}: X \to X^{+}$.
  Dado um morfismo $f: X \to Y$, a comutatividade do quadrado superior no diagrama \eqref{eq:diagrama_adjuncao_ponto_base_em_morfismo} usado na definição de $f^{+}$ mostra que a família $(j^{X}_{1})_{X \in \mathsf{C}}$ depende naturalmente de $X$, ou seja, temos de fato uma transformação natural.
  Note agora que, se $!_{X}: * \to X$ denota o morfismo único decorrente do fato de $*$ ser também um objeto inicial, então a propriedade universal do coproduto fornece um morfismo $\langle \id_{X}, !_{X} \rangle: X^{+} \to X$ tal que $\langle \id_{X}, !_{X} \rangle \circ j^{X}_{1} = \id_{X}$; mostrando então que $j^{X}_{1}$ é um isomorfismo e que, portanto, temos um isomorfismo natural de funtores $\id_{\mathsf{C}} \cong \mathcal{E} \circ (-)^{+}$.

  Resta construirmos um isomorfismo natural $(-)^{+} \circ \mathcal{E} \cong \id_{\mathsf{C}_{*}}$.
  Dado um objeto pontuado $(X,x) \in \mathsf{C}_{*}$ qualquer, afirmamos que a injeção canônica $j^{X}_{1}: X \to X^{+}$ define na verdade um morfismo pontuado do tipo $(X,x) \to (X^{+},j^{X}_{2})$.
  De fato, isso segue simplesmente do fato de que os morfismos $j^{X}_{1} \circ x$ e $j^{X}_{2}$ ambos têm o objeto inicial $*$ como domínio.
  \begin{displaymath}
    \begin{tikzcd}
      X
      \arrow[rr, "j^{X}_{1}"]
      & & X^{+}
      \\ & *
      \arrow[lu, "x"]
      \arrow[ru, "j^{X}_{2}" swap]
    \end{tikzcd}
  \end{displaymath}
  Dado um morfismo pontuado $f: (X,x) \to (Y,y)$, a comutatividade do diagrama usado na definição de $f^{+}: (X^{+},j^{X}_{2}) \to (Y^{+},j^{Y}_{2})$ mais uma vez mostra que temos o quadrado comutativo abaixo,
  \begin{displaymath}
    \begin{tikzcd}
      (X,x)
      \arrow[r, "f"]
      \arrow[d, "j^{X}_{1}" swap]
      & (Y,y)
      \arrow[d, "j^{Y}_{1}"]
      \\ (X^{+},j^{X}_{2})
      \arrow[r, "f^{+}" swap]
      & (Y^{+}, j^{Y}_{2})
    \end{tikzcd}
  \end{displaymath}
  portanto a família de morfismos $(j^{X}_{1})_{(X,x) \in \mathsf{C}}$ define uma transformação natural do funtor identidade $\id_{\mathsf{C}_{*}}$ para a composição $(-)^{+} \circ \mathcal{E}$.

  Dado agora um objeto pontuado $(X,x) \in \mathsf{C}^{*}$, o morfismo $x$ que escolhe ponto base dá origem por meio da propriedade universal do coproduto a um único morfismo $\langle \id_{X},x \rangle: X^{+} \to X$ fazendo comutar o diagrama abaixo.
  \begin{displaymath}
    \begin{tikzcd}[column sep=1.5cm]
      X
      \arrow[rrd, bend left=15, "\id_{X}"]
      \arrow[rd, "j^{X}_{1}" swap]
      \\ & X^{+}
      \arrow[r, dashed, "{\langle \id_{X},x \rangle}" description]
      & X
      \\ *
      \arrow[ru, "j^{X}_{2}"]
      \arrow[rru, bend right=15, "x" swap]
    \end{tikzcd}
  \end{displaymath}
  Ora, a comutatividade da parte inferior do diagrama diz que $\langle \id_{X},x \rangle$ define um morfismo pontuado do tipo $(X^{+},j^{X}_{2}) \to (X,x)$, enquanto a comutatividade da parte superior diz precisamente que esse morfismo é o inverso do morfismo $j^{X}_{1}: (X,x) \to (X^{+},j^{X}_{2})$ considerado anteriormente.
  Concluímos assim que a família $(j^{X}_{1})_{(X,x) \in \mathsf{C}_{*}}$ define na verdade um isomorfismo natural $\id_{\mathsf{C}_{*}} \cong (-)^{+} \circ \mathcal{E}$.
\end{proof}

Veremos agora como a existência de um objeto zero em uma categoria pontuada nos permite fazer uma série de construções especiais.

Dados dois objetos quaisquer $X$ e $Y$ de uma categoria pontuada $\mathsf{C}$, a existência de um objeto simultaneamente inicial e final nos permite definir um morfismo especial $\ct_{X,Y}: X \to Y$ chamado \textbf{morfismo constante de $X$ para $Y$} por meio da composição mostrada abaixo.
\begin{displaymath}
  \begin{tikzcd}
    X
    \arrow[rd, "!_{X}" swap]
    \arrow[rr, dashed]
    & & Y
    \\ & *
    \arrow[ru, "!_{Y}" swap]
  \end{tikzcd}
\end{displaymath}

\begin{exem}
  Vejamos a interpretação dessa noção de morfismo constante em alguns exemplos concretos de categorias pontuadas.

  \begin{enumerate}
  \item[(i)] No caso da categoria de grupos $\mathsf{Grp}$, dados grupos $G$ e $H$, como o morfismo $!_{G}: G \to \{e\}$ manda todos os elementos de $G$ para $e$, enquanto o morfismo $!_{H}: \{e\} \to H$ manda $e$ para a identidade $e_{H}$ do grupo $H$, o morfismo constante $\ct_{G,H}: G \to H$ é constante e igual a $e_{H}$.

  \item[(ii)] No caso da categoria de conjuntos pontuados $\mathsf{Set_{*}}$, dados $(A,a)$ e $(B,b)$, o morfismo constante $\ct_{(A,a),(B,b)}: (A,a) \to (B,b)$ manda todos os elementos de $A$ para o ponto base $b \in B$.
    A interpretação é exatamente a mesma no caso da categoria $\mathsf{Top_{*}}$ de espaços pontuados.
  \end{enumerate}
\end{exem}

Introduduziremos agora algumas construções envolvendo escolhas de pontos base que serão especialmente relevantes no estudo de categorias pontuadas.

\begin{defin}\label{defin:fibra_sobre_ponto_base}
  Sejam $f: X \to Y$ um morfismo em uma categoria $\mathsf{C}$ qualquer e $y: * \to Y$ uma escolha de ponto base no codomínio do mesmo.
  Caso o diagrama
  \begin{displaymath}
    \begin{tikzcd}
      & X
      \arrow[d, "f"]
      \\ *
      \arrow[r, "y" swap]
      & Y
    \end{tikzcd}
  \end{displaymath}
  admita um pullback $(F,i,!_{F})$, nos referiremos a este pullback por \textbf{fibra de $f$ sobre $y$}.
  \begin{displaymath}
    \begin{tikzcd}
      F
      \arrow[r, "i"]
      \arrow[d, "!_{F}" swap]
      & X
      \arrow[d, "f"]
      \\ *
      \arrow[r, "y" swap]
      & Y
    \end{tikzcd}
  \end{displaymath}
\end{defin}

Um fato que conforta o coração é que, conforme esperado, mesmo em uma categoria qualquer, a fibra de um morfismo ``vive dentro'' do domínio deste morfismo em um sentido categórico adequado como mostra o Lema abaixo.

\begin{lema}\label{lema:fibra_define_subobjeto}
  Sejam $f: X \to Y$ um morfismo em uma categoria $\mathsf{C}$ qualquer e $y: * \to Y$ um ponto base.
  Se $(F,i,!_{F})$ é uma fibra de $f$ sobre o ponto $y$, então $i: F \to X$ é um monomorfismo, ou seja, a fibra $F$ define um subobjeto do domínio $X$.
\end{lema}

\begin{proof}
  Afirmamos que o morfismo $y: * \to Y$ que define o ponto base é sempre um monomorfismo.
  Isso pode parecer surpreendente, mas não passa de uma trivialidade absoluta: se $\alpha,\, \beta: Z \to *$ são dois morfismos tais que $y \circ \alpha = y \circ \beta$, necessariamente devemos ter $\alpha = \beta$ simplesmente pelo fato de $*$ ser um objeto final da categoria $\mathsf{C}$.
  Sabendo disso, o resultado em questão segue diretamente do fato de monomorfismos serem preservados por pullbacks (veja \cref{lema:pullback_de_mono_eh_mono}).
\end{proof}

\begin{exem}
  Vejamos que essa noção categórica de fibra faz sentido nas categoriais usuais com as quais estamos acostumados.

  \begin{enumerate}
  \item[(i)] Suponha que $f: X \to Y$ seja um morfismo na categoria de conjuntos $\mathsf{Set}$.
    Dado um ponto $y \in Y$, seja $p_{y}: \{*\} \to Y$ a função associada que escolhe esse elemento, ou seja, $p_{y}(*) \coloneqq y$.
    Vamos mostrar que a fibra usual $f^{-1}(y)$ juntamente com o mapa de inclusão $i: f^{-1}(y) \hookrightarrow X$ e a função terminal $f^{-1}(y) \to \{*\}$ definem uma fibra também no sentido categórico, ou seja, vamos mostrar que o diagrama abaixo é um pullback em $\mathsf{Set}$.
    \begin{displaymath}
      \begin{tikzcd}
        f^{-1}(y)
        \arrow[r, "i"]
        \arrow[d]
        & X
        \arrow[d, "f"]
        \\ \{*\}
        \arrow[r, "p_{y}" swap]
        & Y
      \end{tikzcd}
    \end{displaymath}

    Suponha então que $W$ seja outro conjunto e que $\alpha: W \to X$ seja uma função que juntamente com a função terminal $!_{W}: W \to \{*\}$ faz comutar a camada externa do diagrama abaixo.
    \begin{displaymath}
      \begin{tikzcd}
        W
        \arrow[rrd, bend left=20, "\alpha"]
        \arrow[rdd, bend right=20, "!_{W}" swap]
        \\ & f^{-1}(y)
        \arrow[r, "i"]
        \arrow[d]
        & X
        \arrow[d, "f"]
        \\ & \{*\}
        \arrow[r, "p_{y}" swap]
        & Y
      \end{tikzcd}
    \end{displaymath}
    Veja então que $\alpha$ necessariamente toma valores na fibra $f^{-1}(y)$, já que pela condição de comutatividade acima temos as igualdades
    \begin{displaymath}
      f(\alpha(w)) = p_{y}(!_{W}(w)) = p_{y}(*) = y
    \end{displaymath}
    para qualquer elemento $w \in W$.
    Podemos então fatorar unicamente $\alpha$ pela fibra e obtermos uma função $\overline{\alpha}: W \to f^{-1}(y)$ que faz comutar todo o diagrama abaixo.
    \begin{displaymath}
      \begin{tikzcd}
        W
        \arrow[rrd, bend left=20, "\alpha"]
        \arrow[rdd, bend right=20, "!_{W}" swap]
        \arrow[rd, dashed, "\overline{\alpha}" description]
        \\ & f^{-1}(y)
        \arrow[r, "i"]
        \arrow[d]
        & X
        \arrow[d, "f"]
        \\ & \{*\}
        \arrow[r, "p_{y}" swap]
        & Y
      \end{tikzcd}
    \end{displaymath}
    Exatamente o mesmo raciocínio mostra que a noção categórica de fibra coincide com a noção usual na categoria $\mathsf{Top}$.
    
  \item[(ii)] Seja $R$ um anel qualquer e consider a categoria $R-\mathsf{Mod}$ de $R$-módulos.
    Como o objeto terminal dessa categoria é dado pelo $R$-módulo trivial $\mathbf{0}$, e morfismos de $R$-módulos sempre levam zero em zero, o único ponto base que um $R$-módulo $M$ qualquer possui é seu elemento zero $0_{M}$.
    Assim, só podemos falar categoricamente de fibras sobre o zero, e como era de se esperar, dado um morfismo de $R$-módulos $f: M \to N$, um modelo concreto para tal fibra é dada pelo núcleo $\ker f$, juntamente com o morfismo de inclusão $i: \ker f \hookrightarrow M$ e o morfismo terminal $!: \ker f \to \mathbf{0}$; ou em outras palavras, o quadrado comutativo abaixo é um pullback na categoria $R-\mathsf{Mod}$.
    \begin{displaymath}
      \begin{tikzcd}
        \ker f
        \arrow[r, "i"]
        \arrow[d, "!" swap]
        & M
        \arrow[d, "f"]
        \\ \mathbf{0}
        \arrow[r, "0" swap]
        & N
      \end{tikzcd}
    \end{displaymath}
  \end{enumerate}
\end{exem}

No caso de uma categoria pontuada, como cada objeto admite um único ponto base, já que o objeto terminal é também inicial, só faz sentido falarmos da fibra de um morfismo $f: X \to Y$ sobre o ponto base único $!_{Y}: * \to Y$.
Nos referiremos a essa fibra sobre o único ponto base simplesmente como \textbf{a fibra} do morfismo $f$ e a denotaremos por $\Fib(f)$.
\begin{displaymath}
  \begin{tikzcd}
    \Fib(f)
    \arrow[r, "i_{f}"]
    \arrow[d, "!_{\Fib(f)}" swap]
    & X
    \arrow[d, "f"]
    \\ *
    \arrow[r, "!_{Y}" swap]
    & Y
  \end{tikzcd}
\end{displaymath}

\begin{exem}[Fibra do morfismo constante]
  \label{exem:fibra_morfismo_constate}
  Sejam $X$ e $Y$ objetos quaisquer de uma categoria pontuada $\mathsf{C}$, e considere o morfismo constante $\ct_{X,Y}: X \to Y$.
  Seguindo a experiência que temos com morfismos constantes em categorias concretas, parece razoável esperarmos que a fibra deste morfismo seja o próprio objeto $X$, e vamos mostrar que isso é de fato verdade, ou seja, que o diagram abaixo define um pullback em $\mathsf{C}$.
  \begin{displaymath}
    \begin{tikzcd}
      X
      \arrow[r, "\id_{X}"]
      \arrow[d, "!_{X}" swap]
      & X
      \arrow[d, "{\ct_{X,Y}}"]
      \\ *
      \arrow[r, "!_{Y}" swap]
      & Y
    \end{tikzcd}
  \end{displaymath}

  Note primeiro que o quadrado acima é comutativo, pois pela definição do morfismo constante temos
  \begin{displaymath}
    \ct_{X,Y} \circ \id_{X} = \ct_{X,Y} = !_{Y} \circ !_{X}.
  \end{displaymath}
  Se $W$ é outro objeto e $\alpha: W \to X$ é um morfismo tal que $\alpha \circ \ct_{X,Y} = !_{Y} \circ \alpha$, afirmamos que o próprio morfismo $\alpha: X \to Y$ pode ser usado para fazer o diagrama abaixo comutar.
  \begin{displaymath}
    \begin{tikzcd}
      W
      \arrow[rrd, bend left=20, "\alpha"]
      \arrow[rdd, bend right=20, "!_{W}" swap]
      \arrow[rd, dashed, "\alpha" description]
      \\ & X
      \arrow[d, "!_{X}" swap]
      \arrow[r, "\id_{X}"]
      & X
      \arrow[d, "{\ct_{X,Y}}"]
      \\ & *
      \arrow[r, "!_{Y}" swap]
      & Y
    \end{tikzcd}
  \end{displaymath}
  É claro que triângulo superior é comutativo, e a comutatividade do triângulo inferior segue simplesmente do fato de $!_{X} \circ \alpha$ e $!_{W}$ serem ambos morfismos para o objeto terminal $*$.
  É claro também que $\alpha$ é o único morfismo satisfazendo tais condições de comutatividade, já que se $\alpha': W \to X$ é outro morfismo satisfazendo as mesmas condições, então em particular $\id_{X} \circ \alpha' = \alpha$, portanto $\alpha' = \alpha$.
\end{exem}

\begin{exem}
  Suponha que $f: X \to Y$ seja um monomorfismo em uma categoria pontuada $\mathsf{C}$, sendo $*$ seu objeto zero.
  Afirmamos então que a fibra de $f$ é dada por $*$.
  De fato, no primeiro que o quadrado abaixo é trivialmente comutativo, já que $*$ é em particular um objeto inicial.
  \begin{displaymath}
    \begin{tikzcd}
      *
      \arrow[r, "!_{X}"]
      \arrow[d, "\id_{*}" swap]
      & X
      \arrow[d, "f"]
      \\ *
      \arrow[r, "!_{Y}" swap]
      & Y
    \end{tikzcd}
  \end{displaymath}
  Suponha agora que $W$ seja outro objeto de $\mathsf{C}$ e que $\alpha: W \to X$ seja tal que $f \circ \alpha = !_{Y} \circ !_{W}$, ou seja, a parte externa do diagrama abaixo é comutativa.
  Afirmamos então que o morfismo único $!_{W}: W \to *$ faz comutar o diagrama todo.
  \begin{displaymath}
    \begin{tikzcd}
      W
      \arrow[rrd, bend left=20, "\alpha"]
      \arrow[rdd, bend right=20, "!_{W}" swap]
      \arrow[rd, dashed, "!_{W}" description]
      \\ & *
      \arrow[r, "!_{X}"]
      \arrow[d, "\id_{*}" swap]
      & X
      \arrow[d, "f"]
      \\ & *
      \arrow[r, "!_{Y}" swap]
      & Y
    \end{tikzcd}
  \end{displaymath}
  A comutatividade do triângulo inferior é imediata.
  Já com relação à comutatividade do triângulo superior, note que
  \begin{displaymath}
    f \circ (!_{X} \circ !_{W}) = (f \circ !_{X}) \circ !_{Y} = !_{Y} \circ !_{W} = f \circ \alpha,
  \end{displaymath}
  e sendo $f$ um monomorfismo, podemos cancelá-lo na igualdade acima para obtermos a igualdade desejada $!_{X} \circ !_{W} = \alpha$.

  A recíproca desse fato me parece ser falsa, ou seja, existem morfismos em categorias pontuadas cuja fibra é trivial mas que não são monomorfismos.
  Isso é verdade, por exemplo, em categorias abelianas, mas nesse contexto podemos formar a diferença entre dois morfismos de forma que possamos utilizar a propriedade universal do pullback, mas não vejo como fazer algo análogo em um contexto geral.
\end{exem}

\section{Categorias modelo pontuadas}

Após introduzirmos algumas das noções básicas associadas a categorias pontuadas, vejamos como essa estrutura se combina com a estrutura de uma categoria modelo.
Uma \textbf{categoria modelo pontuada} é nada mais que uma categoria modelo $(\mathsf{M},\mathcal{W},\mathcal{C},\mathcal{F})$ cuja categoria subjacente $\mathsf{M}$ é pontuada no sentido da seção anterior.

A principal fonte de exemplo de categorias de modelos pontuadas para nós serão categorias de objetos pontuados em uma categoria de modelos inicial.

\begin{exem}\label{exem:estrutura_modelo_objetos_pontuados}
  Seja $(\mathsf{M},\mathcal{W},\mathcal{C},\mathcal{F})$ uma categoria modelo.
  Vejamos como definir uma estrutura modelo na categoria de objetos pontuados $\mathsf{M_{*}}$ introduzida no \cref{exem:categoria_objetos_pontuados}.
  Diremos que um morfismo pontuado $f: (A,a) \to (B,b)$ é uma equivalência fraca (respectivamente uma cofibração, uma fibração) se o morfismo subjacente $f: A \to B$ na categoria $\mathsf{M}$ é uma equivalência fraca (respectivamente uma cofibração, uma fibração).
  Vamos denotar as classes de morfismos resultantes em $\mathsf{M}_{*}$ por $\mathcal{W}_{*}$, $\mathcal{C}_{*}$ e $\mathcal{F}_{*}$, respectivamente.

  Vejamos que essas três classes de morfismos em $\mathsf{M}_{*}$ satisfazem os axiomas que definem uma estrutura modelo.
  \begin{enumerate}
  \item[(M1)] Não vamos pensar na bicompletude de $\mathsf{M}_{*}$ por enquanto...
    
  \item[(M2)] Como a composição de morfismos em $\mathsf{M}_{*}$ é dada pela composição em $\mathsf{M}$, é imediato que as equivalências fracas pontuadas satisfazem a propriedade 2-de-3.
    
  \item[(M3)] Suponha que o morfismo pontuado $f: (A,a) \to (B,b)$ seja um retrato do da equivalência fraca pontuada $g: (X,x) \to (Y,y)$.
    Vale então que $f: A \to B$ é uma retração da equivalência fraca $g: X \to Y$ na categoria $\mathsf{M}$ subjacente, logo $f$ é também uma equivalência fraca e, portanto, uma equivalência fraca pontuada.
    Um argumento análogo mostra que as classes de cofibrações e fibrações pontuadas são também fechadas por retrações.
    
  \item[(M4)] Considere o quadrado comutativo abaixo em $\mathsf{M}_{*}$, onde $i: (A,a) \to (B,b)$ é uma cofibração trivial pontuada e $p: (X,x) \to (Y,y)$ é uma fibração pontuada.
    \begin{displaymath}
      \begin{tikzcd}
        (A,a)
        \arrow[r, "\alpha"]
        \arrow[d, tail, "i" {swap}, "\sim" {sloped}]
        & (X,x)
        \arrow[d, two heads, "p", "\sim" {swap,sloped}]
        \\ (B,b)
        \arrow[r, "\beta" swap]
        & (Y,y)
      \end{tikzcd}
    \end{displaymath}
    Esquecendo os pontos base obtemos o quadrado comutativo abaixo na categoria $\mathsf{M}$ onde $i$ e $p$ são agora uma cofibração trivial e uma fibração trivial, respectivamente.
    Usando o axioma de levantamento em $\mathsf{M}$ obtemos um morfismo $h: B \to X$ fazendo comutar o diagrama todo.
    \begin{displaymath}
      \begin{tikzcd}
        A
        \arrow[r, "\alpha"]
        \arrow[d, tail, "i" {swap}, "\sim" {sloped}]
        & X
        \arrow[d, two heads, "p", "\sim" {swap,sloped}]
        \\ B
        \arrow[r, "\beta" swap]
        \arrow[ru, dashed, "h" description]
        & Y
      \end{tikzcd}
    \end{displaymath}
    Veja que $h$ é um morfismo pontuado pois
    \begin{displaymath}
      h \circ b = h \circ (i \circ a) = (h \circ i) \circ a = \alpha \circ a = x.
    \end{displaymath}
    Consequentemente, podemos ver $h$ como um morfismo do tipo $(B,b) \to (X,x)$ na categoria $\mathsf{M}_{*}$, e então obtemos o levantamento necessário para o quadrado considerado inicialmente.
    \begin{displaymath}
      \begin{tikzcd}
        (A,a)
        \arrow[r, "\alpha"]
        \arrow[d, tail, "i" {swap}, "\sim" {sloped}]
        & (X,x)
        \arrow[d, two heads, "p", "\sim" {swap,sloped}]
        \\ (B,b)
        \arrow[r, "\beta" swap]
        \arrow[ru, dashed, "h" description]
        & (Y,y)
      \end{tikzcd}
    \end{displaymath}
    Um raciocínio completamente análogo mostra que a existência de um levantamento pontuado no caso onde $i$ é apenas uma cofibração pontuada, mas $p$ é uma fibração trivial pontuada.
    
  \item[(M5)] Dado um morfismo pontuado $f: (A,a) \to (B,b)$, podemos fatorar o morfismo subjacente $f: A \to B$ como uma cofibração trivial $i: A \to C$ seguida de uma fibração $p: C \to B$.
    Considere no objeto $C$ o ponto base $c: * \to C$ dado pela composição $c \coloneqq i \circ a$.
    É óbvio que com essa escolha de ponto base em $C$ o morfismo $i$ se torna pontuado, e o mesmo é verdade para o morfismo $p$ já que
    \begin{displaymath}
      p \circ c = p \circ i \circ a = f \circ a = b.
    \end{displaymath}
    Temos então a fatoração de $f$ como uma cofibração trivial pontuada $i: (A,a) \to (C,c)$ seguida de uma fibração pontuada $p: (C,c) \to (B,b)$, conforme indicado abaixo.
    \begin{displaymath}
      \begin{tikzcd}
        (A,a)
        \arrow[rr, "f"]
        \arrow[rd, tail, "i" {swap}, "\sim" {sloped}]
        & & (B,b)
        \\ & (C,c)
        \arrow[ru, two heads, "p" swap]
      \end{tikzcd}
    \end{displaymath}

    Um argumento análogo mostra que $f$ também pode ser fatorado como uma cofibração pontuada $j: (A,a) \to (D,d)$ seguida de uma fibração trivial pontuada $q: (D,d) \to (B,b)$.
  \end{enumerate}
\end{exem}

\begin{obs}
  O leitor percebeu que não mencionamos nada sobre a bicompletude da categoria de objetos pontuados $\mathsf{M}_{*}$ associada a uma categoria $\mathsf{M}$.
  Isso é porque essa é uma questão puramente categórica que não tem relação com a estrutura modelo de $\mathsf{M}$.

  É um fato geral que, se $\mathsf{C}$ é uma categoria completa, então a categoria de objetos pontuados $\mathsf{C}_{*}$ é completa também.
  Isso é relativamente tranquilo de demonstrar.
  Dado um diagrama pequeno $F: \mathsf{J} \to \mathsf{C}_{*}$, se $\mathcal{E}: \mathsf{C}_{*} \to \mathsf{C}$ é o morfismo de esquecimento evidente, temos o diagrama pequeno $\mathcal{E} \circ F: \mathsf{J} \to \mathsf{C}$ na categoria inicial, o qual admite um limite $(L,(p_{j}: L \to F(j))_{j \in \mathsf{J}})$ pela hipótese de completude.
  Veja que $L$ possui uma escolha natural de ponto base, pois como cada $F(j)$ possui um ponto base $x_{j}: * \to F(j)$, e a coleção $(p_{j}: * \to F(j))_{j \in \mathsf{J}}$ define um cone sobre $\mathcal{E} \circ F$ com vértice no objeto terminal $*$, a propriedade universal do limite garante a existência de um único morfismo $x: * \to L$ tal que $p_{j} \circ x = x_{j}$ para todo $j \in J$.
  Essa coleção de igualdades diz que os morfismos estruturais $p_{j}$ do limite podem ser vistos como morfismos pontuados $p_{j}: (L,x) \to (F(j),x_{j})$, e podemos então mostrar que o cone $((L,x),p_{j})$ é o limite procurado para o funtor $F$.

  Também é um fato geral que, quando a categoria $\mathsf{C}_{*}$ é cocompleta, o mesmo é válido para a categoria de objetos pontuados $\mathsf{C}_{*}$, mas isso é um pouco mais difícil de demonstrar.
  O início da demonstranção é o mesmo, começamos com um diagrama pequeno $F: \mathsf{J} \to \mathsf{C}_{*}$ na categoria de objetos pontuados e por meio do funtor de esquecimento obtemos um diagrama pequeno $\mathcal{E} \circ F: \mathsf{J} \to \mathsf{C}$ na categoria original.
  Esse diagrama admite um colimite $Q$, mas o problema é que, diferentemente do limite no caso anterior que vinha com uma escolha natural de ponto base, nesse caso temos várias escolhas diferentes de ponto base.
  De fato, se $(\lambda_{j}: F(j) \to Q)_{j \in \mathsf{J}}$ são os morfismos estruturais do colimite, então cada ponto base $x_{j}: * \to F(j)$ dá origem a um ponto base $\lambda_{j} \circ x_{j}: * \to Q$ no colimite.
  A ideia é que o colimite correto é obtido identificando todos esses candidatos a ponto base para que não tenhamos que fazer uma escolha que não seja canônica, e essa identificação é feita por meio de um \emph{outro} colimite.
  Consideramos a categoria $\mathsf{J^{+}}$ obtida de $\mathsf{J}$ pela adjunção de um objeto disjunto $0$ e de um morfismo $!_{j}: 0 \to j$ para cada objeto que já existia previamente.
  O diagrama $F$ pode ser \emph{estendido} a um diagrama $F^{+}: \mathsf{J^{+}} \to \mathsf{C}$ definindo $F^{+}(0) \coloneqq *$ e $F^{+}(!_{j}) \coloneqq x_{j}: * \to F(j)$.
  Podemos então mostrar que o colimite do funtor estendido $F^{+}$ pode ser usado para obtermos o colimite do funtor $F: \mathsf{J} \to \mathsf{C}_{*}$ considerado inicialmente.
\end{obs}

A existência de morfismos constantes em uma categoria pontuada nos permite definir a noção de morfismos homotopicamente nulos em categorias de modelos pontuadas.

\begin{defin}
  Um morfismo $f: X \to Y$ em uma categoria de modelos pontuada $\mathsf{M}$ é dito
  \begin{enumerate}
  \item[(a)] \textbf{homotopicamente nulo à esquerda} se $f$ é homotópico à esquerda ao morfismo constante $\ct_{X,Y}: X \to Y$;
    
  \item[(b)] \textbf{homotopicamente nulo à direita} se $f$ é homotópico à direita ao morfismo constante $\ct_{X,Y}: X \to Y$;
    
  \item[(c)] \textbf{homotopicamente nulo} se $f$ é homotópico ao morfismo constante $\ct_{X,Y}$.
  \end{enumerate}
\end{defin}

Uma noção que é bem comportada em uma categoria modelo pontuada é a de objeto fracamento contrátil.

\begin{defin}\label{defin:objeto_fracamente_contratil}
  Seja $\mathsf{M}$ uma categoria modelo pontuada tendo $*$ como objeto zero.
  Um objeto $X \in \mathsf{M}$ é dito \textbf{fracamente contrátil} se o morfismo inicial $* \to X$ é uma equivalência fraca.
\end{defin}

O lema simples abaixo mostra que a definição acima também pode ser formulada em termos de morfismos terminais.

\begin{lema}\label{lema:fracamente_contratil_sse_morfismo_terminal_eh_equiv_fraca}
  Seja $\mathsf{M}$ uma categoria modelo pontuada tendo $*$ como objeto zero.
  Dado um objeto $X \in \mathsf{M}$, as seguintes condições são equivalentes:
  \begin{enumerate}
  \item[(i)] $X$ é fracamento contrátil;
    
  \item[(ii)] o morfismo terminal $X \to *$ é uma equivalência fraca.
  \end{enumerate}
\end{lema}

\begin{proof}
  Veja que temos o diagrama comutativo abaixo envolvendo os morfismos inicial e terminal associados ao objeto $X$.
  \begin{displaymath}
    \begin{tikzcd}
      *
      \arrow[rr, "\id_{*}", "\sim" {swap}]
      \arrow[rd]
      & & *
      \\ & X
      \arrow[ru]
    \end{tikzcd}
  \end{displaymath}
  Sendo o morfismo idêntico $\id_{*}: * \to *$ uma equivalência fraca, segue diretamente da propriedade 2-de-3 que o morfismo inicial $* \to X$ é uma equivalência fraca se, e somente se, o morfismo terminal $X \to *$ também o é.
\end{proof}

Assim como no caso clássico, as noções de objetos contráteis e morfismos homotopicamente nulos estão relacionadas, embora aqui tenhamos as sutilezas usuais envolvendo homotopias à esquerda e à direita, assim como condições de cofibrância e fbrância sobre os objetos.

\section{Teorias de Obstrução}

Temos enfim todos os ingredientes à nossa disposição para definirmos o que é uma teoria de obstrução em uma categoria modelo.

\begin{defin}\label{defin:teoria_de_obstrucao}
  Dizemos que uma cofibração $i: A \cofib B$ em uma categoria modelo pontuada $\mathsf{M}$ \textbf{admite uma teoria de obstrução} se existe um objeto $W$ tal que para todo diagrama comutativo da forma
  \begin{displaymath}
    \begin{tikzcd}
      A
      \arrow[d, tail, "i" swap]
      \arrow[r]
      & X
      \arrow[d, two heads, "p"]
      \\ B
      \arrow[r]
      & Y
    \end{tikzcd}
  \end{displaymath}
  onde $p: X \fib Y$ é uma fibração, exista uma classe de homotopia $\theta_{p}\in \Hom_{\Ho(\mathsf{M})}(W,\Fib(p))$ com as seguintes propriedades:
  \begin{enumerate}
  \item O problema de levantamento admite solução se e somente se $\theta_{p}$ é trivial, ou seja, se e somente se $\theta_{p} = \eta(\ct_{W,\Fib(p)})$, onde $\eta: \mathsf{M} \to \Ho(\mathsf{M})$ é o funtor de localização.
    
  \item A classe $\theta_{p}$ depende funtorialmente da fibração $p$.
  \end{enumerate}
\end{defin}

Vamos analisar algumas sutilezas por trás da definição acima.
Lembremos que, dados dois objetos \emph{quaisquer} $X,\,Y \in \mathsf{M}$ de uma categoria de modelos, a relação de homotopia, seja à esquerda ou à direita, em geral \emph{não} define uma relação de equivalência no conjunto de morfismos $\Hom_{\mathsf{M}}(X,Y)$, logo não existe um conjunto bem-definido de classes de homotopia de morfismos.
Podemos obter conjuntos bem-definidos de classes de homotopia impondo condições adicionais de (co)fibrância sobre os objetos: se $X$ é cofibrante, então a relação de homotopia à esquerda é uma relação de equivalência, logo temos um conjunto quociente $[X,Y]_{\ell}$ formado por classes de homotopia à esquerda; e se $Y$ é fibrante, então a relação de homotopia à direita é uma relação de equivalência, dessa forma temos um conjunto quociente $[X,Y]_{r}$ formado por classes de homotopia à direita.
Quando $X$ é cofibrante e $Y$ é fibrante, então as duas relações de homotopia coincidem e temos um conjunto quociente único $[X,Y]$ cujos elementos chamamos simplesmente de classes de homotopia.
Se ademais $X$ e $Y$ forem ambos \emph{bifibrantes}, então as operações de pós-composição e pré-composição de morfismos respeitam a relação de homotopia, induzindo operações análogas a nível de classes de homotopia.

Sabendo desse bom comportamento da relação de homotopia para morfismos entre objetos bifibrantes, podemos construir a categoria homotópica $\Ho(\mathsf{M})$ por meio de substituições cofibrantes/fibrantes fortes funtoriais.
Mais precisamente, assumindo que categoria de modelos $\mathsf{M}$ admite fatorações funtoriais, podemos associar a cada objeto $X \in \mathsf{M}$ uma substituição cofibrante forte $p_{X}: X_{c} \overset{\sim}{\fib} X$ e a cada morfismo $\alpha: X \to Y$ um morfismo $\alpha_{c}: X_{c} \to Y_{c}$ entre as substituições cofibrantes que faz comutar o quadrado abaixo.
\begin{displaymath}
  \begin{tikzcd}
    X_{c}
    \arrow[r, dashed, "\alpha_{c}"]
    \arrow[d, two heads, "\sim" {sloped}, "p_{X}" swap]
    & Y_{c}
    \arrow[d, two heads, "\sim" {swap,sloped}, "p_{Y}"]
    \\ X
    \arrow[r, "\alpha" swap]
    & Y
  \end{tikzcd}
\end{displaymath}
Analogamente, associamos a cada objeto $X \in \mathsf{M}$ uma substituição fibrante forte $j_{X}: X \overset{\sim}{\cofib} X_{f}$ e a cada morfismo $\alpha: X \to Y$ um morfismo $\alpha_{f}: X_{f} \to Y_{f}$ entre as subsittuições fibrantes fazendo comutar o diagrama abaixo.
\begin{displaymath}
  \begin{tikzcd}
    X_{f}
    \arrow[r, dashed, "\alpha_{f}"]
    & Y_{f}
    \\ X
    \arrow[u, tail, "j_{X}", "\sim" {swap,sloped}]
    \arrow[r, "\alpha" swap]
    & Y
    \arrow[u, tail, "\sim" {sloped}, "j_{Y}" {swap}]
  \end{tikzcd}
\end{displaymath}

Fixados tais funtores de substituição, definimos a categoria homotópica $\Ho(\mathsf{M})$ da seguinte forma: os objetos de $\Ho(\mathsf{M})$ são precisamente os objetos de $\mathsf{M}$, e dados dois objetos $X,\, Y \in \mathsf{M}$, definimos o conjunto de morfismos $\Hom_{\Ho(\mathsf{M})}(X,Y)$ como sendo o conjunto de classes de homotopia de morfismos $[X_{cf},Y_{cf}]$.
O funtor de localização $\eta: \mathsf{M} \to \Ho(\mathsf{M})$ mapeia então um objeto $X$ para si mesmo, e um morfismo $\alpha: X \to Y$ para a classe de homotopia $[\alpha_{cf}]$ do morfismo $\alpha_{cf}: X_{cf} \to Y_{cf}$ obtido por aplicação dos funtores de subsituição cofibrante e fibrante nessa ordem.
O Teorema de Whitehead diz então precisamente que $\eta$ inverte equivalências fracas, uma das condições necessárias para que $\Ho(\mathsf{M})$ seja uma localização de $\mathsf{M}$ nesta classe de morfismos.

Assim, a rigor a classe $\theta_{p} \in \Hom_{\Ho(\mathsf{M})}(W,\Fib(p))$ que aparece na definição acima não é a classe de homotopia de um morfismo do tipo $W \to \Fib(p)$, mas sim a classe de homotopia de um morfismo do tipo $W_{cf} \to \Fib(p)_{cf}$ entre objetos bifibrantes equivalentes aos originais.
Embora isso seja um pouco incômodo, em muitos casos podemos reinterpretar essa classe de homotopia em termos de outra classe involvendo objetos que estejam mais ``próximos'' dos originais do que as substituições bifibrantes consideradas.
A razão por trás disso é o conteúdo do resultado abaixo.

\begin{prop}
  Sejam $X$ e $Y$ objetos de uma categoria de modelos $\mathsf{M}$.
  Se $p: \widetilde{X} \overset{\sim}{\to} X$ é uma substituição cofibrante qualquer, e $j: Y \overset{\sim}{\to} \widehat{Y}$ é uma substituição fibrante qualquer, existe uma bijeção
  \begin{displaymath}
    \Hom_{\Ho(\mathsf{M})}(X,Y) \cong \left[ \widetilde{X}, \widehat{Y} \right].
  \end{displaymath}
  Em particular, se $X$ já é cofibrante, e $Y$ já é fibrante, então existe uma bijeção
  \begin{displaymath}
    \Hom_{\Ho(\mathsf{M})}(X,Y) \cong [X,Y].
  \end{displaymath}
\end{prop}

Assim, embora os funtores de substituição cofibrante/fibrante não façam necessariamente as escolhas mais convenientes, o resultado acima diz que, se estivermos interessados apenas em classes de homotopia de morfismos, então podemos trabalhar com substituições cofibrantes/fibrantes quaisquer, e não apenas aquelas determinadas pelos funtores.

Essa independência a nível de homotopias também é verdade para a ação dos funtores de substituição nos morfismos.
O funtor de substituição fibrante associa a cada morfismo $\alpha: X \to Y$ um morfismo $\alpha_{c}: X_{c} \to Y_{c}$ entre as substituições cofibrantes que faz comutar o quadrado abaixo.
\begin{displaymath}
  \begin{tikzcd}
    X_{c}
    \arrow[r, dashed, "\alpha_{c}"]
    \arrow[d, two heads, "\sim" {sloped}, "p_{X}" {swap}]
    & Y_{c}
    \arrow[d, two heads, "p_{Y}", "\sim" {swap,sloped}]
    \\ X
    \arrow[r, "\alpha" swap]
    & Y
  \end{tikzcd}
\end{displaymath}
Entretanto, se estivermos interessados unicamente na classe de homotopia à esquerda de $\alpha_{c}$, podemos considerar \emph{qualquer} morfismo $\widetilde{\alpha}: X_{c} \to Y_{c}$ fazendo comutar o quadrado acima.
De fato, se esse é o caso, então as classes de homotopia à esquerda $[\alpha_{c}]_{\ell},\, [\widetilde{\alpha}]_{\ell} \in [X_{c},Y_{c}]_{\ell}$ são ambas mapeadas para a classe de homotopia à esquerda $[\alpha \circ p_{X}]_{\ell} \in [X_{c},Y]_{\ell}$ pela função de pushforward
\begin{displaymath}
  [X_{c},p_{Y}]_{\ell}: [X_{c},Y_{c}]_{\ell} \to [X_{c},Y]_{\ell},
\end{displaymath}
mas como $p_{Y}$ é uma fibração trivial, e $X_{c}$ é cofibrante, segue do CITAR RESULTADO que tal função é uma bijeção, portanto devemos ter $[\alpha_{c}]_{\ell} = [\widetilde{\alpha}]_{\ell}$.

Um argumento completamente análogo mostra que a classe de homotopia à esquerda das substituições cofibrantes depende apenas da classe de homotopia à esquerda do morfismo original quando os objetos envolvidos são fibrantes.
Mais precisamente, se $X$ e $Y$ são fibrantes, e $\alpha,\, \beta: X \to Y$ são homotópicos à direita, então as substituições cofibrantes correspondentes $\alpha_{c},\, \beta_{c}: X_{c} \to Y_{c}$ são homotópicas.\footnote{Aqui não precisamos dizer se à direita ou à esquerda, já que os objetos $X_{c}$ e $Y_{c}$ são bifibrantes nesse caso, portanto todas as noções de homotopia coincidem.}
De fato, basta ver que
\begin{displaymath}
  [X_{c},p_{Y}]([\alpha_{c}]) = [p_{X} \circ \alpha_{c}] = [\alpha \circ p_{Y}]_{r} = [\beta \circ p_{Y}]_{r} = [p_{X} \circ \beta_{c}] = [X_{c},p_{Y}]([\beta_{c}]),
\end{displaymath}
mas $[X_{c},p_{Y}]$ é novamente uma bijeção graças à cofibrância de $X_{c}$ e à trivialidade da fibração $p_{Y}$, portanto devemos ter $[\alpha_{c}] = [\beta_{c}]$ nesse caso também.

É claro que existe toda uma discussão dual à discussão acima que diz respeito ao comportamento do funtor de substituição fibrante nas classes de homotopia.
Uma consequência agradável dessa discussão é que temos uma certa ``liberdade de escolha'' para descrevermos as classes de homotopia associadas a morfismos por meio do funtor de localização $\eta: \mathsf{M} \to \Ho(\mathsf{M})$.

\begin{prop}
  Dado um morfismo $\alpha: X \to Y$, se $\alpha': X_{c} \to Y_{c}$ e $\alpha'': X_{cf} \to Y_{cf}$ são morfismos que fazem comutar o diagrama abaixo,
  \begin{displaymath}
    \begin{tikzcd}
      X_{cf}
      \arrow[r, "\alpha''"]
      & Y_{cf}
      \\ X_{c}
      \arrow[r, "\alpha'"]
      \arrow[u, tail, "j_{X_{c}}"]
      \arrow[d, two heads, "\sim" {sloped}, "p_{X}" {swap}]
      & Y_{c}
      \arrow[u, tail, "j_{Y_{c}}" {swap}]
      \arrow[d, two heads, "p_{Y}", "\sim" {swap,sloped}]
      \\ X
      \arrow[r, "\alpha" swap]
      & Y
    \end{tikzcd}
  \end{displaymath}
  então vale a igualdade $\eta(\alpha) = [\alpha'']$.
\end{prop}

\begin{proof}
  Segue da discussão acima que $[\alpha']_{\ell} = [\alpha_{c}]_{\ell}$, portanto pela cofibrância de $X_{c}$ e $Y_{c}$ e pela discussão dual sobre substituições fibrantes também teremos $[\alpha'_{f}] = [\alpha_{cf}]$, mas pela comutatividade do diagrama também vale que $[\alpha'_{f}] = [\alpha'']$.
  Juntando todas as igualdades temos então
  \begin{displaymath}
    \eta(\alpha) = [\alpha_{cf}] = [\alpha'_{f}] = [\alpha'']. \qedhere
  \end{displaymath}
\end{proof}

Isso nos permite então obter classes de homotopia triviais a partir de morfismos triviais.

\begin{corol}\label{corol:classe_de_homotopia_morfismo_constante}
  Dados objetos $X$ e $Y$ quaisquer de uma categoria de modelos pontuada $\mathsf{M}$, vale a igualdade $\eta(\ct_{X,Y}) = [\ct_{X_{cf},Y_{cf}}]$.
\end{corol}

\begin{proof}
  Basta ver que temos o diagrama comutativo abaixo e aplicar o resultado anterior.
  \begin{displaymath}
    \begin{tikzcd}[column sep=1.25cm]
      X_{cf}
      \arrow[r, "{\ct_{X_{cf},Y_{cf}}}"]
      & Y_{cf}
      \\ X_{c}
      \arrow[r, "{\ct_{X_{c},Y_{c}}}"]
      \arrow[u, tail, "j_{X_{c}}"]
      \arrow[d, two heads, "\sim" {sloped}, "p_{X}" {swap}]
      & Y_{c}
      \arrow[u, tail, "j_{Y_{c}}" {swap}]
      \arrow[d, two heads, "p_{Y}", "\sim" {swap,sloped}]
      \\ X
      \arrow[r, "{\ct_{X,Y}}" swap]
      & Y
    \end{tikzcd} \qedhere
  \end{displaymath}
\end{proof}

\begin{obs}[Funtorialidade das classes de obstrução]
  \label{obs:funtorialidade_classe_de_obstrucao}
  Vamos precisar o significado da funtorialidade na definição de teoria de obstrução.
  O processo de formação de fibras pode ser visto como um funtor na categoria de setas.
  Mais precisamente, considere o quadrado comutativo abaixo em uma categoria pontuada qualquer.
  \begin{equation}
    \label{eq:morfismo_fibracoes}
    \begin{tikzcd}
      X
      \arrow[d, "p" swap]
      \arrow[r, "\varphi"]
      & X'
      \arrow[d, "p'"]
      \\ Y
      \arrow[r, "\psi" swap]
      & Y'
    \end{tikzcd}
  \end{equation}
  Podemos então tomar a fibra dos morfismos $p$ e $p'$ e obter os dois diagrama de pullback mostrados abaixo.
  \begin{displaymath}
    \begin{tikzcd}
      \Fib(p)
      \arrow[d]
      \arrow[r, "i_{p}"]
      & X
      \arrow[d, "p"]
      \\ *
      \arrow[r]
      & Y
    \end{tikzcd}
    \qquad
    \begin{tikzcd}
      \Fib(p')
      \arrow[d]
      \arrow[r, "i_{p'}"]
      & X'
      \arrow[d, "p'"]
      \\ *
      \arrow[r]
      & Y'
    \end{tikzcd}
  \end{displaymath}

  Usando a propriedae universal do pullback obtemos um morfismo $\Fib(\varphi,\psi): \Fib(p) \to \Fib(p')$ caracterizado unicamente por satisfazer a igualdade $\alpha \circ i_{p} \circ \Fib(\varphi,\psi) = i_{p'}$ conforme mostrado no diagrama abaixo.
  \begin{displaymath}
    \begin{tikzcd}
      \Fib(p)
      \arrow[rrd, bend left=30, "\alpha \circ i_{p}"]
      \arrow[rdd, bend right=30]
      \arrow[rd, dashed, "{\Fib(\varphi,\psi)}"]
      \\ & \Fib(p')
      \arrow[r, "i_{p'}"]
      \arrow[d]
      & X'
      \arrow[d, "p'"]
      \\ & *
      \arrow[r]
      & Y'
    \end{tikzcd}
  \end{displaymath}

  A funtorialidade da teoria de obstrução deve então ser entendida da seguinte maneira: dadas fibrações $p: X \fib Y$ e $p': X' \to Y'$ e um quadrado comutativo como em \eqref{eq:morfismo_fibracoes}, deve valer a igualdade de morfismos
  \begin{displaymath}
    \theta_{p'} = \eta(\Fib(\varphi,\psi)) \circ \theta_{p}
  \end{displaymath}
  na categoria homotópica $\Ho(\mathsf{M})$, onde $\theta_{p} \in \Hom_{\Ho(\mathsf{M})}(W,\Fib(p))$ e $\theta_{p'} \in \Hom_{\Ho(\mathsf{M})}(W,\Fib(p'))$ são as classes de obstrução associadas aos problemas de levantamento
  \begin{displaymath}
    \begin{tikzcd}
      A
      \arrow[r, "\alpha"]
      \arrow[d, tail, "i" swap]
      & X
      \arrow[d, two heads, "p"]
      \\ B
      \arrow[r, "\beta" swap]
      & Y
    \end{tikzcd}
    \qquad \text{e} \qquad
    \begin{tikzcd}
      A
      \arrow[r, "\varphi \circ \alpha"]
      \arrow[d, tail ,"i" swap]
      & X'
      \arrow[d, two heads, "p'"]
      \\ B
      \arrow[r, "\beta" swap]
      & Y'
    \end{tikzcd}
  \end{displaymath}
  respectivamente.
\end{obs}

Nosso objetivo é entender um pouco quais tipos de cofibrações admitem uma teoria de obstrução no sentido acima.
Dentre os problemas de levantamento não-triviais que nos interessam, um ponto de partida razoável é entendermos primeiro aqueles envolvendo fibrações do tipo $X \fib *$, ou seja, aqueles envolvendo objetos fibrantes.
Vimos no \cref{exem:extensao_ao_longo_de_cofibracao_como_levantamento} que tais problemas de levantamento são nada mais do que problemas de extensão ao longo de cofibrações.
Vamos entender qual é o conteúdo homotópico destes problemas no caso clássico das cofibrações geradoras $S^n \to D^{n+1}$ na estrutura modelo de Quillen em $\mathsf{Top}$.

O interessante é que esse resultado pode ser generalizado de certa forma para o contexto de categorias de modelos, mas a categoria $\mathsf{Top}$ possui algumas boas propriedades que não são válidas em uma categoria de modelos geral, portanto essa generalização só é possível após enfraquecermos um pouco as hipóteses.
Esse enfraquecimento tem a ver com os tipos de fibrações que aparecerão nos problemas de levantamento.

\begin{defin}\label{defin:teoria_de_obstrucao_fibrante}
  Uma cofibração $i: A \cofib B$ em uma categoria de modelos pontuada $\mathsf{C}$ \textbf{admite uma teoria de obstrução fibrante} se as propriedades da \cref{defin:teoria_de_obstrucao} são válidas para toda fibração $p: X \fib Y$ onde $Y$ é um objeto fibrante.\footnote{Veja que nesse caso $X$ é automticamente fibrante também, portanto temos uma fibração entre objetos fibrantes, e tais tipos de morfismos costumam ser bem comportados.}
\end{defin}

\section{Resultados auxiliares}

Nessa seção coletamos alguns resultados auxiliares que serão úteis em diferentes partes do texto.
Resolvi incluir algumas demonstrações por motivos de completude, especialmente para aqueles resultados cujas demonstrações na literatura foram deixadas para o leitor.

Nosso primeiro objetivo é demonstrar o famoso \emph{Lema de Ken Brown}, um resultado que fornece condições suficientes para que um funtor preserve uma certa classe de equivalências fracas.
Tal lema é uma consequência relativamente simples de um outro lema que fornece fatorações especiais para certas equivalências fracas.

\begin{prop}[Lema de Fatoração]
  \label{prop:lema_de_fatoracao}
  Se $f: X \to Y$ uma equivalência fraca entre objetos fibrantes de uma categoria de modelos, então existem uma cofibração trivial $i: X \overset{\sim}{\cofib} Z$, uma fibração trivial $p: Z \overset{\sim}{\fib} Y$, e uma fibração trivial $q: Z \overset{\sim}{\fib} X$ tais que $p \circ i = f$ e $q \circ i = \id_{X}$.
  \begin{displaymath}
    \begin{tikzcd}
      X
      \arrow[rr, "f"]
      \arrow[rd, tail, "i", "\sim" {swap,sloped}]
      & & Y
      \\ & Z
      \arrow[ru, two heads, "p", "\sim" {swap,sloped}]
      \arrow[lu, two heads, bend left=45, "q", "\sim" {sloped}]
    \end{tikzcd}
  \end{displaymath}
  De maneira dual, se $f: X \to Y$ é uma equivalência fraca entre objetos cofibrantes, então existem uma fibração trivial $q: X \overset{\sim}{\fib} Z$, uma cofibração trivial $j: Z \overset{\sim}{\cofib} Y$ e uma cofibração trivial $i: Z \overset{\sim}{\cofib} X$ tais que $q \circ j = f$ e $q \circ i = \id_{X}$.
  \begin{displaymath}
    \begin{tikzcd}
      X
      \arrow[rr, "f"]
      \arrow[rd, two heads, "q", "\sim" {swap,sloped}]
      & & Y
      \\ & Z
      \arrow[ru, tail, "j", "\sim" {swap,sloped}]
      \arrow[lu, tail, bend left=45, "i", "\sim" {sloped}]
    \end{tikzcd}
  \end{displaymath}
\end{prop}

\begin{proof}
  Vamos demonstrar o primeiro caso, já que este é que nos será útil.
  A demonstração do segundo caso é dual.
  Podemos fatorar o morfismo induzido $(\id_{X},f): X \to X \times Y$ como uma cofibração trivial $i: X \overset{\sim}{\to} Z$ seguida de uma fibração $\theta: Z \overset{\sim}{\to} X \times Y$.
  \begin{displaymath}
    \begin{tikzcd}
      X
      \arrow[rr, "{(\id_{X},f)}"]
      \arrow[rd, tail, "\sim" {sloped}, "\lambda" {swap}]
      & & X \times Y
      \\ & Z
      \arrow[ru, two heads, "\theta" swap]
    \end{tikzcd}
  \end{displaymath}

  Se $\pi_{1}: X \times Y \to X$ e $\pi_{2}: X \times Y \to Y$ são as projeções canônicas, definimos então
  \begin{displaymath}
    q \coloneqq \pi_{1} \circ \theta: Z \to X \quad \text{e} \quad p \coloneqq \pi_{2} \circ \theta: Z \to Y.
  \end{displaymath}
  Veja que tais morfismos satisfazem as igualdades necessárias, já que por um lado
  \begin{displaymath}
    p \circ i = \pi_{2} \circ \theta \circ i = \pi_{2} \circ (\id_{X},f) = f,
  \end{displaymath}
  e por outro
  \begin{displaymath}
    q \circ i = \pi_{1} \circ \theta \circ i = \pi_{1} \circ (\id_{X},f) = \id_{X}.
  \end{displaymath}

  Resta apenas mostrarmos que $p$ e $q$ são fibrações triviais.
  O fato de $p$ ser uma equivalência fraca segue da propriedade 2-de-3, já que temos a igualdade $p \circ i = f$ onde tanto $i$ quanto $f$ são equivalências fracas.
  O fato de $q$ ser uam equivalência fraca também segue da propriedade 2-de-3, pois temos a igualdade $q \circ i = \id_{X}$ onde $i$ e $\id_{X}$ são equivalências fracas.
  Por fim, para ver que $p$ e $q$ são fibrações, veja que temos um diagrama de pullback
  \begin{displaymath}
    \begin{tikzcd}
      X \times Y
      \arrow[r, "\pi_{1}"]
      \arrow[d, "\pi_{2}" swap]
      & X
      \arrow[d, two heads, "!_{X}"]
      \\ Y
      \arrow[r, two heads, "!_{Y}" swap]
      & *
    \end{tikzcd}
  \end{displaymath}
  onde $!_{X}$ e $!_{Y}$ são fibrações graças ao fato de $X$ e $Y$ serem fibrantes.
  Como fibrações são preservadas por pullbacks, segue que as projeções canônicas $\pi_{1}$ e $\pi_{2}$ são fibrações também, logo $p$ e $q$ são ambos composições de fibrações e, portanto, fibrações também.
\end{proof}

\begin{corol}[Lema de Ken Brown]
  \label{corol:lema_de_brown}
  Sejam $(\mathsf{M},\mathcal{W},\mathcal{C},\mathcal{F})$ uma categoria de modelos e $(\mathcal{D},\mathcal{W}')$ uma categoria com equivalências fracas, ou seja, $\mathcal{W}' \subseteq Mor(\mathsf{D})$ é uma classe de morfismos fechada por composições, contendo todos os isomorfismos, e satisfazendo a propriedade 2-de-3.
  Se $F: \mathsf{C} \to \mathsf{D}$ é um funtor que transforma fibrações triviais entre objetos fibrates de $\mathsf{M}$ em equivalências fracas de $\mathsf{D}$, então $F$ também tansforma equivalências fracas entre objetos fibrantes de $\mathsf{M}$ em equivalências fracas de $\mathsf{D}$.
  De maneira dual, se $F$ transforma cofibrações triviais entre objetos cofibrantes de $\mathsf{M}$ em equivalências fracas de $\mathsf{D}$, então $F$ também transforma equivalências fracas entre objetos cofibrantes de $\mathsf{M}$ em equivalências fracas de $\mathsf{D}$.
\end{corol}

\begin{proof}
  Seja $p: X \overset{\sim}{\to} Y$ uma equivalência fraca entre objetos fibrantes de $\mathsf{M}$.
  Aplicando o \cref{prop:lema_de_fatoracao} obtemos uma cofibração trivial $i: X \overset{\sim}{\cofib} Z$, uma fibração trivial $p: Z \overset{\sim}{\fib} Y$ e uma fibração trivial $q: Z \overset{\sim}{\fib} X$ tais que $p \circ i = f$ e $q \circ i = \id_{X}$.
  Veja que $Z$ é um objeto fibrante, já que ele manda uma fibração para um objeto fibrante.
  Assim, $p$ é uma fibração trivial entre objetos fibrantes, portanto $F(p): F(Z) \to F(Y)$ é uma equivalência fraca em $\mathsf{D}$.
  Se conseguirmos mostrar que $F(i): F(X) \to F(Z)$ é também uma equivalência fraca, o resultado desejado seguirá então da igualdade $F(p) \circ F(i) = F(f)$ e da propriedade 2-de-3 em $\mathsf{M}$.
  Ora, sendo $q: Z \to X$ uma fibração trivial entre objetos fibrantes também, vale que $F(q): F(Z) \to F(X)$ é uma equivalência fraca, e a propriedade 2-de-3 aplicada à igualdade $F(q) \circ F(i) = \id_{F(X)}$ nos permite concluir que $F(i)$ é uma equivalência fraca, já que o morfismo idêntico $\id_{F(X)}$ é também uma equivalência fraca.
\end{proof}

\section{Construindo teorias de obstrução}

Nessa seção mostramos enfim como obter teorias de obstrução para algumas classes de cofibrações.
Mostramos inicialmente que cofibrações com condições de contratibilidade sempre admitem teorias de obstrução fibrantes, e em seguida mostramos que certas operações na classe de cofibrações preservam a existência de teorias de obstrução.

\begin{prop}
  Em uma categoria de modelos pontuada, considere uma cofibração $i: A \cofib B$ onde $A$ é cofibrante e $B$ é fibrante e fracamente contrátil.
  Dado um morfismo $f: A \to X$, onde $X$ é um objeto fibrante, existe um morfismo $F: B \to X$ tal que $F \circ i = f$ se, e somente se, $f$ é homotopicamente nulo.
  \begin{displaymath}
    \begin{tikzcd}
      A
      \arrow[r, "f"]
      \arrow[d, tail, "i" swap]
      & X
      \arrow[d, two heads, "t_{X}"]
      \\ B
      \arrow[r, "t_{B}" swap]
      \arrow[ru, dashed, "F" description]
      & *
    \end{tikzcd}
  \end{displaymath}
\end{prop}

\begin{proof}
  
\end{proof}

% \appendix

% \section{Resultados categóricos}

% Esse apêndice contém alguns resultados auxiliares de natureza puramente categórica que são usados ao longo das notas.
% Resolvi incluí-los aqui para tornar o documento um pouquinho mais completo e (espero) um pouco mais útil para um eventual leitor.

% \begin{lema}\label{lema:pullback_de_mono_eh_mono}
%   Monomorfismos são preservados por pullbacks.
%   Mais precisamente, se o diagrama abaixo é um pullback em uma categoria $\mathsf{C}$ qualquer, e $g: X \to Y$ é um monomorfismo, então $f: A \to B$ também é um monomorfismo.
%   \begin{displaymath}
%     \begin{tikzcd}
%       A
%       \arrow[r, "\alpha"]
%       \arrow[d, "f" swap]
%       & X
%       \arrow[d, "g"]
%       \\ B
%       \arrow[r, "\beta" swap]
%       & Y
%     \end{tikzcd}
%   \end{displaymath}
% \end{lema}

% \begin{proof}
%   Suponha que tenhamos morfismos $r,\,s: Z \to A$ tais que $f \circ r = f \circ s$.
%   Nosso objetivo é mostrar que necessariamente temos então $r = s$.

%   Note primeiro que a propriedade universal do pullback garante que $r: Z \to A$ é o único morfismo de seu tipo que faz comutar o diagrama abaixo.
%   \begin{displaymath}
%     \begin{tikzcd}
%       Z
%       \arrow[rd, dashed, "r" description]
%       \arrow[rrd, bend left=20, "\alpha \circ r"]
%       \arrow[rdd, bend right=20, "f \circ r" swap]
%       \\ & A
%       \arrow[r, "\alpha"]
%       \arrow[d, "f" swap]
%       & X
%       \arrow[d, "g"]
%       \\ & B
%       \arrow[r, "\beta" swap]
%       & Y
%     \end{tikzcd}
%   \end{displaymath}
%   Afirmamos que $s: Z \to A$ também satisfaz as mesmas condições de comutatividade.
%   De fato, por um lado a igualdade $f \circ s = f \circ r$ segue direta da hipótese feita sobre os morfismos $r$ e $s$; enquanto por outro temos
%   \begin{displaymath}
%     g \circ \alpha \circ s = \beta \circ f \circ s = \beta \circ f \circ r = g \circ \alpha \circ r,
%   \end{displaymath}
%   mas $g$ é um monomorfismo por hipótese, logo podemos cancelá-lo na sequência de igualdades acima para concluirmos que $\alpha \circ s = \alpha \circ r$ como desejado.

%   A unicidade sobre $r$ vinda da propriedade universal do pullback implica então $s = r$ como queríamos mostrar.
% \end{proof}

% \begin{teo}
%   Sejam $\mathsf{C}$ uma categoria qualquer, $A \in \mathsf{C}$ um objeto qualquer e $A / \mathsf{C}$ a categoria de objetos sob $A$.
%   \begin{enumerate}
%   \item Se $\mathsf{C}$ é completa, então $A / \mathsf{C}$ é completa também.
    
%   \item Se $\mathsf{C}$ é co-completa, então $A / \mathsf{C}$ é co-completa também.
%   \end{enumerate}
% \end{teo}

% \begin{proof}
%   1. Denote por $\mathcal{E}: A / \mathsf{C} \to \mathsf{C}$ o funtor de esquecimento evidente.
%   Considere então uma categoria pequena $\mathsf{J}$ e um funtor $F: \mathsf{J} \to A / \mathsf{C}$.
%   Esse funtor associa a cada $j \in \mathsf{J}$ um par $(F_{j},f_{j})$, onde $F_{j}$ é um objeto de $\mathsf{C}$ e $f_{j}: A to F_{j}$ é um morfismo.
%   Além disso, dado um morfismo $\alpha: i \to j$ em $\mathsf{J}$ temos um morfismo correspondente $F(\alpha): (F_{i},f_{i}) \to (F_{j},f_{j})$, ou seja, um morfismo $F(\alpha): F_{i} \to F_{j}$ em $\mathsf{C}$ satisfazendo a condição de compatibilidade $F(\alpha) \circ f_{i} = f_{j}$.

%   Considere então o funtor composto $\mathcal{E} \circ F: \mathsf{J} \to \mathsf{C}$.
%   Sendo $\mathsf{C}$ completa por hipótese, tal funtor admite um limite $L \in \mathsf{C}$ o qual vem equipado com os morfismos estruturais $(p_{j}: L \to F_{j})_{j \in \mathsf{J}}$ que exibem tal limite como um cone terminal sobre o diagrama determinado pelo funtor $\mathcal{E} \circ F$.
%   Queremos enxergar $L$ como um objeto da categoria $A / \mathsf{C}$, e para isso precisamos de um morfismo do tipo $A \to L$ que obteremos por meio da propriedade universal do limite.
%   Note que a coleção de morfismos $(f_{j})_{j \in \mathsf{C}}$ determina um cone sobre $\mathcal{E} \circ F$ com vértice em $A$ graças às condições de compatibilidade satisfeitas pelos $f_{j}$.
%   Ora, como $L$ é um cone terminal, existe um único morfismo $f: A \to L$ satisfazendo as condições $p_{j} \circ f = f_{j}$ para todo $j \in \mathsf{J}$.

%   O par $(L,f)$ determina então um objeto da categoria $A / \mathsf{C}$, e graças às igualdades mencionadas no final do parágrafo anterior, os morfismos $p_{j}: L \to F_{j}$ podem ser vistos como morfismos $p_{j}: (L,f) \to (F_{j},f_{j})$ na categoria $\mathsf{C}$.
%   Veja que, se $\alpha: i \to j$ é um morfismo em $\mathsf{J}$, como os morfismos $(p_{j})_{j \in \mathsf{J}}$ determinavam um cone sobre $\mathcal{E} \circ F$ e vale que $(\mathcal{E} \circ F)(\alpha) = F(\alpha)$, temos a igualdade $F(\alpha) \circ p_{i} = p_{j}$; portanto a coleção $(p_{j})_{j \in \mathsf{J}}$ define também um cone \emph{sobre o funtor $F$} com vértice em $(L,f)$.

%   Nosso objetivo é mostrar que esse cone é na verdade um cone terminal, o que naturalmente será uma consequência de $L$ determinar um cone terminal sobre o funtor $\mathcal{E} \circ F$.
%   Suponha que $(p_{j}': (L',f') \to (F_{j},f_{j}))_{j \in \mathsf{J}}$ seja um outro cone sobre $F$ com vértice no objeto $(L',f')$.
%   Aplicando o funtor de esquecimento $\mathcal{E}$ obtemos então o cone $(p_{j'}: L' \to F_{j})_{j \in \mathsf{J}}$ sobre $\mathcal{E} \circ F$ com vértice em $L'$.
%   Sendo $L$ o limite deste funtor, existe um morfismo único $\theta: L' \to L$ tal que $p_{j} \circ \theta = p_{j}'$ para todo $j \in \mathsf{J}$.
%   Note que $\theta$ pode ser visto como um morfismo do tipo $(L',f') \to (L,f)$.
%   De fato, por construção $f: A \to L$ é o único morfismo de seu tipo a satisfazer a igualdade $p_{j} \circ f = f_{j}$ para todo $j \in \mathsf{J}$, mas usando que os $p_{j}$'s eram morfismos do tipo $(L',f') \to (F_{j},f_{j})$, temos
%   \begin{displaymath}
%     p_{j} \circ \theta \circ f' = p_{j}' \circ f' = f_{j},
%   \end{displaymath}
%   portanto pela unicidade devemos ter $\theta \circ f' = f$.
%   Esse raciocínio mostra que eixste um único morfismo do cone $(p_{j}': (L',f') \to (F_{j},f_{j}))_{j \in \mathsf{J}}$ para o cone $(p_{j}: (L,f) \to (F_{j},f_{j}))_{j \in \mathsf{J}}$, ou seja, o objeto $(L,f) \in A / \mathsf{C}$ determina um limite para o funtor $F$.
% \end{proof}

\end{document}

%%% Local Variables:
%%% mode: latex
%%% TeX-master: t
%%% End:
