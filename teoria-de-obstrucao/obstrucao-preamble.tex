\documentclass[brazilian]{article}

\usepackage{babel}
\usepackage[utf8]{inputenc}
\usepackage{csquotes}
\usepackage[margin=1.35in]{geometry}
\usepackage[backend=biber,style=alphabetic]{biblatex}
\usepackage{mathtools,amssymb,amsthm}
\usepackage{indentfirst}
\usepackage{hyperref}
\usepackage{tikz-cd}
\usepackage{graphicx}
\usepackage[capitalize,noabbrev]{cleveref}

\usetikzlibrary{babel}

\addbibresource{bibliografia.bib}

\swapnumbers
\newtheorem{teo}{Teorema}[section]
\newtheorem{prop}[teo]{Proposição}
\newtheorem{lema}[teo]{Lema}
\newtheorem{corol}[teo]{Corolário}

\theoremstyle{definition}
\newtheorem{defin}[teo]{Definição}
\newtheorem{obs}[teo]{Observação}
\newtheorem{exem}[teo]{Exemplo}

\DeclarePairedDelimiter{\abs}{\lvert}{\rvert}
\DeclarePairedDelimiter{\Abs}{\lVert}{\rVert}

\newcommand{\id}{\mathrm{id}}
\newcommand{\ct}{\mathrm{ct}}
\newcommand{\Mor}{\mathrm{Mor}}
\newcommand{\Hom}{\mathrm{Hom}}
\newcommand{\Arr}{\mathrm{Arr}}
\newcommand{\cofib}{\rightarrowtail}
\newcommand{\Cofib}{\mathrm{Cofib}}
\newcommand{\fib}{\twoheadrightarrow}
\newcommand{\Fib}{\mathrm{Fib}}
\newcommand{\dom}{\mathrm{dom}}
\newcommand{\cod}{\mathrm{cod}}
\newcommand{\comp}{\mathrm{comp}}
\newcommand{\fac}{\mathrm{fac}}
\newcommand{\init}{\mathrm{init}}
\newcommand{\Fun}{\mathrm{Fun}}
\newcommand{\Cyl}{\mathrm{Cyl}}

\renewcommand{\qedsymbol}{$\blacksquare$}

\title{Teoria de Obstrução Categórica\\Notas para o seminário}
\author{Edmundo Martins}
\date{\today}

%%% Local Variables:
%%% mode: latex
%%% TeX-master: "main"
%%% End:
