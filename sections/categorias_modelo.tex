\section{Categorias de modelos}

\begin{defin}\label{defin:estrutura_de_modelos}
  Seja $\mathsf{M}$ uma categoria localmente pequena.
  Uma \textbf{estrutura de modelos} em $\mathsf{M}$ consiste de três classes de morfismos $\mathcal{W},\, \mathcal{C},\, \mathcal{F} \subseteq \mathrm{Mor}(\mathsf{M})$ cujos elementos são chamados, respectivamente, \textbf{equivalências fracas}, \textbf{cofibrações} e \textbf{fibrações}, as quais devem satisfazer os seguintes axiomas:
  \begin{enumerate}
  \item[(M1)] A categoria $\mathsf{M}$ é bicompleta, ou seja, admite todos os limites e colimites indexados por categorias pequenas.
    
  \item[(M2)] (Propriedade 2-de-3) Dados morfismos $f: X \to Y$ e $g: Y \to Z$ em $\mathsf{M}$, se dois dos morfismos do conjunto $\{f,\,g,\, g \circ f\}$ estiverem em $\mathcal{W}$, então o terceiro também deve estar.
    
  \item[(M3)] (Propriedade de retração) Se um morfismo $f: A \to X$ é retração de um outro morfismo $g: B \to Y$, ou seja, se existe um diagrama comutativo como abaixo,
    \begin{equation}\label{eq:diagrama_axioma_de_retracao}
      \begin{tikzcd}
        A
        \arrow[r]
        \arrow[d, "f" swap]
        \arrow[rr, bend left=25, "\id_{A}"]
        & B
        \arrow[r]
        \arrow[d, "g"]
        & A
        \arrow[d, "f"]
        \\ X
        \arrow[r]
        \arrow[rr, bend right=25, "\id_{X}" swap]
        & Y
        \arrow[r]
        & X
      \end{tikzcd}
    \end{equation}
    e $g$ pertence a $\mathcal{W}$ (ou a $\mathcal{C}$, ou a $\mathcal{F}$), então $f$ também pertence a $\mathcal{W}$ (ou a $\mathcal{C}$, ou a $\mathcal{F}$, respectivamente).
    Em suma, as classes $\mathcal{W}$, $\mathcal{C}$ e $\mathcal{F}$ são fechadas por retrações.
    
  \item[(M4)] (Propriedade de levantamento) Dado um diagrama comutativo como abaixo,
    \begin{equation}\label{eq:diag_prob_levantamento}
      \begin{tikzcd}
        A
        \arrow[r, "\alpha"]
        \arrow[d, "i" swap]
        & X
        \arrow[d, "p"]
        \\ B
        \arrow[r, "\beta" swap]
        & Y
      \end{tikzcd}
    \end{equation}
    onde $i$ é uma cofibração, e $p$ é uma fibração; se um dos dois morfismos $i$ ou $p$ é também uma equivalência fraca, então o diagrama admite um \emph{levantamento}, ou seja, existe um morfismo diagonal $f: B \to X$ que faz comutar o diagrama abaixo.
    \begin{equation}\label{eq:diag_solucao_levantamento}
      \begin{tikzcd}
        A
        \arrow[r, "\alpha"]
        \arrow[d, "i" swap]
        & X
        \arrow[d, "p"]
        \\ B
        \arrow[r, "\beta" swap]
        \arrow[ru, dashed, "f" description]
        & Y
      \end{tikzcd}
    \end{equation}
    
  \item[(M5)] (Propriedade de fatoração) Qualquer morfismo $f: X \to Y$ em $\mathsf{M}$ pode ser na forma mostrada abaixo,
    \begin{equation}\label{eq:diag_fatoracao_cofib_fib_trivial}
      \begin{tikzcd}
        X
        \arrow[rr, "f"]
        \arrow[rd, "i" swap]
        & & Y
        \\ & Z
        \arrow[ru, "p" swap]
      \end{tikzcd}
    \end{equation}
    onde $i$ é uma cofibração, e $p$ é simultaneamente uma fibração e uma equivalência fraca.
    Além disso, todo morfismo também pode ser fatorado na forma mostrada abaixo,
    \begin{equation}\label{eq:diag_fatoracao_cofib_trivial_fib}
      \begin{tikzcd}
        X
        \arrow[rr, "f"]
        \arrow[rd, "j" swap]
        & & Y
        \\ & Z
        \arrow[ru, "q" swap]
      \end{tikzcd}
    \end{equation}
    onde nesse caso $j$ é simultaneamente uma cofibração e uma equivalência, e $q$ é uma fibração.
  \end{enumerate}
\end{defin}

Vamos introduzir um pouco de terminologia antes de fazermos alguns comentários sobre a definição acima.
Os morfismos de $\mathsf{M}$ que pertencem à classe $\mathcal{W} \cap \mathcal{C}$ são chamados de \textbf{cofibrações triviais} ou \textbf{cofibrações acíclicas}, enquanto os morfismos que pertencem à classe $\mathcal{W} \cap \mathcal{F}$ são chamados de \textbf{fibrações triviais} ou \textbf{fibrações acíclicas}.
Usando essa terminologia o axioma de fatoração (M5) pode ser enunciado da seguinte forma: todo morfismo em uma categoria de modelos pode ser fatorado como uma cofibração seguido de uma fibração trivial e também como uma cofibração trivial seguida de uma fibração.

Um quadrado comutativo tendo uma cofibração na aresta esquerda e uma fibração na aresta direita - como no diagrama \eqref{eq:diag_prob_levantamento} - será chamado de um \textbf{problema de levantamento}.
Caso um desses dois morfismos seja trivial, usaremos então o termo \textbf{problema de levantamento trivial}.
Tendo em vista essa terminologia, o axioma de levantamento (M4) pode ser enunciado da seguinte forma: em uma categoria de modelos, todo problema de levantamento trivial admite uma solução.

\begin{obs}\label{obs:axioma_retracao_explicacao}
  Lembremos que, dados objetos $X$ e $Y$ de uma categoria $\mathsf{C}$ qualquer, dizemos que $X$ é um \textbf{retrato} de $Y$ se existem morfismos $s: X \to Y$ e $r: Y \to X$ tais que $r \circ s = \id_{X}$.
  Comumente nos referimos ao morfismo $s$ por \textbf{seção} e ao morfismo $r$ por \textbf{retração}.
  A condição $r \circ s = \id_{X}$ garante que $s$ seja um monomorfismo.
  De fato, se $f,\, g: W \to X$ são morfismos tais que $s \circ f = s \circ g$, então
  \begin{displaymath}
    f
    = \id_{X} \circ f
    = r \circ s \circ f
    = r \circ s \circ g
    = \id_{X} \circ g
    = g.
  \end{displaymath}
  Isso nos permite encarar $X$ como um subobjeto de $Y$, e o morfismo $r$ então intuitivamente deforma $Y$ para esse subobjeto, mas de forma a mantê-lo fixado.
  Note que a condição $r \circ s = \id_{X}$ garante também que o morfismo $r$ seja um epimorfismo.

  A noção de retração que aparece no axioma (M3) de uma estrutura de modelos enunciado acima pode ser interpretada nesse sentido se introduzirmos uma categoria adequada para isso.
  Lembremos que toda categoria $\mathsf{C}$ dá origem a uma categoria de setas $\Arr(\mathsf{C})$.
  Os objetos dessa categorias são precisamente morfismos $f: A \to B$ na categoria incial $\mathsf{C}$, e dados dois tais objetos $f: A \to B$ e $g: X \to Y$, um morfismo do tipo $(f: A \to B) \to (g: X \to Y)$ na categoria de setas $\Arr(\mathsf{C})$ é dado por um par de morfismos $(\alpha: A \to X,\, \beta: B \to Y)$ satisfazendo a igualdade $\beta \circ f = g \circ \alpha$.
  Podemos então visualizar esse morfismo em $\Arr(\mathsf{C})$ na forma de um quadrado comutativo como mostrado abaixo.
  \begin{displaymath}
    \begin{tikzcd}
      A
      \arrow[d, "f" swap]
      \arrow[r, "\alpha"]
      & X
      \arrow[d, "g"]
      \\ B
      \arrow[r, "\beta" swap]
      & Y
    \end{tikzcd}
  \end{displaymath}
  A composição de morfismos é definida ``colando'' quadrados comutativos adjacentes.
  Mais precisamente, dados três objetos $f: X_{1} \to Y_{1},\, g: X_{2} \to Y_{2}$ e $h: X_{3} \to Y_{3}$ na categoria $\Arr(\mathsf{C})$, e dados também dois morfismos componíveis
  \begin{displaymath}
    (\alpha_{1}: X_{1} \to X_{2},\, \beta_{1}: Y_{1} \to Y_{2}) \qquad (\alpha_{2}: X_{2} \to X_{3},\, \beta_{2}: Y_{2} \to Y_{3}),
  \end{displaymath}
  sua composição é o morfismo
  \begin{displaymath}
    (\alpha_{2}, \beta_{2}) \circ (\alpha_{1},\beta_{1}): (f: X_{1} \to Y_{1}) \to (h: X_{3} \to Y_{3})
  \end{displaymath}
  em $\Arr(\mathsf{C})$ definido pelo par
  \begin{displaymath}
    (\alpha_{2},\beta_{2}) \circ (\alpha_{1},\beta_{1}) \coloneqq (\alpha_{2} \circ \alpha_{1}: X_{1} \to X_{3},\, \beta_{2} \circ \beta_{1}: Y_{1} \to Y_{3}).
  \end{displaymath}
  Essa composição pode também ser visualizada como mostrado abaixo.
  \begin{displaymath}
    \begin{tikzcd}
      X_{1}
      \arrow[d, "f" swap]
      \arrow[r, "\alpha_{1}"]
      & X_{2}
      \arrow[d, "g"]
      \arrow[r, "\alpha_{2}"]
      & X_{3}
      \arrow[d, "h"]
      \\ Y_{1}
      \arrow[r, "\beta_{1}" swap]
      & Y_{2}
      \arrow[r, "\beta_{2}" swap]
      & Y_{3}
    \end{tikzcd}
    \quad \Longrightarrow \quad
    \begin{tikzcd}
      X_{1}
      \arrow[d, "f" swap]
      \arrow[r, "\alpha_{2} \circ \alpha_{1}"]
      & X_{3}
      \arrow[d, "h"]
      \\ Y_{1}
      \arrow[r, "\beta_{2} \circ \beta_{1}" swap]
      & Y_{3}
    \end{tikzcd}
  \end{displaymath}

  A associatividade dessa composição via colagem segue diretamente da associatividade da composição na categoria inicial $\mathsf{C}$.
  Por fim, dado um objeto $f: X \to Y$ qualquer, o morfismo idêntico associado a ele é dado pelo par $\id_{f} \coloneqq (\id_{X},\id_{Y})$, conforme mostrado no quadrado comutativo abaixo.
  \begin{displaymath}
    \begin{tikzcd}
      X
      \arrow[r, "\id_{X}"]
      \arrow[d, "f" swap]
      & X
      \arrow[d, "f"]
      \\ Y
      \arrow[r, "\id_{Y}" swap]
      & Y
    \end{tikzcd}
  \end{displaymath}

  Note agora que, se o objeto $f: A \to B$ é um retrato do objeto $g: X \to Y$ \emph{na categoria de setas $\Arr(\mathsf{M})$}, então por definição existem morfismos $s_{1}: A \to X,\, s_{2}: B \to Y,\, r_{1}: X \to A$ e $r_{2}: Y \to B$ tais que $(r_{1},r_{2}) \circ (s_{1},s_{2}) = \id_{f}$, o que também pode ser expresso pelo diagrama comutativo abaixo.
  \begin{displaymath}
    \begin{tikzcd}
      A
      \arrow[r, "s_{1}"]
      \arrow[d, "f" swap]
      \arrow[rr, bend left=45, "\id_{A}"]
      & X
      \arrow[r, "r_{1}"]
      \arrow[d, "g"]
      & A
      \arrow[d, "f"]
      \\ B
      \arrow[r, "s_{2}" swap]
      \arrow[rr, bend right=45, "\id_{B}" swap]
      & Y
      \arrow[r, "r_{2}" swap]
      & B
    \end{tikzcd}
  \end{displaymath}

   Esse é precisamente o diagrama que aparece no axioma de retração na definição de uma estrutura modelo.
  Podemos então reformular tal axioma dizendo que as classes de equivalências fracas, fibrações e cofibrações são todas fechadas por \emph{retrações na categoria de setas $\Arr(\mathsf{C})$}.
\end{obs}

\begin{obs}
  Quando trabalhamos com categorias de modelos, ao invés de dizermos explicitamente que um morfismo é uma equivalência fraca, uma cofibração, ou uma fibração; é comum indicarmos isso decorando de alguma forma a seta que representa tal morfismo.
  A convenção notacional que seguiremos nesse aspecto é a seguinte:
  \begin{itemize}
  \item uma equivalência fraca será denotada por $\overset{\sim}{\rightarrow}$;
    
  \item uma cofibração será denotada por $\cofib$;
    
  \item uma fibração será denotada por $\fib$.
  \end{itemize}
  Combinando os símbolos acima obtemos outros que utilizaremos para denotar cofibrações triviais ou fibrações triviais:
  \begin{itemize}
  \item uma cofibração trivial será denotada por $\overset{\sim}{\cofib}$;
    
  \item uma fibração trivial será denotada por $\overset{\sim}{\fib}$.
  \end{itemize}
\end{obs}

\subsection{Fatorações em categorias}

Antes de investigarmos mais a fundo as propriedades de categorias modelo, vamos investigar parte de sua estrutura sob uma perspectiva mais geral.
O ponto central da discussão é que a definição de uma categoria modelo pode ser encapsulada totalmente pela existência de fatorações em cofibrações e fibrações que estão relacionadas por condições de levantamento.

Inicialmente, definimos a noção de levantamento de forma mais geral.

\begin{defin}
  Sejam $\mathsf{C}$ uma categoria e $\mathcal{A} \subseteq \Mor(\mathsf{C})$ uma classe qualquer de morfismos.
  Dizemos que um morfismo $f: A \to B$ em $\mathsf{C}$ \textbf{satisfaz a propriedade de levantamento à esquerda com relação a $\mathcal{A}$} se todo quadrado comutativo como abaixo,
  \begin{displaymath}
    \begin{tikzcd}
      A
      \arrow[r]
      \arrow[d, "f" swap]
      & X
      \arrow[d, "p"]
      \\ B
      \arrow[r]
      & Y
    \end{tikzcd}
  \end{displaymath}
  onde $p: X \to Y$ pertence a $\mathcal{A}$, admite um levantamento, ou seja, existe um morfismo $h: B \to X$ que faz comutar o diagrama abaixo.
  \begin{displaymath}
    \begin{tikzcd}
      A
      \arrow[r]
      \arrow[d, "f" swap]
      & X
      \arrow[d, "p"]
      \\ B
      \arrow[r]
      \arrow[ru, dashed, "h" description]
      & Y
    \end{tikzcd}
  \end{displaymath}

  Dualmente, dizemos que um morfismo $g: X \to Y$ \textbf{satisfaz a propriedade de levantamento à direita com relação a $\mathcal{A}$} se todo quadrado comutativo como abaixo,
  \begin{displaymath}
    \begin{tikzcd}
      A
      \arrow[d, "i" swap]
      \arrow[r]
      & X
      \arrow[d, "g"]
      \\ B
      \arrow[r]
      & Y
    \end{tikzcd}
  \end{displaymath}
  onde $i: A \to B$ pertence a $\mathcal{A}$, admite um levantamento $h: B \to X$ como mostrado abaixo.
  \begin{displaymath}
    \begin{tikzcd}
      A
      \arrow[r]
      \arrow[d, "i" swap]
      & X
      \arrow[d, "g"]
      \\ B
      \arrow[r]
      \arrow[ru, dashed, "h" description]
      & Y
    \end{tikzcd}
  \end{displaymath}
\end{defin}

Tendo a definição acima em mãos, podemos formular uma noção categórica de fatoração geral o suficiente para englobar a situação que aparece no estudo de categorias modelo.

\begin{defin}
  Um \textbf{sistema de fatoração fraco} em uma categoria $\mathsf{C}$ consiste de um par $(\mathcal{L},\mathcal{R})$, onde $\mathcal{L},\, \mathcal{R} \subseteq \Mor(\mathsf{C})$ são duas classes de morfismos, satisfazendo as seguintes condições:
  \begin{enumerate}
  \item[(i)] Todo morfismo $f \in \Mor(\mathsf{C})$ pode ser escrito na forma $f = f_{L} \circ f_{R}$ com $f_{L} \in \mathcal{L}$ e $f_{R} \in \mathcal{R}$;
    \begin{displaymath}
      \begin{tikzcd}[column sep=1.25cm]
        X
        \arrow[r, "f_{L} \in \mathcal{L}"]
        \arrow[rr, bend right=20, "f" swap]
        & Y
        \arrow[r, "f_{R} \in \mathcal{R}"]
        & Z
      \end{tikzcd}
    \end{displaymath}
    
  \item[(ii)] $\mathcal{L}$ consiste precisamente dos morfismos de $\mathsf{C}$ que satisfazem a propriedade de levantamente à esquerda com relação a $\mathcal{R}$;
    
  \item[(iii)] $\mathcal{R}$ consiste precisamente dos morfismos de $\mathsf{C}$ que satisfazem a propriedade de levantamento à direita com relação a $\mathcal{L}$.
  \end{enumerate}
\end{defin}

Os principais exemplos de sistemas de fatoração fracos nos quais estaremos interessados envolvem as cofibrações e fibrações triviais em uma categoria modelo, embora talves ainda não seja claro como essas classes dão origem a um sistema de fatoração.
Antes de detalharmos esse exemplo, entretanto, vamos demonstrar algumas propriedades gerais de sistemas de fatoração fracos.

\begin{prop}\label{prop:propriedades_fatoracao_fraca}
  Suponha que $(\mathcal{L},\mathcal{R})$ seja um sistema de fatoração fraco em uma categoria $\mathsf{C}$.
  Valem as seguintes propriedades:
  \begin{enumerate}
  \item Ambas as classes contêm todos os isomorfismos de $\mathsf{C}$.
    
  \item Ambas as classes são fechadas por composição.
    
  \item Ambas as classes são fechadas por retratos na categoria de setas $\Arr(\mathsf{C})$.
    
  \item $\mathcal{L}$ é fechada pela formação de pushouts, enquanto $\mathcal{R}$ é fechada pela formação de pullbacks.
    
  \item $\mathcal{L}$ é fechada por coprodutos, enquanto $\mathcal{R}$ é fechada por produtos.
  \end{enumerate}
\end{prop}

%TODO: Eu provavelmente deveria numerar alguns diagramas nessa demonstração e então referenciá-los nas sequências de igualdades a fim de tornar a demonstração mais fácil de acompanhar.

\begin{proof}
  1. Suponha que $f: A \to B$ seja um isomorfismo.
  Sabemos da definição de sistema de fatoração fraco que $\mathcal{L}$ consiste precisamente dos morfismos de $\mathsf{C}$ que satisfazem a propriedade de levantamento à esquerda com relação a $\mathcal{R}$.
  Considere então um quadrado comutativo como abaixo, onde $g: X \to Y$ é um morfismo pertencente à classe $\mathcal{R}$.
  \begin{displaymath}
    \begin{tikzcd}
      A
      \arrow[r, "\alpha"]
      \arrow[d, "f" swap]
      & X
      \arrow[d, "g"]
      \\ B
      \arrow[r, "\beta" swap]
      & Y
    \end{tikzcd}
  \end{displaymath}
  Sendo $f$ um isomorfismo por hipótese, podemos considerar o morfismo inverso $f^{-1}: B \to A$, e definir então um morfismo $h: B \to X$ por meio da composição $h \coloneqq \alpha \circ f^{-1}$.
  Note então que por um lado
  \begin{displaymath}
    h \circ f
    = \alpha \circ f^{-1} \circ f
    = \alpha \circ \id_{A}
    = \alpha,
  \end{displaymath}
  e por outro
  \begin{displaymath}
    g \circ h
    = g \circ \alpha \circ f^{-1}
    = \beta \circ f \circ f^{-1}
    = \beta \circ \id_{B}
    = \beta;
  \end{displaymath}
  mostando que $h$ faz comutar o diagrama abaixo, definindo então um levantamento para o quadrado comutativo original.
  \begin{displaymath}
    \begin{tikzcd}
      A
      \arrow[r, "\alpha"]
      \arrow[d, "f" swap]
      & X
      \arrow[d, "g"]
      \\ B
      \arrow[r, "\beta" swap]
      \arrow[ru, dashed, "h" description]
      & Y
    \end{tikzcd}
  \end{displaymath}

  A demonstração de que $\mathcal{R}$ contém todos os isomorfismos é análoga.
  Se $g: X \to Y$ é um isomorfismo, considere o quadrado comutativo abaixo onde $f: A \to B$ pertence à classe $\mathcal{L}$.
  \begin{displaymath}
     \begin{tikzcd}
      A
      \arrow[r, "\alpha"]
      \arrow[d, "f" swap]
      & X
      \arrow[d, "g"]
      \\ B
      \arrow[r, "\beta" swap]
      & Y
    \end{tikzcd}
  \end{displaymath}
  Dessa vez definimos um morfismo $h: B \to Y$ pela composição $h \coloneqq g^{-1} \circ \beta$, e notamos que esse morfismo satisfaz a igualdade
  \begin{displaymath}
    g \circ h
    = g \circ g^{-1} \circ \beta
    = \id_{Y} \circ \beta
    = \beta,
  \end{displaymath}
  e também a igualdade
  \begin{displaymath}
    h \circ f
    = g^{-1} \circ \beta \circ f
    = g^{-1} \circ g \circ \alpha
    = \id_{X} \circ \alpha
    = \alpha;
  \end{displaymath}
  portanto $h$ define um levantamento neste caso também.

  \smallskip
  2. Suponha que $f_{1}: A \to B$ e $f_{2}: B \to C$ sejam dois morfismos pertencentes à classe $\mathcal{L}$.
  A fim de mostrarmos que sua composição $f_{2} \circ f_{1}: A \to C$ também pertence a $\mathcal{L}$, vamos mostrar que essa composição satisfaz a condição de levantamento à esquerda com relação à $\mathcal{R}$.
  Considere então um quadrado comutativo como abaixo, onde $g: X \to Y$ pertence à classe $\mathcal{R}$.
  \begin{displaymath}
    \begin{tikzcd}
      A
      \arrow[r, "\alpha"]
      \arrow[d, "f_{2} \circ f_{1}" swap]
      & X
      \arrow[d, "g"]
      \\ C
      \arrow[r, "\beta" swap]
      & Y
    \end{tikzcd}
  \end{displaymath}
  A partir do quadrado acima podemos obter o quadrado comutativo mostrado abaixo, o qual admite um levantamento $h_{1}: B \to Y$ pois $f_{1} \in \mathcal{L}$.
  \begin{displaymath}
    \begin{tikzcd}
      A
      \arrow[r, "\alpha"]
      \arrow[d, "f_{1}" swap]
      & X
      \arrow[d, "g"]
      \\ B
      \arrow[r, "\beta \circ f_{2}" swap]
      \arrow[ru, dashed, "h_{1}" description]
      & Y
    \end{tikzcd}
  \end{displaymath}
  Usando o levantamento $h_{1}$ obtemos um terceiro quadrado comutativo como mostrado abaixo, o qual admite um levantamento $h_{2}: C \to X$ pois $f_{2} \in \mathcal{L}$.
  \begin{displaymath}
    \begin{tikzcd}
      B
      \arrow[r, "h_{1}"]
      \arrow[d, "f_{2}" swap]
      & X
      \arrow[d, "g"]
      \\ C
      \arrow[r, "\beta" swap]
      \arrow[ru, dashed, "h_{2}" description]
      & Y
    \end{tikzcd}
  \end{displaymath}
  Afirmamos que $h_{2}: C \to X$ define também um levantamento para o quadrado comutativo considerado inicialmente.
  De fato, por um lado a igualade $g \circ h_{2} = \beta$ segue diretamente da comutatividade do último quadrado acima, e por outro temos a sequência de igualdades
  \begin{displaymath}
    h_{2} \circ f_{2} \circ f_{1} = h_{1} \circ f_{1} = \alpha;
  \end{displaymath}
  portanto $h_{2}$ satisfaz as condições de comutatividades necessárias.

  A demonstração da segunda parte é análoga.
  Suponha que $g_{1}: X \to Y$ e $g_{2}: Y \to Z$ sejam dois morfismos pertencentes à classe $\mathcal{R}$, e considere o quadrado comutativo abaixo, onde $f: A \to B$ pertence à classe $\mathcal{L}$.
  \begin{displaymath}
    \begin{tikzcd}
      A
      \arrow[r, "\alpha"]
      \arrow[d, "f" swap]
      & X
      \arrow[d, "g_{2} \circ g_{1}"]
      \\ B
      \arrow[r, "\beta" swap]
      & Z
    \end{tikzcd}
  \end{displaymath}
  Considere então o quadrado comutativo abaixo, o qual admite um levantamento $h_{2}: B \to Y$ pois $g_{2}$ pertence a $\mathcal{R}$.
  \begin{displaymath}
    \begin{tikzcd}
      A
      \arrow[r, "g_{1} \circ \alpha"]
      \arrow[d, "f" swap]
      & Y
      \arrow[d, "g_{2}"]
      \\ B
      \arrow[r, "\beta" swap]
      \arrow[ru, dashed, "h_{2}" description]
      & Z
    \end{tikzcd}
  \end{displaymath}
  Usando $h_{2}$ consideramos então o quadrado comutativo abaixo, o qual também admite um levantamento $h_{1}: B \to X$ pois $g_{1} \in \mathcal{R}$.
  \begin{displaymath}
    \begin{tikzcd}
      A
      \arrow[r, "\alpha"]
      \arrow[d, "f" swap]
      & X
      \arrow[d, "g_{1}"]
      \\ B
      \arrow[r, "h_{2}" swap]
      \arrow[ru, dashed, "h_{1}" description]
      & Y
    \end{tikzcd}
  \end{displaymath}
  O morfismo $h_{1}$ é precisamente o procurado, já que por um lado a igualdade $h_{1} \circ f = \alpha$ segue diretamente da comutatividade acima, e por outro temos a sequência de igualdades
  \begin{displaymath}
    g_{2} \circ g_{1} \circ h_{1} = g_{2} \circ h_{2} = \beta;
  \end{displaymath}
  mostrando então que $h_{1}$ define um levantamento para o quadrado comutativo inicial.

  \smallskip
  3. Suponha que o morfismo $f: A \to B$ seja um retrato do morfismo $g: X \to Y$ o qual pertence à classe $\mathcal{L}$.
  Temos então por definição o diagrama comutativo abaixo
  \begin{displaymath}
    \begin{tikzcd}
      A
      \arrow[r, "s_{1}"]
      \arrow[d, "f" swap]
      \arrow[rr, bend left=45, "\id_{A}"]
      & X
      \arrow[r, "r_{1}"]
      \arrow[d, "g"]
      & A
      \arrow[d, "f"]
      \\ B
      \arrow[r, "s_{2}" swap]
      \arrow[rr, bend right=45, "\id_{B}" swap]
      & Y
      \arrow[r, "r_{2}" swap]
      & B
    \end{tikzcd}
  \end{displaymath}
  A fim de mostrarmos que $f$ também pertence a $\mathcal{L}$, considere o quadrado comutativo abaixo onde $p: P \to Q$ é um morfismo qualquer da classe $\mathcal{R}$.
  \begin{displaymath}
    \begin{tikzcd}
      A
      \arrow[r, "\alpha"]
      \arrow[d, "f" swap]
      & P
      \arrow[d, "p"]
      \\ B
      \arrow[r, "\beta" swap]
      & Q
    \end{tikzcd}
  \end{displaymath}
  A partir deste quadrado e do diagrama anterior produzimos o quadrado comutativo, o qual admite um levantamento $h: Y \to P$ já que $g \in \mathcal{L}$ por hipótese.
  \begin{displaymath}
    \begin{tikzcd}
      X
      \arrow[r, "\alpha \circ r_{1}"]
      \arrow[d, "g" swap]
      & P
      \arrow[d, "p"]
      \\ Y
      \arrow[r, "\beta \circ r_{2}" swap]
      \arrow[ru, dashed, "h" description]
      & Q
    \end{tikzcd}
  \end{displaymath}
  Afirmamos então que o morfismo $H: B \to P$ dado pela composição $H \coloneqq h \circ s_{2}$ define um levantamento para o quadrado inicial.
  De fato, por um lado temos
  \begin{displaymath}
    H \circ f = h \circ s_{2} \circ f = h \circ g \circ s_{1} = \alpha \circ r_{1} \circ s_{1} = \alpha \circ \id_{A} = \alpha,
  \end{displaymath}
  e por outro temos também
  \begin{displaymath}
    p \circ H = p \circ h \circ s_{2} = \beta \circ r_{2} \circ s_{2} = \beta \circ \id_{B} = \beta.
  \end{displaymath}

  Supondo ainda que $f$ seja um retrato de $g$, considere agora o caso em que $g$ pertence à classe $\mathcal{R}$.
  A fim de mostrarmos que $f$ também pertence a $\mathcal{R}$, considere o quadrado comutativo abaixo onde $j: M \to N$ é um morfismo qualquer da classe $\mathcal{L}$.
  \begin{displaymath}
    \begin{tikzcd}
      M
      \arrow[r, "\varphi"]
      \arrow[d, "j" swap]
      & A
      \arrow[d, "f"]
      \\ N
      \arrow[r, "\psi" swap]
      & B
    \end{tikzcd}
  \end{displaymath}
  A partir disso obtemos o quadrado comutativo abaixo, o qual admite um levantamento $h: N \to X$ pois $g$ pertence à classe $\mathcal{R}$ por hipótese.
  \begin{displaymath}
    \begin{tikzcd}
      M
      \arrow[r, "s_{1} \circ \varphi"]
      \arrow[d, "j" swap]
      & X
      \arrow[d, "g"]
      \\ N
      \arrow[r, "s_{2} \circ \psi" swap]
      \arrow[ru, dashed, "h" description]
      & Y
    \end{tikzcd}
  \end{displaymath}
  Afirmamos então que $H: N \to A$ definido por $H \coloneqq r_{1} \circ h$ é o levantamento procurado para o quadrado considerado inicialmente.
  De fato, por um lado temos as igualdades
  \begin{displaymath}
    H \circ j = r_{1} \circ h \circ j = r_{1} \circ s_{1} \circ \varphi = \id_{A} \circ \varphi = \varphi,
  \end{displaymath}
  e por outro temos também as igualdades
  \begin{displaymath}
    f \circ H = f \circ r_{1} \circ h = r_{2} \circ g \circ h = r_{2} \circ s_{2} \circ \psi = \id_{B} \circ \psi = \psi.
  \end{displaymath}

  \smallskip
  4. Suponha que $f: X_{1} \to Y_{1}$ pertença a $\mathcal{L}$ e que o quadrado comutativo abaixo seja um pushout em $\mathsf{C}$.
  \begin{displaymath}
    \begin{tikzcd}
      X_{1}
      \arrow[r, "\alpha"]
      \arrow[d, "f" swap]
      & X_{2}
      \arrow[d, "g"]
      \\ Y_{1}
      \arrow[r, "\beta" swap]
      & Y_{2}
    \end{tikzcd}
  \end{displaymath}
  Nosso objetivo é mostrar que então $g$ também pertence a $\mathcal{L}$, e com esse intuito consideramos o problema de levantamento abaixo, onde $p: P \to Q$ é um morfismo qualquer na classe $\mathcal{R}$.
  \begin{displaymath}
    \begin{tikzcd}
      X_{2}
      \arrow[r, "\varphi"]
      \arrow[d, "g" swap]
      & P
      \arrow[d, "p"]
      \\ Y_{2}
      \arrow[r, "\psi" swap]
      & Q
    \end{tikzcd}
  \end{displaymath}
  Note primeiro que o quadrado comutativo abaixo admite um levantamento $h: Y_{1} \to P$ pois $f \in \mathcal{L}$ por hipótese.
  \begin{displaymath}
    \begin{tikzcd}
      X_{1}
      \arrow[r, "\varphi \circ \alpha"]
      \arrow[d, "f" swap]
      & P
      \arrow[d, "p"]
      \\ Y_{1}
      \arrow[r, "\psi \circ \beta" swap]
      \arrow[ru, dashed, "h" description]
      & Q
    \end{tikzcd}
  \end{displaymath}
  A igualdade $h \circ f = \varphi \circ \alpha$ nos permite então usar a hipótese de que temos um pushout para obtermos um morfismo $H: Y_{2} \to P$ que faz comutar todo o diagrama abaixo.
  \begin{displaymath}
     \begin{tikzcd}
      X_{1}
      \arrow[r, "\alpha"]
      \arrow[d, "f" swap]
      & X_{2}
      \arrow[d, "g"]
      \arrow[rdd, bend left=30, "\varphi"]
      \\ Y_{1}
      \arrow[r, "\beta" swap]
      \arrow[rrd, bend right=30, "h" swap]
      & Y_{2}
      \arrow[rd, dashed, "H" description]
      \\ & & P
    \end{tikzcd}
  \end{displaymath}
  Afirmamos que o morfismo $H$ obtido dessa maneira é o levantamento procurado.
  Note primeiro que a igualdade $H \circ g = \varphi$ segue imediatamente da comutatividade acima.
  Já a igualdade $p \circ H = \psi$ requer um pouco mais de trabalho.
  Repare com carinho que estes dois morfismos fazem comutar o diagrama abaixo,
  \begin{displaymath}
    \begin{tikzcd}
      X_{1}
      \arrow[r, "\alpha"]
      \arrow[d, "f" swap]
      & X_{2}
      \arrow[d, "g"]
      \arrow[rdd, bend left=30, "p \circ \varphi"]
      \\ Y_{1}
      \arrow[r, "\beta" swap]
      \arrow[rrd, bend right=30, "\psi \circ \beta" swap]
      & Y_{2}
      \arrow[rd, shift left=1.2, "\psi"]
      \arrow[rd, shift right=1.2, "p \circ H" swap]
      \\ & & Q
    \end{tikzcd}
  \end{displaymath}
  mas como o quadrado que aparece neste diagrama é um pushout, sua propriedade universal garante que existe um único morfismo do tipo $Y_{2} \to Q$ que faça tudo comutar, portanto deve valer a igualdade $p \circ H = \psi$.

  A demonstração de que morfismos em $\mathcal{R}$ são preservados por pullbacks é novamente análoga.
  Suponha agora que o quadrado comutativo abaixo seja um pullback e que o morfismo $g: X_{2} \to Y_{2}$ seja pertence a $\mathcal{R}$.
  \begin{displaymath}
    \begin{tikzcd}
      X_{1}
      \arrow[r, "\alpha"]
      \arrow[d, "f" swap]
      & X_{2}
      \arrow[d, "g"]
      \\ Y_{1}
      \arrow[r, "\beta" swap]
      & Y_{2}
    \end{tikzcd}
  \end{displaymath}
  A fim de mostrarmos que $f$ também pertence a $\mathcal{R}$, consideramos o problema de levantamento abaixo, onde $j: M \to N$ é um morfismo qualquer pertencente à classe $\mathcal{L}$.
  \begin{displaymath}
    \begin{tikzcd}
      M
      \arrow[r, "\varphi"]
      \arrow[d, "j" swap]
      & X_{1}
      \arrow[d, "f"]
      \\ N
      \arrow[r, "\psi" swap]
      & Y_{1}
    \end{tikzcd}
  \end{displaymath}
  Colando estes dois quadrados obtemos o quadrado comutativo abaixo, o qual admite um levantamento $h: N \to X_{2}$ pois $g \in \mathcal{R}$ por hipótese.
  \begin{displaymath}
    \begin{tikzcd}
      M
      \arrow[r, "\alpha \circ \varphi"]
      \arrow[d, "j" swap]
      & X_{2}
      \arrow[d, "g"]
      \\ N
      \arrow[r, "\beta \circ \psi" swap]
      \arrow[ru, dashed, "h" description]
      & Y_{2}
    \end{tikzcd}
  \end{displaymath}
  A estratégia para obtermos um morfismo do tipo $N \to X_{1}$ é usarmos a propriedade universal do pullback.
  A igualdade $h \circ j = \alpha \circ \varphi$ implicada pela comutatividade acima diz que a ``camada externa'' do diagrama abaixo comuta, portanto segue da propriedade universal do pullback que existe um único morfismo do tipo $H: N \to X_{1}$ fazendo comutar o diagrama todo.
  \begin{displaymath}
    \begin{tikzcd}
      N
      \arrow[rd, dashed, "H" description]
      \arrow[rdd, bend right=30, "\psi" swap]
      \arrow[rrd, bend left=30, "h"]
      \\ & X_{1}
      \arrow[r, "\alpha"]
      \arrow[d, "f" swap]
      & X_{2}
      \arrow[d, "g"]
      \\ & Y_{1}
      \arrow[r, "\beta" swap]
      & Y_{2}
    \end{tikzcd}
  \end{displaymath}

  Resta mostrarmos que $H$ é o levantamento procurado para o quadrado comutativo inicial.
  A igualdade $f \circ H = \psi$ segue imediatamente da comutatividade acima.
  Analogamente ao que ocorreu no caso do pushout, a fim de mostrarmos que a igualdade $H \circ j = \varphi$ também vale, precisamos utilizar a unicidade na propriedade universal do pullback.
  Ambos os morfismos $H \circ j$ e $\varphi$ fazem comutar o diagrama abaixo,
  \begin{displaymath}
    \begin{tikzcd}
      M
      \arrow[rd, shift left=1.2, "H \circ j"]
      \arrow[rd, shift right=1.2, "\varphi" swap]
      \arrow[rrd, bend left=30, "\alpha \circ \varphi"]
      \arrow[rdd, bend right=30, "f \circ \varphi" swap]
      \\ & X_{1}
      \arrow[r, "\alpha"]
      \arrow[d, "f" swap]
      & X_{2}
      \arrow[d, "g"]
      \\ & Y_{1}
      \arrow[r, "\beta" swap]
      & Y_{2}
    \end{tikzcd}
  \end{displaymath}
  mas a propriedade universal do pullback garante a existência de um único morfismo do tipo $M \to X_{1}$ fazendo comutar o diagrama acima, de onde podemos concluímos enfim que a igualdade $H \circ j = \varphi$ deve ser verdadeira.

  \smallskip
  5. Suponha que $f: A_{1} \to B_{1}$ e $g: A_{2} \to B_{2}$ sejam dois morfismos que pertençam a $\mathcal{L}$.
  A fim de mostrarmos que seu coproduto $f \sqcup g: A_{1} \sqcup A_{2} \to B_{1} \sqcup B_{2}$ também pertence a $\mathcal{L}$, basta mostrarmos que o problema de levantamento abaixo admite uma solução, onde $p: X \to Y$ é um morfismo qualquer na classe $\mathcal{R}$.
  \begin{displaymath}
    \begin{tikzcd}
      A_{1} \sqcup A_{2}
      \arrow[r, "\alpha"]
      \arrow[d, "f \sqcup g" swap]
      & X
      \arrow[d, "p"]
      \\ B_{1} \sqcup B_{2}
      \arrow[r, "\beta" swap]
      & Y
    \end{tikzcd}
  \end{displaymath}

  Se $i_{1}: A_{1} \to A_{1} \sqcup A_{2}$, $i_{2}: A_{2} \to A_{1} \sqcup A_{2}$, $j_{1}: B_{1} \to B_{1} \sqcup B_{2}$ e $j_{2}: B_{2} \to B_{1} \sqcup B_{2}$ denotam as várias inclusões canônicas nos respectivos coprodutos, lembre-se que $f \sqcup g$ é por definição o único mapa do tipo $A_{1} \sqcup A_{2} \to B_{1} \sqcup B_{2}$ que faz comutar o diagrama abaixo.
  \begin{displaymath}
    \begin{tikzcd}[column sep=1.5cm]
      A_{1}
      \arrow[r, "f"]
      \arrow[d, "i_{1}" swap]
      & B_{1}
      \arrow[d, "j_{1}"]
      \\ A_{1} \sqcup A_{2}
      \arrow[r, dashed, "f \sqcup g" description]
      & B_{1} \sqcup B_{2}
      \\ A_{2}
      \arrow[u, "i_{2}"]
      \arrow[r, "g" swap]
      & B_{2}
      \arrow[u, "j_{2}" swap]
    \end{tikzcd}
  \end{displaymath}

  A partir do quadrado comutativo inicial obtemos o quadrado comutativo abaixo o qual admite um levantamento pois $f \in \mathcal{L}$ por hipótese.
  \begin{displaymath}
    \begin{tikzcd}
      A_{1}
      \arrow[r, "\alpha \circ i_{1}"]
      \arrow[d, "f" swap]
      & X
      \arrow[d, "p"]
      \\ B_{1}
      \arrow[r, "\beta \circ j_{1}" swap]
      \arrow[ru, dashed, "h_{1}" description]
      & Y
    \end{tikzcd}
  \end{displaymath}
  Veja que o quadrado é realmente comutativo pois
  \begin{displaymath}
    \beta \circ j_{1} \circ f = \beta \circ (f \sqcup g) \circ i_{1} = p \circ \alpha \circ i_{1}.
  \end{displaymath}
  Analogamento, temos também o quadrado comutativo abaixo que admite um levantamento pelo fato que $g \in \mathcal{L}$ também.
  \begin{displaymath}
    \begin{tikzcd}
      A_{2}
      \arrow[r, "\alpha \circ i_{2}"]
      \arrow[d, "g" swap]
      & X
      \arrow[d, "p"]
      \\ B_{2}
      \arrow[r, "\beta \circ j_{2}" swap]
      \arrow[ru, dashed, "h_{2}" description]
      & Y
    \end{tikzcd}
  \end{displaymath}

  Combinando estes dois levantamentos obtemos por meio da propriedade universal do coproduto um morfismo $\langle h_{1},h_{2} \rangle: B_{1} \sqcup B_{2} \to X$ o qual afirmamos ser o levantamento procurado para o quadrado original.
  Por um lado, a fim de mostrarmos a igualdade
  \begin{displaymath}
    p \circ \langle h_{1},h_{2} \rangle = \beta,
  \end{displaymath}
  pela unicidade na propriedade universal do coproduto basta mostrarmos as duas igualdades abaixo,
  \begin{displaymath}
    \begin{cases}
      p \circ \langle h_{1}, h_{2} \rangle \circ j_{1} = \beta \circ j_{1}, \\
      p \circ \langle h_{1}, h_{2} \rangle \circ j_{2} = \beta \circ j_{2};
    \end{cases}
  \end{displaymath}
  mas pelas propriedades dos morfismos envolvidos vemos que
  \begin{displaymath}
    p \circ \langle h_{1}, h_{2} \rangle \circ j_{1} =  p \circ h_{1} = \beta \circ j_{1},
  \end{displaymath}
  sendo que a segunda igualdade segue de uma sequência análoga de igualdades.

  Já a igualdade $\langle h_{1}, h_{2} \rangle \circ (f \sqcup g) = \alpha$ seguirá também da unicidade na propriedade do coproduto se mostrarmos as igualdades
  \begin{displaymath}
    \begin{cases}
      \langle h_{1}, h_{2} \rangle \circ (f \sqcup g) \circ i_{1} = \alpha \circ i_{1}, \\
      \langle h_{1}, h_{2} \rangle \circ (f \sqcup g) \circ i_{2} = \alpha \circ i_{2}.
    \end{cases}
  \end{displaymath}
  No primeiro caso basta vermos que
  \begin{displaymath}
    \langle h_{1}, h_{2} \rangle \circ (f \sqcup g) \circ i_{1} = \langle h_{1}, h_{2} \rangle \circ j_{1} \circ f = h_{1} \circ f = \alpha \circ i_{1},
  \end{displaymath}
  enquanto no segundo caso temos uma sequência análoga de igualdades.

  A demonstração para o produto de morfismos na classe $\mathcal{R}$ é dual.
  Suponha que $f: X_{1} \to Y_{1}$ e $g: X_{2} \to Y_{2}$ sejam dois morfismos da classe $\mathcal{R}$.
  A fim de mostrarmos que o produto $f \times g: X_{1} \times X_{2} \to Y_{1} \times Y_{2}$ também pertence a $\mathcal{R}$, consideramos um morfismo $i: A \to B$ qualquer na classe $\mathcal{L}$, e procuramos uma solução para o problema de levantamento mostrado abaixo.
  \begin{displaymath}
    \begin{tikzcd}
      A
      \arrow[r, "\alpha"]
      \arrow[d, "i" swap]
      & X_{1} \times X_{2}
      \arrow[d, "f \times g"]
      \\ B
      \arrow[r, "\beta" swap]
      & Y_{1} \times Y_{2}
    \end{tikzcd}
  \end{displaymath}
  Se $p_{1}: X_{1} \times X_{2} \to X_{1}$, $p_{2}: X_{1} \times X_{2} \to X_{2}$, $q_{1}: Y_{1} \times Y_{2} \to Y_{1}$ e $q_{2}: Y_{1} \times Y_{2} \to Y_{2}$ denotam as projeções canônicas dos respectivos produtos, lembre-se que o morfismo $f \times g$ é por definição o único morfismo do tipo $X_{1} \times X_{2} \to Y_{1} \times Y_{2}$ que faz comutar o diagrama abaixo.
  \begin{displaymath}
    \begin{tikzcd}[column sep=1.5cm]
      X_{1}
      \arrow[r, "f"]
      & Y_{1}
      \\ X_{1} \times X_{2}
      \arrow[u, "p_{1}"]
      \arrow[r, dashed, "f \times g" description]
      \arrow[d, "p_{2}" swap]
      & Y_{1} \times Y_{2}
      \arrow[u, "q_{1}" swap]
      \arrow[d, "q_{2}"]
      \\ Y_{1}
      \arrow[r, "g" swap]
      & Y_{2}
    \end{tikzcd}
  \end{displaymath}

  Usando o quadrado comutativo inicial obtemos os dois problemas de levantamento indicados abaixo, os quais possem soluções $h_{1}$ e $h_{2}$ pois $f$ e $g$ pertencem à classe $\mathcal{R}$ por hipótese.
  \begin{displaymath}
    \begin{tikzcd}
      A
      \arrow[r, "p_{1} \circ \alpha"]
      \arrow[d, "i" swap]
      & X_{1}
      \arrow[d, "f"]
      \\ B
      \arrow[r, "q_{1} \circ \beta" swap]
      \arrow[ru, dashed, "h_{1}" description]
      & Y_{1}
    \end{tikzcd}
    \qquad
    \begin{tikzcd}
      A
      \arrow[r, "p_{2} \circ \alpha"]
      \arrow[d, "i" swap]
      & X_{2}
      \arrow[d, "g"]
      \\ B
      \arrow[r, "q_{2} \circ \beta" swap]
      \arrow[ru, dashed, "h_{2}" description]
      & Y_{2}
    \end{tikzcd}
  \end{displaymath}

  Os morfismos $h_{1}$ e $h_{2}$ combinados induzem o morfismo $(h_{1},h_{2}): B \to X_{1} \times X_{2}$, o qual afirmamos ser uma solução para o problema de levantamento considerado inicialmente.
  De fato, temos as igualdades
  \begin{displaymath}
    p_{1} \circ (h_{1},h_{2}) \circ i = h_{1} \circ i = p_{1} \circ \alpha,
  \end{displaymath}
  e também as igualdades
  \begin{displaymath}
    p_{2} \circ (h_{1},h_{2}) \circ i = h_{2} \circ i = p_{2} \circ \alpha,
  \end{displaymath}
  e juntas essas duas implicam a igualdade $(h_{1},h_{2}) \circ i = \alpha$ desejada.
  Por fim, usando a comutatividade do diagrama que caracteriza o produto $q_{1} \times q_{2}$ vemos que também valem as igualdades
  \begin{displaymath}
    q_{1} \circ (f \times g) \circ (h_{1},h_{2}) = f \circ p_{1} \circ (h_{1},h_{2}) = f \circ h_{1} = q_{1} \circ \beta
  \end{displaymath}
  e as igualdades
  \begin{displaymath}
    q_{2} \circ (f \times g) \circ (h_{1},h_{2}) = g \circ p_{2} \circ (h_{1},h_{2}) = g \circ h_{2} = q_{2} \circ \beta;
  \end{displaymath}
  e juntas essas igualdades implicam $(f \times g) \circ (h_{1},h_{2}) = \beta$ como desejado.
\end{proof}

Agora mostramos como podemos usar a estrutura de uma categoria modelo para construirmos enfim exemplos um pouco mais concretos de sistemas de fatoração fracos.

\begin{prop}\label{prop:categoria_modelo_induz_sistemas_de_fatoracao}
  Suponha que $(\mathsf{M},\mathcal{W},\mathcal{C},\mathcal{F})$ seja uma categoria modelo.
  As seguintes afirmações são verdadeiras:
  \begin{enumerate}
  \item $(\mathcal{C} \cap \mathcal{W},\mathcal{F})$ define um sistema de fatoração fraco em $\mathsf{M}$;
    
  \item $(\mathcal{C},\mathcal{F} \cap \mathcal{W})$ define um sistema de fatoração fraco em $\mathsf{M}$.
  \end{enumerate}
\end{prop}

\begin{proof}
  1. O axioma de fatoração (M5) garante que, dado qualquer morfismo $f: X \to Y$ em $\mathsf{M}$, existe uma cofibração trivial $j: X \overset{\sim}{\cofib} \widetilde{Y}$ e uma fibração $q: \widetilde{Y} \fib Y$ que fatoram $f$ como indicado abaixo.
  \begin{displaymath}
    \begin{tikzcd}
      X
      \arrow[r, tail, "j", "\sim" {swap}]
      \arrow[rr, bend right=30, "f" swap]
      & \widetilde{Y}
      \arrow[r, two heads, "q"]
      & Y
    \end{tikzcd}
  \end{displaymath}
  Isso significa que o par de classes de morfismos $(\mathcal{C} \cap \mathcal{W}, \mathcal{F})$ satisfaz a primeira condição de um sistema de fatoração fraco.

  Resta mostrarmos que $\mathcal{C} \cap \mathcal{W}$ e $\mathcal{F}$ definem um ao outro por meio de propriedades de levantamento.
  Note que o axioma de levantamento (M4) garante que toda fibração satisfaz a propriedade de levantamento à direita com relação à classe $\mathcal{C} \cap \mathcal{W}$ das cofibrações triviais.
  Reciprocamente, suponha que $f: X \to Y$ satisfaça a propriedade de levantamento à direita com relação à classe das cofibrações triviais.
  Queremos mostrar que isso garante que $f$ seja uma fibração, e como o únicos axiomas que temos para lidar com fibrações são os de retração e fatoração, não é surpresa que a estratégia da demonstração seja combinarmos esses dois axiomas com a hipótese de levantamento sobre $f$.

  Usando o axioma de fatoração podemos obter uma cofibração trivial $j: X \overset{\sim}{\cofib} \widetilde{Y}$ e uma fibração $q: \widetilde{Y} \fib Y$ tais que $f = q \circ j$.
  Essa igualdade também pode ser expressa em termos do quadrado comutativo abaixo, o qual admite um levantamento $h: \widetilde{Y} \to X$ já que $f$ satisfaz a propriedade de levantamento à direita com relação às cofibrações triviais.
  \begin{displaymath}
    \begin{tikzcd}
      X
      \arrow[r, "\id_{X}"]
      \arrow[d, tail, "j" {swap}, "\sim" {sloped}]
      & X
      \arrow[d, "f"]
      \\ \widetilde{Y}
      \arrow[r, two heads, "q" swap]
      \arrow[ru, dashed, "h" description]
      & Y
    \end{tikzcd}
  \end{displaymath}
  Podemos organizar estes morfismos todos no diagrama comutativo mostrado abaixo,
  \begin{displaymath}
    \begin{tikzcd}
      X
      \arrow[r, tail, "j", "\sim" {swap}]
      \arrow[d, "f" swap]
      \arrow[rr, bend left=45, "\id_{X}"]
      & \widetilde{Y}
      \arrow[r, "h"]
      \arrow[d, two heads, "q"]
      & X
      \arrow[d, "f"]
      \\ Y
      \arrow[r, "\id_{Y}" swap]
      \arrow[rr, bend right=45, "\id_{Y}" swap]
      & Y
      \arrow[r, "\id_{Y}" swap]
      & Y
    \end{tikzcd}
  \end{displaymath}
  o qual mostra que $f$ é uma retração da fibração $q$ na categoria de setas $\Arr(\mathsf{M})$ e, portanto, uma fibração também.

  Vejamos agora a descrição das cofibrações triviais em termos das fibrações.
  Note inicialmente que o axioma de levantamento (M4) mais uma vez já garante que toda cofibração trivial satisfaz a propriedade de levantamento à esquerda com relação à classe de fibrações, a questão aqui é mostrar que essa propriedade de levantamento \emph{caracteriza} as cofibrações triviais.
  Suponha então que $f: X \to Y$ satisfaça a propriedade de levantamento à esquerda com relação às fibrações.
  Como a classe $\mathcal{C} \cap \mathcal{W}$ é fechada por retrações, pois tanto $\mathcal{C}$ quanto $\mathcal{W}$ o são, basta mostrarmos que $f$ é uma retração de uma cofibração trivial.
  Novamente consideramos a fatoração abaixo em termos de uma cofibração trivial seguida de uma fibração.
  \begin{displaymath}
    \begin{tikzcd}
      X
      \arrow[r, "j" {swap}, "\sim"]
      \arrow[rr, bend left=30, "f"]
      & \widetilde{Y}
      \arrow[r, two heads, "q" swap]
      & Y
    \end{tikzcd}
  \end{displaymath}

  A condição $f = q \circ j$ pode ser expressa em termos do quadrado comutativo abaixo, e tal quadrado admite um levantamento $h: Y \to \widetilde{Y}$ pois $f$ por hipótese satisfaz a condição de levantamento à esquerda com relação às fibrações.
  \begin{displaymath}
    \begin{tikzcd}
      X
      \arrow[d, "f" swap]
      \arrow[r, tail, "j", "\sim" {swap}]
      & \widetilde{Y}
      \arrow[d, two heads, "q"]
      \\ Y
      \arrow[r, "\id_{Y}" swap]
      \arrow[ru, dashed, "h" description]
      & Y
    \end{tikzcd}
  \end{displaymath}
  Os morfismos todos em questão podem então ser combinados no diagrama comutativo abaixo que expressa $f$ como uma retração da cofibração trivial $j$ como queríamos.
  \begin{displaymath}
    \begin{tikzcd}
      X
      \arrow[r, "\id_{X}"]
      \arrow[rr, bend left=45, "\id_{X}"]
      \arrow[d, "f" swap]
      & X
      \arrow[d, tail, "j", "\sim" {swap, sloped}]
      \arrow[r, "\id_{X}"]
      & X
      \arrow[d, "f"]
      \\ Y
      \arrow[r, "h" swap]
      \arrow[rr, bend right=45, "\id_{Y}" swap]
      & \widetilde{Y}
      \arrow[r, two heads, "q" swap]
      & Y
    \end{tikzcd}
  \end{displaymath}

  \smallskip
  2. Novamente pelo axioma de fatoração (M5) podemos fatorar um morfismo $f: X \to Y$ qualquer de $\mathsf{M}$ da forma indicada abaixo.
  \begin{equation}
    \label{eq:fatoracao_cofib_fib_triv}
    \begin{tikzcd}
      X
      \arrow[r, tail, "i" swap]
      \arrow[rr, bend left=45, "f"]
      & \widehat{X}
      \arrow[r, two heads, "p" {swap}, "\sim"]
      & Y
    \end{tikzcd}
  \end{equation}
  Em outras palavras, o par $(\mathcal{C}, \mathcal{F} \cap \mathcal{W})$ satisfaz a primeira condição na definição de sistema de fatoração fraco.
  Resta verificarmos que as classes se definem mutuamente em termos de propriedades de levantamento.

  Inicialmente, sabemos do axioma de levantamento (M4) que toda cofibração satisfaz a propriedade de levantamento à esquerda com relação à classe das fibrações triviais.
  Suponha agora que $f: X \to Y$ satisfaça tal propriedade de levantamento, e vamos então mostrar que $f$ é necessariamente uma cofibração trivial também, o que como nos dois casos anteriores será feito por meio do axioma de retração.
  A condição $f = p \circ i$ pode ser expressa pela comutatividade do quadrado abaixo, e a condição de levantamento sobre $f$ nos permite obter o mapa $h: Y \to \widehat{X}$ indicado.
  \begin{displaymath}
    \begin{tikzcd}
      X
      \arrow[r, tail, "i"]
      \arrow[d, "f" swap]
      & \widehat{X}
      \arrow[d, two heads, "p", "\sim" {swap, sloped}]
      \\ Y
      \arrow[r, "\id_{Y}" swap]
      \arrow[ru, dashed, "h" description]
      & Y
    \end{tikzcd}
  \end{displaymath}
  Usando os morfismos à disposição construímos o diagrama abaixo expressando $f$ como uma retração da cofibração $i$, de onde concluímos que $f$ é também uma cofibração.
  \begin{displaymath}
    \begin{tikzcd}
      X
      \arrow[r, "\id_{X}"]
      \arrow[d, "f" swap]
      \arrow[rr, bend left=45, "\id_{X}"]
      & X
      \arrow[r, "\id_{X}"]
      \arrow[d, tail, "i"]
      & X
      \arrow[d, "f"]
      \\ Y
      \arrow[r, "h" swap]
      \arrow[rr, bend right=45, "\id_{Y}" swap]
      & \widehat{X}
      \arrow[r, two heads, "p" {swap}, "\sim"]
      & Y
    \end{tikzcd}
  \end{displaymath}

  Por fim, resta apenas mostrarmos que as fibrações triviais são precisamente os morfismos satisfazendo a condição de levantamento à direita com relação às cofibrações.
  O axioma de levantamento (M4) nos diz que toda fibração trivial satisfaz tal propriedade de levantamento, a questão é justamente a recíproca.
  Suponha então que $f: X \to Y$ satisfaça essa propriedade também, e seguindo a notação da fatoração em \eqref{eq:fatoracao_cofib_fib_triv}, considere o quadrado comutativo abaixo, e o mapa $h: \widehat{X} \to X$ obtido da hipótese feita sobre $f$.
  \begin{displaymath}
    \begin{tikzcd}
      X
      \arrow[r, "\id_{X}"]
      \arrow[d, tail, "i" swap]
      & X
      \arrow[d, "f"]
      \\ \widehat{X}
      \arrow[r, two heads, "p" {swap}, "\sim"]
      \arrow[ru, dashed, "h" description]
      & Y
    \end{tikzcd}
  \end{displaymath}
  Como o leitor já há muito deve ter previsto, combinando todos esses ingredientes obtemos o diagrama comutativo abaixo exibindo $f$ como uma retração da fibração trivial $p$, o que nos permite concluir enfim que $f$ é também uma fibração trivial.
  \begin{displaymath}
    \begin{tikzcd}
      X
      \arrow[r, tail, "i"]
      \arrow[d, "f" swap]
      \arrow[rr, bend left=45, "\id_{X}"]
      & \widehat{X}
      \arrow[r, "h"]
      \arrow[d, two heads, "p", "\sim" {swap,sloped}]
      & X
      \arrow[d, "f"]
      \\ Y
      \arrow[r, "\id_{Y}" swap]
      \arrow[rr, bend right=45, "\id_{Y}" swap]
      & Y
      \arrow[r, "\id_{Y}" swap]
      & Y
    \end{tikzcd} \qedhere
  \end{displaymath}
\end{proof}

\begin{obs}
  É comum dizer que os resultados da \cref{prop:categoria_modelo_induz_sistemas_de_fatoracao} mostram que a estrutura de uma categoria modelo é \emph{sobredeterminada}, pois o conhecimento de duas das três classes de morfismos $\mathcal{W}$, $\mathcal{C}$ e $\mathcal{F}$ nos permite determinar totalmente a terceira.
  De fato, se conhecemos as classes $\mathcal{W}$ e $\mathcal{C}$, então conhecemos também a classe das cofibrações triviais $\mathcal{C} \cap \mathcal{W}$, e da proposição anterior sabemos que $\mathcal{F}$ pode ser descrita então como a classe dos morfismos que satisfazem a propriedade de levantamento à direita cmo relação a $\mathcal{C} \cap \mathcal{W}$.
  Analogamente, se conhecemos as classes $\mathcal{W}$ e $\mathcal{F}$, então conhecemos a classe das fibrações triviais $\mathcal{F} \cap \mathcal{W}$, e pelo resultado anterior recuperamos $\mathcal{C}$ como a classe dos morfismos que satisfazem a propriedade de levantamento à esquerda com relação a $\mathcal{F} \cap \mathcal{W}$.
  Fica a pergunta se as equivalências fracas podem ser recuperadas a partir das fibrações e cofibrações.
  A resposta é que em certo sentido sim, como mostra o próximo resultado, embora mais precisamente tenhamos que conhecer as fibrações e cofibrações triviais.

  Essa propriedade de sobredeterminação é por vezes utilizada para \emph{definirmos} uma estrutura modelo completa partindo apenas de alguns componentes de sua estrutura.
  Um exemplo onde esse fenômenos ocorre, e que abordaremos em detalhes mais tarde, é no estudo das categorias modelo \emph{cofibrantemente geradas}, onde uma estrutura modelo é gerada a partir de uma coleção de morfismos que queremos considerar como cofibrações triviais.
  Um exemplo concreto é a estrutura modelo usual na categoria $\mathsf{Top}$, onde a classe de cofibrações triviais inicial usada para gerar o resto das estruturas é o dos mapas $D^{n} \hookrightarrow D^{n} \times I$ incluindo o disco $D^{n}$ na base inferior do cilindro associado.
\end{obs}

Aplicando as propriedades gerais demonstradas na \cref{prop:propriedades_fatoracao_fraca} aos sistemas explícitos construídos na \cref{prop:categoria_modelo_induz_sistemas_de_fatoracao} obtemos as propriedades abaixo a respeito dos vários tipos de morfismos em uma categoria modelo.

\begin{corol}\label{corol:propriedades_de_preservacao_categoria_modelo}
  Em uma categoria modelo qualquer valem as seguintes propriedades:
  \begin{enumerate}
  \item Um isomorfismo é simultaneamento uma cofibração trivial e uma fibração trivial.
    
  \item Cofibrações, cofibrações triviais, fibrações e fibrações triviais são todas preservadas por composições.
    
  \item Cofibrações e cofibrações triviais são preservadas por pushouts, enquanto fibrações e fibrações triviais são preservadas por pullbacks.
    
  \item Cofibrações e cofibrações triviais são preservadas por coprodutos, enquanto fibrações e fibrações triviais são preservadas por produtos.
  \end{enumerate}
\end{corol}

Por fim, temos o resultado simples abaixo mostrando como ``detectar'' equivalências fracas usando outras classes de morfismos.

\begin{lema}\label{lema:detectando_equivalencia_fraca}
  Em uma categoria modelo, um morfismo $f: X \to Y$ é uma equivalência fraca se, e somente se, ele pode ser fatorado como uma cofibração trivial seguida de uma fibração trivial.
\end{lema}

\begin{proof}
  Uma das implicações é simples: se $f$ pode ser fatorado na forma descrita no enunciado, então $f$ é em particular a composição de duas equivalências fracas e, portanto, uma equivalência fraca também.
  Reciprocamente, se $f$ é uma equivalência fraca, pelo axioma de fatoração podemos encontrar uma cofibração trivial $i: X \overset{\sim}{\cofib} \widehat{X}$ e uma fibração $p: \widehat{X} \fib Y$ tais que $f = p \circ i$.
  Ora, como $f$ e $i$ são equivalências fracas, segue do axioma 2-de-3 que $p$ também o é, ou seja, $p$ é uma fibração trivial, e a fatoração $f = p \circ i$ exibe $f$ como uma cofibração trivial seguida de uma fibração trivial.
\end{proof}

Encerramos essa subseção mostrando como os axiomas de uma categoria modelo também pode ser enunciados apenas em termos de sistemas de fatoração fracos.

\begin{prop}
  Sejam $\mathsf{M}$ uma categoria bicompleta e $\mathcal{W}$, $\mathcal{C}$ e $\mathcal{F}$ três classes de morfismos em $\mathsf{M}$.
  Suponha ainda que a classe $\mathcal{W}$ satisfaça as seguintes condições:
  \begin{enumerate}
  \item[(i)] $\mathcal{W}$ contém todos os isomorfismos;
    
  \item[(ii)] $\mathcal{W}$ satisfaz a propriedade 2-de-3.
  \end{enumerate}
  Nessas condições, as seguintes afirmações são equivalentes:
  \begin{enumerate}
  \item $(\mathsf{M},\mathcal{W},\mathcal{C},\mathcal{F})$ é uma categoria modelo;
    
  \item $(\mathcal{W} \cap \mathcal{C},\mathcal{F})$ e $(\mathcal{C},\mathcal{W} \cap \mathcal{F})$ são sistemas de fatoração fracos em $\mathsf{M}$.
  \end{enumerate}
\end{prop}

\begin{proof}
  A implicação $1 \implies 2$ é precisamente o conteúdo da \cref{prop:categoria_modelo_induz_sistemas_de_fatoracao}.

  Reciprocamente, suponha que os pares de classes $(\mathcal{W} \cap \mathcal{C},\mathcal{F})$ e $(\mathcal{C},\mathcal{W} \cap \mathcal{F})$ definam sistemas de fatoração fracos em $\mathsf{M}$, e vamos verificar então que os axiomas de uma categoria modelo são satisfeitos.
  
  O axioma de completude (M1) e o axioma 2-de-3 (M2) seguem das hipóteses impostas no enunciado.
  Os axiomas de lavantamento (M4) e de fatoração (M5) seguem diretamente das duas propriedades que definem um sistema de fatoração fraco.
  Resta apenas verificarmos então o axioma de retração (M3).
  Note que a validade deste para as classes de cofibrações e fibrações segue de uma aplicação direta do item (c) da \cref{prop:propriedades_fatoracao_fraca} aos dois sistemas de fatoração fracos em questão.
  Falta então mostrarmos que a classe das equivalências fracas é também fechada por retrações, e isso será um tanto mais emocionante.
  Suponha que o morfismo $f: X \to Y$ seja retração de uma equivalência fraca $g: A \to B$ como mostrado no diagrama abaixo.
  \begin{displaymath}
    \begin{tikzcd}
      X
      \arrow[d, "f" swap]
      \arrow[r, "s_{1}"]
      \arrow[rr, bend left=45, "\id_{X}"]
      & A
      \arrow[d, "g"]
      \arrow[r, "r_{1}"]
      & X
      \arrow[d, "f"]
      \\ Y
      \arrow[r, "s_{2}" swap]
      \arrow[rr, bend right=45, "\id_{Y}" swap]
      & B
      \arrow[r, "r_{2}" swap]
      & Y
    \end{tikzcd}
  \end{displaymath}
  Suponha inicialmente que $f$ seja uma fibração.
  O morfismo $g$ pode ser fatorado na forma $g = p \circ i$, onde $i: A \overset{\sim}{\cofib} C$ é uma cofibração trivial, e $p: C \fib B$ é uma fibração.
  Note que, na verdade, a propriedade 2-de-3 garante que $p$ é também uma equivalência fraca e, portanto, uma fibração trivial.
  Nosso primeiro objetivo é mostrar que $f$ é uma retração de $p$, o que garante que $f$ seja também uma fibração trivial pelo item (c) da \cref{prop:propriedades_fatoracao_fraca} e em particular uma equivalência fraca.
  Como temos a sequência de igualdades
  \begin{displaymath}
    r_{2} \circ p \circ i = r_{2} \circ g = f \circ r_{1},
  \end{displaymath}
  o quadrado indicado abaixo é comutativo, e o fato de $i$ ser uma cofibração trivial e $f$ ser uma fibração garantem a existência do levantamento $r: C \to X$.
  \begin{displaymath}
    \begin{tikzcd}
      A
      \arrow[r, "r_{1}"]
      \arrow[d, tail, "i" {swap}, "\sim" {sloped}]
      & X
      \arrow[d, two heads, "f"]
      \\ C
      \arrow[r, "r_{2} \circ p" swap]
      \arrow[ru, dashed, "r" description]
      & Y
    \end{tikzcd}
  \end{displaymath}
  Usando esse morfismo $r$ construímos o diagrama comutativo abaixo, o qual expressa $f$ como uma retração da fibração trivial $p$, conforme desejado.
  \begin{displaymath}
    \begin{tikzcd}
      X
      \arrow[r, "i \circ s_{1}"]
      \arrow[d, two heads, "f" swap]
      \arrow[rr, bend left=45, "\id_{X}"]
      & C
      \arrow[r, "r"]
      \arrow[d, two heads, "p", "\sim" {swap, sloped}]
      & X
      \arrow[d, two heads, "p"]
      \\ Y
      \arrow[r, "s_{2}" swap]
      \arrow[rr, bend right=45, "\id_{Y}" swap]
      & B
      \arrow[r, "r_{2}" swap]
      & Y
    \end{tikzcd}
  \end{displaymath}

  Voltemos agora ao caso geral onde $f: X \to Y$ é um morfismo qualquer.
  Sabemos da definição de um sistema de fatoração fraca que podemos escrever $f = q \circ j$, onde $j: X \overset{\sim}{\cofib} Z$ é uma cofibração trivial, e $q: Z \fib Y$ é uma fibração.
  Lembrando que a categoria $\mathsf{C}$ em questão é por hipótese bicompleta, podemos em particular formar o pushout do diagrama
  \begin{displaymath}
    \begin{tikzcd}
      X
      \arrow[r, "s_{1}"]
      \arrow[d, tail, "j" {swap}, "\sim" {sloped}]
      & A
      \\ Z
    \end{tikzcd}
  \end{displaymath}
  de forma a obtermos o quadrado comutativo mostrado abaixo.
  Note que a propriedade (d) da \cref{prop:propriedades_fatoracao_fraca} aplicado ao sistema de fatoração fraco $(\mathcal{C} \cap \mathcal{W},\mathcal{F})$ garante que o morfismo $J$ que aparece no pushout abaixo seja de fato uma cofibração trivial.
  \begin{displaymath}
    \begin{tikzcd}
      X
      \arrow[r, "s_{1}"]
      \arrow[d, tail, "j" {swap}, "\sim" {sloped}]
      & A
      \arrow[d, tail , "J", "\sim" {swap,sloped}]
      \\ Z
      \arrow[r, "S" swap]
      & P
    \end{tikzcd}
  \end{displaymath}
  Uma primeira aplicação da propriedade universal do pushout fornece um morfismo $R: P \to Z$ fazendo comutar o diagram abaixo.
  \begin{displaymath}
    \begin{tikzcd}
       X
      \arrow[r, "s_{1}"]
      \arrow[d, tail, "j" {swap}, "\sim" {sloped}]
      & A
      \arrow[d, tail , "J", "\sim" {swap,sloped}]
      \arrow[rdd, bend left=30, "j \circ r_{1}"]
      \\ Z
      \arrow[r, "S" swap]
      \arrow[rrd, bend right=30, "\id_{Z}" swap]
      & P
      \arrow[rd, dashed, "R" description]
      \\ & & Z
    \end{tikzcd}
  \end{displaymath}
  Agora, uma segunda aplicação da propriedade universal do pushout nos fornece o morfismo $Q: P \to B$ fazendo comutar o diagrama abaixo.
  \begin{displaymath}
    \begin{tikzcd}
      X
      \arrow[r, "s_{1}"]
      \arrow[d, tail, "j" {swap}, "\sim" {sloped}]
      & A
      \arrow[d, tail , "J", "\sim" {swap,sloped}]
      \arrow[rdd, bend left=30, "g", "\sim" {swap,sloped}]
      \\ Z
      \arrow[r, "S" swap]
      \arrow[rrd, bend right=30, "s_{2} \circ q" swap]
      & P
      \arrow[rd, dashed, "Q" description]
      \\ & & B
    \end{tikzcd}
  \end{displaymath}

  Os morfismos envolvendo o pushout construídos acima podem então ser combinados no diagrammaso comutativo mostrado abaixo.
  \begin{displaymath}
    \begin{tikzcd}
      X
      \arrow[r, "s_{1}"]
      \arrow[d, tail, "j" {swap}, "\sim" {sloped}]
      \arrow[rr, bend left=45, "\id_{X}"]
      \arrow[dd, bend right=45, "f" swap]
      & A
      \arrow[d, tail, "J", "\sim" {swap,sloped}]
      \arrow[r, "r_{1}"]
      & X
      \arrow[d, tail, "j", "\sim" {swap,sloped}]
      \\ Z
      \arrow[r, "S"]
      \arrow[d, two heads, "q" swap]
      & P
      \arrow[d, "Q", "\sim" {swap,sloped}]
      \arrow[r, "R"]
      & Z
      \arrow[d, two heads, "q"]
      \\ Y
      \arrow[r, "s_{2}" swap]
      \arrow[rr, bend right=45, "\id_{Y}" swap]
      & B
      \arrow[r, "r_{2}" swap]
      & Y
    \end{tikzcd}
  \end{displaymath}
  Veja que a propriedade 2-de-3 garante que o morfismo $Q$ é realmente uma equivalência fraca como indicado acima, pois $J$ e $g$ o são, e por construção temos a igualdade $Q \circ J = g$.

  Temos que justificar ainda de alguma forma a comutatividade do quadrado inferior direito, mas vamos assumir isso momentaneamente e dar continuidade ao argumento.
  Como o morfismo $R$ por construção satisfaz a igualdade $R \circ S = \id_{Z}$, os dois quadrados comutativos inferiores exibem a fibração $q: Z \fib Y$ como uma retração da equivalência fraca $Q: P \overset{\sim}{\to} B$, e o caso demonstrado anteriormente nos permite concluir então que $q$ é na verdade uma fibração trivial, sendo em particular uma equivalência fraca também.
  Ora, segue da igualdade $f = q \circ j$ que $f$ é a composição de duas equivalências fracas, sendo portanto uma equivalência fraca também conforme queríamos mostrar.

  A fim de fecharmos a lacuna restante no argumento acima, note que a comutatividade do quadrado inferior direito segue imediatamente da unicidade na propriedade universal do pushout, já que os morfismos $q \circ R,\, r_{2} \circ Q: P \to Y$ ambos fazem comutar o diagrama abaixo.
  \begin{displaymath}
    \begin{tikzcd}[column sep=1.25cm]
       X
      \arrow[r, "s_{1}"]
      \arrow[d, tail, "j" {swap}, "\sim" {sloped}]
      & A
      \arrow[d, tail , "J", "\sim" {swap,sloped}]
      \arrow[rdd, bend left=30, "f \circ r_{1}"]
      \\ Z
      \arrow[r, "S" swap]
      \arrow[rrd, bend right=30, "q" swap]
      & P
      \arrow[rd, shift left=1.2, "q \circ R"]
      \arrow[rd, shift right=1.2, "r_{2} \circ Q" swap]
      \\ & & Y
    \end{tikzcd} \qedhere
  \end{displaymath}
\end{proof}

\subsection{Fatorações funtoriais}

Nessa subseção formalizamos a noção de uma fatoração funtorial de morfismos em uma categoria.
O conceito de fatoração funtorial é um tanto peculiar: na teoria ele oferece uma vantagem para descrevermos certas construções, mas na prática o processo de fatoração funtorial pode não ser necessariamente o mais simples ou o mais vantajoso em uma dada situação.

Antes de definirmos uma fatoração propriamente dita, precisamos introduzir um outro tipo de categoria de setas relacionado à categoria $\Arr(\mathsf{C})$ introduzida anteriormente.
Dada então uma categoria $\mathsf{C}$ qualquer, considere a categoria $\mathsf{C}^{[2]}$ cujos objetos são pares $(g,f)$ de \emph{setas componíveis em $\mathsf{C}$}, ou seja, morfismos satisfazendo a condição $\dom g = \cod f$, situação esta que também representaremos pelo diagrama da forma abaixo.
\begin{displaymath}
  \begin{tikzcd}
    X
    \arrow[d, "f" swap]
    \\ Y
    \arrow[d, "g" swap]
    \\ Z
  \end{tikzcd}
\end{displaymath}

Os morfismos em $\mathsf{C}^{[2]}$, como esperado, são triplas de morfismos horizontais fazendo comutar os dois quadrados evidentes.
Mais precisamente, dados dois pares de setas componíveis $(g,f)$ e $(g',f')$, um morfismo do tipo $(g,f) \to (g',f')$ em $\mathsf{C}^{[2]}$ é dado por uma tripla de morfismos $(\alpha: \dom f \to \dom f',\, \beta: \cod f \to \cod f',\, \gamma: \cod g \to \cod g')$ fazendo comutar o diagrama a seguir.
\begin{displaymath}
  \begin{tikzcd}
    X
    \arrow[r, "\alpha"]
    \arrow[d, "f" swap]
    & X'
    \arrow[d, "f'"]
    \\ Y
    \arrow[r, "\beta"]
    \arrow[d, "g" swap]
    & Y'
    \arrow[d, "g'"]
    \\ Z
    \arrow[r, "\gamma" swap]
    & Z'
  \end{tikzcd}
\end{displaymath}
A composição de morfismos é análoga àquela existente na categoria $\Arr(\mathsf{C})$: colamos os dois pares de quadrados comutativos ao longo da ``aresta'' comum para obtermos um terceiro par de quadrados comutativos representando a composição dos dois morfismos.

\begin{obs}
  A notação $\mathsf{C}^{[2]}$ escolhida para denotar a categoria introduzida acima não é por acaso.
  Lembre-se que todo conjunto parcialmente ordenado pode ser encarado como uma categoria.
  Quando fazemos isso com o conjunto parcialmente ordenado $\{0 \leq 1 \leq 2\}$, obtemos uma categoria comumente denotada por $[2]$, e que pode ser visualizada pictoricamente da forma abaixo (omitindo os morfismos identidade).
  \begin{displaymath}
    \begin{tikzcd}
      0
      \arrow[r, "0 \leq 1"]
      \arrow[rr, bend right=25, "0 \leq 2" swap]
      & 1
      \arrow[r, "1 \leq 2"]
      & 2      
    \end{tikzcd}
  \end{displaymath}

  A variante da categoria de setas que introduzimos anteriormente é então \emph{precisamente} a categoria de funtores do tipo $[2] \to \mathsf{C}$, o que justifica o uso da notação $\mathsf{C}^{[2]}$.
  Note também que, se no lugar de $[2]$ considerarmos a categoria $[1]$ obtida a partir do conjunto parcialmente ordenado $0 \leq 1$, a qual está representada abaixo,
  \begin{displaymath}
    \begin{tikzcd}
      0
      \arrow[r, "0 \leq 1"]
      & 1
    \end{tikzcd}
  \end{displaymath}
  então a categoria $\mathsf{C}^{[1]}$ formada por funtores do tipo $[1] \to \mathsf{C}$ é exatamente a categoria de setas usual $\Arr(\mathsf{C})$ introduzida anteriormente.
\end{obs}

Lembremos que a categoria de setas $\Arr(\mathsf{C})$ vem equipada com os funtores domínio e codomínio $\dom,\, \cod: \Arr(\mathsf{C}) \to \mathsf{C}$.
Analogamente, a categoria de setas componíveis $\mathsf{C}^{[2]}$ vem equipada com três funtores $d_{0},\, d_{1},\, \comp: \mathsf{C}^{[2]} \to \Arr(\mathsf{C})$ que vamos agora definir.
O funtor $d_{0}$ nos objetos registra o primeiro morfismo de um par componível, ou seja, $d_{0}(g,f) \coloneqq f$, o que podemos visualizar da forma indicada a seguir.
\begin{displaymath}
  \begin{tikzcd}
    X
    \arrow[d, "f" swap]
    \\ Y
    \arrow[d, "g" swap]
    \\ Z
  \end{tikzcd}
  \quad
  \overset{d_{0}}{\rightsquigarrow}
  \quad
  \begin{tikzcd}
    X
    \arrow[d, "f" swap]
    \\ Y
  \end{tikzcd}
\end{displaymath}
Já nos morfismos, dado um morfismo $(\alpha,\beta,\gamma): (g,f) \to (g',f')$, temos por definição $d_{0}(\alpha,\beta,\gamma) \coloneqq (\alpha,\beta)$.
Visualmente, $d_{0}$ seleciona o primeiro quadrado comutativo do par de quadrados comutativos que representa o morfismo $(\alpha,\beta,\gamma)$ em $\mathsf{C}^{[2]}$.
\begin{displaymath}
  \begin{tikzcd}
    X
    \arrow[d, "f" swap]
    \arrow[r, "\alpha"]
    & X'
    \arrow[d, "f'"]
    \\ Y
    \arrow[d, "g" swap]
    \arrow[r, "\beta"]
    & Y'
    \arrow[d, "g'"]
    \\ Z
    \arrow[r, "\gamma" swap]
    & Z'
  \end{tikzcd}
  \quad \overset{d_{0}}{\rightsquigarrow} \quad
  \begin{tikzcd}
    X
    \arrow[d, "f" swap]
    \arrow[r, "\alpha"]
    & X'
    \arrow[d, "f'"]
    \\ Y
    \arrow[r, "\beta" swap]
    & Y'
  \end{tikzcd}
\end{displaymath}
O funtor $d_{1}$ é análogo: nos objetos ele seleciona o segundo morfismo do par componível, logo $d_{1}(g,f) \coloneqq g$, e nos morfismos ele seleciona o segundo quadrado comutativo, de forma que $d_{1}(\alpha,\beta,\gamma) \coloneqq (\beta,\gamma)$.
Já o funtor $\comp: \mathsf{C}^{[2]} \to \Arr(\mathsf{C})$, como o nome muito sugere, é dado exatamente pela composição das setas componíveis.
Dessa forma, nos objetos temos $\comp(g,f) \coloneqq g \circ f$,
\begin{displaymath}
  \begin{tikzcd}
    X
    \arrow[d, "f" swap]
    \\ Y
    \arrow[d, "g" swap]
    \\ Z
  \end{tikzcd}
  \quad \overset{\comp}{\rightsquigarrow} \quad
  \begin{tikzcd}
    X
    \arrow[d, "g \circ f" swap]
    \\ Z
  \end{tikzcd}
\end{displaymath}
enquanto nos morfismos temos $\comp(\alpha,\beta,\gamma) \coloneqq (\alpha,\gamma)$.
\begin{displaymath}
  \begin{tikzcd}
    X
    \arrow[d, "f" swap]
    \arrow[r, "\alpha"]
    & X'
    \arrow[d, "f'"]
    \\ Y
    \arrow[d, "g" swap]
    \arrow[r, "\beta"]
    & Y'
    \arrow[d, "g'"]
    \\ Z
    \arrow[r, "\gamma" swap]
    & Z'
  \end{tikzcd}
  \quad \overset{\comp}{\rightsquigarrow} \quad
  \begin{tikzcd}
    X
    \arrow[d, "g \circ f" swap]
    \arrow[r, "\alpha"]
    & X'
    \arrow[d, "g' \circ f'"]
    \\ Z
    \arrow[r, "\gamma" swap]
    & Z'
  \end{tikzcd}
\end{displaymath}

Tendo introduzido a categoria $\mathsf{C}^{[2]}$, podemos enfim falar de fatoraçõs funtoriais.

\begin{defin}\label{defin:fatoracao_funtorial}
  Uma \textbf{fatoração funtorial} em uma categoria $\mathsf{C}$é  um funtor $\fac: \Arr(\mathsf{C}) \to \mathsf{C}^{[2]}$ que é uma seção do funtor de composição $\comp: \mathsf{C}^{[2]} \to \Arr(\mathsf{C})$, ou seja, que satisfaz a equação $\comp \circ \fac = \id_{\mathsf{\Arr(\mathsf{C})}}$.
\end{defin}

Vamos procurar entender o significado prático da existência de fatorações funtoriais em uma categoria.
O funtor $\fac$ associa a um morfismo $f: X \to Z$ em $\mathsf{C}$ um par de morfismos $(f_{1},f_{0})$ cuja composição $f_{1} \circ f_{0}$ deve ser igual ao morfismo $f$.
Essa última igualdade garante então que tenhamos $\dom f_{0} = \dom f = X$ e $\cod f_{1} = \cod f = Z$.
Se denotarmos $Y \coloneqq \cod f_{0} = \dom f_{1}$, temos a situação indicada abaixo.
\begin{displaymath}
  \begin{tikzcd}
    X
    \arrow[d, "f" swap]
    \\ Z
  \end{tikzcd}
  \quad \overset{\fac}{\rightsquigarrow} \quad
  \begin{tikzcd}
    X
    \arrow[d, "f_{0}" swap]
    \arrow[dd, bend left=45, "f_{1} \circ f_{0} = f"]
    \\ Y
    \arrow[d, "f_{1}" swap]
    \\ Z
  \end{tikzcd}
\end{displaymath}
Dado outro morfismo $f': X' \to Z'$, e um morfismo $(\alpha: X \to X',\, \gamma: Z \to Z')$ de $f$ para $f'$ em $\Arr(\mathsf{C})$, a fatoração funtorial $\fac$ associa a esse morfismo do tipo $(f_{1},f_{0}) \to (f_{1}',f_{0}')$ em $\mathsf{C}^{[2]}$, ou seja, uma tripla de morfismos $(\varphi: X \to X',\, \psi: Y \to Y',\, \theta: Z \to Z')$ fazendo comutar os dois quadrados adjacentes.
Note que, como $\fac$ é uma seção de $\comp$, devemos ter $(\alpha,\beta) = \comp(\fac(\alpha,\beta)) = \comp(\varphi,\psi,\theta) = (\varphi,\theta)$, de forma que os morfismos $\alpha$ e $\beta$ são em certo sentido preservados na fatoração funtorial, conforme indicado abaixo.
\begin{displaymath}
  \begin{tikzcd}
    X
    \arrow[d, "f" swap]
    \arrow[r, "\alpha"]
    & X'
    \arrow[d, "f'"]
    \\ Z
    \arrow[r, "\beta" swap]
    & Z'
  \end{tikzcd}
  \quad \overset{\fac}{\rightsquigarrow} \quad
  \begin{tikzcd}
    X
    \arrow[r, "\alpha"]
    \arrow[d, "f_{0}" swap]
    & X'
    \arrow[d, "f_{0}'"]
    \\ Y
    \arrow[d, "f_{1}" swap]
    \arrow[r, "\psi"]
    & Y'
    \arrow[d, "f_{1}'"]
    \\ Z
    \arrow[r, "\beta" swap]
    & Z'
  \end{tikzcd}
\end{displaymath}

Por fim, introduzimos a noção de um sistema de fatoração fraca funtorial, a noção que será de fato de nosso interesse no estudo de categorias modelo.

\begin{defin}\label{defin:fatoracao_fraca_funtorial}
  Um sistema de fatoração fraco $(\mathcal{L},\mathcal{R})$ uma categoria $\mathsf{C}$ é dito \textbf{funtorial} se existe um funtor $\fac: \Arr(\mathsf{C}) \to \mathsf{C}^{[2]}$ satisfazendo ass seguintes condições:
  \begin{enumerate}
  \item $\comp \circ \fac = \id_{\Arr(\mathsf{C})}$;
    
  \item $d_{0}(\fac(f)) \in \mathcal{L}$ e $d_{1}(\fac(f)) \in \mathcal{R}$ para todo $f \in \Arr(\mathsf{C})$.
  \end{enumerate}
\end{defin}

A primeira condição acima diz simplesmente que o funtor $\fac$ define uma fatoração funtorial em $\mathsf{C}$, mas a segunda condição impõe que os morfismos que aparecem nessa fatoração interajam adequadamente com o sistema de fatoração fraco existente: o primeiro morfismo da fatoração deve necessariamente pertence à classe $\mathcal{L}$, enquanto o segundo deve necessariamente pertencer à classe $\mathcal{R}$.
\begin{displaymath}
  \begin{tikzcd}
    X
    \arrow[d, "f" swap]
    \\ Z
  \end{tikzcd}
  \quad \overset{\fac}{\rightsquigarrow} \quad
  \begin{tikzcd}
    X
    \arrow[d, "f_{0} \in \mathcal{L}" swap]
    \arrow[dd, bend left=45, "f_{1} \circ f_{0} = f"]
    \\ Y
    \arrow[d, "f_{1} \in \mathcal{R}" swap]
    \\ Z
  \end{tikzcd}
\end{displaymath}

Muitos autores exigem que os sistemas de fatoração fracos $(\mathcal{C} \cap \mathcal{W},\mathcal{F})$ e $(\mathcal{C},\mathcal{F} \cap \mathcal{W})$ de uma categoria modelo sejam funtoriais no sentido acima.
Isso nem sempre é estritamente necessário, mas é algo que facilita a nossa vida em alguns momentos.
É importante comentar que, embora nem toda estrutura modelo admita fatorações funtoriais, uma grande parte delas admite, incluindo aquelas obtidas por meio do famigerado Argumento do Objeto Pequeno que estudaremos mais adiante.

%%% Local Variables:
%%% mode: latex
%%% TeX-master: "../main"
%%% End:
