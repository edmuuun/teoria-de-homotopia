\section{Objetos fibrantes e cofibrantes}

Nessa seção introduzimos o conceito de objetos cofibrantes e fibrantes.
Estas duas classes de objetos em uma categoria modelo serão de extrema importância em diferentes pontos da teoria, como na construção de um modelo mais simples para a localização de uma categoria modelo e na descrição de uma noção puramente categórica de homotopia.

\begin{defin}
  Sejam $\mathsf{M}$ uma categoria modelo, $\varnothing \in \mathsf{M}$ um objeto inicial qualquer, e $*$ um objeto final qualquer.\footnote{O fato de $\mathsf{M}$ ser cocompleta e completa garante a existência de objetos iniciais e finais.} Um objeto $X \in \mathsf{M}$ é dito \textbf{cofibrante} se o morfismo único $!_{X}: \varnothing \to X$ é uma cofibração, e é dito \textbf{fibrante} se o morfismo único $!_{X}: X \to *$ é uma fibração.
  Quando um objeto é simultaneamente cofibrante e fibrante, dizemos também que ele é \textbf{bifibrante}.
\end{defin}

\begin{obs}\label{obs:cofibrante_fibrante_independe_objeto_inicial_final}
  Veja que a noção de objeto cofibrante não depende do objeto inicial $\varnothing \in \mathsf{M}$ fixado acima.
  De fato, se $\varnothing$ e $\varnothing'$ são dois objetos iniciais, $X$ é um objeto qualquer, podemos considerar o diagrama comutativo abaixo onde todos os morfismos são aqueles dados pela propriedade que caracteriza um objeto inicial.
  \begin{displaymath}
    \begin{tikzcd}
      \varnothing
      \arrow[rd, "!_{X}"]
      \arrow[dd, bend right=15, "!_{\varnothing'}" swap]
      \\ & X
      \\ \varnothing'
      \arrow[ru, "!'_{X}" swap]
      \arrow[uu, bend right=15, "!'_{\varnothing}" swap]
    \end{tikzcd}
  \end{displaymath}
  Como objetos iniciais possuem apenas a identidade como automorfismo, $!_{\varnothing'}$ e $!'_{\varnothing}$ são isomorfismos, sendo em particular cofibrações.
  Segue disso que $!_{X}: \varnothing \to X$ é uma cofibração se, e somente se, $!'_{X}: \varnothing' \to X$ também o é.

  É claro que um raciocínio dual mostra que a noção de um objeto ser fibrante também independe do objeto final considerado na categoria modelo.
\end{obs}

Vamos mostrar agora algumas propriedade simples a respeito de objetos cofibrantes e fibrantes.

\begin{lema}\label{lema:props_obj_cofib_fib}
  Em uma categoria modelo $\mathsf{M}$ qualquer valem as seguintes propriedades:
  \begin{enumerate}
  \item[(a)] Se $X$ é um objeto cofibrante, e $i: X \cofib Y$ é uma cofibração, então $Y$ também é cofibrante.
    
  \item[(b)] Se $Y$ é um objeto fibrante, e $p: X \fib Y$ é uma fibração, então $X$ também é fibrante.
  \end{enumerate}
\end{lema}

\begin{proof}
  (a) Queremos mostrar que o morfismo $!_{Y}: \varnothing \to Y$ é uma cofibração.
  Ora, como esse morfismo é \emph{único}, certamente vale a igualdade $!_{Y} = f \circ !_{X}$, e como $!_{X}: \varnothing \to X$ é por hipótese uma cofibração, $!_{Y}$ é a composição de duas cofibrações e, portanto uma cofibração também.

  \smallskip
  (b) Queremos mostrar que o morfismo $!_{X}: X \to *$ é uma fibração.
  Ora, como esse morfismo é \emph{único}, certamente vale a igualdade $!_{X} = !_{Y} \circ p$, e como $!_{Y}: Y \to *$ é por hipótese uma fibração, vemos que $!_{X}$ é a composição de duas fibrações e, portanto, uma fibração também.
\end{proof}

Veremos mais adiante que muitas construções categóricas usuais só se comportam de forma homotopicamente adequada quando aplicadas a objetos que são fibrantes ou cofibrantes.
Assim, para que possamos realizar tais construções com objetos quaisquer, precisamos antes \emph{substituí-los} por outros que sejam fibrantes ou cofibrantes, uma noção que tornamos precisa na definição abaixo.

\begin{defin}
  Sejam $\mathsf{M}$ uma categoria modelo e $X \in \mathsf{M}$ um objeto qualquer.
  \begin{enumerate}
  \item Uma \textbf{substituição cofibrante} para $X$ é um par $(X_{c},\varphi)$, onde $X_{c}$ é um objeto cofibrante, e $\varphi: X_{c} \overset{\sim}{\to} X$ é uma equivalência fraca.
    Quando $\varphi$ é também uma fibração, ou seja, quando $\varphi$ é uma fibração trivial, dizemos que o par $(X_{c},\varphi)$ define uma substituição cofibrante forte.
    
  \item Uma \textbf{substituição fibrante} para $X$ é um par $(X_{f},\psi)$, onde $X_{f}$ é um objeto fibrante, e $\psi: X \overset{\sim}{\to} X_{f}$ é uma equivalência fraca.
    Quando $\psi$ é também uma cofibração, ou seja, quando $\psi$ é uma cofibração trivial, dizemos que o par $(X_{f},\psi)$ define uma substituição fibrante \textbf{forte}.
  \end{enumerate}
\end{defin}

\begin{obs}
  Em alguns trabalhos da Literatura, é comum se referir a uma aproximação cofibrante forte $X_{c} \overset{\sim}{\fib} X$ como uma \emph{aproximação cofibrante fibrante}.
  Perceba então que nessa terminologia o adjetivo cofibrante se refere ao objeto $X_{c}$, equanto o adjetivo fibrante se refere ao morfismo $X_{c} \to X$.
  Analogamente, uma aproximação fibrante forte $X_{f} \overset{\sim}{\cofib} X$ é também chamada de uma \emph{aproximação fibrante cofibrante}.
  Não usarei essa terminologia absolutamente horrível em nenhum outro lugar dessas notas, e na verdade espero muito que meu cérebro a esqueça o quanto antes.
\end{obs}

Naturalmente somos levados a nos indagar quanto à existência e à unicidade de substituições cofibrantes e fibrantes, e são essas duas questões que iremos investigar agora.

\begin{prop}
  As seguintes afirmações são válidas em uma categoria modelo $\mathsf{M}$ qualquer:
  \begin{enumerate}
  \item Todo objeto possui uma aproximação cofibrante forte e também uma aproximação fibrante forte.
    
  \item Se $\varphi: X_{c} \overset{\sim}{\fib} X$ é uma aproximação cofibrante forte, dada qualquer outra aproximação cofibrante $\psi: X_{c}' \overset{\sim}{\to} X$, existe uma equivalência fraca $\theta: X_{c}' \overset{\sim}{\to} X_{c}$.
    
  \item Se $\varphi: X \overset{\sim}{\cofib} X_{f}$ é uma aproximação fibrante forte, dada qualquer outra aproximação fibrante $\psi: X \overset{\sim}{\to} X_{f}'$, existe uma equivalência fraca $\theta: X_{f} \overset{\sim}{\to} X_{f}'$.
  \end{enumerate}
\end{prop}

\begin{proof}
  1. Dado um objeto qualquer $X \in \mathsf{M}$, usando o axioma de fatoração (M5) podemos fatorar o morfismo único $!_{X}: \varnothing \to X$ como uma cofibração seguinda de uma fibração trivial conforme indicado abaixo.
  \begin{displaymath}
    \begin{tikzcd}
      \varnothing
      \arrow[r, tail, "!_{X_{c}}" swap]
      \arrow[rr, bend right=45, "!_{X}" swap]
      & X_{c}
      \arrow[r, two heads, "\varphi" {swap}, "\sim"]
      & X
    \end{tikzcd}
  \end{displaymath}
  Note então que o objeto $X_{c}$ é cofibrante e que o par $(X_{c},\varphi)$ define uma aproximação cofibrante forte para $X$.

  Usando novamente o axioma de fatoração podemos reescrever o morfismo único $!_{X}: X \to *$ como uma cofibração trivial seguida de uma fibração conforme indicado abaixo.
  \begin{displaymath}
    \begin{tikzcd}
      X
      \arrow[r, tail, "\sim", "\psi" {swap}]
      \arrow[rr, bend right=45, "!_{X}" swap]
      & X_{f}
      \arrow[r, two heads, "!_{X_{f}}" swap]
      & *
    \end{tikzcd}
  \end{displaymath}
  Basta notar então que o objeto $X_{f}$ é fibrante e que o par $(X_{f},\psi)$ define uma aproximação fibrante forte para $X$.

  \smallskip
  2. Usando os morfismos dados podemos montar o quadrado comtuativo mostrado abaixo.
  \begin{displaymath}
    \begin{tikzcd}
      \varnothing
      \arrow[r, tail, "!_{X_{c}}"]
      \arrow[d, tail, "!_{X_{c}'}" swap]
      & X_{c}
      \arrow[d, two heads, "\sim" {swap, sloped}, "\varphi"]
      \\ X_{c}'
      \arrow[r, "\sim", "\psi" {swap}]
      & X
    \end{tikzcd}
  \end{displaymath}
  Note que o quadrado é realmente comutativo, já que as composições $\varphi \circ !_{X_{c}}$ e $\psi \circ !_{X_{c}'}$ ambas definem morfismos do tipo $\varnothing \to X$, mas só existe um único morfismo desse tipo.
  Como $!_{X_{c}'}$ é uma cofibração, e $\varphi$ é uma fibração trivial, usando o axioma de levantamento (M4) podemos obter um morfismo diagonal $\theta: X_{c}' \to X_{c}$ fazendo comutar o diagrama todo.
  Em particular, temos a igualdade $\varphi \circ \theta = \psi$, e como $\varphi$ e $\psi$ são equivalências fracas, segue do axioma 2-de-3 que $\theta$ também é uma equivalência fraca.

  \smallskip
  3. A demonstração é análoga ao que fizemos no item 2.
  Usando os morfismos dados montamos o quadrado comutativo abaixo,
  \begin{displaymath}
    \begin{tikzcd}
      X
      \arrow[r, "\psi", "\sim" {swap}]
      \arrow[d, tail, "\varphi" {swap}, "\sim" {sloped}]
      & X_{f}'
      \arrow[d, two heads, "!_{X_{f}'}"]
      \\ X_{f}
      \arrow[r, "!_{X_{f}}" swap]
      & *
    \end{tikzcd}
  \end{displaymath}
  o qual admite um levantamento $\theta: X_{f} \to X_{f}'$ já que $\varphi$ é uma cofibraçao trivial, e $\psi$ é uma fibração.
  Esse levantamento satisfaz a equação $\theta \circ \varphi = \psi$, portanto pelo axioma 2-de-3 concluímos que $\theta$ é uma equivalência fraca.
\end{proof}

Os itens 2 e 3 acima são o mais próximo que temos de uma unicidade para aproximações cofibrantes ou fibrantes.
Podemos resumi-lo dizendo que toda aproximação cofibrante ou fibrante é fracamente equivalente a outra aproximação que é forte.
Isso certamente não significa que as duas aproximações sejam \emph{isomorfas}, já que equivalências fracas não são invertíveis em geral.
É verdade, entretanto, que aproximações diferentes se tornam isomorfas ao passarmos para a categoria homotópica de $\mathsf{M}$, já que esta é obtida por localização na classe das equivalêncas fracas.

\begin{obs}[Substituições funtoriais]
  \label{obs:substituicao_cofibrante_fibrante_funtorial}
  Suponha que a categoria de modelos $\mathsf{M}$ admita fatorações funtoriais, ou seja, que os dois sistemas de fatoração fracos $(\mathcal{C} \cap \mathcal{W},\mathcal{F})$ e $(\mathcal{C},\mathcal{F} \cap \mathcal{W})$ sejam funtoriais, sendo $\fac,\, \fac': \Arr(\mathsf{M}) \to \mathsf{M}^{[2]}$ os respectivos funtores de fatoração.
  Usando tais fatorações funtoriais podemos obter funtores de substituição cofibrante e fibrante.
  Considere inicialmente o funtor $\init: \mathsf{M} \to \Arr(\mathsf{M})$ que associa a um objeto $X$ o morfismo único $!_{X}: \varnothing \to X$, e que associa a um morfismo $f: X \to Y$ o par de morfismos $(\id_{\varnothing},f)$, o qual faz comutar o diagrama abaixo
  \begin{displaymath}
    \begin{tikzcd}
      \varnothing
      \arrow[d, "!_{X}" swap]
      \arrow[r, "\id_{\varnothing}"]
      & \varnothing
      \arrow[d, "!_{Y}"]
      \\ X
      \arrow[r, "f" swap]
      & Y
    \end{tikzcd}
  \end{displaymath}
  e portanto define um morfismo do tipo $!_{X} \to !_{Y}$ na categoria de setas $\Arr(\mathsf{M})$.

  Usando o funtor $\init$ introdudizo acima, definirmos um funtor $\mathrm{subcof}: \mathsf{M} \to \Arr(\mathsf{M})$ por meio da composição mostrada abaixo.
  \begin{displaymath}
    \begin{tikzcd}
      \mathsf{M}
      \arrow[r, "\init"]
      \arrow[rrr, dashed, bend right=20, "\mathrm{subcof}" swap]
      & \Arr(\mathsf{M})
      \arrow[r, "\fac'"]
      & \mathsf{M}^{[2]}
      \arrow[r, "d_{1}"]
      & \Arr(\mathsf{M})
    \end{tikzcd}
  \end{displaymath}
  Veja que, ao aplicarmos o funtor de fatoração $\fac'$ ao morfismo $!_{X}: \varnothing \to X$, obtemos uma fatoração dada por uma cofibração $!_{X_{c}}: \varnothing \to X_{c}$ seguida de uma fibração trivial $p_{X}: X_{c} \to X$ conforme indicado abaixo.
  \begin{displaymath}
    \begin{tikzcd}
      \varnothing
      \arrow[rr, "!_{X}"]
      \arrow[rd, tail, "!_{X_{c}}" swap]
      & & X
      \\ & X_{c}
      \arrow[ru, two heads, "p_{X}" {swap}, "\sim" {sloped}]
    \end{tikzcd}
  \end{displaymath}
  Note então que o par $(X_{c},p_{X})$ define uma substituição cofibrante forte para $X$.
  A funtorialidade da construção pode ser entendida da seguinte maneira: se $f: X \to Y$ é um morfismo em $\mathsf{M}$, existe também um morfismo induzido $f_{c}: X_{c} \to Y_{c}$ entre as substituições cofibrantes que faz comutar o diagrama abaixo.
  \begin{equation}\label{eq:funtorialidade_substituicao_cofibrante}
    \begin{tikzcd}
      X_{c}
      \arrow[r, dashed, "f_{c}"]
      \arrow[d, two heads, "p_{X}" {swap}, "\sim" {sloped}]
      & Y_{c}
      \arrow[d, two heads, "p_{Y}", "\sim" {swap,sloped}]
      \\ X
      \arrow[r, "f" swap]
      & Y
    \end{tikzcd}
  \end{equation}

  Denotando por $\mathsf{M}_{c}$ a subcategoria plena de $\mathsf{M}$ gerada pelos objetos cofibrantes, temos um funtor
  \begin{displaymath}
    \dom \circ \mathrm{subcof}: \mathsf{M} \to \mathsf{M}_{c},
  \end{displaymath}
  e a comutatividade do quadrado acima diz que a coleção de morfismos $(p_{X})_{X \in \mathsf{M}}$ define uma transformação natural do tipo $i \circ (\dom \circ \mathrm{subcof}) \Rightarrow \id_{\mathsf{M}}$, enquanto a coleção $(p_{X})_{X \in \mathsf{M}_{c}}$ define uma transformação natural do tipo $(\dom \circ \mathrm{subcof}) \circ i \Rightarrow \id_{\mathsf{M}_{c}}$, onde $i: \mathsf{M}_{c} \to \mathsf{M}$ é o funtor de inclusão.

  Existe também um processo análogo para obtermos substituições fibrantes funtoriais em uma categoria de modelos equipada com fatorações funtoriais.
  Consideramos inicialmente o funtor $\mathrm{term}: \mathsf{M} \to \Arr(\mathsf{M})$ que associa a cada objeto $X \in \mathsf{M}$ o morfismo terminal $!_{X}: X \to *$ e que associa a cada morfismo $f: X \to Y$ o par $(f,\id_{*})$, o qual faz comutar o quadrado indicado abaixo e, portanto, define um morfismo do tipo $!_{X} \to !_{Y}$ na categoria de setas.
  \begin{displaymath}
    \begin{tikzcd}
      X
      \arrow[r, "f"]
      \arrow[d, "!_{X}" swap]
      & Y
      \arrow[d, "!_{Y}"]
      \\ *
      \arrow[r, "\id_{*}" swap]
      & *
    \end{tikzcd}
  \end{displaymath}

  Consideramos então o funtor $\mathrm{subfib}: \mathsf{M} \to \Arr(\mathsf{M})$ dado pela composição indicada abaixo.
  \begin{displaymath}
    \begin{tikzcd}
      \mathsf{M}
      \arrow[r, "\mathrm{term}"]
      \arrow[rrr, dashed, bend right=20, "\mathrm{subfib}" swap]
      & \Arr(\mathsf{M})
      \arrow[r, "\mathrm{fac}"]
      & \mathsf{M^{[2]}}
      \arrow[r, "d_{0}"]
      & \Arr(\mathsf{M})
    \end{tikzcd}
  \end{displaymath}
  Tal funtor associa a cada objeto $X \in \mathsf{M}$ uma cofibração trivial $j_{X}: X \overset{\sim}{\cofib} X_{f}$ que faz comutar o diagrama abaixo,
  \begin{displaymath}
    \begin{tikzcd}
      X
      \arrow[rr, "!_{X}"]
      \arrow[rd, tail, "j_{X}" {swap}, "\sim" {sloped}]
      & & *
      \\ & X_{f}
      \arrow[ru, two heads, "!_{X_{f}}" swap]
    \end{tikzcd}
  \end{displaymath}
  portanto o par $(X_{f},j_{X})$ define uma substituição fibrante forte para $X$.
  Além disso, o funtor $\mathrm{subfib}$ em questão associa também a cada morfismo $g: X \to Y$ em $\mathsf{M}$ um morfismo correspondente $g_{f}: X_{f} \to Y_{f}$ entre as substituições fibrantes que faz comutar o quadrado abaixo.
  \begin{equation}\label{eq:funtorialidade_sub_fib}
    \begin{tikzcd}
      X
      \arrow[d, tail, "j_{X}" {swap}, "\sim" {sloped}]
      \arrow[r, "g"]
      & Y
      \arrow[d, tail, "j_{Y}", "\sim" {swap,sloped}]
      \\ X_{f}
      \arrow[r, dashed, "g_{f}" swap]
      & Y_{f}
    \end{tikzcd}
  \end{equation}

  Denotando por $\mathsf{M}_{f}$ a subcategoria plena de $\mathsf{M}$ gerada pelos objeos fibrantes, temos o funtor composto $\cod \circ \mathrm{subfib}: \mathsf{M} \to \mathsf{M}_{f}$, e a comutatividade do diagrama \eqref{eq:funtorialidade_sub_fib} diz que a coleção de morfismos $(j_{X})_{X \in \mathsf{M}}$ define uma transformação natural do tipo $\id_{\mathsf{M}} \Rightarrow i \circ (\cod \circ \mathrm{subfib})$, enquanto a coleção de morfismos $(j_{X})_{X \in \mathsf{M}_{f}}$ define uma transformação natural $\id_{\mathsf{M}_{f}} \Rightarrow (\cod \circ \mathrm{subfib}) \circ i$, onde $i: \mathsf{M}_{f} \to \mathsf{M}$ denota o morfismo de inclusão.
\end{obs}

%%% Local Variables:
%%% mode: latex
%%% TeX-master: "../main"
%%% End:
