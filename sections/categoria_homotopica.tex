\section{A categoria homotópica}

Nessa seção utilizamos as ferramentas homotópicas desenvolvidas na seção anterior para descrevermos uma construção explícita da localização de uma categoria de modelos na sua classe de equivalências fracas.
O arcabouço técnico que sustenta tal construção é uma versão do celebrado Teorema de Whitehead válido para categorias de modelos que relaciona equivalências fracas e equivalências homotópicas.

\begin{lema}[Levantamentos a menos de homotopia]
  Em uma categoria de modelos $\mathsf{M}$, suponha que $p: X \fib Y$ seja uma fibração, $B$ seja um objeto cofibrante, e $f: B \to Y$ seja um morfismo qualquer.
  Se existe um morfismo $f': B \to X$ tal que a composição $p \circ f'$ seja homotópica à esquerda a $f$, então existe também um morfismo $\widetilde{f}: B \to X$ satisfazendo a igualdade $p \circ \widetilde{f} = f$.
\end{lema}

\begin{proof}
  Seja $H: \Cyl(B) \to Y$ uma homotopia à esquerda de $p \circ f'$ para $f$, e considere o quadrado comutativo abaixo.
  \begin{displaymath}
    \begin{tikzcd}
      B
      \arrow[d, tail, "\sim" {sloped}, "i_{0}" {swap}]
      \arrow[r, "f'"]
      & X
      \arrow[d, two heads, "p"]
      \\ \Cyl(B)
      \arrow[r, "H" swap]
      & Y
    \end{tikzcd}
  \end{displaymath}
  Veja que o fato de $i_{0}$ ser uma cofibração trivial segue da hipótese de cofibrância de $B$ e do \cref{lema:props_obj_cilindro}.
  Aplicando o axioma de levantamento obtemos um morfismo $\widetilde{H}: \Cyl(B) \to X$ conforme indicado no diagrama comutativo abaixo.
  \begin{displaymath}
    \begin{tikzcd}
      B
      \arrow[d, tail, "\sim" {sloped}, "i_{0}" {swap}]
      \arrow[r, "f'"]
      & X
      \arrow[d, two heads, "p"]
      \\ \Cyl(B)
      \arrow[r, "H" swap]
      \arrow[ru, dashed, "\widetilde{H}" description]
      & Y
    \end{tikzcd}
  \end{displaymath}
  Note então que $\widetilde{f} \coloneqq H \circ i_{1}: B \to X$ é o morfismo procurado pois
  \begin{displaymath}
    p \circ \widetilde{f} = p \circ (\widetilde{H} \circ i_{1}) = (p \circ \widetilde{H}) \circ i_{1} = H \circ i_{1} = f. \qedhere
  \end{displaymath}
\end{proof}

%%% Local Variables:
%%% mode: latex
%%% TeX-master: "../main"
%%% End:
