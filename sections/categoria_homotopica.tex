\section{A categoria homotópica}

Nessa seção utilizamos as ferramentas homotópicas desenvolvidas na seção anterior para descrevermos uma construção explícita da localização de uma categoria de modelos na sua classe de equivalências fracas.
O arcabouço técnico que sustenta tal construção é uma versão do celebrado Teorema de Whitehead válido para categorias de modelos que relaciona equivalências fracas e equivalências homotópicas.

\begin{lema}[Levantamentos a menos de homotopia]
  \label{lema:levantamento_a_menos_de_homotopia}
  Em uma categoria de modelos $\mathsf{M}$, suponha que $p: X \fib Y$ seja uma fibração, $B$ seja um objeto cofibrante, e $f: B \to Y$ seja um morfismo qualquer.
  Se existe um morfismo $f': B \to X$ tal que a composição $p \circ f'$ seja homotópica à esquerda a $f$, então existe também um morfismo $\widetilde{f}: B \to X$ satisfazendo a igualdade $p \circ \widetilde{f} = f$.
\end{lema}

\begin{proof}
  Seja $H: \Cyl(B) \to Y$ uma homotopia à esquerda de $p \circ f'$ para $f$, e considere o quadrado comutativo abaixo.
  \begin{displaymath}
    \begin{tikzcd}
      B
      \arrow[d, tail, "\sim" {sloped}, "i_{0}" {swap}]
      \arrow[r, "f'"]
      & X
      \arrow[d, two heads, "p"]
      \\ \Cyl(B)
      \arrow[r, "H" swap]
      & Y
    \end{tikzcd}
  \end{displaymath}
  Veja que o fato de $i_{0}$ ser uma cofibração trivial segue da hipótese de cofibrância de $B$ e do \cref{lema:props_obj_cilindro}.
  Aplicando o axioma de levantamento obtemos um morfismo $\widetilde{H}: \Cyl(B) \to X$ conforme indicado no diagrama comutativo abaixo.
  \begin{displaymath}
    \begin{tikzcd}
      B
      \arrow[d, tail, "\sim" {sloped}, "i_{0}" {swap}]
      \arrow[r, "f'"]
      & X
      \arrow[d, two heads, "p"]
      \\ \Cyl(B)
      \arrow[r, "H" swap]
      \arrow[ru, dashed, "\widetilde{H}" description]
      & Y
    \end{tikzcd}
  \end{displaymath}
  Note então que $\widetilde{f} \coloneqq H \circ i_{1}: B \to X$ é o morfismo procurado pois
  \begin{displaymath}
    p \circ \widetilde{f} = p \circ (\widetilde{H} \circ i_{1}) = (p \circ \widetilde{H}) \circ i_{1} = H \circ i_{1} = f. \qedhere
  \end{displaymath}
\end{proof}

\begin{lema}[Extensão a menos de homotopia]
  \label{lema:extensao_a_menos_de_homotopia}
  Em uma categoria de modelos $\mathsf{M}$, suponha que $i: A \cofib B$ seja uma cofibração, $X$ seja um objeto fibrante, e $f: A \to X$ seja um morfismo qualquer.
  Se existe um morfismo $f': B \to X$ tal que $f' \circ i \simeq_{r} f$, então existe também um morfismo $\widetilde{f}: B \to X$ tal que $\widetilde{f} \circ i = f$.
\end{lema}

\begin{proof}
  Seja $h: A \to P(X)$ uma homotopia à direita de $f' \circ i $ para $f$.
  Sabemos do \cref{lema:props_obj_de_caminhos} que o morfismo $p_{0}: P(X) \to X$ é uma fibração trivial, e aplicando então o axioma de levantamento obtemos o morfismo $H: B \to P(X)$ mostrado no diagrama comutativo abaixo.
  \begin{displaymath}
    \begin{tikzcd}
      A
      \arrow[d, tail, "i" swap]
      \arrow[r, "h"]
      & P(X)
      \arrow[d, two heads, "p_{0}", "\sim" {swap,sloped}]
      \\ B
      \arrow[r, "f'" swap]
      \arrow[ru, dashed, "H" description]
      & X
    \end{tikzcd}
  \end{displaymath}
  Basta notar agora que $\widetilde{f} \coloneqq p_{1} \circ H$ é o morfismo procurado pois temos a sequência de igualdades
  \begin{displaymath}
    \widetilde{f} \circ i = (p_{1} \circ H) \circ i = p_{1} \circ (H \circ i) = p_{1} \circ h = f. \qedhere
  \end{displaymath}
\end{proof}

Vimos na seção anterior que, se $B$ é um objeto cofibrante, então qualquer morfismo $\beta: X \to Y$ dá origem a uma função $[B,\beta]_{\ell}: [B,X]_{\ell} \to [B,Y]_{\ell}$.
Inicialmente estudamos condições suficientes para que a função de pushforward
\begin{displaymath}
  [B,p]_{\ell}: [B,X]_{\ell} \to [B,Y]_{\ell}
\end{displaymath}
nas classes de homotopia (possivelmente à esquerda apenas) seja uma bijeção.
A demonstração que apresentamos aqui é de certa forma uma combinação das demonstrações apresentadas em \cite[Proposição 7.25]{heuts-moerdijk} e em \cite[Proposição 1.2.5]{hovey}.
Um ponto interessante da demonstração apresentada na segunda referência é que elas dá a oportunidade de apresentar o famigerado \emph{Lema de Brown}.
Inicialmente provamos uma versão mais simples deste resultado que é usada para provar a versão usual.

\begin{prop}[Lema de Fatoração]
  \label{prop:lema_de_fatoracao}
  Seja $f: X \to Y$ uma equivalência fraca entre objetos fibrantes de uma categoria de modelos.
  Então $f$ pode ser fatorada como uma cofibração trivial $i: X \overset{\sim}{\cofib} Z$ seguida de uma fibração trivial $p: Z \overset{\sim}{\fib} Y$, e além disso existe também uma fibração trivial $q: Z \overset{\sim}{\fib} X$ tal que $q \circ i = \id_{X}$.
\end{prop}

\begin{proof}
  Podemos fatorar o morfismo induzido $(\id_{X},f): X \to X \times Y$ como uma cofibração trivial $i: X \overset{\sim}{\to} Z$ seguida de uma fibração $\theta: Z \overset{\sim}{\to} X \times Y$.
  \begin{displaymath}
    \begin{tikzcd}
      X
      \arrow[rr, "{(\id_{X},f)}"]
      \arrow[rd, tail, "\sim" {sloped}, "\lambda" {swap}]
      & & X \times Y
      \\ & Z
      \arrow[ru, two heads, "\theta" swap]
    \end{tikzcd}
  \end{displaymath}

  Se $\pi_{1}: X \times Y \to X$ e $\pi_{2}: X \times Y \to Y$ são as projeções canônicas, definimos então
  \begin{displaymath}
    q \coloneqq \pi_{1} \circ \theta: Z \to X \quad \text{e} \quad p \coloneqq \pi_{2} \circ \theta: Z \to Y.
  \end{displaymath}
  Veja que tais morfismos satisfazem as igualdades necessárias, já que por um lado
  \begin{displaymath}
    p \circ i = \pi_{2} \circ \theta \circ i = \pi_{2} \circ (\id_{X},f) = f,
  \end{displaymath}
  e por outro
  \begin{displaymath}
    q \circ i = \pi_{1} \circ \theta \circ i = \pi_{1} \circ (\id_{X},f) = \id_{X}.
  \end{displaymath}

  Resta apenas mostrarmos que $p$ e $q$ são fibrações triviais.
  O fato de $p$ ser uma equivalência fraca segue da propriedade 2-de-3, já que temos a igualdade $p \circ i = f$ onde tanto $i$ quanto $f$ são equivalências fracas.
  O fato de $q$ ser uam equivalência fraca também segue da propriedade 2-de-3, pois temos a igualdade $q \circ i = \id_{X}$ onde $i$ e $\id_{X}$ são equivalências fracas.
  Por fim, para ver que $p$ e $q$ são fibrações, veja que temos um diagrama de pullback
  \begin{displaymath}
    \begin{tikzcd}
      X \times Y
      \arrow[r, "\pi_{1}"]
      \arrow[d, "\pi_{2}" swap]
      & X
      \arrow[d, two heads, "!_{X}"]
      \\ Y
      \arrow[r, two heads, "!_{Y}" swap]
      & *
    \end{tikzcd}
  \end{displaymath}
  onde $!_{X}$ e $!_{Y}$ são fibrações graças ao fato de $X$ e $Y$ serem fibrantes.
  Como fibrações são preservadas por pullbacks, segue que as projeções canônicas $\pi_{1}$ e $\pi_{2}$ são fibrações também, logo $p$ e $q$ são ambos composições de fibrações e, portanto, fibrações também.
\end{proof}

O tradicional Lema de Brown é então uma consequência direta da \cref{prop:lema_de_fatoracao}.

\begin{corol}[Lema de Brown]
  \label{corol:lema_de_brown}
  Sejam $(\mathsf{M},\mathcal{W},\mathcal{C},\mathcal{F})$ uma categoria de modelos e $(\mathcal{D},\mathcal{W}')$ uma categoria com equivalências fracas, ou seja, $\mathcal{W}' \subseteq Mor(\mathsf{D})$ é uma classe de morfismos satisfazendo a propriedade 2-de-3.
  Suponha que $F: \mathsf{M} \to \mathsf{D}$ seja um funtor que transforma fibrações triviais entre objetos fibrantes de $\mathsf{M}$ em equivalências fracas de $\mathsf{D}$.
  Então $F$ também transforma equivalências fracas entre objetos fibrantes de $\mathsf{M}$ em equivalências fracas de $\mathsf{D}$.
\end{corol}

\begin{proof}
  Seja $p: X \overset{\sim}{\to} Y$ uma equivalência fraca entre objetos fibrantes de $\mathsf{M}$.
  Aplicando o \cref{prop:lema_de_brown_1} obtemos uma cofibração trivial $i: X \overset{\sim}{\cofib} Z$, uma fibração trivial $p: Z \overset{\sim}{\fib} Y$ e uma fibração trivial $q: Z \overset{\sim}{\fib} X$ tais que $p \circ i = f$ e $q \circ i = \id_{X}$.
  Veja que $Z$ é um objeto fibrante, já que ele manda uma fibração para um objeto fibrante.
  Assim, $p$ é uma fibração trivial entre objetos fibrantes, portanto $F(p): F(Z) \to F(Y)$ é uma equivalência fraca em $\mathsf{D}$.
  Se cosneguirmos mostrar que $F(i): F(X) \to F(Z)$ é também uma equivalência fraca, o resultado desejado seguirá então da igualdade $F(p) \circ F(i) = F(f)$ e da propriedade 2-de-3 em $\mathsf{M}$.
  Ora, sendo $q: Z \to X$ uma fibração trivial entre objetos fibrantes também, vale que $F(q): F(Z) \to F(X)$ é uma equivalência fraca, e a propriedade 2-de-3 aplicada à igualdade $F(q) \circ F(i) = \id_{F(X)}$ nos permite concluir que $F(i)$ é uma equivalência fraca, já que o morfismo idêntico $\id_{F(X)}$ é também uma equivalência fraca.
\end{proof}

Uma aplicação bastante bacana do Lema de Brown é uma demonstração simples de que equivalências fracas entre objetos fibrantes dão origem a bijeções entre classes de homotopia.

\begin{prop}
  Seja $B$ um objeto cofibrante de uma categoria de modelos $\mathsf{M}$.
  \begin{enumerate}
  \item Se $p: X \overset{\sim}{\fib} Y$ é uma fibração trivial, então $[B,p]_{\ell}: [B,X]_{\ell} \to [B,Y]_{\ell}$ é uma bijeção.
    
  \item Se $p: X \overset{\sim}{\to} Y$ é uma equivalência fraca entre objetos fibrantes, então $[B,p]: [B,X] \to [B,Y]$ é uma bijeção.
  \end{enumerate}
\end{prop}

\begin{proof}
  1. Nesse caso, a função de pushoforward $\mathsf{M}(B,p): \mathsf{M}(B,X) \to \mathsf{M}(B,Y)$ a nível de morfismos já é sobrejetora.
  De fato, dado um morfismo $g: B \to Y$ qualquer,  aplicando o axioma de levantamento ao quadrado abaixo, onde $!_{B}: \varnothing \to B$ é uma cofibração e $p: X \to Y$ é uma fibração trivial,
  \begin{displaymath}
    \begin{tikzcd}
      \varnothing
      \arrow[r, "!_{X}"]
      \arrow[d, tail, "!_{B}" swap]
      & X
      \arrow[d, two heads, "p", "\sim" {swap,sloped}]
      \\ B
      \arrow[r, "g" swap]
      \arrow[ru, dashed, "f" description]
      & Y
    \end{tikzcd}
  \end{displaymath}
  obtemos um morfismo $f: B \to X$ tal que $p \circ f = g$.
  Consequentemente, a nível de classes de homotopia à esquerda temos
  \begin{displaymath}
    [B,p]_{\ell}([f]_{\ell}) = [p \circ f]_{\ell} = [g]_{\ell},
  \end{displaymath}
  portanto $[B,p]_{\ell}$ é também uma função sobrejetora.

  Vejamos agora a questão da injetividade.
  Suponha que $[f_{1}]_{\ell},\,[f_{2}]_{\ell} \in [B,X]_{\ell}$ sejam duas classes de homotopia à esquerda tais que $[B,p]_{\ell}([f_{1}]_{\ell}) = [B,p]_{\ell}([f_{2}]_{\ell})$, ou seja, existe uma homotopia à esquerda $p \circ f_{1} \simeq_{\ell} p \circ f_{2}$ dada pelo morfismo $h: \Cyl(B) \to Y$.
  Usando a trivialidade da fibração $p$ podemos obter um levantamento $H: \Cyl(B) \to X$ para o quadrado comutativo indicado abaixo.
  \begin{displaymath}
    \begin{tikzcd}[column sep=1.25cm]
      B \sqcup B
      \arrow[d, tail, "i" swap]
      \arrow[r, "{\langle f_{1},f_{2} \rangle}"]
      & X
      \arrow[d, two heads, "p", "\sim" {swap,sloped}]
      \\ \Cyl(B)
      \arrow[r, "h" swap]
      \arrow[ru, dashed, "H" description]
      & Y
    \end{tikzcd}
  \end{displaymath}
  A condição $H \circ i = \langle f_{1},f_{2} \rangle$ diz precisamente que $H$ define uma homotopia à esquerda $f_{1} \simeq_{\ell} f_{2}$, portanto $[f_{1}]_{\ell} = [f_{2}]_{\ell}$, e $[B,p]_{\ell}$ é uma função injetora.

  \smallskip
  2. Vamos aplicar o Lema de Brown (\cref{corol:lema_de_brown}) de uma forma espertinha.
  Considere o funtor $[B,-]_{\ell}: \mathsf{M} \to \mathsf{Set}$ que associa a cada objeto $X \in \mathsf{M}$ a classe de homotopia à esquerda $[B,X]_{\ell}$ e que associa a um morfismo $f: X \to Y$ a função de pushforward $[B,f]_{\ell}: [B,X]_{\ell} \to [B,Y]_{\ell}$ correspondente.
  Considere $\mathsf{Set}$ como uma categoria com equivalências fracas no sentido do \cref{corol:lema_de_brown} tomando a classe $\mathsf{W}' \subseteq \Mor(\mathsf{Set})$ formada por todos as bijeções.
  O item 1 diz então que o funtor $[B,-]_{\ell}: \mathsf{M} \to \mathsf{Set}$ introduzido transforma qualquer fibração trivial de $\mathsf{M}$ em uma equivalência fraca de $\mathsf{Set}$ no sentido acima.
  Em particular, $[B,-]_{\ell}$ transforma também fibrações triviais entre objetos fibrantes em equivalências fracas, portanto pelo Lema de Brown concluímos que $[B,-]_{\ell}$ também transforma equivalência fracas entre objetos fibrantes em equivalências fracas; ou seja, se $p: X \overset{\sim}{\to} Y$ é uma equivalência fraca entre objetos fibrantes, a função de pushforward correspondente $[B,p]: [B,X] \to [B,Y]$ é uma bijeção.
\end{proof}

%%% Local Variables:
%%% mode: latex
%%% TeX-master: "../main"
%%% End:
