\section{Localizações}

Nessa seção discutimos inicialmente a noção de localização de uma categoria $\mathsf{C}$ em uma classe qualquer de morfismos, e em seguida discutimos como as estrutura de uma categoria de modelos nos permite de certa forma simplificar o processo de localização na classe das equivalências fracas, simplificação esta que nos permitirá mais adiante construir modelos muito mais manejáveis para essa localização em termos de classes de homotopia.

Em geral, a única forma possível de compararmos dois morfismos em uma categoria $\mathsf{C}$ qualquer é por meio de igualdades.
Note, entretanto, que a situação é mais sútil para morfismos na categoria de categorias $\mathsf{Cat}$, já que dois funtores podem estar relacionados por uma igualdade ou também por um isomorfismo natural.
Isso significa que construções universais envolvendo \emph{categorias em si} podem ser formuladas de duas formas: uma versão \emph{estrita} usando apenas igualdades entre funtores, e uma versão \emph{fraca} usando isomorfismos naturais entre funtores.
A primeira definição de localização que daremos, a qual aparece por exemplo nas referências \cite[Definição 7.30]{heuts-moerdijk} e \cite{nlab:localization}, é a versão fraca.
Essa é a versão que realmente aparece na teoria de categoria de modelos, já que a construção de uma localização na classe de equivalências fracas por meio das classes de homotopia de morfismos entre objetos bifibrantes satisfaz apenas essa versão fraca da definição.

\begin{defin}
  Sejam $\mathsf{C}$ uma categoria e $\mathcal{W} \Mor(\mathsf{C})$ uma classe qualquer de morfismos.
  Uma \textbf{localização fraca de $\mathsf{C}$ em $\mathcal{W}$} é um par $(\mathsf{L},\gamma)$, onde $\mathsf{L}$ é uma categoria e $\gamma: \mathsf{C} \to \mathsf{L}$ é um funtor, satisfazendo as seguintes condições:
  \begin{enumerate}
  \item[(i)] O funtor $\gamma$ transforma os morfismo de $\mathcal{W}$ em isomorfismos de $\mathsf{L}$.
    
  \item[(ii)] Dada uma outra categoria $\mathsf{D}$ qualquer, o funtor de pré-composição com $\gamma$ define uma \emph{equivalência} de categorias
    \begin{displaymath}
      \gamma^{*}: \Fun(\mathsf{L},\mathsf{D}) \overset{\simeq}{\longrightarrow} \Fun_{\mathcal{W}}(\mathsf{C},\mathsf{D}),
    \end{displaymath}
    onde $\Fun_{\mathcal{W}}(\mathsf{C},\mathsf{D})$ denota a subcategoria plena da categoria de funtores $\Fun(\mathsf{C},\mathsf{D})$ gerada por aqueles funtores que transformam morfismos de $\mathcal{W}$ em isomorfismos de $\mathsf{D}$.
  \end{enumerate}
\end{defin}

%%% Local Variables:
%%% mode: latex
%%% TeX-master: "../main"
%%% End:
