\section{Teoria de Homotopia em categorias modelo}

Nessa seção introduzimos enfim noções homotópicas que podem ser descritas em uma categoria modelo qualquer.
Veremos, entretanto, que mesmo a noção básica de homotopia entre dois morfismos possui sutilezas que a tornam mais complexa do que a noção clássica de homotopia entre mapas contínuos de espaços topológicos.
Felizmente, também veremos que a noção categórica de homotopia se aproxima muito maisda clássica quando trabalhamos apenas com objetos cofibrantes ou fibrantes, e nesse caso podemos usar a construção usual da categoria homotópica para definirmos uma espécie de localização de uma categoria modelo.

A fim de imitarmos a noção topológica de homotopia, o primeiro passo será darmos uma descrição categórica para a construção do cilindro $B \times I$ associado a um espaço topológico $B$ qualquer, onde é claro que $I$ denota o intervalo unitário da reta.

Lembremos inicialmente que, dado um objeto $B$ de uma categoria $\mathsf{C}$ qualquer que admita coprodutos, a propriedade universal dessa construção garante a existência de um único mapa $\nabla: B \sqcup B \to B$ fazendo comutar o diagrama abaixo.
\begin{displaymath}
  \begin{tikzcd}
    B
    \arrow[rd, "j_{1}"]
    \arrow[rrd, bend left=15, "\id_{B}"]
    \\ & B \sqcup B
    \arrow[r, dashed, "\nabla" description]
    & B
    \\ B
    \arrow[ru, "j_{2}" swap]
    \arrow[rru, bend right=15, "\id_{B}" swap]
  \end{tikzcd}
\end{displaymath}
Tal morfismo é comumente chamado de \textbf{morfismo codiagonal} ou também de \textbf{morfismo de dobra}\footnote{Uma tradução direta do inglês \emph{fold map}.}.
Intuitivamente, esse morfismo simplesmente cola duas cópias exatamente uma sobre a outra.
Por vezes, se precisarmos distinguir entre os morfismos codiagonais associados a diferentes objetos, utilizaremos também a notação $\nabla_{B}$ para o morfismo descrito acima.

\begin{defin}
  Sejam $\mathsf{M}$ uma categoria modelo e $B \in \mathsf{M}$ um objeto qualquer.
  Um \textbf{objeto cilindro} para $B$ é uma fatoração do morfismo codiagonal $\nabla: B \sqcup B \to B$ como uma cofibração seguida de uma equivalência fraca.
  Mais explicitamente, um objeto cilindro para $B$ é uma tripla $(\Cyl(B),i,\varepsilon)$, onde $i: B \sqcup B \cofib \Cyl(B)$ é uma cofibração e $\varepsilon: \Cyl(B) \overset{\sim}{\to} B$ é uma equivalência fraca tais que $\varepsilon \circ i = \nabla$, conforme mostrado no diagrama comutativo abaixo.
  \begin{displaymath}
    \begin{tikzcd}
      B \sqcup B
      \arrow[d, tail, "i" swap]
      \arrow[rd, "\nabla"]
      \\ \Cyl(B)
      \arrow[r, "\varepsilon" {swap}, "\sim"]
      & B
    \end{tikzcd}
  \end{displaymath}
  Dizemos que a tripla $(\Cyl(B),i,\varepsilon)$ define um objeto cilindro \textbf{forte} se $\varepsilon$ é também uma fibração, ou seja, se $\varepsilon$ é uma fibração trivial.
\end{defin}

\begin{obs}
  O axioma de fatoração (M5) garante que o morfismo $\nabla: B \sqcup B \to B$ pode ser fatorado como uma cofibração seguida de uma fibração trivial, portanto todo objeto de uma categoria modelo admite um objeto cilindro forte.
  Entretanto, por vezes esse modelo para o objeto cilindro forte dado pelo axioma de fatoração pode ser muito complicado, por isso é vantajoso trabalharmos com cilindros fracos também, cuja descrição é por vezes mais simples.
\end{obs}

Vamos introduzir mais um pouco de terminologia.
Dado  um objeto cilindro $(\Cyl(B),i,\varepsilon)$ para $B$, se $j_{1},\, j_{2}: B \to B \sqcup B$ são as injeções canônicas, frequentemente denotaremos os morfismos do tipo $B \to \Cyl(B)$ dados pelas composições $i \circ j_{1}$ e $i \circ j_{2}$ por $i_{0}$ e $i_{1}$, respectivamente.
Essa diferença nos índices pode parecer estranho, mas a motivação para tal vem do caso topológico clássico, onde um modelo para o objeto cilindro $\Cyl(B)$ é o produto $B \times I$.
Nesse caso, a composição $i_{0} = i \circ j_{1}$ mapeia $B$ para a face inferior $B \times \{0\}$ de $B \times I$, enquanto a composição $i_{1} = i \circ j_{2}$ mapeia $B$ para a face superior $B \times \{1\}$ do cilindro $B \times I$, o que justifica a razoabilidade dos índices aparecendo em cada uma das composições.

Antes de vermos como a noção de objeto cilindro nos permite definir uma noção categórica de homotopia, vejamos algumas propriedades simples satisfeitas por tais objetos.

\begin{lema}\label{lema:propriedades_objeto_cilindro}
  Sejam $\mathsf{M}$ uma categoria modelo, $B \in \mathsf{M}$ um objeto qualquer, e $(\Cyl(B),i,\varepsilon)$ um objeto cilindro qualquer para $B$.
  \begin{enumerate}
  \item[(i)] Os morfismos $i_{0},\, i_{1}: B \to \Cyl(B)$ são equivalências fracas.
    
  \item[(ii)] Se $B$ é um objeto cofibrante, então os morfismos $i_{0},\, i_{1}: B \to \Cyl(B)$ são também cofibrações e, portanto, cofibrações triviais.
  \end{enumerate}
\end{lema}

\begin{proof}
  (i) Note que pela definição de $i_{0}$ temos
  \begin{displaymath}
    \varepsilon \circ i_{0} = \varepsilon \circ i \circ j_{1} = \nabla \circ j_{1} = \id_{B}.
  \end{displaymath}
  Ora, como $\id_{B}$ é uma equivalência fraca, o mesmo valendo para $\varepsilon$ pela definição de objeto cilindro, segue da propriedade 2-de-3 que $i_{0}$ é também uma equivalência fraca.
  A demonstração de que $i_{1}$ é uma equivalência fraca segue de um raciocínio completamente análogo.

  \smallskip
  (ii) Lembremos que em qualquer categoria, o coproduto pode ser interpretado também como um pushout sobre o objeto inicial.
  Mais precisamente, se $\varnothing$ denota um objeto inicial de $\mathsf{M}$, então o quadrado comutativo abaixo é um pushout.
  \begin{displaymath}
    \begin{tikzcd}
      \varnothing
      \arrow[r, "!_{B}"]
      \arrow[d, "!_{B}" swap]
      & B
      \arrow[d, "j_{1}"]
      \\ B
      \arrow[r, "j_{2}" swap]
      & B \sqcup B
    \end{tikzcd}
  \end{displaymath}
  Por hipótese $B$ é cofibrante, ou seja, o morfismo único $!_{B}$ é uma cofibração, mas já sabemos do \cref{corol:propriedades_de_preservacao_categoria_modelo} que cofibrações são preservadas por pushouts, portanto $j_{1}$ e $j_{2}$ são cofibrações.
  Mas segue então que os morfismos $i_{0}$ e $i_{1}$ são composições de cofibrações e, portanto, cofibrações também de acordo com o \cref{corol:propriedades_de_preservacao_categoria_modelo}.
\end{proof}

Tendo em mãos essas propriedades básicas, podemos enfim definir uma primeira noção de homotopia entre morfismos em uma categoria modelo.

\begin{defin}
  Dois morfismos $f,\,g: B \to X$ em uma categoria modelo $\mathsf{M}$ são ditos \textbf{homotópicos à esquerda} se existe um objeto cilindro $(\Cyl(B),i,\varepsilon)$ para $B$ e um morfismo $h: \Cyl(B) \to X$ tal que $h \circ i = \langle f,g \rangle$, conforme mostrado no diagrama comutativo abaixo.
  Nesse caso, denotamos essa relação por $f \sim_{L} g$.
  \begin{displaymath}
    \begin{tikzcd}
      B
      & B \sqcup B
      \arrow[l, "\nabla" swap]
      \arrow[r, "{\langle f,g \rangle}"]
      \arrow[d, tail, "i"]
      & X
      \\ & \Cyl(B)
      \arrow[ru, dashed, "h" swap]
      \arrow[lu, "\varepsilon", "\sim" {sloped}]
    \end{tikzcd}
  \end{displaymath}
\end{defin}

Se interpretarmos os morfismos $i_{0},\, i_{1}: B \to \Cyl(B)$ como sendo as faces inferior e superior do cilindro como no caso topológico clássico, então as composições $h \circ i_{0}$ e $h \circ i_{1}$ determinam os estágios inicial e final da homotopia $h$.
Usando a definição de tais morfismos e a comutatividade acima vemos que por um lado
\begin{displaymath}
  h \circ i_{0} = h \circ i \circ j_{1} = \langle f,g \rangle \circ j_{1} = f,
\end{displaymath}
e por outro
\begin{displaymath}
  h \circ i_{1} = h \circ i \circ j_{2} = \langle f,g \rangle \circ j_{2} = g.
\end{displaymath}
Assim, recuperamos em certo sentido a intuição clássica de uma família de morfismos que começa em $f$ e termina em $g$.

Veja que a definição acima possui uma sutileza: exigimos que a homotopia $h$ esteja definida em \emph{algum} objeto cilindro $\Cyl(B)$, mas $B$ pode muito bem admitir diversos objetos cilindros distintos.
Isso pode representar uam dificuldade para ``combinarmos'' homotopias à esquerda, já que elas podem não estar definidas nos mesmos objetos cilindros.

O resultado abaixo mostra que homotopias à esquerda são preservadas por composição de morfismos à esquerda, o que possivelmente justifica a terminologia usada.
Além disso, mostramos que homotopias à esquerda também são preservadas por composição de morfismos à direita quando supomos que o codomínio é fibrante, e essa hipótese também garante a independência do objeto cilindro no qual a homotopia está definida.

\begin{prop}\label{prop:propriedades_de_preservacao_homotopia_a_esquerda}
  Suponha que $f,\,g : B \to X$ sejam dois morfismos homotópicos à esquerda em uma categoria modelo $\mathsf{M}$.

  \begin{enumerate}
  \item Dado um morfismo $\beta: X \to Y$, os morfismos compostos $\beta \circ f,\, \beta \circ g: B \to Y$ são homotópicos à esquerda.
    
  \item Se $X$ é fibrante, então a homotopia entre $f$ e $g$ independe do objeto cilindro usado.
    
  \item Se $X$ é fibrante, então dado qualquer morfismo $\alpha: A \to B$, os morfismos compostos $f \circ \alpha,\, g \circ \alpha: A \to X$ são homotópicos à esquerda.
  \end{enumerate}
\end{prop}

\begin{proof}
  1. Por hipótese existe um objeto cilindro $(\Cyl(B),i,\varepsilon)$ para $B$ e um morfismo $h: \Cyl(B) \to X$ satisfazendo $h \circ i = \langle f,g \rangle$.
  Afirmamos que $\beta \circ h$ é a homotopia à esquerda procurada de $\beta \circ f$ para $\beta \circ g$.
  De fato, basta notar que
  \begin{displaymath}
    \beta \circ h \circ i = \beta \circ \langle f,g \rangle = \langle \beta \circ f, \beta \circ g \rangle,
  \end{displaymath}
  onde a última igualdade segue diretamente da propriedade universal do coproduto, já que a composição $\beta \circ \langle f,g \rangle$ satisfaz as igualdades
  \begin{displaymath}
    (\beta \circ \langle f,g \rangle) \circ j_{1} = \beta \circ f \quad \text{e} \quad (\beta \circ \langle f,g \rangle) \circ j_{2} = \beta \circ g.
  \end{displaymath}

  \smallskip
  2. Por hipótese sabemos que existe \emph{algum} objeto cilindro $(\Cyl(B),i,\varepsilon)$ para $B$ e um morfismo $h: \Cyl(B) \to X$ tal que $h \circ i = \langle f,g \rangle$.
  Afirmamos que existe também uma homotopia de $f$ para $g$ definida em um cilindro \emph{forte}.
  De fato, sabemos do axioma de fatoração que a equivalência fraca $\varepsilon: \Cyl(B) \to B$ pode ser fatorada como uma cofibração seguida de uma fibração trivial conforme indicado abaixo.
  \begin{displaymath}
    \begin{tikzcd}
      \Cyl(B)
      \arrow[rr, "\varepsilon", "\sim" {swap}]
      \arrow[rd, tail, "j" swap]
      & & B
      \\ & C
      \arrow[ru, two heads, "p" {swap}, "\sim" {sloped}]
    \end{tikzcd}
  \end{displaymath}
  Afirmamos então que a tripla $(C,j \circ i,p)$ é um objeto cilindro forte para $B$.
  Veja que $j \circ i$ e $p$ fatoram o morfismo de dobra $\nabla: B \to B \sqcup B$ pois
  \begin{displaymath}
    p \circ (j \circ i) = (p \circ j) \circ i = \varepsilon \circ i = \nabla,
  \end{displaymath}
  e como $p$ é uma fibração trivial pelo axioma de fatoração, a tripla em questão define de fato um cilindro forte.

  Vamos agora mostrar que existe uma homotopia de $f$ para $g$ definida nesse cilindro forte.
  Note inicialmente que o morfismo $j: \Cyl(B) \to C$ que apareceu na fatoração acima é uma cofibração trivial pela propriedade 2-de-3, já que $\varepsilon$ e $p$ são ambos equivalências fracas.
  Como o morfismo único $!_{X}: X \to *$ é por hipótese uma fibração, segue do axioma de levantamento que existe um morfismo diagonal $\widetilde{h}: C \to X$ fazendo comutar o diagrama abaixo.
  \begin{displaymath}
    \begin{tikzcd}
      \Cyl(B)
      \arrow[r, "h"]
      \arrow[d, tail, "j" {swap}, "\sim" {sloped}]
      & X
      \arrow[d, two heads, "!_{X}"]
      \\ C
      \arrow[r, "!_{C}" swap]
      \arrow[ru, dashed, "\widetilde{h}" description]
      & *
    \end{tikzcd}
  \end{displaymath}
  Note então que $\widetilde{h}$ define uma homotopia de $f$ para $g$ com relação ao cilindro forte $(C,j \circ i,p)$ pois
  \begin{displaymath}
    \widetilde{h} \circ (j \circ i)
    = (\widetilde{h} \circ j) \circ i
    = h \circ i
    = \langle f,g \rangle.
  \end{displaymath}

  Vejamos agora como mostrar que a homotopia entre $f$ e $g$ independe do cilindro usado.
  Segue da discussão acima que podemos assumir sem perda de generalidade que o objeto cilindro $(\Cyl(B),i,\varepsilon)$ no qual está definida a homotopia $h$ por hipótese é forte.
  Suponha agora que $(S,i',\varepsilon')$ seja um outro objeto cilindro qualquer.
  Queremos construir um morfismo $h': S \to X$ tal que $h' \circ i' = \langle f,g \rangle$.
  A ideia é construirmos um morfismo do tipo $S \to \Cyl(B)$ que, ao ser composto com a homotopia $h$ já existente, forneça uma outra homotopia definida agora no objeto cilindro $S$.
  A fim de construirmos tal morfismo, note que o quadrado mostrado abaixo é comutativo, pois ambos os pares $(i,\varepsilon)$ e $(i',\varepsilon')$ são fatorações do morfismo de dobra $\nabla$.
  Ademais, como $i'$ é uma cofibração, pois $S$ é um objeto cilindro, e $\varepsilon$ é uma fibração trivial, pois $\Cyl(B)$ é um cilindro forte; segue do axioma de levantamento que existe um morfismo diagonal $\phi: S \to \Cyl(B)$ como indicado.
  \begin{displaymath}
    \begin{tikzcd}
      B \sqcup B
      \arrow[r, tail, "i"]
      \arrow[d, tail, "i'" swap]
      & \Cyl(B)
      \arrow[d, two heads, "\varepsilon", "\sim" {swap,sloped}]
      \\ S
      \arrow[r, "\sim", "\varepsilon'" {swap}]
      \arrow[ru, dashed, "\phi" description]
      & B
    \end{tikzcd}
  \end{displaymath}
  Veja então que a composição $h' \coloneqq h \circ \phi$ define a homotopia desejada, pois da comutatividade acima vemos que
  \begin{displaymath}
    h' \circ i' = h \circ \phi \circ i' = h \circ i = \langle f,g \rangle.
  \end{displaymath}

  \smallskip
  3. Suponha que $(\Cyl(A),i_{A},\varepsilon_{A})$ seja um objeto cilindro qualquer para $A$ e que $(\Cyl(B),i_{B},\varepsilon_{B})$ seja um objeto cilindro \emph{forte} para $B$.
  Sabemos pelo item 2 que certamente existe uma homotopia $h: \Cyl(B) \to X$ de $f$ para $g$ definida nesse cilindro forte.
  Considere o diagrama de levantamento exibido abaixo.
  \begin{displaymath}
    \begin{tikzcd}[column sep=1.25cm]
      A \sqcup A
      \arrow[r, "{i_{B} \circ \alpha \sqcup \alpha}"]
      \arrow[d, tail, "i_{A}" swap]
      & \Cyl(B)
      \arrow[d, two heads, "\varepsilon_{B}", "\sim" {swap,sloped}]
      \\ \Cyl(A)
      \arrow[r, "\alpha \circ \varepsilon_{A}" swap]
      & B
    \end{tikzcd}
  \end{displaymath}
  Veja que o quadrado externo é de fato comutativo pois por um lado
  \begin{displaymath}
    \alpha \circ \varepsilon_{A} \circ i_{A} = \alpha \circ \nabla_{A} = \alpha \circ \langle \id_{A},\id_{A} \rangle = \langle \alpha,\alpha \rangle,
  \end{displaymath}
  e por outro
  \begin{displaymath}
    \varepsilon_{B} \circ i_{B} \circ \alpha \sqcup \alpha = \nabla_{B} \circ \alpha \sqcup \alpha = \langle \alpha,\alpha \rangle.
  \end{displaymath}
  Segue do axioma de levantamento que existe o morfismo diagonal $\phi: \Cyl(A) \to \Cyl(B)$ mostrado acima.
  Note então que $h \circ \phi: \Cyl(A) \to X$ é a homotopia procurada pois
  \begin{displaymath}
    h \circ \phi \circ i_{A} = h \circ i_{B} \circ (\alpha \sqcup \alpha) = \langle f,g \rangle \circ (\alpha \sqcup \alpha) = \langle f \circ \alpha, g \circ \alpha \rangle. \qedhere
  \end{displaymath}
\end{proof}

Introduzimos agora uma noção de certa forma dual àquela de objeto cilindro.

\begin{defin}\label{defin:objeto_de_caminhos}
  Sejam $\mathsf{M}$ uma categoria modelo e $B \in \mathsf{M}$ um objeto qualquer.
  Um \textbf{objeto de caminhos} para $B$ é uma fatoração do morfismo diagonal $\Delta: B \to B \times B$ como uma equivalência fraca seguida de uma fibração.
  Mais explicitamente, um objeto de caminhos para $B$ é uma tripla $(P(B),c,p)$, onde $P(B) \in \mathsf{M}$ é um objeto da categoria, $c: B \overset{\sim}{\to} P(B)$ é uma equivalência fraca, e $p: P(B) \fib B \times B$ é uma fibração tais que $\Delta = p \circ c$, conforme mosrado no diagrama abaixo.
  \begin{displaymath}
    \begin{tikzcd}
      & P(B)
      \arrow[d, two heads, "p"]
      \\ B
      \arrow[ru, "c", "\sim" {swap,sloped}]
      \arrow[r, "\Delta" swap]
      & B \times B
    \end{tikzcd}
  \end{displaymath}
  Quando o morfismo $c$ é também uma equivalência fraca, portanto uma cofibração trivial, dizemos que a tripla $(P(B),c,p)$ define um objeto de caminhos \textbf{forte} para $B$.
\end{defin}

\begin{obs}
  O axioma de fatoração (M5) garante que o morfismo diagonal $\Delta: B \times B \to B$ pode ser fatorado como uma cofibração trivial seguida de uma fibração, portanto todo objeto de uma categoria modelo admite um objeto de caminhos que é até mesmo forte.
  Como no caso do objetos cilindro, nem sempre esse modelo obtido pelo axioma de fatoração é o mais conveniente para trabalharmos, de forma que é vantajoso considerarmos também objetos de caminhos que não sejam fortes.
\end{obs}

Neste caso temos também algumas notações associadas.
Dado um objeto cilindro $(P(B),c,p)$ para $B$, se $\pi_{1},\,\pi_{2}: B \times B \to B$ são as projeções canônicas associadas ao produto, denotaremos os morfismos compostos $\pi_{1} \circ p,\,pi_{2} \circ p: P(B) \to B$ por $p_{0}$ e $p_{1}$, respectivamente.
Note então que a fibração $p$ é precisamente o morfismo induzido por $p_{0}$ e $p_{1}$ por meio da propriedade universal do produto, ou seja, $p = (p_{0},p_{1})$.
Como no caso de objetos cilindros, os índices usados na notação ficam claros quando examinamos o caso topológico clássico, onde o objeto de caminhos é dado pelo espaço de mapas $B^{I}$ munido da topologia compacto-aberta, a equivalência fraca $c: B \to B^{I}$ associa a cada ponto $b \in B$ o caminho $c(b): I \to B$ constante naquele ponto, e a fibração $p: B^{I} \to B \times B$ associa a um caminho $\gamma \in B^{I}$ seus pontos inicial e final, ou seja, $p(\gamma) \coloneqq (\gamma(0),\gamma(1))$.
Nesse caso, os mapas $p_{0},\,p_{1}: B^{I} \to B$ introduzidos acima são simplesmente os mapas de avaliação no instante inicial $0$ e no instante final $1$, respectivamente.

Vamos agora percorrer um caminho completamente análogo ao que percorremos para objetos cilindros.
Começamos inicialmente verificando algumas de suas propriedades básicas.

\begin{lema}\label{lema:propriedades_objeto_de_caminhos}
  Sejam $\mathsf{M}$ uma categoria modelo, $B \in \mathsf{M}$ um objeto, e $(P(B),c,p)$ um objeto de caminhos qualquer para $B$.
  \begin{enumerate}
  \item[(i)] Os morfismos $p_{0},\, p_{1}: P(B) \to B$ são equivalência fracas.
    
  \item[(ii)] Se $B$ é um objeto fibrante, então os morfismos $p_{0},\, p_{1}: P(B) \to B$ são também fibrações e, portanto, fibrações triviais.
  \end{enumerate}
\end{lema}

\begin{proof}
  (i) Note que pela definição de $p_{0}$ e do morfismo diagonal $\Delta$ temos
  \begin{displaymath}
    p_{0} \circ c = \pi_{1} \circ p \circ c = \pi_{1} \circ \Delta = \id_{B}.
  \end{displaymath}
  Ora, como $\id_{B}$ é uma equivalência fraca, o mesmo valendo para $c$ pela definição de objeto de caminhos, segue da propriedade 2-de-3 que $p_{0}$ é uma equivalência fraca.
  A demonstração de que $p_{1}$ também é uma equivalência fraca segue de um raciocínio completamente análogo.

  \smallskip
  (ii) Lembremos que em qualquer categoria, um produto por der interpretado como um pullback sobre o objeto final.
  Mais precisamente, se $* \in \mathsf{M}$ denota um objeto final qualquer de $\mathsf{M}$, então o quadrado comutativa abaixo é um pullback.
  \begin{displaymath}
    \begin{tikzcd}
      B \times B
      \arrow[r, "\pi_{1}"]
      \arrow[d, "\pi_{2}" swap]
      & B
      \arrow[d, "!_{B}"]
      \\ B
      \arrow[r, "!_{B}" swap]
      & *
    \end{tikzcd}
  \end{displaymath}
  Como $B$ é por hipótese fibrante, ou seja, o morfismo $!_{B}: B \to *$ é uma fibração, e como fibrações são preservadas por pullbacks de acordo com a \cref{corol:propriedades_de_preservacao_categoria_modelo}, concluímos que as projeções canônicas $\pi_{1}$ e $\pi_{2}$ são fibrações também.
  Segue então que os morfismos $p_{0} = \pi_{1} \circ p$ e $p_{1} = \pi_{2} \circ p$ são composições de fibrações e, portanto, fibrações também de acordo com o \cref{corol:propriedades_de_preservacao_categoria_modelo}.
\end{proof}

Objetos de caminhos dão origem a uma noção de homotopia entre morfismos que é dual à noção de homotopia à esquerda introduzida anteriormente em termos de objetos cilindros.

\begin{defin}\label{defin:homotopia_a_direita}
  Dois morfismos $f,\, g: B \to X$ em uma categoria modelo $\mathsf{M}$ são ditos \textbf{homotópicos à direita} se existe um objeto de caminhos $(P(X),c,p)$ para $X$ e um morfismo $h: B \to P(X)$ tal que $p \circ h = (f,g)$, conforme mostrado no diagrama abaixo.
  Neste caso, denotamos essa relação por $f \sim_{R} g$.
  \begin{displaymath}
    \begin{tikzcd}
      & P(X)
      \arrow[d, two heads, "p"]
      \\ B
      \arrow[ru, dashed, "h"]
      \arrow[r, "{(f,g)}" swap]
      & X \times X
      & X
      \arrow[lu, "\sim" {sloped, swap}, "c" swap]
      \arrow[l, "\Delta"]
    \end{tikzcd}
  \end{displaymath}
\end{defin}

Veja que a definição acima recupera a definição clássica da Topologia.
O mapa $h: B \to X^{I}$ define uma família de caminhos em $X$ parametrizada pelos pontos do espaço $B$.
Para cada $b \in B$, o caminho associada $h(b)$ tem $f(b)$ como ponto inicial pois
\begin{displaymath}
  [h(b)](0) = p_{0}(h(b)) = \pi_{1}(p(h(b))) = \pi_{1}(f(b),g(b)) = f(b),
\end{displaymath}
e analogamente, $h(b)$ tem como ponto final a imagem $g(b)$.
Assim, a família de caminhos definida pela homotopia $h$ em certo sentido ``conecta'' a imagem do estágio inicial $f$ da homotopia ao estágio final $g$ da mesma.

Nosso objetivo agora é demonstrar um resultado análogo à \cref{prop:propriedades_de_preservacao_homotopia_a_esquerda} para homotopias à direita.

\begin{prop}\label{prop:propriedades_de_preservacao_homotopia_a_direita}
  Suponha que $f,\,g: B \to X$ sejam dois morfismos homotópicos à direita em uma categoria modelo $\mathsf{M}$.
  \begin{enumerate}
  \item Dado um morfismo $\alpha: A \to B$, os morfismos compostos $f \circ \alpha,\, g \circ \alpha: A \to X$ são homotópicos à direita.
    
  \item Se $B$ é cofibrante, então a homotopia entre $f$ e $g$ independe do objeto de caminhos para $X$.
    
  \item Se $B$ é cofibrante, então dado qualquer morfismo $\beta: X \to Y$, os morfismos compostos $\beta \circ f,\, \beta \circ g: B \to Y$ são homotópicos à direita.
  \end{enumerate}
\end{prop}

\begin{proof}
  1. Como $f$ e $g$ são homotópicos à direita, existe \emph{algum} objeto de caminhos $(P(X),c,p)$ para $X$ juntamente com um morfismo $h: B \to P(X)$ tal que $p \circ h = (f,g)$.
  Afirmamos que o morfismo composto $h \circ \alpha: A \to P(X)$ define uma homotopia à direita de $f \circ \alpha$ para $g \circ \alpha$.
  De fato, basta para isso notarmos que
  \begin{displaymath}
    p \circ h \circ \alpha = (f,g) \circ \alpha = (f \circ \alpha, g \circ \alpha),
  \end{displaymath}
  sendo que a última igualdade é uma consequência direta da propriedade universal que caracteriza o produto $X \times X$.

  \smallskip
  2. Considere um objeto de caminhos $(P(X),c,p)$ para $X$ e uma homotopia à direita $h: B \to P(X)$ como acima.
  Afirmamos inicialmente que podemos encontrar uma outra homotopia à direita entre $f$ e $g$ que esteja definida em um objeto de caminhos \emph{forte} para $X$.
  Inicialmente, aplicamos o axioma de fatoração (M5) à equivalência fraca $c: X \to P(X)$ para obtermos uma cofibração trivial $i: X \overset{\sim}{\cofib} P$ seguida de uma fibração $q: P \fib P(X)$ tais que $c = q \circ i$, conforme mostrado abaixo.
  \begin{displaymath}
    \begin{tikzcd}
      X
      \arrow[rr, "c", "\sim" {swap}]
      \arrow[rd, tail, "i" {swap}, "\sim" {sloped}]
      & & P(X)
      \\ & P
      \arrow[ru, two heads, "q" swap]
    \end{tikzcd}
  \end{displaymath}
  Afirmamos que a tripla $(P,i,p \circ q)$ define um objeto de caminhos forte para $X$.
  De fato, $i$ é uma cofibração trivial pela fatoração acima, enquanto $p \circ q$ é a composição de duas fibrações e, portanto, uma fibração também.
  Além disso, esses morfismos fatoram o morfismo diagonal de $X$ já que
  \begin{displaymath}
    (p \circ q) \circ i = p \circ (q \circ i) = p \circ c = \Delta.
  \end{displaymath}

  Tendo em mãos o objeto de caminhos forte acima, vejamos como obter uma homotopia que tome valores nele a partir da homotopia $h$ já existente.
  Veja inicialmente que o problema de levantamento abaixo admite uma solução $H: B \to P$, já que $!_{B}: \varnothing \to B$ é uma cofibração pela hipótese de $B$ ser cofibrante, e $q: P \to P(X)$ é uma fibração trivial pela fatoração acima e pela propriedade 2-de-3.
  \begin{displaymath}
    \begin{tikzcd}
      \varnothing
      \arrow[d, tail, "!_{B}" swap]
      \arrow[r, "!_{P}"]
      & P
      \arrow[d, two heads, "q", "\sim" {sloped, swap}]
      \\ B
      \arrow[r, "h" swap]
      \arrow[ru, dashed, "H" description]
      & P(X)
    \end{tikzcd}
  \end{displaymath}
  Esse morfismo $H$ é precisamente a homotopia à direita procurada, pois
  \begin{displaymath}
    (p \circ q) \circ H = p \circ (q \circ H) = p \circ h = (f,g).
  \end{displaymath}

  Podemos assumir então sem perda de generalidade que o objeto de caminhos $(P(X),c,p)$ no qual está definida a homotopia à direita entre $f$ e $g$ é forte.
  Se $(P',c',p')$ é outro objeto de caminhos \emph{qualquer} para $X$, vejamos como construir uma homotopia à direita $h': B \to P'$ tomando valores nesse objeto.
  Note primeiro que o problema de levantamento dado pelo quadrado comtuativo abaixo admite uma solução $\phi$, já que $c$ é uma cofibração trivial, e $p'$ é uma fibração.
  \begin{displaymath}
    \begin{tikzcd}
      X
      \arrow[r, "c'"]
      \arrow[d, tail, "c" {swap}, "\sim" {sloped}]
      & P'
      \arrow[d, two heads, "p'"]
      \\ P(X)
      \arrow[ru, dashed, "\phi" description]
      \arrow[r, "p" swap]
      & X \times X
    \end{tikzcd}
  \end{displaymath}
  Basta notar agora que a composição $\phi \circ h: B \to P'$ define a homotopia à direita desejada, pois
  \begin{displaymath}
    p' \circ \phi \circ h = p \circ h = (f,g).
  \end{displaymath}

  \smallskip
  3. Suponha que $(P(X),c_{X},p_{X})$ seja um objeto de caminhos forte para $X$ e que $(P(Y),c_{Y},p_{Y})$ seja um objeto de caminhos qualquer para $Y$.
  Sendo $B$ cofibrante, segue do item 2 que certamente existe uma homotopia de $f$ para $g$ em termos do objeto de caminhos $P(X)$, ou seja, um morfismo $h: B \to P(X)$ tal que $p_{X} \circ h = (f,g)$.
  A ideia é obtermos um morfismo $\phi: P(X) \to P(Y)$ que nos permita ``empurrar'' a homotopia $h$ ao longo do morfismo $\beta$, e para isso é claro que vamos usar o axioma de levantamento.

  Note que o quadrado abaixo é comutativo, pois
  \begin{displaymath}
    p_{Y} \circ c_{Y} \circ \beta = \Delta_{Y} \circ \beta = \beta \times \beta = (\beta \times \beta) \circ \id_{X} = (\beta \times \beta) \circ \Delta_{X} = (\beta \times \beta) \circ c_{X} \circ p_{X}.
  \end{displaymath}
  Aplicando o axioma de levantamento (M4) obtemos então o levantamento $\phi: P(X) \to P(Y)$ conforme indicado.
  \begin{displaymath}
    \begin{tikzcd}[column sep=1.35cm]
      X
      \arrow[r, "c_{Y} \circ \beta"]
      \arrow[d, tail, "\sim" {sloped}, "c_{X}" {swap}]
      & P(Y)
      \arrow[d, two heads,"p_{Y}"]
      \\ P(X)
      \arrow[r, "(\beta \times \beta) \circ p_{X}" swap]
      \arrow[ru, dashed, "\phi" description]
      & Y \times Y
    \end{tikzcd}
  \end{displaymath}

  Afirmamos que o morfismo composto $\phi \circ h: B \to P(Y)$ define a homotopia à direita desejada então $\beta \circ f$ e $\beta \circ g$.
  De fato, basta notar que
  \begin{displaymath}
    p_{Y} \circ \phi \circ h = (\beta \times \beta) \circ p_{X} \circ h = (\beta \times \beta) \circ (f,g) = (\beta \circ f,\beta \circ g). \qedhere
  \end{displaymath}
\end{proof}

%%% Local Variables:
%%% mode: latex
%%% TeX-master: "../main"
%%% End:
