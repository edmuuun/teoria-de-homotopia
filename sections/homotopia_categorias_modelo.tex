\section{Teoria de Homotopia em categorias modelo}

Nessa seção introduzimos enfim noções homotópicas que podem ser descritas em uma categoria modelo qualquer.
Veremos, entretanto, que mesmo a noção básica de homotopia entre dois morfismos possui sutilezas que a tornam mais complexa do que a noção clássica de homotopia entre mapas contínuos de espaços topológicos.
Felizmente, também veremos que a noção categórica de homotopia se aproxima muito maisda clássica quando trabalhamos apenas com objetos cofibrantes ou fibrantes, e nesse caso podemos usar a construção usual da categoria homotópica para definirmos uma espécie de localização de uma categoria modelo.

A fim de imitarmos a noção topológica de homotopia, o primeiro passo será darmos uma descrição categórica para a construção do cilindro $B \times I$ associado a um espaço topológico $B$ qualquer, onde é claro que $I$ denota o intervalo unitário da reta.

Lembremos inicialmente que, dado um objeto $B$ de uma categoria $\mathsf{C}$ qualquer que admita coprodutos, a propriedade universal dessa construção garante a existência de um único mapa $\nabla: B \sqcup B \to B$ fazendo comutar o diagrama abaixo.
\begin{displaymath}
  \begin{tikzcd}
    B
    \arrow[rd, "j_{1}"]
    \arrow[rrd, bend left=15, "\id_{B}"]
    \\ & B \sqcup B
    \arrow[r, dashed, "\nabla" description]
    & B
    \\ B
    \arrow[ru, "j_{2}" swap]
    \arrow[rru, bend right=15, "\id_{B}" swap]
  \end{tikzcd}
\end{displaymath}
Tal morfismo é comumente chamado de \textbf{morfismo codiagonal} ou também de \textbf{morfismo de dobra}\footnote{Uma tradução direta do inglês \emph{fold map}.}.
Intuitivamente, esse morfismo simplesmente cola duas cópias exatamente uma sobre a outra.
Por vezes, se precisarmos distinguir entre os morfismos codiagonais associados a diferentes objetos, utilizaremos também a notação $\nabla_{B}$ para o morfismo descrito acima.

\begin{defin}
  Sejam $\mathsf{M}$ uma categoria modelo e $B \in \mathsf{M}$ um objeto qualquer.
  Um \textbf{objeto cilindro} para $B$ é uma fatoração do morfismo codiagonal $\nabla: B \sqcup B \to B$ como uma cofibração seguida de uma equivalência fraca.
  Mais explicitamente, um objeto cilindro para $B$ é uma tripla $(\Cyl(B),i,\varepsilon)$, onde $i: B \sqcup B \cofib \Cyl(B)$ é uma cofibração e $\varepsilon: \Cyl(B) \overset{\sim}{\to} B$ é uma equivalência fraca tais que $\varepsilon \circ i = \nabla$, conforme mostrado no diagrama comutativo abaixo.
  \begin{displaymath}
    \begin{tikzcd}
      B \sqcup B
      \arrow[d, tail, "i" swap]
      \arrow[rd, "\nabla"]
      \\ \Cyl(B)
      \arrow[r, "\varepsilon" {swap}, "\sim"]
      & B
    \end{tikzcd}
  \end{displaymath}
  Dizemos que a tripla $(\Cyl(B),i,\varepsilon)$ define um objeto cilindro \textbf{forte} se $\varepsilon$ é também uma fibração, ou seja, se $\varepsilon$ é uma fibração trivial.
\end{defin}

\begin{obs}
  O axioma de fatoração (M5) garante que o morfismo $\nabla: B \sqcup B \to B$ pode ser fatorado como uma cofibração seguida de uma fibração trivial, portanto todo objeto de uma categoria modelo admite um objeto cilindro forte.
  Entretanto, por vezes esse modelo para o objeto cilindro forte dado pelo axioma de fatoração pode ser muito complicado, por isso é vantajoso trabalharmos com cilindros fracos também, cuja descrição é por vezes mais simples.
\end{obs}

Vamos introduzir mais um pouco de terminologia.
Dado  um objeto cilindro $(\Cyl(B),i,\varepsilon)$ para $B$, se $j_{1},\, j_{2}: B \to B \sqcup B$ são as injeções canônicas, frequentemente denotaremos os morfismos do tipo $B \to \Cyl(B)$ dados pelas composições $i \circ j_{1}$ e $i \circ j_{2}$ por $i_{0}$ e $i_{1}$, respectivamente.
Essa diferença nos índices pode parecer estranho, mas a motivação para tal vem do caso topológico clássico, onde um modelo para o objeto cilindro $\Cyl(B)$ é o produto $B \times I$.
Nesse caso, a composição $i_{0} = i \circ j_{1}$ mapeia $B$ para a face inferior $B \times \{0\}$ de $B \times I$, enquanto a composição $i_{1} = i \circ j_{2}$ mapeia $B$ para a face superior $B \times \{1\}$ do cilindro $B \times I$, o que justifica a razoabilidade dos índices aparecendo em cada uma das composições.

Antes de vermos como a noção de objeto cilindro nos permite definir uma noção categórica de homotopia, vejamos algumas propriedades simples satisfeitas por tais objetos.

\begin{lema}\label{lema:props_obj_cilindro}
  Sejam $\mathsf{M}$ uma categoria modelo, $B \in \mathsf{M}$ um objeto qualquer, e $(\Cyl(B),i,\varepsilon)$ um objeto cilindro qualquer para $B$.
  \begin{enumerate}
  \item[(i)] Os morfismos $i_{0},\, i_{1}: B \to \Cyl(B)$ são equivalências fracas.
    
  \item[(ii)] Se $B$ é um objeto cofibrante, então os morfismos $i_{0},\, i_{1}: B \to \Cyl(B)$ são também cofibrações e, portanto, cofibrações triviais.
  \end{enumerate}
\end{lema}

\begin{proof}
  (i) Note que pela definição de $i_{0}$ temos
  \begin{displaymath}
    \varepsilon \circ i_{0} = \varepsilon \circ i \circ j_{1} = \nabla \circ j_{1} = \id_{B}.
  \end{displaymath}
  Ora, como $\id_{B}$ é uma equivalência fraca, o mesmo valendo para $\varepsilon$ pela definição de objeto cilindro, segue da propriedade 2-de-3 que $i_{0}$ é também uma equivalência fraca.
  A demonstração de que $i_{1}$ é uma equivalência fraca segue de um raciocínio completamente análogo.

  \smallskip
  (ii) Lembremos que em qualquer categoria, o coproduto pode ser interpretado também como um pushout sobre o objeto inicial.
  Mais precisamente, se $\varnothing$ denota um objeto inicial de $\mathsf{M}$, então o quadrado comutativo abaixo é um pushout.
  \begin{displaymath}
    \begin{tikzcd}
      \varnothing
      \arrow[r, "!_{B}"]
      \arrow[d, "!_{B}" swap]
      & B
      \arrow[d, "j_{1}"]
      \\ B
      \arrow[r, "j_{2}" swap]
      & B \sqcup B
    \end{tikzcd}
  \end{displaymath}
  Por hipótese $B$ é cofibrante, ou seja, o morfismo único $!_{B}$ é uma cofibração, mas já sabemos do \cref{corol:propriedades_de_preservacao_categoria_modelo} que cofibrações são preservadas por pushouts, portanto $j_{1}$ e $j_{2}$ são cofibrações.
  Mas segue então que os morfismos $i_{0}$ e $i_{1}$ são composições de cofibrações e, portanto, cofibrações também de acordo com o \cref{corol:propriedades_de_preservacao_categoria_modelo}.
\end{proof}

Tendo em mãos essas propriedades básicas, podemos enfim definir uma primeira noção de homotopia entre morfismos em uma categoria modelo.

\begin{defin}
  Dois morfismos $f,\,g: B \to X$ em uma categoria modelo $\mathsf{M}$ são ditos \textbf{homotópicos à esquerda} se existe um objeto cilindro $(\Cyl(B),i,\varepsilon)$ para $B$ e um morfismo $h: \Cyl(B) \to X$ tal que $h \circ i = \langle f,g \rangle$, conforme mostrado no diagrama comutativo abaixo.
  Nesse caso, denotamos essa relação por $f \sim_{L} g$.
  \begin{displaymath}
    \begin{tikzcd}
      B
      & B \sqcup B
      \arrow[l, "\nabla" swap]
      \arrow[r, "{\langle f,g \rangle}"]
      \arrow[d, tail, "i"]
      & X
      \\ & \Cyl(B)
      \arrow[ru, dashed, "h" swap]
      \arrow[lu, "\varepsilon", "\sim" {sloped}]
    \end{tikzcd}
  \end{displaymath}
\end{defin}

Se interpretarmos os morfismos $i_{0},\, i_{1}: B \to \Cyl(B)$ como sendo as faces inferior e superior do cilindro como no caso topológico clássico, então as composições $h \circ i_{0}$ e $h \circ i_{1}$ determinam os estágios inicial e final da homotopia $h$.
Usando a definição de tais morfismos e a comutatividade acima vemos que por um lado
\begin{displaymath}
  h \circ i_{0} = h \circ i \circ j_{1} = \langle f,g \rangle \circ j_{1} = f,
\end{displaymath}
e por outro
\begin{displaymath}
  h \circ i_{1} = h \circ i \circ j_{2} = \langle f,g \rangle \circ j_{2} = g.
\end{displaymath}
Assim, recuperamos em certo sentido a intuição clássica de uma família de morfismos que começa em $f$ e termina em $g$.

Veja que a definição acima possui uma sutileza: exigimos que a homotopia $h$ esteja definida em \emph{algum} objeto cilindro $\Cyl(B)$, mas $B$ pode muito bem admitir diversos objetos cilindros distintos.
Isso pode representar uam dificuldade para ``combinarmos'' homotopias à esquerda, já que elas podem não estar definidas nos mesmos objetos cilindros.

O resultado abaixo mostra que homotopias à esquerda são preservadas por composição de morfismos à esquerda, o que possivelmente justifica a terminologia usada.
Além disso, mostramos que homotopias à esquerda também são preservadas por composição de morfismos à direita quando supomos que o codomínio é fibrante, e essa hipótese também garante a independência do objeto cilindro no qual a homotopia está definida.

\begin{prop}\label{prop:props_homotopia_esquerda}
  Suponha que $f,\,g : B \to X$ sejam dois morfismos homotópicos à esquerda em uma categoria modelo $\mathsf{M}$.

  \begin{enumerate}
  \item Dado um morfismo $\beta: X \to Y$, os morfismos compostos $\beta \circ f,\, \beta \circ g: B \to Y$ são homotópicos à esquerda.
    
  \item Se $X$ é fibrante, então a homotopia entre $f$ e $g$ independe do objeto cilindro usado.
    
  \item Se $X$ é fibrante, então dado qualquer morfismo $\alpha: A \to B$, os morfismos compostos $f \circ \alpha,\, g \circ \alpha: A \to X$ são homotópicos à esquerda.
  \end{enumerate}
\end{prop}

\begin{proof}
  1. Por hipótese existe um objeto cilindro $(\Cyl(B),i,\varepsilon)$ para $B$ e um morfismo $h: \Cyl(B) \to X$ satisfazendo $h \circ i = \langle f,g \rangle$.
  Afirmamos que $\beta \circ h$ é a homotopia à esquerda procurada de $\beta \circ f$ para $\beta \circ g$.
  De fato, basta notar que
  \begin{displaymath}
    \beta \circ h \circ i = \beta \circ \langle f,g \rangle = \langle \beta \circ f, \beta \circ g \rangle,
  \end{displaymath}
  onde a última igualdade segue diretamente da propriedade universal do coproduto, já que a composição $\beta \circ \langle f,g \rangle$ satisfaz as igualdades
  \begin{displaymath}
    (\beta \circ \langle f,g \rangle) \circ j_{1} = \beta \circ f \quad \text{e} \quad (\beta \circ \langle f,g \rangle) \circ j_{2} = \beta \circ g.
  \end{displaymath}

  \smallskip
  2. Por hipótese sabemos que existe \emph{algum} objeto cilindro $(\Cyl(B),i,\varepsilon)$ para $B$ e um morfismo $h: \Cyl(B) \to X$ tal que $h \circ i = \langle f,g \rangle$.
  Afirmamos que existe também uma homotopia de $f$ para $g$ definida em um cilindro \emph{forte}.
  De fato, sabemos do axioma de fatoração que a equivalência fraca $\varepsilon: \Cyl(B) \to B$ pode ser fatorada como uma cofibração seguida de uma fibração trivial conforme indicado abaixo.
  \begin{displaymath}
    \begin{tikzcd}
      \Cyl(B)
      \arrow[rr, "\varepsilon", "\sim" {swap}]
      \arrow[rd, tail, "j" swap]
      & & B
      \\ & C
      \arrow[ru, two heads, "p" {swap}, "\sim" {sloped}]
    \end{tikzcd}
  \end{displaymath}
  Afirmamos então que a tripla $(C,j \circ i,p)$ é um objeto cilindro forte para $B$.
  Veja que $j \circ i$ e $p$ fatoram o morfismo de dobra $\nabla: B \to B \sqcup B$ pois
  \begin{displaymath}
    p \circ (j \circ i) = (p \circ j) \circ i = \varepsilon \circ i = \nabla,
  \end{displaymath}
  e como $p$ é uma fibração trivial pelo axioma de fatoração, a tripla em questão define de fato um cilindro forte.

  Vamos agora mostrar que existe uma homotopia de $f$ para $g$ definida nesse cilindro forte.
  Note inicialmente que o morfismo $j: \Cyl(B) \to C$ que apareceu na fatoração acima é uma cofibração trivial pela propriedade 2-de-3, já que $\varepsilon$ e $p$ são ambos equivalências fracas.
  Como o morfismo único $!_{X}: X \to *$ é por hipótese uma fibração, segue do axioma de levantamento que existe um morfismo diagonal $\widetilde{h}: C \to X$ fazendo comutar o diagrama abaixo.
  \begin{displaymath}
    \begin{tikzcd}
      \Cyl(B)
      \arrow[r, "h"]
      \arrow[d, tail, "j" {swap}, "\sim" {sloped}]
      & X
      \arrow[d, two heads, "!_{X}"]
      \\ C
      \arrow[r, "!_{C}" swap]
      \arrow[ru, dashed, "\widetilde{h}" description]
      & *
    \end{tikzcd}
  \end{displaymath}
  Note então que $\widetilde{h}$ define uma homotopia de $f$ para $g$ com relação ao cilindro forte $(C,j \circ i,p)$ pois
  \begin{displaymath}
    \widetilde{h} \circ (j \circ i)
    = (\widetilde{h} \circ j) \circ i
    = h \circ i
    = \langle f,g \rangle.
  \end{displaymath}

  Vejamos agora como mostrar que a homotopia entre $f$ e $g$ independe do cilindro usado.
  Segue da discussão acima que podemos assumir sem perda de generalidade que o objeto cilindro $(\Cyl(B),i,\varepsilon)$ no qual está definida a homotopia $h$ por hipótese é forte.
  Suponha agora que $(S,i',\varepsilon')$ seja um outro objeto cilindro qualquer.
  Queremos construir um morfismo $h': S \to X$ tal que $h' \circ i' = \langle f,g \rangle$.
  A ideia é construirmos um morfismo do tipo $S \to \Cyl(B)$ que, ao ser composto com a homotopia $h$ já existente, forneça uma outra homotopia definida agora no objeto cilindro $S$.
  A fim de construirmos tal morfismo, note que o quadrado mostrado abaixo é comutativo, pois ambos os pares $(i,\varepsilon)$ e $(i',\varepsilon')$ são fatorações do morfismo de dobra $\nabla$.
  Ademais, como $i'$ é uma cofibração, pois $S$ é um objeto cilindro, e $\varepsilon$ é uma fibração trivial, pois $\Cyl(B)$ é um cilindro forte; segue do axioma de levantamento que existe um morfismo diagonal $\phi: S \to \Cyl(B)$ como indicado.
  \begin{displaymath}
    \begin{tikzcd}
      B \sqcup B
      \arrow[r, tail, "i"]
      \arrow[d, tail, "i'" swap]
      & \Cyl(B)
      \arrow[d, two heads, "\varepsilon", "\sim" {swap,sloped}]
      \\ S
      \arrow[r, "\sim", "\varepsilon'" {swap}]
      \arrow[ru, dashed, "\phi" description]
      & B
    \end{tikzcd}
  \end{displaymath}
  Veja então que a composição $h' \coloneqq h \circ \phi$ define a homotopia desejada, pois da comutatividade acima vemos que
  \begin{displaymath}
    h' \circ i' = h \circ \phi \circ i' = h \circ i = \langle f,g \rangle.
  \end{displaymath}

  \smallskip
  3. Suponha que $(\Cyl(A),i_{A},\varepsilon_{A})$ seja um objeto cilindro qualquer para $A$ e que $(\Cyl(B),i_{B},\varepsilon_{B})$ seja um objeto cilindro \emph{forte} para $B$.
  Sabemos pelo item 2 que certamente existe uma homotopia $h: \Cyl(B) \to X$ de $f$ para $g$ definida nesse cilindro forte.
  Considere o diagrama de levantamento exibido abaixo.
  \begin{displaymath}
    \begin{tikzcd}[column sep=1.25cm]
      A \sqcup A
      \arrow[r, "{i_{B} \circ \alpha \sqcup \alpha}"]
      \arrow[d, tail, "i_{A}" swap]
      & \Cyl(B)
      \arrow[d, two heads, "\varepsilon_{B}", "\sim" {swap,sloped}]
      \\ \Cyl(A)
      \arrow[r, "\alpha \circ \varepsilon_{A}" swap]
      & B
    \end{tikzcd}
  \end{displaymath}
  Veja que o quadrado externo é de fato comutativo pois por um lado
  \begin{displaymath}
    \alpha \circ \varepsilon_{A} \circ i_{A} = \alpha \circ \nabla_{A} = \alpha \circ \langle \id_{A},\id_{A} \rangle = \langle \alpha,\alpha \rangle,
  \end{displaymath}
  e por outro
  \begin{displaymath}
    \varepsilon_{B} \circ i_{B} \circ \alpha \sqcup \alpha = \nabla_{B} \circ \alpha \sqcup \alpha = \langle \alpha,\alpha \rangle.
  \end{displaymath}
  Segue do axioma de levantamento que existe o morfismo diagonal $\phi: \Cyl(A) \to \Cyl(B)$ mostrado acima.
  Note então que $h \circ \phi: \Cyl(A) \to X$ é a homotopia procurada pois
  \begin{displaymath}
    h \circ \phi \circ i_{A} = h \circ i_{B} \circ (\alpha \sqcup \alpha) = \langle f,g \rangle \circ (\alpha \sqcup \alpha) = \langle f \circ \alpha, g \circ \alpha \rangle. \qedhere
  \end{displaymath}
\end{proof}

Uma consequência particularmente útil é que a noção de homotopia à esquerda se torna uma relação de equivalência quando trabalhamos com morfismos tendo codomínio fibrante.

\begin{corol}\label{corol:homotopia_esquerda_rel_equiv_se_dom_cofib}
  Em uma categoria de modelos $(\mathsf{M},\mathcal{W},\mathcal{C},\mathcal{F})$, se $B$ é um objeto cofibrante e $X$ é um objeto qualquer, então a relação de homotopia à esquerda $\simeq_{L}$ define uma relação de equivalência no conjunto de morfismos $\mathsf{M}(B,X)$.
\end{corol}

\begin{proof}
  Vejamos primeiro a reflexividade da relação de homotopia à esquerda.
  Dado um morfismo $f: B \to X$, considere $(\Cyl(B),i,\varepsilon)$ um objeto cilindro qualquer para $B$.
  Veja que o quadrado indicado abaixo é comutativo pois
  \begin{displaymath}
    f \circ \varepsilon \circ i = f \circ \nabla = f \circ \langle \id_{B},\id_{B} \rangle = \langle f,f \rangle,
  \end{displaymath}
  e como o morfismo idêntico $\id_{X}: X \to X$ é uma fibração trivial, e $i: B \sqcup B \to \Cyl(B)$ é uma cofibração por hipótese, segue do axioma de levantamento que existe o morfismo diagonal $h: \Cyl(B) \to X$ mostrado, e a comuatividade mostra que tal morfismo define uma homotopia à esquerda de $f$ para si mesmo.
  \begin{displaymath}
    \begin{tikzcd}
      B \sqcup B
      \arrow[r, "{\langle f,f \rangle}"]
      \arrow[d, tail, "i" swap]
      & X
      \arrow[d, two heads, "\id_{X}", "\sim" {swap, sloped}]
      \\ \Cyl(B)
      \arrow[r, "f \circ \varepsilon" swap]
      \arrow[ru, dashed, "h" description]
      & X
    \end{tikzcd}
  \end{displaymath}
  Note que a demonstração acima não usa em nenhum momento a hipótese de cofibrância sobre $B$, ou seja, a relação de homotopia à esquerda é \emph{sempre} reflexiva.

  Vejamos agora a demonstração da simetria da relação.
  Dados dois morfismos $f,\,g: B \to X$ que sejam homotópicos à esquerda, suponhamos que tal homotopia $h: \Cyl(B) \to X$ esteja definido em um objeto cilindro $(\Cyl(B),i,\varepsilon)$ qualquer.
  A ideia é montarmos um outro objeto cilindro que tenha as ``faces trocadas'', e faremos isso trocando a ordem das parcelas do coproduto $B \sqcup B$.
  Mais precisamente, sabemos da propriedade universal do coproduto que existe um único morfismo $\Sigma: B \sqcup B$ tal que $\Sigma \circ j_{1} = j_{2}$ e $\Sigma \circ j_{2} = j_{1}$, onde $j_{1},\, j_{2}: B \to B \sqcup B$ denotam as injeções canônicas no coproduto.
  Intuitivamente, $\Sigma$ troca as duas cópias de $B$ que compõem o coproduto $B \sqcup B$.
  Uma propriedade do morfismo $\Sigma$ que será importante para nós é sua interação com o morfismo codiagonal $\nabla: B \sqcup B \to B$.
  Mais precisamente, afirmamos que vale a igualdade $\nabla \circ \Sigma = \nabla$.
  Faz bastante sentido que essa igualdade seja verdadeira, pois se $\nabla$ intuitivamente idetifica as duas cópias de $B$ que formam o coproduto $B \sqcup B$, não faz diferença trocarmos essas cópias de lugar antes de fazermos a identificação.
  A demonstração dessa igualdade é uma aplicação direta da propriedade universal do coproduto, pois $\nabla$ é caracterizado unicamente por satisfazer as igualdades
  \begin{displaymath}
    \nabla \circ j_{1} = \nabla \circ j_{2} = \id_{B},
  \end{displaymath}
  mas a composição $\nabla \circ \Sigma$ satisfaz as mesmas igualdades, pois por um lado
  \begin{displaymath}
    (\nabla \circ \Sigma) \circ j_{1} = \nabla \circ (\Sigma \circ j_{1}) = \nabla \circ j_{2} = \id_{B}
  \end{displaymath}
  e por outro
  \begin{displaymath}
    (\nabla \circ \Sigma) \circ j_{2} = \nabla \circ (\Sigma \circ j_{2}) = \nabla \circ j_{1} = \id_{B}.
  \end{displaymath}

  Sabendo das propriedades acima, afirmamos que a tripla $(\Cyl(B),i \circ \Sigma,\varepsilon)$ define um outro objeto cilindro para $B$.
  Note primeiro que o morfismo de troca $\Sigma$ é um automorfismo de $B \sqcup B$, já que uma aplicação direta da propriedade universal do coproduto mostra que vale a igualdade $\Sigma \circ \Sigma = \id_{B}$.
  Segue em particular que $\Sigma$ é uma cofibração, portanto o mesmo é válido para a composição $i \circ \Sigma: B \cofib \Cyl(B)$.
  Por fim, usando a propriedade de $\Sigma$ discutida no parágrafo anterior vemos que $i \circ \Sigma$ e $\varepsilon$ fatoram o morfismo codiagonal, pois
  \begin{displaymath}
    \varepsilon \circ (i \circ \Sigma) = (\varepsilon \circ i) \circ \Sigma = \nabla \circ \Sigma = \nabla.
  \end{displaymath}
  
  Afirmamos então que o próprio morfismo $h: \Cyl(B) \to X$ define a homotopia desejada de $g$ para $f$ com respeito ao objeto cilindro ``trocado'' $(\Cyl(B),i \circ \Sigma,\varepsilon)$.
  De fato, note que por um lado
  \begin{displaymath}
    h \circ (i \circ \Sigma) \circ j_{1}
    = (h \circ i) \circ (\Sigma \circ j_{1})
    = \langle f,g \rangle \circ j_{2}
    = g,
  \end{displaymath}
  e por outro
  \begin{displaymath}
    h \circ (i \circ \Sigma) \circ j_{2}
    = (h \circ i) \circ (\Sigma \circ j_{2})
    = \langle f,g \rangle \circ j_{1}
    = f;
  \end{displaymath}
  portanto $h \circ (i \circ \Sigma) = \langle g,f \rangle$ conforme desejado.
  Note que essa demonstração também não necessita da hipótese de cofibrância sobre $B$, ou seja, a relação de homotopia à esquerda é \emph{sempre} simétrica.

  Resta verificarmos a transitividade da relação, e é aqui onde finalmente usamos a hipótese de cofibrância sobre $B$.
  Suponha que tenhamos três morfismos $f_{1},\,f_{2},\,f_{3}: B \to X$ tais que $f_{1} \simeq_{L} f_{2}$ e $f_{2} \simeq_{L} f_{3}$.
  A primeira homotopia à esquerda é dada por um morfismo $h_{1}: \Cyl(B) \to X$, enquanto a segunda é dada por um morfismo $h_{2}: \Cyl'(B) \to X$, sendo que $(\Cyl(B),i,\varepsilon)$ e $(\Cyl'(B),i',\varepsilon')$ são objetos cilindros possivelmente \emph{distintos}.\footnote{Aqui não podemos usar o resultado de independência de objetos cilindro pois isso requer que o codomínio dos morfismos seja fibrante.}
  A ideia é que podemos colar os dois cilindros juntos de uma forma que vai nos permitir ``concatenar'' as homotopias $h_{1}$ e $h_{2}$ de forma a obtermos uma homotopia de $f_{1}$ para $f_{3}$.
  Inicialmente, consideramos o objeto $Q \in \mathsf{M}$ dado pelo diagrama de pushout abaixo, onde $i_{1}: B \to \Cyl(B)$ e $i_{0}': B \to \Cyl'(B)$ são componentes das cofibrações $i: B \sqcup B \to \Cyl(B)$ e $i': B \sqcup B \to \Cyl'(B)$, respectivamente.
  \begin{displaymath}
    \begin{tikzcd}
      B
      \arrow[r, "i_{1}"]
      \arrow[d, "i_{0}'" swap]
      & \Cyl(B)
      \arrow[d, "b"]
      \\ \Cyl'(B)
      \arrow[r, "t" swap]
      & Q
    \end{tikzcd}
  \end{displaymath}
  A hipótese de cofibrância sobre $B$ garante por meio do \cref{lema:props_obj_cilindro} que os morfismos $i_{1}$ e $i_{0}'$ são cofibrações triviais, e como esse tipo de morfismo é preservado por pushouts, seque que os morfismos $b: \Cyl(B) \to Q$ e $t: \Cyl'(B) \to Q$ que aparecem acima são também cofibrações triviais.
  Como os morfismos $\varepsilon: \Cyl(B) \to B$ e $\varepsilon': \Cyl'(B) \to B$ satisfazem as igualdades $\varepsilon \circ i_{1} = \varepsilon' \circ i_{0}' = \id_{B}$, segue da propriedade universal do pushout a existência de um morfismo $\widetilde{\varepsilon}: Q \to B$ fazendo comutar o diagrama mostrado abaixo.
  \begin{displaymath}
    \begin{tikzcd}
      B
      \arrow[r, "i_{1}"]
      \arrow[d, "i_{0}'" swap]
      & \Cyl(B)
      \arrow[d, "b"]
      \arrow[rdd, bend left=20, "\varepsilon"]
      \\ \Cyl'(B)
      \arrow[r, "t" swap]
      \arrow[rrd, bend right=20, "\varepsilon'" swap]
      & Q
      \arrow[rd, dashed, "\widetilde{\varepsilon}" description]
      \\ & & B
    \end{tikzcd}
  \end{displaymath}
  Note que, como $\varepsilon$ e $b$ são equivalências fracas, o mesmo é válido para $\widetilde{\varepsilon}$ por conta da igualdade $\widetilde{\varepsilon} \circ b = \varepsilon$ e da propriedade 2-de-3.

  Considere agora o morfismo $\widetilde{i}: B \sqcup B \to Q$ definido como $\widetilde{i} \coloneqq \langle b \circ i_{0}, t \circ i_{1}' \rangle$ por meio da propriedade universal do coproduto $B \sqcup B$.
  Afirmamos que $\widetilde{\varepsilon}$ e $\widetilde{i}$ juntos fatoram o morfismo codiagonal.
  De fato, por um lado temos
  \begin{displaymath}
    \widetilde{\varepsilon} \circ \widetilde{i} \circ j_{1}
    = \widetilde{\varepsilon} \circ b \circ i_{0}
    = \varepsilon \circ i_{0}
    = \id_{B},
  \end{displaymath}
  e por outro temos também
  \begin{displaymath}
    \widetilde{\varepsilon} \circ \widetilde{i} \circ j_{2}
    = \widetilde{\varepsilon} \circ t \circ i_{1}'
    = \varepsilon' \circ i_{1}'
    = \id_{B};
  \end{displaymath}
  portanto a propriedade universal do coproduto implica a igualdade $\widetilde{\varepsilon} \circ \widetilde{i} = \nabla$ desejada.

  Veja que como a homotopia $h_{1}: \Cyl(B) \to X$ ``termina'' em $f_{2}$, enquanto a homotopia $h_{2}: \Cyl'(B) \to X$ ``começa'' em $f_{2}$, ou seja, temos as igualdades $h \circ i_{1} = h \circ i_{0}' = f_{2}$, a camada externa do diagrama abaixo comuta, portanto pela propriedade universal do pushout obtemos o morfismo $h: Q \to X$ indicado.
  \begin{displaymath}
     \begin{tikzcd}
      B
      \arrow[r, "i_{1}"]
      \arrow[d, "i_{0}'" swap]
      & \Cyl(B)
      \arrow[d, "b"]
      \arrow[rdd, bend left=20, "h_{1}"]
      \\ \Cyl'(B)
      \arrow[r, "t" swap]
      \arrow[rrd, bend right=20, "h_{2}" swap]
      & Q
      \arrow[rd, dashed, "h" description]
      \\ & & B
    \end{tikzcd}
  \end{displaymath}
  Afirmamos que o morfismo $h$ construído dessa forma satisfaz a igualdade $h \circ \widetilde{i} = \langle f_{1},f_{3} \rangle$.
  De fato, pré-compondo com a injeção $j_{1}$ no coproduto vemos que
  \begin{displaymath}
    h \circ \widetilde{i} \circ j_{1} = h \circ b \circ i_{0} = h_{1} \circ i_{0} = f_{1},
  \end{displaymath}
  e pré-compondo com a outra injeção $j_{2}$ vemos também que
  \begin{displaymath}
    h \circ \widetilde{i} \circ j_{2} = h \circ t \circ i_{1}' = h_{2} \circ i_{1}' = f_{3};
  \end{displaymath}
  de forma que a igualdade desejada segue da propriedade universal do coproduto novamente.

  Infelizmente, embora isso pareça muito ser verdade, $h$ \emph{não} define uma homotopia de $f_{1}$ para $f_{3}$, pois a tripla $(Q,\widetilde{i},\widetilde{\varepsilon})$ \emph{não} define um objeto cilindro para $B$ já que $\widetilde{i}$ pode não ser uma cofibração.
  Felizmente, podemos usar o axioma de fatoração para corrigirmos esse defeito sem grandes dores de cabeça.
  Mais precisamente, usando tal axioma podemos reescrever $\widetilde{i}$ como uma cofibração $\widehat{i}: B \sqcup B \to \widehat{Q}$ seguida de uma fibração trivial $p: \widehat{Q} \overset{\sim}{\fib} C$ conforme mostrado abaixo.
  \begin{displaymath}
    \begin{tikzcd}
      B \sqcup B
      \arrow[rr, "\widetilde{i}"]
      \arrow[rd, tail, "\widehat{i}" swap]
      & & Q
      \\ & \widehat{Q}
      \arrow[ru, two heads, "\sim" {sloped}, "p" {swap}]
    \end{tikzcd}
  \end{displaymath}
  Veja que a tripla $(\widehat{Q},\widehat{i},\widetilde{\varepsilon} \circ p)$ define realmente um objeto cilindro: $\widehat{i}$ é uma cofibração por hipótese, $\widetilde{\varepsilon} \circ p$ é a composição de duas equivalências fracas e, portanto, uma equivalência fraca também; e os dois morfismos em questão fatoram a codiagonal pois
  \begin{displaymath}
    \widetilde{\varepsilon} \circ p \circ \widehat{i} = \widetilde{\varepsilon} \circ \widetilde{i} = \nabla.
  \end{displaymath}
  Note agora que o morfismo composto $h \circ p: \widehat{Q} \to X$ define enfim a homotopia procurada com relação a esse verdadeiro objeto cilindro como mostra a sequeência de igualdades
  \begin{displaymath}
    h \circ p \circ \widehat{i} = h \circ \widetilde{i} = \langle f_{1},f_{3} \rangle. \qedhere
  \end{displaymath}
\end{proof}

Tendo em vista o \cref{corol:homotopia_esquerda_rel_equiv_se_dom_cofib}, se $X$ é um objeto cofibrante, então para qualquer outro objeto $Y$ podemos formar o conjunto quociente
\begin{displaymath}
  [X,Y]_{\ell} \coloneqq \mathsf{M}(X,Y)/\simeq_{\ell}
\end{displaymath}
cujos elementos chamaremos de \textbf{classes de homotopia à esquerda} de morfismos.
Dado um morfismo qualquer$f: X \to Y$, denotaremos por $[f]_{\ell}$ sua imagem no quociente $[X,Y]_{\ell}$.

Note que, se $\beta: Y \to Z$ é um morfismo \emph{qualquer}, podemos temos a função $\mathsf{M}(X,\beta): \mathsf{M}(X,Y) \to \mathsf{M}(X,Z)$ de pushforward (pós-composição) ao longo de $\beta$.
Sabemos da \cref{prop:props_homotopia_esquerda} que homotopias à esquerda são preservadas por pushforwards, a composição $\pi_{X,Z} \circ \mathsf{M}(X,\beta): \mathsf{M}(X,Y) \to [X,Z]_{\ell}$ é constante nas classes de homotopia à esquerda, portanto existe uma únicafunção induzida $[X,\beta]_{\ell}: [X,Y]_{\ell} \to [X,Z]_{\ell}$ fazendo comutar o diagrama abaixo,
\begin{displaymath}
  \begin{tikzcd}[column sep=1.25cm]
    \mathsf{M}(X,Y)
    \arrow[r, "{\mathsf{M}(X,\beta)}"]
    \arrow[d, "{\pi_{X,Y}}" swap]
    & \mathsf{M}(X,Z)
    \arrow[d, "{\pi_{X,Z}}"]
    \\ {[X,Y]_{\ell}}
    \arrow[r, dashed, "{[X,\beta]_{\ell}}" swap]
    & {[X,Z]_{\ell}}
  \end{tikzcd}
\end{displaymath}
onde $\pi_{X,Y}: \mathsf{M}(X,Y) \to [X,Y]_{\ell}$ e $\pi_{X,Z}: \mathsf{M}(X,Z) \to [X,Z]_{\ell}$ denotam as projeções canônicas no quociente.
Explicitamente, dada uma classe de homotopia à esquerda $[f]_{\ell} \in [X,Y]_{\ell}$, pela comutatividade do quadrado acima temos
\begin{displaymath}
  [X,\beta]_{\ell}([f]_{\ell}) = [\beta \circ f]_{\ell}.
\end{displaymath}

\begin{defin}\label{defin:obj_de_caminhos}
  Sejam $\mathsf{M}$ uma categoria modelo e $X \in \mathsf{M}$ um objeto qualquer.
  Um \textbf{objeto de caminhos} para $B$ é uma fatoração do morfismo diagonal $\Delta: X \to X \times X$ como uma equivalência fraca seguida de uma fibração.
  Mais explicitamente, um objeto de caminhos para $B$ é uma tripla $(P(X),c,p)$, onde $P(X) \in \mathsf{M}$ é um objeto da categoria, $c: X \overset{\sim}{\to} P(X)$ é uma equivalência fraca, e $p: P(X) \fib X \times X$ é uma fibração tais que $\Delta = p \circ c$, conforme mosrado no diagrama abaixo.
  \begin{displaymath}
    \begin{tikzcd}
      & P(X)
      \arrow[d, two heads, "p"]
      \\ X
      \arrow[ru, "c", "\sim" {swap,sloped}]
      \arrow[r, "\Delta" swap]
      & X \times X
    \end{tikzcd}
  \end{displaymath}
  Quando o morfismo $c$ é também uma cofibração, portanto uma cofibração trivial, dizemos que a tripla $(P(X),c,p)$ define um objeto de caminhos \textbf{forte} para $X$.
\end{defin}

\begin{obs}
  O axioma de fatoração (M5) garante que o morfismo diagonal $\Delta: X \times X \to X$ pode ser fatorado como uma cofibração trivial seguida de uma fibração, portanto todo objeto de uma categoria modelo admite um objeto de caminhos que é até mesmo forte.
  Como no caso do objetos cilindro, nem sempre esse modelo obtido pelo axioma de fatoração é o mais conveniente para trabalharmos, de forma que é vantajoso considerarmos também objetos de caminhos que não necessariamente sejam fortes.
\end{obs}

Neste caso temos também algumas notações associadas.
Dado um objeto de caminhos $(P(X),c,p)$ para $X$, se $\pi_{1},\,\pi_{2}: X \times X \to X$ são as projeções canônicas associadas ao produto, denotaremos os morfismos compostos $\pi_{1} \circ p,\,pi_{2} \circ p: P(X) \to X$ por $p_{0}$ e $p_{1}$, respectivamente.
Note então que a fibração $p$ é precisamente o morfismo induzido por $p_{0}$ e $p_{1}$ por meio da propriedade universal do produto, ou seja, $p = (p_{0},p_{1})$.
Como no caso de objetos cilindros, os índices usados na notação ficam claros quando examinamos o caso topológico clássico, onde o objeto de caminhos é dado pelo espaço de mapas $X^{I}$ munido da topologia compacto-aberta, a equivalência fraca $c: X \to X^{I}$ associa a cada ponto $x \in X$ o caminho $c(b): I \to X$ constante naquele ponto, e a fibração $p: X^{I} \to X \times X$ associa a um caminho $\gamma \in B^{X}$ seus pontos inicial e final, ou seja, $p(\gamma) \coloneqq (\gamma(0),\gamma(1))$.
Nesse caso, os mapas $p_{0},\,p_{1}: X^{I} \to B$ introduzidos acima são simplesmente os mapas de avaliação no instante inicial $0$ e no instante final $1$, respectivamente, o que justifica nossa escolha de índice na notação usada para tais morfismos.

Vamos agora percorrer um caminho completamente análogo ao que percorremos para objetos cilindros.
Começamos inicialmente verificando algumas das propriedades básicas satisfeitas por objetos de caminhos.

\begin{lema}\label{lema:props_obj_de_caminhos}
  Sejam $\mathsf{M}$ uma categoria modelo, $X \in \mathsf{M}$ um objeto, e $(P(X),c,p)$ um objeto de caminhos qualquer para $X$.
  \begin{enumerate}
  \item[(i)] Os morfismos $p_{0},\, p_{1}: P(X) \to X$ são equivalência fracas.
    
  \item[(ii)] Se $X$ é um objeto fibrante, então os morfismos $p_{0},\, p_{1}: P(X) \to X$ são também fibrações e, portanto, fibrações triviais.
  \end{enumerate}
\end{lema}

\begin{proof}
  (i) Note que pela definição de $p_{0}$ e do morfismo diagonal $\Delta$ temos
  \begin{displaymath}
    p_{0} \circ c = \pi_{1} \circ p \circ c = \pi_{1} \circ \Delta = \id_{X}.
  \end{displaymath}
  Ora, como $\id_{X}$ é uma equivalência fraca, o mesmo valendo para $c$ pela definição de objeto de caminhos, segue da propriedade 2-de-3 que $p_{0}$ é uma equivalência fraca.
  A demonstração de que $p_{1}$ também é uma equivalência fraca segue de um raciocínio completamente análogo.

  \smallskip
  (ii) Lembremos que em qualquer categoria, um produto por der interpretado como um pullback sobre o objeto final.
  Mais precisamente, se $* \in \mathsf{M}$ denota um objeto final qualquer de $\mathsf{M}$, então o quadrado comutativa abaixo é um pullback.
  \begin{displaymath}
    \begin{tikzcd}
      X \times X
      \arrow[r, "\pi_{1}"]
      \arrow[d, "\pi_{2}" swap]
      & B
      \arrow[d, "!_{B}"]
      \\ B
      \arrow[r, "!_{B}" swap]
      & *
    \end{tikzcd}
  \end{displaymath}
  Como $X$ é por hipótese fibrante, ou seja, o morfismo $!_{X}: X \to *$ é uma fibração, e como fibrações são preservadas por pullbacks de acordo com a \cref{corol:propriedades_de_preservacao_categoria_modelo}, concluímos que as projeções canônicas $\pi_{1}$ e $\pi_{2}$ são fibrações também.
  Segue então que os morfismos $p_{0} = \pi_{1} \circ p$ e $p_{1} = \pi_{2} \circ p$ são composições de fibrações e, portanto, fibrações também de acordo com o \cref{corol:propriedades_de_preservacao_categoria_modelo} novamente.
\end{proof}

Objetos de caminhos dão origem a uma noção de homotopia entre morfismos que é dual à noção de homotopia à esquerda introduzida anteriormente em termos de objetos cilindros.

\begin{defin}\label{defin:homotopia_direita}
  Dois morfismos $f,\, g: B \to X$ em uma categoria modelo $\mathsf{M}$ são ditos \textbf{homotópicos à direita} se existe algum objeto de caminhos $(P(X),c,p)$ para $X$ e um morfismo $h: B \to P(X)$ tal que $p \circ h = (f,g)$, conforme mostrado no diagrama abaixo.
  Neste caso, denotamos essa relação por $f \sim_{r} g$.
  \begin{displaymath}
    \begin{tikzcd}
      & P(X)
      \arrow[d, two heads, "p"]
      \\ B
      \arrow[ru, dashed, "h"]
      \arrow[r, "{(f,g)}" swap]
      & X \times X
      & X
      \arrow[lu, "\sim" {sloped, swap}, "c" swap]
      \arrow[l, "\Delta"]
    \end{tikzcd}
  \end{displaymath}
\end{defin}

Veja que a definição acima recupera a definição clássica da Topologia.
O mapa $h: B \to X^{I}$ define uma família de caminhos em $X$ parametrizada pelos pontos do espaço $B$.
Para cada $b \in B$, o caminho associada $h(b)$ tem $f(b)$ como ponto inicial pois
\begin{displaymath}
  [h(b)](0) = p_{0}(h(b)) = \pi_{1}(p(h(b))) = \pi_{1}(f(b),g(b)) = f(b),
\end{displaymath}
e analogamente, $h(b)$ tem como ponto final a imagem $g(b)$.
Assim, a família de caminhos definida pela homotopia $h$ em certo sentido ``conecta'' a imagem do estágio inicial $f$ da homotopia ao estágio final $g$ da mesma.

Nosso objetivo agora é demonstrar um resultado análogo à \cref{prop:props_homotopia_esquerda} para homotopias à direita.

\begin{prop}\label{prop:props_homotopia_direita}
  Suponha que $f,\,g: B \to X$ sejam dois morfismos homotópicos à direita em uma categoria modelo $\mathsf{M}$.
  \begin{enumerate}
  \item Dado um morfismo $\alpha: A \to B$, os morfismos compostos $f \circ \alpha,\, g \circ \alpha: A \to X$ são homotópicos à direita.
    
  \item Se $B$ é cofibrante, então a homotopia entre $f$ e $g$ independe do objeto de caminhos para $X$.
    
  \item Se $B$ é cofibrante, então dado qualquer morfismo $\beta: X \to Y$, os morfismos compostos $\beta \circ f,\, \beta \circ g: B \to Y$ são homotópicos à direita.
  \end{enumerate}
\end{prop}

%TODO: Revisar a demonstração abaixo e deixá-la mais clara.

\begin{proof}
  1. Como $f$ e $g$ são homotópicos à direita, existe \emph{algum} objeto de caminhos $(P(X),c,p)$ para $X$ juntamente com um morfismo $h: B \to P(X)$ tal que $p \circ h = (f,g)$.
  Afirmamos que o morfismo composto $h \circ \alpha: A \to P(X)$ define uma homotopia à direita de $f \circ \alpha$ para $g \circ \alpha$.
  De fato, basta para isso notarmos que
  \begin{displaymath}
    p \circ h \circ \alpha = (f,g) \circ \alpha = (f \circ \alpha, g \circ \alpha),
  \end{displaymath}
  sendo que a última igualdade é uma consequência direta da propriedade universal que caracteriza o produto $X \times X$.

  \smallskip
  2. Considere um objeto de caminhos $(P(X),c,p)$ para $X$ juntamente com uma homotopia à direita $h: B \to P(X)$ de $f$ para $g$ com relação a esse objeto de caminhos.
  Afirmamos inicialmente que podemos encontrar uma outra homotopia à direita entre $f$ e $g$ que esteja definida em um objeto de caminhos \emph{forte} para $X$.
  Inicialmente, aplicamos o axioma de fatoração (M5) à equivalência fraca $c: X \to P(X)$ para obtermos uma cofibração trivial $i: X \overset{\sim}{\cofib} P$ seguida de uma fibração $q: P \fib P(X)$ tais que $c = q \circ i$, conforme mostrado abaixo.
  \begin{displaymath}
    \begin{tikzcd}
      X
      \arrow[rr, "c", "\sim" {swap}]
      \arrow[rd, tail, "i" {swap}, "\sim" {sloped}]
      & & P(X)
      \\ & P
      \arrow[ru, two heads, "q" swap]
    \end{tikzcd}
  \end{displaymath}
  Afirmamos que a tripla $(P,i,p \circ q)$ define um objeto de caminhos forte para $X$.
  De fato, $i$ é uma cofibração trivial pela fatoração acima, enquanto $p \circ q$ é a composição de duas fibrações e, portanto, uma fibração também.
  Além disso, esses morfismos fatoram o morfismo diagonal de $X$ já que
  \begin{displaymath}
    (p \circ q) \circ i = p \circ (q \circ i) = p \circ c = \Delta.
  \end{displaymath}

  Tendo em mãos o objeto de caminhos forte acima, vejamos como obter uma homotopia que tome valores nele a partir da homotopia $h$ já existente.
  Veja inicialmente que o problema de levantamento abaixo admite uma solução $H: B \to P$, já que $!_{B}: \varnothing \to B$ é uma cofibração pela hipótese de $B$ ser cofibrante, e $q: P \to P(X)$ é uma fibração trivial pela fatoração acima e pela propriedade 2-de-3.
  \begin{displaymath}
    \begin{tikzcd}
      \varnothing
      \arrow[d, tail, "!_{B}" swap]
      \arrow[r, "!_{P}"]
      & P
      \arrow[d, two heads, "q", "\sim" {sloped, swap}]
      \\ B
      \arrow[r, "h" swap]
      \arrow[ru, dashed, "H" description]
      & P(X)
    \end{tikzcd}
  \end{displaymath}
  Esse morfismo $H$ é precisamente a homotopia à direita procurada, pois
  \begin{displaymath}
    (p \circ q) \circ H = p \circ (q \circ H) = p \circ h = (f,g).
  \end{displaymath}

  Podemos assumir então sem perda de generalidade que o objeto de caminhos $(P(X),c,p)$ no qual está definida a homotopia à direita entre $f$ e $g$ é forte.
  Se $(P',c',p')$ é outro objeto de caminhos \emph{qualquer} para $X$, vejamos como construir uma homotopia à direita $h': B \to P'$ tomando valores nesse objeto.
  Note primeiro que o problema de levantamento dado pelo quadrado comutativo abaixo admite uma solução $\phi$, já que $c$ é uma cofibração trivial, e $p'$ é uma fibração.
  \begin{displaymath}
    \begin{tikzcd}
      X
      \arrow[r, "c'"]
      \arrow[d, tail, "c" {swap}, "\sim" {sloped}]
      & P'
      \arrow[d, two heads, "p'"]
      \\ P(X)
      \arrow[ru, dashed, "\phi" description]
      \arrow[r, "p" swap]
      & X \times X
    \end{tikzcd}
  \end{displaymath}
  Basta notar agora que a composição $\phi \circ h: B \to P'$ define a homotopia à direita desejada, pois
  \begin{displaymath}
    p' \circ \phi \circ h = p \circ h = (f,g).
  \end{displaymath}

  \smallskip
  3. Suponha que $(P(X),c_{X},p_{X})$ seja um objeto de caminhos forte para $X$ e que $(P(Y),c_{Y},p_{Y})$ seja um objeto de caminhos qualquer para $Y$.
  Sendo $B$ cofibrante, segue do item 2 que certamente existe uma homotopia de $f$ para $g$ em termos do objeto de caminhos $P(X)$, ou seja, um morfismo $h: B \to P(X)$ tal que $p_{X} \circ h = (f,g)$.
  A ideia é obtermos um morfismo $\phi: P(X) \to P(Y)$ que nos permita ``empurrar'' a homotopia $h$ ao longo do morfismo $\beta$, e para isso é claro que vamos usar o axioma de levantamento.

  Note que o quadrado abaixo é comutativo, pois
  \begin{displaymath}
    p_{Y} \circ c_{Y} \circ \beta = \Delta_{Y} \circ \beta = \beta \times \beta = (\beta \times \beta) \circ \id_{X} = (\beta \times \beta) \circ \Delta_{X} = (\beta \times \beta) \circ c_{X} \circ p_{X}.
  \end{displaymath}
  Aplicando o axioma de levantamento (M4) obtemos então o levantamento $\phi: P(X) \to P(Y)$ conforme indicado.
  \begin{displaymath}
    \begin{tikzcd}[column sep=1.35cm]
      X
      \arrow[r, "c_{Y} \circ \beta"]
      \arrow[d, tail, "\sim" {sloped}, "c_{X}" {swap}]
      & P(Y)
      \arrow[d, two heads,"p_{Y}"]
      \\ P(X)
      \arrow[r, "(\beta \times \beta) \circ p_{X}" swap]
      \arrow[ru, dashed, "\phi" description]
      & Y \times Y
    \end{tikzcd}
  \end{displaymath}

  Afirmamos que o morfismo composto $\phi \circ h: B \to P(Y)$ define a homotopia à direita desejada então $\beta \circ f$ e $\beta \circ g$.
  De fato, basta notar que
  \begin{displaymath}
    p_{Y} \circ \phi \circ h = (\beta \times \beta) \circ p_{X} \circ h = (\beta \times \beta) \circ (f,g) = (\beta \circ f,\beta \circ g). \qedhere
  \end{displaymath}
\end{proof}

O próximo passo no nosso estudo de homotopias em categorias de modelos é dualizarmos o \cref{corol:homotopia_esquerda_rel_equiv_se_dom_cofib} para obtermos um resultado análogo para homotopias à direita.

\begin{corol}\label{corol:homotopia_direita_rel_equiv_se_cod_fib}
  Em uma categoria de modelos $(\mathsf{M},\mathcal{W},\mathcal{C},\mathcal{F})$, se $B$ é um objeto qualquer, e $X$ é um objeto fibrante, então a relação de homotopia à direita $\simeq_{r}$ define uma relação de equivalência no conjunto de morfismos.
\end{corol}

\begin{proof}
  Dado um morfismo $f: B \to X$ qualquer, note primeiro que o morfismo idêntico $\id_{X}$ é certamente homotópico à direita a si mesmo, pois se $(P(X),c,p)$ é um objeto de caminhos qualquer para $X$, o próprio morfismo $c: X \to P(X)$ define a homotopia à direita mencionada já que $p \circ c = \Delta = (\id_{X},\id_{X})$.
  Ora, sabendo então que $\id_{X} \simeq_{r} \id_{X}$ e que homotopias à direita são preservadas por composições à direita pela \cref{prop:props_homotopia_direita}, segue que $\id_{X} \circ f \simeq_{r} \id_{X} \circ f$, ou seja, $f \simeq_{r} f$; mostrando assim a reflexividade da relação.
  De forma similar ao que ocorreu com homotopias à esquerda, note que essa demonstração não depende da hipótese de fibrância sobre $X$, ou seja, a relação de homotopia à direita é \emph{sempre} reflexiva.

  Vejamos agora a questão da simetria da relação.
  A estratégia é completamente análoga à que empregamos na demonstração da simetria da relação de homotopia à esquerda.
  Dados dois morfismos $f,\,g: B \to X$ e uma homotopia $h: B \to P(X)$ de $f$ para $g$ tomando valores em um objeto de caminhos $(P(X),c,p)$ qualquer para $X$, precisamos produzir um outro objeto de caminhos no qual os pontos inicial e final estejam trocados, e a estratégia para conseguirmos isso é trocarmos a ordem dos fatores do produto $B \times B$.
  Mais precisamente, se $\pi_{1},\, \pi_{2}: X \times X  \to X$ denotam as projeções canônicas, consideramos o morfismo $\Sigma \coloneqq (\pi_{2},\pi_{1}): X \times X \to X \times X$.
  Note que as igualdades
  \begin{displaymath}
    \pi_{1} \circ (\Sigma \circ \Delta) = (\pi_{1} \circ \Sigma) \circ \Delta = \pi_{2} \circ \Delta = \id_{X}
  \end{displaymath}
  juntamente com as igualdades
  \begin{displaymath}
    \pi_{2} \circ (\Sigma \circ \Delta) = (\pi_{2} \circ \Sigma) \circ \Delta = \pi_{1} \circ \Delta = \id_{X}
  \end{displaymath}
  mostram que o morfismo diagonal é invariante pelo morfismo $\Delta$, ou seja, vale a igualdade $\Sigma \circ \Delta = \Delta$.
  Uma aplicação similar da propriedade universal do produto mostra que o morfismo $\Sigma$ satisfaz também a igualdade $\Sigma \circ \Sigma = \id_{X \times X}$, de onde concluímos, em particular, que $\Sigma$ é um isomorfismo e portanto uma fibração trivial.

  Afirmamos agora que a tripla $(P(X),c,\Sigma \circ p)$ define um objeto de caminhos para $X$.
  De fato, já sabemos que $c$ é uma equivalência fraca, $\Sigma \circ p: P(X) \to X \times X$ é a composição de duas fibrações, logo uma fibração também, e usando as propriedades acima vemos que
  \begin{displaymath}
    (\Sigma \circ p) \circ c = \Sigma \circ (p \circ c) = \Sigma \circ \Delta = \Delta;
  \end{displaymath}
  portanto temos uma outra fatoração do morfismo diagonal.
  Basta notar agora que $h: B \to P(X)$ satisfaz as igualdades
  \begin{displaymath}
    \pi_{1} \circ (\Sigma \circ p) \circ h = (\pi_{1} \circ \Sigma) \circ (p \circ h) = \pi_{2} \circ (f,g) = g
  \end{displaymath}
  e também
  \begin{displaymath}
    \pi_{2} \circ (\Sigma \circ p) \circ h = (\pi_{2} \circ \Sigma) \circ (p \circ h) = \pi_{1} \circ (f,g) = f;
  \end{displaymath}
  portanto $h$ define uma homotopia à direita de $g$ para $f$ com relação ao objeto de caminhos ``trocado'' $(P(X),c, \Sigma \circ p)$.
  Veja que essa demonstração também não exige a condição de fibrância sobre $X$, ou seja, a relação de homotopia à direita é \emph{sempre} simétrica.

  Resta apenas mostrarmos a transitividade da relação, e aqui a condição de fibrância de $X$ será essencial.
  Suponha qu tenhamos três morfismos $f_{1},\, f_{2},\, f_{3}: B \to X$ juntamente com homotopias à direita $f_{1} \simeq_{r} f_{2}$ e $f_{2} \simeq_{r} f_{3}$ dadas, respectivamente, por morfismos $h_{1}: B \to P(X)$ e $h_{2}: B \to P'(X)$, onde $(P(X),c,p)$ e $(P'(X),c',p')$ são dois objetos de caminhos possivelmente distintos.
  Analogamente ao que fizemos  no caso de homotopias à esquerda, precisamos de alguma forma combinar os objetos $P(X)$ e $P'(X)$ de forma a obtermos um outro objeto de caminhos onde faça sentido a concatenação das homotopias $h_{1}$ e $h_{2}$.
  O primeiro passo nesse sentido é formarmos o pullback $P$ indicado abaixo,
  \begin{displaymath}
    \begin{tikzcd}
      P
      \arrow[r, "\phi"]
      \arrow[d, "\phi'" swap]
      & P(X)
      \arrow[d, "p_{1}"]
      \\ P'(X)
      \arrow[r, "p_{0}'" swap]
      & X
    \end{tikzcd}
  \end{displaymath}
  onde $p_{1}: P(X) \to X$ e $p_{0}': P'(X) \to X$ são componentes das fibrações $p: P(X) \to X \times X$ e $p': P'(X) \to X \times X$, respectivamente.
  A hipótese de fibrância de $X$ garante que tais morfismos sejam fibrações triviais de acordo com o \cref{lema:props_obj_cilindro}, portanto os morfismos $\phi$ e $\phi'$ que aparecem no pullback acima são também fibrações triviais graças ao \cref{corol:propriedades_de_preservacao_categoria_modelo}.

  As igualdades $p_{1} \circ c = p_{0}' \circ  = \id_{X}$ dão origem por meio da propriedade universal do pullback a um morfismo $\overline{c}: X \to P$ fazendo comutar o diagrama mostrado abaixo.
  \begin{displaymath}
    \begin{tikzcd}
      X
      \arrow[rrd, bend left=20, "c"]
      \arrow[rdd, bend right=20, "c'" swap]
      \arrow[rd, dashed, "\overline{c}" description]
      \\ & P
      \arrow[r, "\phi"]
      \arrow[d, "\phi'" swap]
      & P(X)
      \arrow[d, "p_{1}"]
      \\ & P'(X)
      \arrow[r, "p_{0}'" swap]
      & X
    \end{tikzcd}
  \end{displaymath}
  Note que o fato de $\phi$ e $c$ serem equivalências fracas, juntamente com a igualdade $\phi \circ \overline{c} = c$ garantem que $\overline{c}$ seja uma equivalência fraca também graças à propriedade 2-de-3.

  O pullback $P$ vem equipado também com um morfismo $q: P \to X \times X$ devido por meio da propriedade universal do produto como $q \coloneqq (p_{0} \circ \phi, p_{1}' \circ \phi')$.
  Esse morfiso $q$ define juntamente com o morfismo $\overline{c}$ obtido acima uma fatoração do morfismo diagonal de $X$, fato este que segue diretamente das igualdades
  \begin{displaymath}
    \pi_{1} \circ q \circ \overline{c} = p_{0} \circ \phi \circ \overline{c} = p_{0} \circ c = \id_{X}
  \end{displaymath}
  e também das igualdades
  \begin{displaymath}
    \pi_{2} \circ q \circ \overline{c} = p_{1}' \circ \phi' \circ \overline{c} = p_{1}' \circ c' = \id_{X}.
  \end{displaymath}

  As homotopias $h_{1}$ e $h_{2}$ consideradas inicialmente satisfazem as igualdades $p_{1} \circ h_{1} = p_{0}' \circ h_{2} = f_{2}$, logo a propriedade universal do pullback fornece então um morfismo $h: B \to P$ fazendo comutar o diagrama mostrado abaixo.
  \begin{displaymath}
    \begin{tikzcd}
      B
      \arrow[rrd, bend left=20, "h_{1}"]
      \arrow[rdd, bend right=20, "h_{2}" swap]
      \arrow[rd, dashed, "h" description]
      \\ & P
      \arrow[r, "\phi"]
      \arrow[d, "\phi'" swap]
      & P(X)
      \arrow[d, "p_{1}"]
      \\ & P'(X)
      \arrow[r, "p_{0}'" swap]
      & X
    \end{tikzcd}
  \end{displaymath}
  Afirmamos que esse morfismo $h$ satisfaz a igualdade $q \circ h = (f_{1},f_{3})$.
  De fato, por um lado temos
  \begin{displaymath}
    \pi_{1} \circ q \circ h = p_{0} \circ \phi \circ h = p_{0} \circ h_{1} = f_{1},
  \end{displaymath}
  enquanto por outro temos
  \begin{displaymath}
    \pi_{2} \circ q \circ h = p_{1}' \circ \phi' \circ h = p_{1}' \circ h_{2} = f_{3};
  \end{displaymath}
  portanto a igualdade desejada segue da propriedade universal do produto.

  Infelizmente, embora o morfismo $h$ satisfaça a igualdade acima, ele \emph{não} define uma homotopia à direita de $f_{1}$ para $f_{3}$ pois a tripla $(P,\overline{c},q)$ \emph{não} necessariamente define um objeto de caminhos para $X$ já que $q$ pode não ser uma fibração.
  Felizmente podemos corrigir isso facilmente como no caso da homotopia à esquerda.
  Usando o axioma de fatoração podemos reescrever $q: P \to X$ como uma cofibração trivial $i: P \to \widetilde{P}$ seguida de uma fibração $\widetilde{q}: \widetilde{P} \to X \times X$ conforme indicado abaixo.
  \begin{displaymath}
    \begin{tikzcd}
      P
      \arrow[rr, "q"]
      \arrow[rd, tail, "i" {swap}, "\sim" {sloped}]
      & & X \times X
      \\ & \widetilde{P}
      \arrow[ru, two heads, "\widetilde{p}" swap]
    \end{tikzcd}
  \end{displaymath}
  A tripla $(\widetilde{P},i \circ \overline{c},\widetilde{q})$ sim define um objeto de caminhos para $X$.
  De fato, o morfismo $i \circ \overline{c}: X \to \widetilde{P}$ é a composição de duas equivalências fracas e, portanto, uma equivalência fraca também, $\widetilde{q}: \widetilde{P} \to X \times X$ é uma fibração por construção; e temos também a igualdade
  \begin{displaymath}
    \widetilde{q} \circ (i \circ \overline{c}) = (\widetilde{q} \circ i) \circ \overline{c} = q \circ \overline{c} = \Delta.
  \end{displaymath}
  Notamos enfim que o morfismo composto $i \circ h: B \to \widetilde{P}$ define a homotopia à direita de $f_{1}$ para $f_{3}$ desejada como mostram as igualdades
  \begin{displaymath}
    \widetilde{q} \circ (i \circ h) = (\widetilde{q} \circ i) \circ h = q \circ h = (f_{1},f_{3}). \qedhere
  \end{displaymath}
\end{proof}

Como no caso da relação de homotopia à esquerda, se $Y$ é um objeto fibrante, então para qualquer outro objeto $X$ podemos formar o conjunto quociente
\begin{displaymath}
  [X,Y]_{r} \coloneqq \mathsf{M}(X,Y)/\simeq_{r}
\end{displaymath}
cujos elementos são chamados de \textbf{classes de homotopia à direita} de morfismos.
Dado um morfismo $f: X \to Y$, denotamos por $[f]_{r}$ sua imagem no conjunto quociente $[f]_{r}$.

Analogamente ao que tínhamos no caso da relação de homotopia à esquerda, se $\alpha: W \to X$ é um morfismo \emph{qualquer}, a compatibilidade de homotopias à direita com pullbacks provada na \cref{prop:props_homotopia_direita} nos permite fatorar a função $\mathsf{M}(\alpha,Y): \mathsf{M}(X,Y) \to \mathsf{M}(W,Y)$ de pullback (pré-composição) ao longo de $\alpha$ pelas projeções canônicas para obtermos uma única função induzida $[\alpha,Y]_{r}: [X,Y]_{r} \to [W,Y]_{r}$ fazendo comutar o quadrado abaixo.
\begin{displaymath}
  \begin{tikzcd}[column sep=1.25cm]
    \mathsf{M}(W,Y)
    \arrow[d, "{\pi_{W,Y}}" swap]
    & \mathsf{M}(X,Y)
    \arrow[d, "{\pi_{X,Y}}"]
    \arrow[l, "{\mathsf{M}(\alpha,Y)}" swap]
    \\ {[W,Y]_{r}}
    & {[X,Y]_{r}}
    \arrow[l, dashed, "{[\alpha,Y]_{r}}"]
  \end{tikzcd}
\end{displaymath}
Explicitamente, dada uma classe de homotopia à direita $[f]_{r} \in [X,Y]_{r}$, pela comutatividade acima temos
\begin{displaymath}
  [\alpha,Y]_{r}([f]_{r}) = [f \circ \alpha]_{r}.
\end{displaymath}

Encerraremos essa seção comparando as noções de homotopias à esquerda e à direita e procurando entender quando as duas coincidem ou não.
Em suma, o resultado abaixo diz que essas duas noções sempre coincidem quando o domínio ou o codomínio são objetos bons do ponto de vista homotópico, ou seja, satisfazem condições de cofibrância ou fibrância.

\begin{prop}\label{prop:comparando_homotopia_esquerda_direita}
  Seja $(\mathsf{M},\mathcal{W},\mathcal{C},\mathcal{F})$ uma categoria de modelos.
  \begin{enumerate}
  \item Se $B$ é um objeto cofibrante, $X$ é um objeto qualquer, e $f,\,g: B \to X$ são dois morfismos homotópicos à esquerda, então $f$ e $g$ são também homotópicos à direita.
    
  \item Se $B$ é um objeto qualquer, $X$ é um objeto fibrante, e $f,\,g: B \to X$ são dois morfismos homotópicos à direita, então $f$ e $g$ são também homotópicos à esquerda.
  \end{enumerate}
\end{prop}

\begin{proof}
  1. Sejam $(\Cyl(B),i,\varepsilon)$ um objeto cilindro qualquer para $B$ e $h: \Cyl(B) \to X$ uma homotopia à esquerda de $f$ para $g$.
  A cofibrância de $B$ garante que $i_{0}: B \to \Cyl(B)$ seja uma cofibração trivial pelo \cref{lema:props_obj_cilindro}.
  Veja que o quadrado mostrado abaixo é comutativo como mostra a sequência de igualdades
  \begin{displaymath}
    (f \circ \varepsilon,h) \circ i_{0} = (f \circ \varepsilon \circ i_{0}, h \circ i_{0}) = (f,f) = \Delta \circ f = p \circ c \circ f.
  \end{displaymath}
  \begin{displaymath}
    \begin{tikzcd}[column sep=1.3cm]
      B
      \arrow[d, tail, "i_{0}" {swap}, "\sim" {sloped}]
      \arrow[r, "c \circ f"]
      & P(X)
      \arrow[d, two heads, "p"]
      \\ \Cyl(B)
      \arrow[r, "{(f \circ \varepsilon, h)}" swap]
      \arrow[ru, dashed, "\theta" description]
      & X \times X,
    \end{tikzcd}
  \end{displaymath}
  O axioma de levantamento garante então a existência do morfismo diagonal $\theta: \Cyl(B) \to P(X)$ fazendo comutar o diagrama todo como mostrado acima.

  Vamos tentar entender a função do morfismo $\theta$.
  Tal morfismo transforma pontos do objeto cilindro $\Cyl(B)$ em pontos do objeto $P(X)$, que pensamos intuitivamnete como caminhos em $X$.
  A igualdade $p \circ \theta = (f \circ \varepsilon,h)$ diz que que o ponto inicial desse caminho é determinada por $f \circ \varepsilon$, enquanto o ponto final é determinado pelo valor da própria homotopia $h$.
  Ora, como nossa homotopia termina em $g$, a ideia é que se compormos $\theta$ com a cofibração $i_{1}: B \overset{\sim}{\cofib} \Cyl(B)$ que determina a parte superior do cilindro, então teremos um caminho de $f$ até $g$.
  De fato, o morfismo composto $H \coloneqq \theta \circ i_{1}$ satisfaz
  \begin{displaymath}
    p \circ H
    = p \circ \theta \circ i_{1}
    = (f \circ \varepsilon,h) \circ i_{1}
    = (f \circ \varepsilon \circ i_{1}, h \circ i_{1})
    = (f,g),
  \end{displaymath}
  definindo, portanto, uma homotopia à direita de $f$ para $g$.

  \smallskip
  2. A demonstração desse segundo item é simplesmente uma dualização da demonstração do primeiro.
  Considere $h: B \to P(X)$ uma homotopia à direita de $f$ para $g$.
  Aplicando o \cref{lema:props_obj_de_caminhos} concluímos que a projeção $p_{0}: P(X) \to X$ é uma fibração trivial.
  A sequência de igualdades
  \begin{displaymath}
    p_{0} \circ \langle c \circ f, h \rangle
    = \langle p_{0} \circ c \circ f, p_{0} \circ h \rangle
    = \langle f,f \rangle
    = f \circ \nabla
    = f \circ \varepsilon \circ i
  \end{displaymath}
  mostra que o quadrado abaixo é comutativo.
  \begin{displaymath}
    \begin{tikzcd}[column sep=1.3cm]
      B \sqcup B
      \arrow[r, "{\langle c \circ f, h \rangle}"]
      \arrow[d, tail, "i" swap]
      & P(X)
      \arrow[d, two heads, "p_{0}", "\sim" {swap, sloped}]
      \\ \Cyl(B)
      \arrow[ru,dashed, "\psi" description]
      \arrow[r, "f \circ \varepsilon" swap]
      & X
    \end{tikzcd}
  \end{displaymath}
  Aplicando o axioma de levantamento obtemos o mapa diagonal $\psi: \Cyl(B) \to P(X)$ também indicado acima e que faz o diagrama todo comutar ainda.
  Basta notar então que o morfismo composto $H \coloneqq p_{1} \circ \psi$ satisfaz as igualdades
  \begin{displaymath}
    H \circ i = p_{1} \circ \psi \circ i = p_{1} \circ \langle c \circ f,h \rangle = \langle p_{1} \circ c \circ f, p_{1} \circ h \rangle = \langle f,g \rangle
  \end{displaymath}
  e define, portanto, uma homotopia à esquerda de $f$ para $g$.
\end{proof}

\begin{defin}\label{defin:homotopia}
  Dois morfismos $f,\,g: B \to X$ em uma categoria de modelos $\mathsf{M}$ são ditos \textbf{homotópicos} quando são simultaneamente homotópicos à esquerda e à direita.
\end{defin}

Tendo a definição acima em mãos, podemos reformular o enunciado da \cref{prop:comparando_homotopia_esquerda_direita} da seguinte forma: quando o domínio é cofibrante, então dois morfismos homotópicos à esquerda são também homotópicos; e quando o codomínio é fibrante, dois morfismos homotópicos à direita são também homotópicos.
O caso que nos será mais útil e que deixamos registrados na forma do próximo corolário é aquele onde as condições de cofibrância e fibrância são ambas satisfeitas, de forma que as várias noções de homotopia vistas até agora coincidam.

\begin{corol}\label{corol:homotopias_coincidem_se_dom_cofib_cod_fib}
  Em uma categoria de modelos, dados um objeto cofibrante $B$, um objeto fibrante $X$, e dois morfismos $f\,\ g: B \to X$, as seguintes afirmações são equivalentes:
  \begin{enumerate}
  \item $f$ e $g$ são homotópicos à esquerda;
    
  \item $f$ e $g$ são homotópicos à direita;
    
  \item $f$ e $g$ são homotópicos.
  \end{enumerate}
\end{corol}

Dados então um objeto cofibrante $X$ e um objeto fibrante $Y$ em uma categoria de modelos $\mathsf{M}$, sabemos pelo  \cref{corol:homotopia_esquerda_rel_equiv_se_dom_cofib}, ou pelo \cref{corol:homotopia_direita_rel_equiv_se_cod_fib} que a relação de homotopia $\simeq$ entre morfismos define uma relação de equivalência no conjunto de morfismos $\mathsf{M}(X,Y)$, de forma que podemos então considerar o conjunto quociente
\begin{equation}
  \label{eq:conjunto_classes_homotopia}
  [X,Y] \coloneqq \mathsf{M}(X,Y)/\simeq
\end{equation}
cujos elementos são chamados de \textbf{classes de homotopia} de morfismos de $X$ para $Y$.
Veremos na próxima seção que podemos usar estes conjuntos de classes de homotopia de morfismos para construirmos um modelo (a menos de equivalência) explícito para a localização de $\mathsf{M}$ na classe das equivalências fracas imitando a construção da categoria homotópica usual vinda da Teoria de Homotopia clássica.

%%% Local Variables:
%%% mode: latex
%%% TeX-master: "../main"
%%% End:
