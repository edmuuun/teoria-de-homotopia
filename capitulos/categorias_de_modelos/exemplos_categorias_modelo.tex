\section{Alguns exemplos e construções}

O objetivo dessa seção é dar alguns exemplos de categorias de modelos e descrever também algumas construções possíveis com tais categorias.
Infelizmente, a maioria dos exemplos que podemos dar no momento são ou muito simples, ou puramente teóricos, não sendo portanto muito úteis.
Na maioria dos exemplos úteis e concretos, a verificação dos axiomas de uma estrutura de modelos costuma ser bastante trabalhosa e exige resultados adicionais que veremos mais adiante.

\begin{exem}
  Seja $\mathsf{M}$ uma categoria bicompleta qualquer, considere $\mathcal{W}$ como sendo a classe dos isomorfismos em $\mathsf{C}$, e considere $\mathcal{C}$ e $\mathcal{F}$ como sendo a classe de todos os morfismos.
  Afirmamos que nesse caso a tupla $(\mathsf{M},\mathcal{W},\mathcal{C},\mathcal{F})$ define uma categoria modelo.

  A validade do axioma (M1) segue da hipótese que fizemos sobre $\mathsf{C}$.
  Considere agora dois morfismos $f: X \to Y$ e $g: Y \to Z$ em $\mathsf{C}$.
  Se $f$ e $g$ pertencem a $\mathcal{W}$, ou seja, $f$ e $g$ são isomorfismos, então $g \circ f$ também pertence a $\mathcal{W}$, já que isomorfismos são preservados por composição.
  Agora, se $f$ e $g \circ f$ pertencem a $\mathcal{W}$, a igualdade $g = (g \circ f) \circ f^{-1}$ expressa $g$ como composição de dois isomorfismos, logo $g \in \mathcal{W}$.
  Por fim, se $g$ e $g \circ f$ pertencem a $\mathcal{W}$, então $f = g^{-1} \circ (g \circ f)$ expressa $f$ como composição de dois isomorfismos, portanto $f \in \mathcal{W}$.
  Isso mostra que $\mathcal{W}$ satisfaz a propriedade 2-de-3.

  A validade do axioma de retração é imediata para as classes $\mathcal{C}$ e $\mathcal{F}$ já que elas consistem de todos os morfismos.
  Suponha agora que $f: X \to Y$ seja uma retração do isomorfismos $g: A \to B$, de forma que tenhamos um diagrama comutativo como abaixo.
  \begin{displaymath}
    \begin{tikzcd}
      X
      \arrow[r, "s_{1}"]
      \arrow[d, "f" swap]
      \arrow[rr, bend left=45, "\id_{X}"]
      & A
      \arrow[r, "r_{1}"]
      \arrow[d, "g"]
      & X
      \arrow[d, "f"]
      \\ Y
      \arrow[r, "s_{2}" swap]
      \arrow[rr, bend right=45, "\id_{Y}" swap] 
      & B
      \arrow[r, "r_{2}" swap]
      & Y
    \end{tikzcd}
  \end{displaymath}
  Afirmamos que, nesse caso, o morfismo $r_{1} \circ g^{-1} \circ s_{2}$ é o inverso de $f$.
  De fato, por um lado temos
  \begin{displaymath}
    r_{1} \circ g^{-1} \circ s_{2} \circ f = r_{1} \circ g^{-1} \circ g \circ s_{1} = r_{1} \circ \id_{A} \circ s_{1} = r_{1} \circ s_{1} = \id_{X},
  \end{displaymath}
  enquanto por outro temos
  \begin{displaymath}
    f \circ r_{1} \circ g^{-1} \circ s_{2}
    = r_{2} \circ g \circ g^{-1} \circ s_{2}
    = r_{2} \circ \id_{B} \circ s_{2}
    = r_{2} \circ s_{2}
    = \id_{Y}.
  \end{displaymath}
  Isso mostra que $f$ é um isomorfismo também, portanto a classe $\mathcal{W}$ é fechada por retrações.

  Vamos checar agora o axioma de levantamento.
  No problema de levantamento abaixo,
  \begin{displaymath}
    \begin{tikzcd}
      A
      \arrow[r, "\alpha"]
      \arrow[d, tail, "i" {swap}, "\sim" {sloped}]
      & X
      \arrow[d, two heads, "p"]
      \\ B
      \arrow[r, "\beta" swap]
      & Y
    \end{tikzcd}
  \end{displaymath}
  dizer que $i$ é uma cofibração trivial significa apenas dizer que $i$ é um isomorfismo.
  Veja então que o morfismo $h: B \to X$ dado por $h \coloneqq \alpha \circ i^{-1}$ soluciona o problema em questão pois
  \begin{displaymath}
    h \circ i = \alpha \circ i^{-1} \circ i = \alpha \circ \id_{A} = \alpha,
  \end{displaymath}
  e também
  \begin{displaymath}
    p \circ h = p \circ \alpha \circ i^{-1} = \beta \circ i \circ i^{-1} = \beta \circ \id_{B} = \beta.
  \end{displaymath}

  Considere agora o problema de levantamento abaixo.
  \begin{displaymath}
    \begin{tikzcd}
      A
      \arrow[r, "\alpha"]
      \arrow[d, tail, "i" swap]
      & X
      \arrow[d, two heads, "p", "\sim" {swap, sloped}]
      \\ B
      \arrow[r, "\beta" swap]
      & Y
    \end{tikzcd}
  \end{displaymath}
  Como no caso anterior, dizer que $p$ é uma fibração trivial significa simplesmente dizer que $p$ é um isomorfismo.
  Podemos então definir $h: B \to X$ pela composição $h \coloneqq p^{-1} \circ \beta$, e afirmamos que com essa definição obtemos um levantamento.
  De fato, por um lado
  \begin{displaymath}
    p \circ h = p \circ p^{-1} \circ \beta = \id_{Y} \circ \beta = \beta,
  \end{displaymath}
  enquanto por outro
  \begin{displaymath}
    h \circ i = p^{-1} \circ \beta \circ i = p^{-1} \circ p \circ \alpha = \id_{X} \circ \alpha = \alpha.
  \end{displaymath}

  Resta agora mostrarmos a existência de fatorações, mas isso é bem simples pela forma como foram definidas as classes $\mathcal{W}$, $\mathcal{C}$ e $\mathcal{F}$.
  Dado um morfismo $f: X \to Y$ qualquer, lembrando que as cofibrações triviais são os isomorfismos da categoria, e que as fibrações são todos os morfismos, a fatoração
  \begin{displaymath}
    f = f \circ \id_{X}
  \end{displaymath}
  exibe $f$ como uma cofibração trivial seguida de uma fibração.
  Analogamente, como todo morfismo é uma cofibração, e todo isomorfismo é uma fibração trivial, a fatoração
  \begin{displaymath}
    f = \id_{Y} \circ f
  \end{displaymath}
  expressa $f$ como uma cofibração seguida de uma fibração trivial.
\end{exem}

\begin{exem}[Produto de categorias de modelos]
  \label{exem:prod_categorias_de_modelos}
  Sejam $\mathsf{M}$ e $\mathsf{N}$ duas categorias de modelos.
  Lembre-se que podemos formar a categoria produto $\mathsf{M} \times \mathsf{N}$ cujos objetos são pares $(X,Y)$, onde $X$ é um objeto de $\mathsf{M}$, e $Y$ é um objeto de $\mathsf{N}$; e um morfismo do tipo $(X,Y) \to (X',Y')$ é dado por um par $(f,g)$, onde $f$ é um morfismo do tipo $X \to X'$ em $\mathsf{M}$, e $g$ é um morfismo do tipo $Y \to Y'$ em $\mathsf{N}$.
  Em suma, definimos
  \begin{displaymath}
    \Hom_{\mathsf{M} \times \mathsf{N}}((X,Y),(X',Y')) \coloneqq \Hom_\mathsf{M}(X,X') \times \Hom_\mathsf{N}(Y,Y').
  \end{displaymath}
  A composição em $\mathsf{M} \times \mathsf{N}$ é dada pelas composições em $\mathsf{M}$ e $\mathsf{N}$ nas respectivas coordenadas, ou seja, dados morfismos $(f_1,g_1): (X_1,Y_1) \to (X_2,Y_2)$ e $(f_2,g_2): (X_2,Y_2) \to (X_3,Y_3)$, definimos sua composição pela fórmula
  \begin{displaymath}
    (f_2,g_2) \circ (f_1,g_1) \coloneqq (f_2 \circ f_1,g_2 \circ g_1): (X_1,Y_1) \to (X_3,Y_3).
  \end{displaymath}

  Afirmamos que o produto $\mathsf{M} \times \mathsf{N}$ também admite uma estrutura de modelos obtida pela combinação das estruturas existentes em $\mathsf{M}$ e $\mathsf{N}$.
  As classes de morfismos em questão são definidas da seguinte maneira: um morfismo $(f,g): (X,Y) \to (X',Y')$ em $\mathsf{M} \times \mathsf{N}$ é dito uma
  \begin{itemize}
    \item equivalência fraca se $f: X \to X'$ e $g: Y \to Y'$ são equivalências fracas em $\mathsf{M}$ e $\mathsf{N}$, respectivamente;

    \item cofibração se $f: X \to X'$ e $g: Y \to Y'$ são cofibrações em $\mathsf{M}$ e $\mathsf{N}$, respectivamente;

    \item fibração se $f: X \to X'$ e $g: Y \to Y'$ são fibrações em $\mathsf{M}$ e $\mathsf{N}$, respectivamente.
  \end{itemize}

  Resta verificarmos a validade dos axiomas.
  A bicompletude de $\mathsf{M} \times \mathsf{N}$ é uma questão puramente categórica: o (co)limite de um funtor $F: \mathsf{J} \to \mathsf{M} \times \mathsf{N}$ pode ser calculado a partir dos (co)limites dos funtores componentes $\mathsf{J} \to \mathsf{M}$ e $\mathsf{J} \to \mathsf{N}$ obtidos a partir dos funtores de projeção $\mathsf{M} \times \mathsf{N} \to \mathsf{M}$ e $\mathsf{M} \times \mathsf{N} \to \mathsf{N}$.

  Vejamos a questão ad propriedade 2-de-3 em $\mathsf{M} \times \mathsf{N}$. Suponha que tenhamos dois morfismos $(f_1,g_1): (X_1,Y_1) \to (X_2,Y_2)$ e $(f_2,g_2): (X_2,Y_2) \to (X_3,Y_3)$ e suponha que $(f_1,g_1)$ e a composição $(f_2,g_2) \circ (f_1,g_1) = (f_2 \circ f_1,g_2 \circ g_1)$ sejam equivalências fracas.
  Segue da definição de equivalência fraca em $\mathsf{M} \times \mathsf{N}$ que $f_1$ e $f_2 \circ f_1$ são equivalências fracas em $\mathsf{M}$, enquanto $g_1$ e $g_2 \circ g_1$ são equivalêncisa fracas em $\mathsf{N}$.
  Aplicando então a propriedade 2-de-3 nas categorias $\mathsf{M}$ e $\mathsf{N}$ concluímos que $f_2$ é uma equivalência fraca em $\mathsf{M}$, e $g_2$ é uma equivalência fraca em $\mathsf{N}$; portanto o par $(f_2,g_2)$ define uma equivalência fraca em $\mathsf{M} \times \mathsf{N}$ como gostaríamos.
  Os outros dois casos possíveis da propriedade 2-de-3 seguem por um raciocínio completamente análogo.
\end{exem}

%%% Local Variables:
%%% mode: latex
%%% TeX-master: "../main"
%%% End: