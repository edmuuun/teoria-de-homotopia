\section{Localizações}

%TODO: Terminar a demonstração do teorema sobre localizações nos objetos cofibrantes/fibrantes/bifibrantes.

Nessa seção discutimos inicialmente a noção de localização de uma categoria $\mathsf{C}$ em uma classe qualquer de morfismos, e em seguida discutimos como as estrutura de uma categoria de modelos nos permite de certa forma simplificar o processo de localização na classe das equivalências fracas, simplificação esta que nos permitirá mais adiante construir modelos muito mais manejáveis para essa localização em termos de classes de homotopia.

Em geral, a única forma possível de compararmos dois morfismos em uma categoria $\mathsf{C}$ qualquer é por meio de igualdades.
Note, entretanto, que a situação é mais sútil para morfismos na categoria de categorias $\mathsf{Cat}$, já que dois funtores podem estar relacionados por uma igualdade ou também por um isomorfismo natural.
Isso significa que construções universais envolvendo \emph{categorias em si} podem ser formuladas de duas formas: uma versão \emph{estrita} usando apenas igualdades entre funtores, e uma versão \emph{fraca} usando isomorfismos naturais entre funtores.
A primeira definição de localização que daremos, a qual aparece por exemplo nas referências \cite[Definição 7.30]{heuts-moerdijk} e \cite{nlab:localization}, é a versão fraca.
Essa é a versão que realmente aparece na teoria de categoria de modelos, já que a construção de uma localização na classe de equivalências fracas por meio das classes de homotopia de morfismos entre objetos bifibrantes satisfaz apenas essa versão fraca da definição.
Uma boa discussão comparando estas duas possíveis definições e também outras existentes na literatura também pode ser encontrada em \cite{overflow:localization}.

\begin{defin}\label{defin:localizacao_fraca}
  Sejam $\mathsf{C}$ uma categoria e $\mathcal{W} \Mor(\mathsf{C})$ uma classe qualquer de morfismos.
  Uma \textbf{localização fraca de $\mathsf{C}$ em $\mathcal{W}$} é um par $(\mathsf{L},\gamma)$, onde $\mathsf{L}$ é uma categoria e $\gamma: \mathsf{C} \to \mathsf{L}$ é um funtor, satisfazendo as seguintes condições:
  \begin{enumerate}
  \item[(i)] O funtor $\gamma$ transforma os morfismos de $\mathcal{W}$ em isomorfismos de $\mathsf{L}$.
    
  \item[(ii)] Dada uma outra categoria $\mathsf{D}$ qualquer, o funtor de pré-composição com $\gamma$ define uma \emph{equivalência} de categorias
    \begin{displaymath}
      \Fun(\gamma,\mathsf{D}): \Fun(\mathsf{L},\mathsf{D}) \overset{\simeq}{\longrightarrow} \Fun_{\mathcal{W}}(\mathsf{C},\mathsf{D}),
    \end{displaymath}
    onde $\Fun_{\mathcal{W}}(\mathsf{C},\mathsf{D})$ denota a subcategoria plena da categoria de funtores $\Fun(\mathsf{C},\mathsf{D})$ gerada por aqueles funtores que transformam morfismos de $\mathcal{W}$ em isomorfismos de $\mathsf{D}$.
  \end{enumerate}
\end{defin}

O resultado abaixo mostra que a definição acima pode ser reformulada de forma a ficar mais similar a outras definições feitas por meio de propriedades universais.

\begin{prop}\label{prop:localizacao_fraca_via_prop_universal}
  Sejam $\mathsf{C}$ uma categoria e $\mathcal{W} \subseteq \Mor(\mathsf{C})$ uma classe de morfismos.
  Dada uma categoria $\mathsf{L}$ e um funtor $\gamma: \mathsf{C} \to \mathsf{L}$ que transforma os morfismos de $\mathcal{W}$ em isomorfismos de $\mathsf{L}$, o par $(\mathsf{L},\gamma)$ define uma localização fraca de $\mathsf{C}$ em $\mathcal{W}$ se, e somente se, as duas condições abaixo são satisfeitas para qualquer categoria $\mathsf{D}$:
  \begin{enumerate}
  \item Se $F: \mathsf{C} \to \mathsf{D}$ é um funtor que transforma morfismos de $\mathcal{W}$ em isomorfismos de $\mathsf{D}$, então existe um funtor $\overline{F}: \mathsf{L} \to \mathsf{D}$ e um isomorfismo natural de funtores $\overline{F} \circ \gamma \cong F$.
    
  \item Dados dois funtores $G_{1},\, G_{2}: \mathsf{L} \to \mathsf{D}$ quaisquer, a função
    \begin{displaymath}
      \Fun(\gamma,\mathsf{D})_{G_{1},G_{2}}: \Hom_{\Fun(\mathsf{L},\mathsf{D})}(G_{1},G_{2}) \to \Hom_{\Fun_{\mathcal{W}}(\mathsf{C},\mathsf{D})}(G_{1} \circ \gamma, G_{2} \circ \gamma)
    \end{displaymath}
  \end{enumerate}
  associada ao funtor de pré-composição $\Fun(\gamma,\mathsf{D}): \Fun(\mathsf{L},\mathsf{D}) \to \Fun_{\mathcal{W}}(\mathsf{C},\mathsf{D})$ é uma bijeção.
\end{prop}

\begin{proof}
  Lembremos que um funtor $F: \mathsf{C} \to \mathsf{D}$ define uma equivalência de categorias se, e somente se, ele é essencialmente sobrejetivo, pleno e fiel, ou seja, se ele satisfaz as seguintes condições:
  \begin{enumerate}
  \item Dado um objeto $Y \in \mathsf{D}$ qualquer, existe um objeto $X \in \mathsf{C}$ e um isomorfismo $F(X) \cong Y$.
    
  \item Dados dois objetos $X_{1},\, X_{2} \in \mathsf{C}$ quaisquer, a função
    \begin{displaymath}
      F_{X_{1},X_{2}}: \Hom_{\mathsf{C}}(X_{1},X_{2}) \to \Hom_{\mathsf{D}}(F(X_{1}),F(X_{2}))
    \end{displaymath}
    associado a funtor $F: \mathsf{C} \to \mathsf{D}$ é uma bijeção.
  \end{enumerate}

  O resultado do enunciado segue então dessa caracterização de equivalência aplicada ao funtor $\Fun(\gamma,\mathsf{D}): \Fun(\mathsf{L},\mathsf{D}) \to \Fun_{\mathcal{W}}(\mathsf{C},\mathsf{D})$.
\end{proof}

Apesar da formulação de localização acima ser mais similar a outras propriedades unviversais com as quias estamos acostumados, existe uma diferença sútil mas importante: as propriedades universais usuais estabelecem a existência sob certas condições de um morfismo \emph{único} satisfazendo certas \emph{igualdades}.
Já a propriedade universal da \cref{prop:localizacao_fraca_via_prop_universal} que caracteriza a localização fraca estabelece a existência sob certas hipóteses de \emph{algum} funtor satisfazendo uma condição de \emph{isomorfismo natural} e não de igualdade.
Veremos agora que, apesar de não termos a unicidade na caracterização acima, temos uma unicidade a menos de isomorfismos naturais de funtores.

\begin{prop}\label{prop:unicidade_fraca_funtor_induzido_localizacao_fraca}
  Sejam $\mathsf{C}$ e $\mathsf{D}$ categorias, $\mathcal{W} \subseteq \Mor(\mathsf{C})$ uma classe qualquer de morfismos, e $F: \mathsf{C} \to \mathsf{D}$ um funtor que inverte os morfismos de $\mathcal{W}$.
  Se $(\mathsf{L},\gamma)$ é uma localização fraca de $\mathsf{C}$ em $\mathcal{W}$, e $\overline{F},\,\widehat{F}: \mathsf{L} \to \mathsf{D}$ são dois funtores tais que $\overline{F} \circ \gamma \cong F$ e $\widehat{F} \circ \gamma \cong F$, então existe também um isomorfismo natural $\overline{F} \cong \widehat{F}$.
\end{prop}

O resultado acima segue facilmente do seguinte resultado mais geral a respeito de equivalências de categorias.

\begin{lema}\label{lema:equiv_categorias_quase_injetivo}
  Suponha que $F: \mathsf{C} \to \mathsf{D}$ seja uma equivalência de categorias.
  Se $X,\, Y \in \mathsf{C}$ são dois objetos tais que $F(X) \cong F(Y)$, então existe também um isomorfismo $X \cong Y$.
\end{lema}

\begin{proof}
   Sejam $G: \mathsf{D} \to \mathsf{C}$ o funtor quase-inverso a $F$ e $\theta: G \circ F \Rightarrow \id_{\mathsf{C}}$ um isomorfismo natural de funtores.
  Existe por hipótese um isomorfiso $\alpha: F(X) \to F(Y)$, e aplicando $G$ obtemos então um isomorfismo $G(\alpha): G(F(X)) \to G(F(Y))$.
  Considerando então os isomorfismos componentes $\theta_{X}: G(F(X)) \to X$ e $\theta_{Y}: G(F(Y)) \to Y$, a composição
  \begin{displaymath}
    \theta_{Y} \circ G(\alpha) \circ \theta_{X}^{-1}: X \to Y
  \end{displaymath}
  define o isomorfismo desejado.
\end{proof}

\begin{proof}[Demonstração da \cref{prop:unicidade_fraca_funtor_induzido_localizacao_fraca}]
  Sabemos da definição de localização fraca que o funtor de pré-composição
  \begin{displaymath}
    \Fun(\gamma,\mathsf{D}): \Fun(\mathsf{L},\mathsf{D}) \to \Fun_{\mathcal{W}}(\mathsf{C},\mathsf{D})
  \end{displaymath}
  é uma equivalência de categorias, mas pelas hipóteses do enunciado temos
  \begin{displaymath}
    \Fun(\gamma,\mathsf{D})(\overline{F}) = \overline{F} \circ \gamma \cong F \cong \widehat{F} \circ \gamma = \Fun(\gamma,\mathsf{D})(\widehat{F});
  \end{displaymath}
  logo o resultado segue diretamente do \cref{lema:equiv_categorias_quase_injetivo} aplicado à equivalência $\Fun(\gamma,\mathsf{D})$.
\end{proof}

Uma consequência importante é que, embora duas localizações fracas possam não ser isomorfas, elas serão sempre equivalentes.
Isso pode parecer mais fraco do que gostaríamos, mas a experiência mostra que em geral a noção de equivalência entre categorias é mais razoável do que a noção de isomorfismo entre categorias.

\begin{corol}\label{corol:localizacao_fraca_unica_a_menos_de_equiv}
  Dada uma categoria $\mathsf{C}$ e uma classe de morfismos $\mathcal{W} \subseteq \mathsf{C}$, quaisquer duas localizações de $\mathsf{C}$ em $\mathcal{W}$ são equivalentes.
\end{corol}

\begin{proof}
  Suponha que $(\mathsf{L},\gamma)$ e $(\mathsf{J},\delta)$ sejam duas localizações de $\mathsf{C}$ em $\mathcal{W}$.
  Como $\delta: \mathsf{C} \to \mathsf{J}$ inverte morfismos de $\mathcal{W}$, aplicando a \cref{prop:localizacao_fraca_via_prop_universal} ao par $(\mathsf{L},\gamma)$ obtemos um funtor $\overline{\delta}: \mathsf{L} \to \mathsf{J}$ juntamente com um isomorfismo natural de funtores $\overline{\delta} \circ \gamma \cong \delta$.
  Ora, como $\gamma: \mathsf{C} \to \mathsf{L}$ também inverte morfismos de $\mathcal{W}$, aplicado o mesmo resultado agora ao par $(\mathsf{J},\delta)$ obtemos um funtor $\overline{\gamma}: \mathsf{J} \to \mathsf{L}$ e um isomorfismo natural de funtores $\delta \circ \overline{\gamma} \cong \gamma$.

  Afirmamos que $\overline{\delta}$ e $\overline{\gamma}$ são funtores quase-inversos, ou seja, que existem isomorfismos naturais de funtores $\overline{\gamma} \circ \overline{\delta} \cong \id_{\mathsf{L}}$ e $\overline{\delta} \circ \overline{\gamma} \cong \id_{\mathsf{J}}$.
  No primeiro caso, veja que, como o funtor idêntico $\id_{\mathsf{L}}$ satisfaz $\id_{\mathsf{L}} \circ \gamma = \gamma$, segue da \cref{prop:unicidade_fraca_funtor_induzido_localizacao_fraca} que, se $G: \mathsf{L} \to \mathsf{L}$ é um funtor tal que $G \circ \gamma \cong \gamma$, então necessariamente devemos ter também um isomorfismo natural $G \cong \id_{\mathsf{L}}$.
  Ora, basta notar agora que
  \begin{displaymath}
    \overline{\gamma} \circ \overline{\delta} \circ \gamma \cong \overline{\gamma} \circ \delta \cong \gamma,
  \end{displaymath}
  de onde concluímos que $\overline{\gamma} \circ \overline{\delta} \cong \id_{\mathsf{L}}$.
  Analogamente, o funtor $\id_{\mathsf{J}}$ é o único a menos de isomorfismo do seu tipo que fatora $\delta$, mas temos os isomorfismos naturais
  \begin{displaymath}
    \overline{\delta} \circ \overline{\gamma} \circ \delta \cong \overline{\delta} \circ \gamma \cong \delta,
  \end{displaymath}
  portanto temos também o isomorfismo natural $\overline{\delta} \circ \overline{\gamma} \cong \id_{\mathsf{J}}$.
\end{proof}

Tendo em vista essa unicidade a menos de equivalências, é comum denotarmos qualquer localização de $\mathsf{C}$ em $\mathcal{W}$ por $\mathsf{C}[\mathcal{W}^{-1}]$.

Tendo estudado algumas propriedade básicas das localizações fracas, a próxima pergunta natural é quanto a sua existência.
Será que podemos sempre construir uma localização de uma categoria $\mathsf{C}$ em uma classe de morfismos $\mathcal{W} \subseteq \Mor(\mathsf{C})$?
A resposta é que sim \emph{a menos de tecnicalidades conjuntistas}.
Usando a construção da categoria livre gerada por um conjunto de setas, odemos construir no braço uma localização fraca, mas a categoria obtida dessa forma não é localmente pequena, o que nos força então a trabalhar com algum universo de Grothendieck grande o suficiente para que a construção faça sentido.
Felizmente, no caso de uma categoria de modelos $(\mathsf{M},\mathcal{W},\mathcal{C},\mathcal{F})$, a estrutura adicional dada pelas cofibrações e fibrações nos permite construir um \emph{modelo alternativo} para uma localização fraca $\mathsf{M}[\mathcal{W}^{-1}]$ nas equivalências fracas.
Essa construção alternativa depende crucialmente do resultado abaixo, o qual diz que podemos obter uma localização fraca para uma categoria de modelos trabalhando apenas com subcategorias de objetos bem comportados.

\begin{teo}
  Seja $(\mathsf{M},\mathcal{W},\mathcal{C},\mathcal{F})$ uma categoria de modelos.
  Denote por $\mathsf{M}_{c}$ (respectivamente $\mathsf{M}_{f}$ e $\mathsf{M}_{cf}$) a subcategoria plena gerada pelos objetos cofibrantes (respectivamente fibrantes e bifibrantes), e denote por $\mathcal{W}_{c}$ (respectivamente $\mathcal{W}_{f}$ e $\mathcal{W}_{cf}$) a classe de morfismos dada pela interseção $\mathcal{W} \cap \Mor(\mathsf{M}_{c})$ (respectivamente $\mathcal{W} \cap \Mor(\mathsf{M}_{f})$ e $\mathcal{W} \cap \Mor(\mathsf{M}_{cf})$).
  As seguintes categorias são todas equivalentes:
  \begin{enumerate}
  \item[(i)] $\mathsf{M}[\mathcal{W}^{-1}]$;
    
  \item[(ii)] $\mathsf{M}_{c}[\mathcal{W}_{c}^{-1}]$;
    
  \item[(iii)] $\mathsf{M}_{f}[\mathcal{W}_{f}^{-1}]$;
    
  \item[(iv)] $\mathsf{M}_{cf}[\mathcal{W}_{cf}^{-1}]$.
  \end{enumerate}
\end{teo}

\begin{proof}
  Vejamos primeiro a equivalência entre as localizações fracas $\mathsf{M}[\mathcal{W}^{-1}]$ e $\mathsf{M}_{c}[\mathcal{W}_{c}^{-1}]$.
  Se $\gamma: \mathsf{M} \to \mathsf{M}[\mathcal{W}^{-1}]$ e $\gamma_{c}: \mathsf{M}_{c} \to \mathsf{M}[\mathcal{W}_{c}^{-1}]$ são so funtores de localização, e $i: \mathsf{M}_{c} \to \mathsf{M}$ denota o funtor de inclusão, note que o funtor composto
  \begin{displaymath}
    \gamma \circ i: \mathsf{M}_{c} \to \mathsf{M}[\mathcal{W}^{-1}]
  \end{displaymath}
  inverte os morfismos de $\mathcal{W}_{c}$, pois $\gamma$ inverte os morfismos de $\mathcal{W}$, e $\mathcal{W}_{c} \subseteq \mathcal{W}$.
  Segue da propriedade universal da localização fraca que existe um funtor induzido $\overline{i}: \mathsf{M}_{c}[\mathcal{W}_{c}^{-1}] \to \mathsf{M}[\mathcal{W}^{-1}]$ juntamente com um isomorfismo natural $\overline{i} \circ \gamma_{c} \cong \gamma \circ i$.

  Vejamos agora como construir um funtor no sentido contrário $\mathsf{M}[\mathcal{W}^{-1}] \to \mathsf{M}_{c}[\mathcal{W}_{c}^{-1}]$.
  Seja $\mathrm{subcof}: \mathsf{M} \to \Arr(\mathsf{M})$ o funtor de substituição cofibrante como descrito na \cref{obs:substituicao_cofibrante_fibrante_funtorial}.
  Lembremos que tal funtor associa a cada objeto $X \in \mathsf{M}$ uma fibração trivial $p_{X}: X_{c} \to X$, onde $X_{c}$ é um objeto cofibrante, e associa a um morfismo $f: X \to Y$ um morfismo correspondente $f_{c}: X_{c} \to Y_{c}$ entre as substituições cofibrantes que faz comutar o diagrama abaixo.
  \begin{displaymath}
    \begin{tikzcd}
      X_{c}
      \arrow[r, "f_{c}"]
      \arrow[d, two heads, "p_{X}" {swap}, "\sim" {sloped}]
      & Y_{c}
      \arrow[d, two heads, "p_{Y}", "\sim" {swap,sloped}]
      \\ X
      \arrow[r, "f" swap]
      & Y
    \end{tikzcd}
  \end{displaymath}
  Afirmamos que o funtor composto
  \begin{displaymath}
    \gamma_{c} \circ \dom \circ \mathrm{subcof}: \mathsf{M} \to \mathsf{M}_{c}[\mathcal{W}_{c}^{-1}]
  \end{displaymath}
  inverte os morfismos de $\mathcal{W}$.
  De fato, note que, se $f: X \to Y$ é uma equivalência fraca, então o mesmo vale para a composição $f \circ p_{X}$, a qual é igual à composição $p_{Y} \circ f_{c}$; mas sendo $p_{Y}$ uma equivalência fraca também, segue da propriedade 2-de-3 que $f_{c}$ é uma equivalência fraca, portanto $\gamma_{c}(f_{c})$ é um isomorfismo.
  Segue da propriedade universal da localização fraca que existe um morfismo induzido $\varphi: \mathsf{M}[\mathcal{W}^{-1}] \to \mathsf{M}_{c}[\mathcal{W}_{c}^{-1}]$ juntamente com um isomorfismo natural de funtores $\varphi \circ \gamma \cong \gamma_{c} \circ \dom \circ \mathrm{subcof}$.

  Nosso objetivo agora é mostrar que os funtores $\overline{i}$ e $\varphi$ são quase-inversos.
  Vimos na \cref{obs:substituicao_cofibrante_fibrante_funtorial} que a coleção $(p_{X})_{X \in \mathsf{M}}$ define uma transformação natural $i \circ \dom \circ \mathrm{subcof} \Rightarrow \id_{\mathsf{M}}$.
  Como $\gamma(p_{X})$ é um isomorfismo para todo $X \in \mathsf{M}$, segue que a coleção $(\gamma(X))_{X \in \mathsf{M}}$ define um \emph{isomorfismo} natural de funtores
  \begin{displaymath}
    \gamma \circ i \circ \dom \circ \mathrm{subcof} \cong \gamma,
  \end{displaymath}
  mas note que pelas propriedades caracterizando os vários funtores acima temos também uma sequência de isomorfismos naturais
  \begin{displaymath}
    \gamma \circ i \circ \dom \circ \mathrm{subcof} \cong \overline{i} \circ \gamma_{c} \circ \dom \circ \mathrm{subcof} \cong \overline{i} \circ \varphi \circ \gamma.
  \end{displaymath}
  Combinando esses vários isomorfismos naturais obtemos o isomorfismo $\overline{i} \circ \varphi \circ \gamma \cong \gamma$, de onde concluímos que $\overline{i} \circ \varphi \cong \id_{\mathsf{M}}$, já que o funtor de pré-composição
  \begin{displaymath}
    \Fun(\gamma,\mathsf{M}[\mathcal{W}^{-1}]): \Fun(\mathsf{M}[\mathcal{W}^{-1}],\mathsf{M}[\mathcal{W}^{-1}]) \to \Fun_{\mathcal{W}}(\mathsf{M},\mathsf{M}[\mathcal{W}^{-1}])
  \end{displaymath}
  define uma equivalência de categorias.

  O outro isomorfismo é obtido de forma similar.
  A coleção $(p_{X})_{X \in \mathsf{M}_{c}}$ define uma transformação natural $\dom \circ \mathrm{subcof} \circ i \Rightarrow \id_{\mathsf{M}_{c}}$, mas $\gamma_{c}(p_{X})$ é um isomorfismo para todo $X \in \mathsf{M}_{c}$, portanto $(\gamma_{c}(p_{X}))_{X \in \mathsf{M}_{c}}$ define um \emph{isomorfismo} natural $\gamma_{c} \circ \dom \circ \mathrm{subcof} \circ i \cong \gamma_{c}$.
  Basta ver agora que temos os isomorfismos naturais
  \begin{displaymath}
    \gamma_{c} \circ \dom \circ \mathrm{subcof} \circ i \cong \varphi \circ \circ i \cong \varphi \circ \overline{i} \circ \gamma_{c},
  \end{displaymath}
  logo $\varphi \circ \overline{i} \circ \gamma_{c} \cong \gamma_{c}$, o que implica o isomorfismo desejado $\varphi \circ \overline{i} \cong \id_{\mathsf{M}_{c}}$ já que o funtor
  \begin{displaymath}
    \Fun(\gamma_{c},\mathsf{M}_{c}[\mathcal{W}_{c}^{-1}]): \Fun(\mathsf{M}_{c}[\mathcal{W}_{c}^{-1}],\mathsf{M}_{c}[\mathcal{W}_{c}^{-1}]) \to \Fun_{\mathcal{W}_{c}}(\mathsf{M}_{c}, \mathsf{M}_{c}[\mathcal{W}_{c}^{-1}])
  \end{displaymath}
  define uma equivalência de categorias.

  O argumento para mostrarmos a equivalência $\mathsf{M}[\mathcal{W}^{-1}] \cong \mathsf{M}_{f}[\mathcal{W}^{-1}_{f}]$ é análogo.
  Por um lado, a composição do funtor de inclusão $i: \mathsf{M}_{f} \to \mathsf{M}$ com o funtor de localização $\gamma: \mathsf{M} \to \mathsf{M}[\mathcal{W}^{-1}]$ inverte os morfismos de $\mathcal{W}_{f}$, portanto nesse caso obtemos também um funtor $\overline{i}: \mathsf{M}_{f}[\mathcal{W}_{f}^{-1}] \to \mathsf{M}[\mathcal{W}^{-1}]$ juntamente com um isomorfismo natural $\overline{i} \circ \gamma_{f} \cong \gamma \circ i$.
  A fim de obtermos um funtor no outro sentido, consideramos inicial o funtor de substituição fibrante
  \begin{displaymath}
    \mathrm{subfib}: \mathsf{M} \to \Arr(\mathsf{M})
  \end{displaymath}
  discutido também na \cref{obs:substituicao_cofibrante_fibrante_funtorial}, e verificamos que o funtor composto
  \begin{displaymath}
    \gamma_{f} \circ \cod \circ \mathrm{subfib}: \mathsf{M} \to \mathsf{M}_{f}[\mathcal{W}_{f}^{-1}]
  \end{displaymath}
  inverte os morfismos de $\mathcal{W}$, induzindo portanto um funtor $\psi: \mathsf{M}[\mathcal{W}^{-1}] \to \mathsf{M}_{f}[\mathcal{W}_{f}^{-1}]$ juntamente com um isomorfismo natural $\psi \circ \gamma \cong \gamma_{f} \circ \cod \circ \mathrm{subfib}$.
  A demonstração de que os funtores $\overline{i}$ e $\psi$ são quase-inversos é análoga ao que fizemos no caso anterior.

  Mostramos por fim a existência de uma equivalência $\mathsf{M}_{cf}[\mathcal{W}_{cf}^{-1}] \cong \mathsf{M}_{c}[\mathcal{W}_{c}^{-1}]$.
  Se $\lambda: \mathsf{M}_{cf} \to \mathsf{M}_{c}$ denota o funtor de inclusão, como nos outros casos temos que a composição
  \begin{displaymath}
    \gamma_{c} \circ \lambda: \mathsf{M}_{cf} \to \mathsf{M}_{c}[\mathcal{W}_{c}^{-1}]
  \end{displaymath}
  inverte os morfismos de $\mathcal{W}_{cf}$, portanto a propriedade universal da localização fraca dá origem a um funtor $\overline{\lambda}: \mathsf{M}_{cf}[\mathcal{W}_{cf}^{-1}] \to \mathsf{M}_{c}[\mathcal{W}_{c}^{-1}]$ juntamente com um isomorfismo natural $\gamma_{c} \circ \lambda \cong \overline{\lambda} \circ \gamma_{cf}$.
  Denote por $\mathrm{subfib'}: \mathsf{M}_{c} \to \Arr(\mathsf{M})$ a restrição do funtor de substituição fibrante à categoria $\mathsf{M}_{c}$ gerada pelos objetos cofibrantes.
  Tal funtor associa a cada objeto cofibrante $X \in \mathsf{M}_{c}$ uma \emph{cofibração trivial} $j_{X}: X \overset{\sim}{\to} X_{f}$, onde $X_{f}$ é um objeto fibrante.
  Note que, sendo $X$ cofibrante e $j_{X}$ uma cofibração, segue do \cref{lema:props_obj_cofib_fib} que $X_{f}$ é ainda cofibrante, sendo portanto um objeto \emph{bifibrante}.
  Dessa forma, a composição $\cod \circ \mathrm{subfib}'$ pode ser vista como um funtor do tipo $\mathsf{M}_{c} \to \mathsf{M}_{cf}$, e podemos então mostrar que a composição $\gamma_{cf} \circ \cod \circ \mathrm{subfib}': \mathsf{M}_{c} \to \mathsf{M}_{cf}[\mathcal{W}_{cf}^{-1}]$ inverte morfismos de $\mathcal{W}_{c}$, dando origem portanto a um funtor induzido $\theta: \mathsf{M}_{c}[\mathcal{W}_{c}^{1}] \to \mathsf{M}_{cf}[\mathcal{W}_{cf}^{-1}]$ juntamente com um isomorfismo natural $\theta \circ \gamma_{c} \cong \gamma_{cf} \circ \cod \circ \mathrm{subfib}'$.
  A demonstração de que os funtores $\overline{\lambda}$ e $\theta$ são quase-inversos segue, de forma similar aos casos anteriores, das propriedades universais das localizações fracas envolvidas e também do fato de que as coleções de morfismos $(\gamma_{c}(j_{X}))_{X \in \mathsf{M}_{c}}$ e $(\gamma_{cf}(j_{X}))_{X \in \mathsf{M}_{cf}}$ definem isomorfismos naturais $\gamma \cong \gamma \circ \lambda \circ \cod \circ \mathrm{subfib}'$ e $\gamma_{cf} \cong \gamma_{cf} \circ \cod \circ \mathrm{subfib}' \circ \lambda$, respectivamente.
\end{proof}

%%% Local Variables:
%%% mode: latex
%%% TeX-master: "../main"
%%% End:
