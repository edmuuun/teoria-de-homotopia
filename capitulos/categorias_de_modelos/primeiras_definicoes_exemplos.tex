\section{Primeiras definições e exemplos}

\begin{defin}\label{defin:estrutura_de_modelos}
  Seja $\mathsf{M}$ uma categoria localmente pequena.
  Uma \textbf{estrutura de modelos} em $\mathsf{M}$ consiste de três classes de morfismos $\mathcal{W},\, \mathcal{C},\, \mathcal{F} \subseteq \mathrm{Mor}(\mathsf{M})$ cujos elementos são chamados, respectivamente, \textbf{equivalências fracas}, \textbf{cofibrações} e \textbf{fibrações}, as quais devem satisfazer os seguintes axiomas:
  \begin{enumerate}
  \item[(M1)] A categoria $\mathsf{M}$ é bicompleta, ou seja, admite todos os limites e colimites indexados por categorias pequenas.
    
  \item[(M2)] (Propriedade 2-de-3) Dados morfismos $f: X \to Y$ e $g: Y \to Z$ em $\mathsf{M}$, se dois dos morfismos do conjunto $\{f,\,g,\, g \circ f\}$ estiverem em $\mathcal{W}$, então o terceiro também deve estar.
    
  \item[(M3)] (Propriedade de retração) Se um morfismo $f: A \to X$ é retração de um outro morfismo $g: B \to Y$, ou seja, se existe um diagrama comutativo como abaixo,
    \begin{equation}\label{eq:diagrama_axioma_de_retracao}
      \begin{tikzcd}
        A
        \arrow[r]
        \arrow[d, "f" swap]
        \arrow[rr, bend left=25, "\id_{A}"]
        & B
        \arrow[r]
        \arrow[d, "g"]
        & A
        \arrow[d, "f"]
        \\ X
        \arrow[r]
        \arrow[rr, bend right=25, "\id_{X}" swap]
        & Y
        \arrow[r]
        & X
      \end{tikzcd}
    \end{equation}
    e $g$ pertence a $\mathcal{W}$ (ou a $\mathcal{C}$, ou a $\mathcal{F}$), então $f$ também pertence a $\mathcal{W}$ (ou a $\mathcal{C}$, ou a $\mathcal{F}$, respectivamente).
    Em suma, as classes $\mathcal{W}$, $\mathcal{C}$ e $\mathcal{F}$ são fechadas por retrações.
    
  \item[(M4)] (Propriedade de levantamento) Dado um diagrama comutativo como abaixo,
    \begin{equation}\label{eq:diag_prob_levantamento}
      \begin{tikzcd}
        A
        \arrow[r, "\alpha"]
        \arrow[d, "i" swap]
        & X
        \arrow[d, "p"]
        \\ B
        \arrow[r, "\beta" swap]
        & Y
      \end{tikzcd}
    \end{equation}
    onde $i$ é uma cofibração, e $p$ é uma fibração; se um dos dois morfismos $i$ ou $p$ é também uma equivalência fraca, então o diagrama admite um \emph{levantamento}, ou seja, existe um morfismo diagonal $f: B \to X$ que faz comutar o diagrama abaixo.
    \begin{equation}\label{eq:diag_solucao_levantamento}
      \begin{tikzcd}
        A
        \arrow[r, "\alpha"]
        \arrow[d, "i" swap]
        & X
        \arrow[d, "p"]
        \\ B
        \arrow[r, "\beta" swap]
        \arrow[ru, dashed, "f" description]
        & Y
      \end{tikzcd}
    \end{equation}
    
  \item[(M5)] (Propriedade de fatoração) Qualquer morfismo $f: X \to Y$ em $\mathsf{M}$ pode ser na forma mostrada abaixo,
    \begin{equation}\label{eq:diag_fatoracao_cofib_fib_trivial}
      \begin{tikzcd}
        X
        \arrow[rr, "f"]
        \arrow[rd, "i" swap]
        & & Y
        \\ & Z
        \arrow[ru, "p" swap]
      \end{tikzcd}
    \end{equation}
    onde $i$ é uma cofibração, e $p$ é simultaneamente uma fibração e uma equivalência fraca.
    Além disso, todo morfismo também pode ser fatorado na forma mostrada abaixo,
    \begin{equation}\label{eq:diag_fatoracao_cofib_trivial_fib}
      \begin{tikzcd}
        X
        \arrow[rr, "f"]
        \arrow[rd, "j" swap]
        & & Y
        \\ & Z
        \arrow[ru, "q" swap]
      \end{tikzcd}
    \end{equation}
    onde nesse caso $j$ é simultaneamente uma cofibração e uma equivalência, e $q$ é uma fibração.
  \end{enumerate}
\end{defin}

Vamos introduzir um pouco de terminologia antes de fazermos alguns comentários sobre a definição acima.
Os morfismos de $\mathsf{M}$ que pertencem à classe $\mathcal{W} \cap \mathcal{C}$ são chamados de \textbf{cofibrações triviais} ou \textbf{cofibrações acíclicas}, enquanto os morfismos que pertencem à classe $\mathcal{W} \cap \mathcal{F}$ são chamados de \textbf{fibrações triviais} ou \textbf{fibrações acíclicas}.
Usando essa terminologia o axioma de fatoração (M5) pode ser enunciado da seguinte forma: todo morfismo em uma categoria de modelos pode ser fatorado como uma cofibração seguido de uma fibração trivial e também como uma cofibração trivial seguida de uma fibração.

Um quadrado comutativo tendo uma cofibração na aresta esquerda e uma fibração na aresta direita - como no diagrama \eqref{eq:diag_prob_levantamento} - será chamado de um \textbf{problema de levantamento}.
Caso um desses dois morfismos seja trivial, usaremos então o termo \textbf{problema de levantamento trivial}.
Tendo em vista essa terminologia, o axioma de levantamento (M4) pode ser enunciado da seguinte forma: em uma categoria de modelos, todo problema de levantamento trivial admite uma solução.

\begin{obs}\label{obs:axioma_retracao_explicacao}
  Lembremos que, dados objetos $X$ e $Y$ de uma categoria $\mathsf{C}$ qualquer, dizemos que $X$ é um \textbf{retrato} de $Y$ se existem morfismos $s: X \to Y$ e $r: Y \to X$ tais que $r \circ s = \id_{X}$.
  Comumente nos referimos ao morfismo $s$ por \textbf{seção} e ao morfismo $r$ por \textbf{retração}.
  A condição $r \circ s = \id_{X}$ garante que $s$ seja um monomorfismo.
  De fato, se $f,\, g: W \to X$ são morfismos tais que $s \circ f = s \circ g$, então
  \begin{displaymath}
    f
    = \id_{X} \circ f
    = r \circ s \circ f
    = r \circ s \circ g
    = \id_{X} \circ g
    = g.
  \end{displaymath}
  Isso nos permite encarar $X$ como um subobjeto de $Y$, e o morfismo $r$ então intuitivamente deforma $Y$ para esse subobjeto, mas de forma a mantê-lo fixado.
  Note que a condição $r \circ s = \id_{X}$ garante também que o morfismo $r$ seja um epimorfismo.

  A noção de retração que aparece no axioma (M3) de uma estrutura de modelos enunciado acima pode ser interpretada nesse sentido se introduzirmos uma categoria adequada para isso.
  Lembremos que toda categoria $\mathsf{C}$ dá origem a uma categoria de setas $\Arr(\mathsf{C})$.
  Os objetos dessa categorias são precisamente morfismos $f: A \to B$ na categoria incial $\mathsf{C}$, e dados dois tais objetos $f: A \to B$ e $g: X \to Y$, um morfismo do tipo $(f: A \to B) \to (g: X \to Y)$ na categoria de setas $\Arr(\mathsf{C})$ é dado por um par de morfismos $(\alpha: A \to X,\, \beta: B \to Y)$ satisfazendo a igualdade $\beta \circ f = g \circ \alpha$.
  Podemos então visualizar esse morfismo em $\Arr(\mathsf{C})$ na forma de um quadrado comutativo como mostrado abaixo.
  \begin{displaymath}
    \begin{tikzcd}
      A
      \arrow[d, "f" swap]
      \arrow[r, "\alpha"]
      & X
      \arrow[d, "g"]
      \\ B
      \arrow[r, "\beta" swap]
      & Y
    \end{tikzcd}
  \end{displaymath}
  A composição de morfismos é definida ``colando'' quadrados comutativos adjacentes.
  Mais precisamente, dados três objetos $f: X_{1} \to Y_{1},\, g: X_{2} \to Y_{2}$ e $h: X_{3} \to Y_{3}$ na categoria $\Arr(\mathsf{C})$, e dados também dois morfismos componíveis
  \begin{displaymath}
    (\alpha_{1}: X_{1} \to X_{2},\, \beta_{1}: Y_{1} \to Y_{2}) \qquad (\alpha_{2}: X_{2} \to X_{3},\, \beta_{2}: Y_{2} \to Y_{3}),
  \end{displaymath}
  sua composição é o morfismo
  \begin{displaymath}
    (\alpha_{2}, \beta_{2}) \circ (\alpha_{1},\beta_{1}): (f: X_{1} \to Y_{1}) \to (h: X_{3} \to Y_{3})
  \end{displaymath}
  em $\Arr(\mathsf{C})$ definido pelo par
  \begin{displaymath}
    (\alpha_{2},\beta_{2}) \circ (\alpha_{1},\beta_{1}) \coloneqq (\alpha_{2} \circ \alpha_{1}: X_{1} \to X_{3},\, \beta_{2} \circ \beta_{1}: Y_{1} \to Y_{3}).
  \end{displaymath}
  Essa composição pode também ser visualizada como mostrado abaixo.
  \begin{displaymath}
    \begin{tikzcd}
      X_{1}
      \arrow[d, "f" swap]
      \arrow[r, "\alpha_{1}"]
      & X_{2}
      \arrow[d, "g"]
      \arrow[r, "\alpha_{2}"]
      & X_{3}
      \arrow[d, "h"]
      \\ Y_{1}
      \arrow[r, "\beta_{1}" swap]
      & Y_{2}
      \arrow[r, "\beta_{2}" swap]
      & Y_{3}
    \end{tikzcd}
    \quad \Longrightarrow \quad
    \begin{tikzcd}
      X_{1}
      \arrow[d, "f" swap]
      \arrow[r, "\alpha_{2} \circ \alpha_{1}"]
      & X_{3}
      \arrow[d, "h"]
      \\ Y_{1}
      \arrow[r, "\beta_{2} \circ \beta_{1}" swap]
      & Y_{3}
    \end{tikzcd}
  \end{displaymath}

  A associatividade dessa composição via colagem segue diretamente da associatividade da composição na categoria inicial $\mathsf{C}$.
  Por fim, dado um objeto $f: X \to Y$ qualquer, o morfismo idêntico associado a ele é dado pelo par $\id_{f} \coloneqq (\id_{X},\id_{Y})$, conforme mostrado no quadrado comutativo abaixo.
  \begin{displaymath}
    \begin{tikzcd}
      X
      \arrow[r, "\id_{X}"]
      \arrow[d, "f" swap]
      & X
      \arrow[d, "f"]
      \\ Y
      \arrow[r, "\id_{Y}" swap]
      & Y
    \end{tikzcd}
  \end{displaymath}

  Note agora que, se o objeto $f: A \to B$ é um retrato do objeto $g: X \to Y$ \emph{na categoria de setas $\Arr(\mathsf{M})$}, então por definição existem morfismos $s_{1}: A \to X,\, s_{2}: B \to Y,\, r_{1}: X \to A$ e $r_{2}: Y \to B$ tais que $(r_{1},r_{2}) \circ (s_{1},s_{2}) = \id_{f}$, o que também pode ser expresso pelo diagrama comutativo abaixo.
  \begin{displaymath}
    \begin{tikzcd}
      A
      \arrow[r, "s_{1}"]
      \arrow[d, "f" swap]
      \arrow[rr, bend left=45, "\id_{A}"]
      & X
      \arrow[r, "r_{1}"]
      \arrow[d, "g"]
      & A
      \arrow[d, "f"]
      \\ B
      \arrow[r, "s_{2}" swap]
      \arrow[rr, bend right=45, "\id_{B}" swap]
      & Y
      \arrow[r, "r_{2}" swap]
      & B
    \end{tikzcd}
  \end{displaymath}

   Esse é precisamente o diagrama que aparece no axioma de retração na definição de uma estrutura modelo.
  Podemos então reformular tal axioma dizendo que as classes de equivalências fracas, fibrações e cofibrações são todas fechadas por \emph{retrações na categoria de setas $\Arr(\mathsf{C})$}.
\end{obs}

\begin{obs}
  Quando trabalhamos com categorias de modelos, ao invés de dizermos explicitamente que um morfismo é uma equivalência fraca, uma cofibração, ou uma fibração; é comum indicarmos isso decorando de alguma forma a seta que representa tal morfismo.
  A convenção notacional que seguiremos nesse aspecto é a seguinte:
  \begin{itemize}
  \item uma equivalência fraca será denotada por $\overset{\sim}{\rightarrow}$;
    
  \item uma cofibração será denotada por $\cofib$;
    
  \item uma fibração será denotada por $\fib$.
  \end{itemize}
  Combinando os símbolos acima obtemos outros que utilizaremos para denotar cofibrações triviais ou fibrações triviais:
  \begin{itemize}
  \item uma cofibração trivial será denotada por $\overset{\sim}{\cofib}$;
    
  \item uma fibração trivial será denotada por $\overset{\sim}{\fib}$.
  \end{itemize}
\end{obs}